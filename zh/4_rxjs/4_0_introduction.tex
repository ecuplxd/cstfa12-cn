% \section{Introduction}
\section{简介}

% RxJS is a wonderful framework to handle observable streams of items. We see its use
% in Angular as event emitters, HTTP responses and services. To become a good Angular
% developer you first have to become a good RxJS developer. It is that critical to the
% correct programming of Angular. RxJS is also the base technology for NgRx, the very
% popular state cache engine for client apps.

RxJS 是一个很棒的处理可观察流的框架。
我们可以看到在 Angular 中像 event emitters、HTTP 响应以及服务都有应用。
要成为一位优秀的 Angular 开发人员你首先不得不成为一位优秀的 RxJS 开发人员。
这对于 Angular 的正确编程至关重要。
RxJS 也是 NgRx 的基础基数,一个客户端 APP 非常流行的状态缓存引擎。

% An observable is a dual of an enumerable. With an enumerable, your code would
% call .next() repeatable to get the next item in a sequence. In contrast, with an
% observable, you supply one method to handle new data items, and optionally a
% method to handle errors and another to handle the completion event (successful end
% of items). The internals of the observable will call your code when it has data items to
% deliver to you. So you could imagine a HTTP request being sent to the server, and a
% blob of data coming back in multiple packets. As more items come in, the observable
% will call the handler in your code to process them.

可观察对象是一个可枚举对象。
使用可枚举,你的代码将调用 \mintinline{ts}{.next()} 以重复获取序列中的下一项。
相比之下,使用 observable,需提供一个函数来处理新数据项,
一个可选的方法来处理错误,以及一个方法来处理完成事件(每一项结束)。
当有新数据要推送的时候,observable 内部会调用你的代码。
所以你可以想象一个 HTTP 请求被发送到服务器,一个数据块以多个数据包的形式返回。
随着更多数据被返回,可观察对象将调用你代码中的回调函数来处理它们。
