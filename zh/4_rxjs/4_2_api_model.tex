% \section{API Model}
\section{API 模型}

% In the links below, we assume <RXJS> refers to the root directory of you RxJS
% source installation. The links below actually link to the relevant document on github.

在下面的链接中,我们假设 <RXJS> 指的是你 RxJS 的根目录位置。
下面的链接实际上链接到 github 上的相关文档。

% The main exports are in:

主要的导出在:

\begin{itemize}
  \item \href{https://github.com/ReactiveX/rxjs/blob/master/src/index.ts}
        {<RXJS>/src/index.ts}
\end{itemize}

% with a large range of operators in:

有大量的操作符:

\begin{itemize}
  \item \href{https://github.com/ReactiveX/rxjs/blob/master/src/operators/index.ts}
        {<RXJS>/src/operators/index.ts}
\end{itemize}

% Application developers who wish to use RxJS will mostly import what they need from
% these two files. There are sub-directories inside src/internal and src/operators/internal
% and as the name suggests, external developers should not be directly importing (or in
% any way depending on) these, as they may (will) change from version to version.

希望使用 RxJS 的应用开发人员大多会从这两个文件中导入他们需要的内容。
在 src/internal 和 src/operators/internal 中有子目录,顾名思义,
外部开发人员不应该直接导入(或以任何方式依赖于)这些,
因为它们可能(将)从一个版本改变到另一个版本。

% Developers with old code base that reply on older versions of RxJS should at some
% stage upgrade to RxJS 6 or later. But as an interim measure, a compatibility layer is
% available to help old code run against RxJS 6. The compat package is at:

使用旧代码库并响应旧版本 RxJS 的开发人员应该在某个阶段升级到 RxJS 6 或更高版本。
但是作为临时措施,可以使用兼容层来帮助旧代码针对 RxJS 6 运行。
compat 包位于:

\begin{itemize}
  \item \href{https://github.com/ReactiveX/rxjs/tree/master/compat}
        {<RXJS>/compat}
\end{itemize}

% If you have a large codebase it is not always convenient to immediately upgrade to
% RxJS 6, so a compat package can be very useful. You should consider use fo compat
% as a temporary measure. Over time you at first should and later must upgrade, as it is
% expected the compat package will disappear at some point in future. We do not
% discuss the compat package further here.

如果你的代码库很大,立即升级到 RxJS 6 并不总是很方便,因此 compat 包可能非常有用。
你应该考虑使用 compat 作为临时措施。
随着时间的推移,你首先应该并且后来必须升级,因为预计 compat 包将在未来的某个时候消失。
我们不在这里进一步讨论 compat 包。

% What
% \href{https://github.com/ReactiveX/rxjs/blob/master/src/index.ts}
% {<RXJS>/src/index.ts}
% exports can be subdivided into the following categories:

\href{https://github.com/ReactiveX/rxjs/blob/master/src/index.ts}
{<RXJS>/src/index.ts}
导出的内容可以细分为以下几类:

\begin{itemize}
  \item Observables
  \item Subjects (these are both observers and observables)
  \item Schedulers
  \item Subscription
  \item Notification
  \item Utils
  \item Error Types
  \item Static observable creation functions
  \item Two constants (EMPTY and NEVER)
  \item config
\end{itemize}

% So where is observer exported? Hmm … There is also one
% \texttt{export *}
% :

那么观察者在哪里导出? 嗯……还有一个导出 \texttt{export *}:

\begin{minted}{typescript}
export * from './internal/types';
\end{minted}


% and if we look at:

如果我们看看:

\begin{itemize}
  \item \href{https://github.com/ReactiveX/rxjs/blob/master/src/internal/types.ts}
        {<RXJS>/src/types.ts}
\end{itemize}

% we see
% \texttt{Observer}
% defined as an interface in there, along with definitions of operator
% interfaces, subscription interfaces, observable interfaces and a range of other
% observer interfaces – we will shortly explore them all.

我们看到 \texttt{Observer} 被定义为一个接口,
以及操作员接口、订阅接口、可观察接口和一系列其他观察者接口的定义 —— 我们将很快探索它们。
