\section{API Model}

In the links below, we assume <RXJS> refers to the root directory of you rRxJS
source installation. The links below actually link to the relevant document on github.

The main exports are in:

\begin{itemize}
  \item \href{https://github.com/ReactiveX/rxjs/blob/master/src/index.ts}
        {<RXJS>/src/index.ts}
\end{itemize}

with a large range of operators in:

\begin{itemize}
  \item \href{https://github.com/ReactiveX/rxjs/blob/master/src/operators/index.ts}
        {<RXJS>/src/operators/index.ts}
\end{itemize}

Application developers who wish to use RxJS will mostly import what they need from
these two files. There are sub-directories inside src/internal and src/operators/internal
and as the name suggests, external developers should not be directly importing (or in
any way depending on) these, as they may (will) change from version to version.

Developers with old code base that reply on older versions of RxJS should at some
stage upgrade to RxJS 6 or later. But as an interim measure, a compatibility layer is
available to help old code run against RxJS 6. The compat package is at:

\begin{itemize}
  \item \href{https://github.com/ReactiveX/rxjs/tree/master/compat}
        {<RXJS>/compat}
\end{itemize}

If you have a large codebase it is not always convenient to immediately upgrade to
RxJS 6, so a compat package can be very useful. You should consider use fo compat
as a temporary measure. Over time you at first should and later must upgrade, as it is
expected the compat package will disappear at some point in future. We do not
discuss the compat package further here.

What
\href{https://github.com/ReactiveX/rxjs/blob/master/src/index.ts}
{<RXJS>/src/index.ts}
exports can be subdivided into the following categories:

\begin{itemize}
  \item Observables
  \item Subjects (these are both observers and observables)
  \item Schedulers
  \item Subscription
  \item Notification
  \item Utils
  \item Error Types
  \item Static observable creation functions
  \item Two constants (EMPTY and NEVER)
  \item config
\end{itemize}

So where is observer exported? Hmm … There is also one
\texttt{export *}
:

\begin{minted}{typescript}
export * from './internal/types';
\end{minted}


and if we look at:

\begin{itemize}
  \item \href{https://github.com/ReactiveX/rxjs/blob/master/src/internal/types.ts}
        {<RXJS>/src/types.ts}
\end{itemize}

we see
\texttt{Observer}
defined as an interface in there, along with definitions of operator
interfaces, subscription interfaces, observable interfaces and a range of other
observer interfaces – we will shortly explore them all.
