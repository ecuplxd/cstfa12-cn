% \section{Overview}
\section{概览}

% Rendering in Angular is undergoing further evolution. It seems it was not too long ago
% that Angular 2’s original Render architecture evolved to Render2, and now along
% comes the very new Render3, quite a different approach to a view engine.

Angular 的渲染正在进行进一步的进化。
似乎还不久前 Angular2 的原始渲染架构演变为 Render2,
现在是新的 Render3,一种完全不同的视图引擎方法。

% The main change can be summed up with just this one function
% (from
% \href{https://github.com/angular/angular/blob/master/packages/core/src/render3/interfaces/renderer.ts}
% {<ANGULAR-MASTER>/packages/core/src/render3/interfaces/renderer.ts}
% ):

主要变化可以用一个函数来总结
(来自
\href{https://github.com/angular/angular/blob/master/packages/core/src/render3/interfaces/renderer.ts}
{<ANGULAR-MASTER>/packages/core/src/render3/interfaces/renderer.ts}):

\begin{minted}{typescript}
export const domRendererFactory3: RendererFactory3 = {
  createRenderer: (
    hostElement: RElement | null,
    rendererType: RendererType2 | null
  ): Renderer3 => {
    return document;
  } %\step{1}%,
};
\end{minted}


% By default, the new renderer is compatible with the
% \texttt{document}
% object from the
% standard DOM. Actually, it is better not just to say “compatible with” but to say “is”.
% When running in the browser UI thread, where the DOM is available, then the
% browser’s native
% \texttt{document}
% object (provided by the web browser itself, not Angular)
% really IS the renderer – it is directly used to render content. An app literally cannot
% perform quicker than that. Regardless of which framework you use and how it
% structures its rendering pipeline, ultimately this
% \texttt{document}
% object in the real DOM will
% have to be called. So why not call it directly in scenarios where it is supported (that
% means inside the browser UI thread)? That is exaclty what Ivy does when possible.
% For other rendering scenarios (web worker, server, or more specialized, such as
% WebRTC) then something that looks like the
% \texttt{document}
% object will need to be provided.

默认情况下,新渲染引擎与标准 DOM 中的 document 对象兼容。
其实,最好不要只说“兼容”,而要说“是”。
当在浏览器 UI 线程中运行时,DOM 可用,然后是浏览器的原生 document 对象
(由 Web 浏览器本身提供,而不是 Angular)
真的是渲染器——它直接用于渲染内容。
从字面上看,应用的执行速度不会比这更快。
无论你使用哪种框架以及它如何构建其渲染管道,最终都必须调用真实 DOM 中的这个 document 对象。
那么为什么不在支持它的场景中直接调用它(即意味着在浏览器 UI 线程内)?
这正是 Ivy 在可能的情况下所做的。
对于其他渲染场景(Web Worker、服务器或更专业的,例如 WebRTC),
则需要提供类似于 document 对象的内容。

% For readers coming from a C++, C\#, Java or similar background, it is very important
% to understand that TypeScript (and JavaScript) uses
% \url{structuralsubtyping}
% (also
% affectionately called “duck typing”) and not nominal typing (named types) as used by
% those other languages. In TypeScript, a type that implements the fields of another
% type can be used in its place – there is no neccessity for implementing common
% interfaces or to have a common parent class. So the fact that the
% \texttt{document}
% object
% from the standard DOM does not implement Angular’s
% \texttt{Render3}
% but yet is being used
% in a function to return such a type (see
% 1
% above), is not a problem, so long as it
% implements all the fields that
% \texttt{Render3}
% needs (which it does, as we shall soon
% discover).

对于来自 C++、C\#、Java 或类似背景的读者来说,
理解 TypeScript(和 JavaScript)使用
\href{https://www.typescriptlang.org/docs/handbook/interfaces.html}{结构子类型}
(也被亲切地称为“鸭子类型”)而不是那些使用的名义类型(命名类型)的其它语言非常重要。
在 TypeScript 中,可以使用实现另一种类型字段的类型来代替它 —— 没有必要实现公共接口或具有公共父类。
因此,标准 DOM 中的 document 对象没有实现 Angular 的 Render3,
但在函数中使用以返回这种类型(参见上面的 \step{1})这一事实不是问题,
只要它实现了 Render3 所需要的字段(它确实需要,我们很快就会发现)。

% Now we will explore in more depth what is happening when we use
% \texttt{enableIvy}
% with
% Angular CLI and how the main Angular project delivers Render3 functionality. We will
% see three of its packages are involved – Compiler-CLI, Compiler and Core.

现在我们将更深入地探索当我们将 \texttt{"enableIvy"} 与 Angular CLI 结合使用时会发生什么,
以及主要的 Angular 项目如何提供 Render3 功能。
我们将看到涉及到它的三个包 —— Compiler-CLI、Compiler 和 Core。

% \subsection{Public Documentation}
\subsection{公共文档}

% If we visit:

如果我们访问:

\begin{itemize}
  \item \url{https://next.angular.io/api?query=render}
\end{itemize}

% we see the following listing for all parts of the public API that contain the term
% “render”:

我们将会看到所有包含 “render” 关键字的公共 API 列表

% So for now, there is no public API to any Renderer3 functionality.

所以目前,没有任何 Renderer3 功能的公共 API。

