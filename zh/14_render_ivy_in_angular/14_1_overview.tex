\section{Overview}

% Rendering in Angular is undergoing further evolution. It seems it was not too long ago
% that Angular 2’s original Render architecture evolved to Render2, and now along
% comes the very new Render3, quite a different approach to a view engine.

% The main change can be summed up with just this one function
% (from
% \href{https://github.com/angular/angular/blob/master/packages/core/src/render3/interfaces/renderer.ts}
% {<ANGULAR-MASTER>/packages/core/src/render3/interfaces/renderer.ts}
% ):

\begin{minted}{typescript}
export const domRendererFactory3: RendererFactory3 = {
  createRenderer: (
    hostElement: RElement | null,
    rendererType: RendererType2 | null
  ): Renderer3 => {
    return document;
  } %\step{1}%,
};
\end{minted}


% By default, the new renderer is compatible with the
% \texttt{document}
% object from the
% standard DOM. Actually, it is better not just to say “compatible with” but to say “is”.
% When running in the browser UI thread, where the DOM is available, then the
% browser’s native
% \texttt{document}
% object (provided by the web browser itself, not Angular)
% really IS the renderer – it is directly used to render content. An app literally cannot
% perform quicker than that. Regardless of which framework you use and how it
% structures its rendering pipeline, ultimately this
% \texttt{document}
% object in the real DOM will
% have to be called. So why not call it directly in scenarios where it is supported (that
% means inside the browser UI thread)? That is exaclty what Ivy does when possible.
% For other rendering scenarios (web worker, server, or more specialized, such as
% WebRTC) then something that looks like the
% \texttt{document}
% object will need to be provided.

% For readers coming from a C++, C\#, Java or similar background, it is very important
% to understand that TypeScript (and JavaScript) uses
% \url{structuralsubtyping}
% (also
% affectionately called “duck typing”) and not nominal typing (named types) as used by
% those other languages. In TypeScript, a type that implements the fields of another
% type can be used in its place – there is no neccessity for implementing common
% interfaces or to have a common parent class. So the fact that the
% \texttt{document}
% object
% from the standard DOM does not implement Angular’s
% \texttt{Render3}
% but yet is being used
% in a function to return such a type (see
% 1
% above), is not a problem, so long as it
% implements all the fields that
% \texttt{Render3}
% needs (which it does, as we shall soon
% discover).

% Now we will explore in more depth what is happening when we use
% \texttt{enableIvy}
% with
% Angular CLI and how the main Angular project delivers Render3 functionality. We will
% see three of its packages are involved – Compiler-CLI, Compiler and Core.

% \subsection{Public Documentation}
\subsection{公共文档}

% If we visit:

如果我们访问:

\begin{itemize}
  \item \url{https://next.angular.io/api?query=render}
\end{itemize}

% we see the following listing for all parts of the public API that contain the term
% “render”:

我们将会看到所有包含 “render” 关键字的公共 API 列表

% So for now, there is no public API to any Renderer3 functionality.

所以目前,没有任何 Renderer3 功能的公共 API。


% \subsection{Public Documentation}
\subsection{公共文档}

% If we visit:

如果我们访问:

\begin{itemize}
  \item \url{https://next.angular.io/api?query=render}
\end{itemize}

% we see the following listing for all parts of the public API that contain the term
% “render”:

我们将会看到所有包含 “render” 关键字的公共 API 列表

% So for now, there is no public API to any Renderer3 functionality.

所以目前,没有任何 Renderer3 功能的公共 API。

