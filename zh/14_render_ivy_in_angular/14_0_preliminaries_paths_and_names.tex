% \section{Preliminaries – Paths and names}
\section{前置 - 路径和名称}

% This document is a guided tour of the new “Ivy” rendering functionality within the
% modern Angular source tree. We mostly look at the main Angular repository (paths
% within which we prefix with ANGULAR-MASTER) and also have some discussion of the
% Angular CLI repository (paths within which we prefix with ANGULAR-CLI-MASTER).
% You may wish to clone these repositories to your local machine or you may wish to
% refer to them on Github. If the latter, you should expand these placeholders to:

本章探讨现代 Angular 源码中新的渲染引擎 —— “Ivy”。
我们查看了主要的 Angular 仓库(以 ANGULAR-MASTER 为前缀的路径),
并且还讨论了 Angular CLI 仓库(以 ANGULAR-CLI-MASTER 为前缀的路径)。
你可能希望将这些仓库克隆到您的本地机器上,
或者你可能希望在 Github 上参考它们。
如果是后者,你应该将这些占位符替换为:

\begin{itemize}
  \item ANGULAR-MASTER - https://github.com/angular/angular
  \item ANGUALR-CLI-MASTER - https://github.com/angular/angular-cli
\end{itemize}

% We like to use the name “Render3” (instead of “Render 3” with a space), as it helps
% with Google search, etc. The code-name for this is “Ivy”, we will see that used in
% places in the code and with the new
% \texttt{enableIvy}
% option for Compiler-CLI, which can be
% added via Angular CLI’s
% \texttt{ng new}
% command, as described here:

我们更倾向于使用名称 “Render3”(而不是带空格的 “Render 3”),
因为它有助于 Google 搜索等。
它的代号是“Ivy”,
我们将看到它在代码中的地方使用,
并带有用于 Compiler-CLI 的新 \texttt{"enableIvy"} 选项,
可以通过 Angular CLI 的 \texttt{"ng new"} 命令添加,如下所述:

\begin{itemize}
  \item \url{https://next.angular.io/guide/ivy}
\end{itemize}
