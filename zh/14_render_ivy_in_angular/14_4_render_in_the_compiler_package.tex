\section{Render3 in The Compiler Package}

Inside the Angular Compiler package:

\begin{itemize}
  \item \href{https://github.com/angular/angular/tree/master/packages/compiler/}
        {<ANGULAR\_MASTER>/packages/compiler}
\end{itemize}

the Render3 feature resides mainly in four files. They are:

\begin{itemize}
  \item \href{https://github.com/angular/angular/blob/master/packages/compiler/src/aot/partial_module.ts}
        {<ANGULAR-MASTER>/packages/compiler/src/aot/partial\_module.ts}
  \item \href{https://github.com/angular/angular/blob/master/packages/compiler/src/aot/compiler.ts}
        {<ANGULAR-MASTER>/packages/compiler/src/aot/compiler.ts}
  \item \href{https://github.com/angular/angular/blob/master/packages/compiler/src/render3/r3_identifiers.ts}
        {<ANGULAR-MASTER>/packages/compiler/src/render3/r3\_identifiers.ts}
  \item \href{https://github.com/angular/angular/blob/master/packages/compiler/src/render3/r3_view_compiler.ts}
        {<ANGULAR-MASTER>/packages/compiler/src/render3/r3\_view\_compiler.ts}
\end{itemize}

Let’s start with partial\_module.ts. It has these few lines, to describe what a partial
module type is:

\begin{minted}{typescript}
import * as o from '../output/output_ast';

export interface PartialModule {
  fileName: string;
  statements: o.Statement[];
}
\end{minted}


We have seen from our coverage of Compiler-CLI that it makes a call to
\texttt{emitAllPartialModules}
inside the Compiler package. This is to be found in the
\url{src/aot/compiler.ts}
file and so is exported by this line from
\url{src/compiler.ts}
(yes: same
files names, different directories):

\begin{minted}{typescript}
export * from './aot/compiler';
\end{minted}


it is defined as:

\begin{minted}{typescript}
  emitAllPartialModules({
    ngModuleByPipeOrDirective,
    files,
  }: NgAnalyzedModules): PartialModule[] {
    // Using reduce like this is a select
    // many pattern (where map is a select pattern)
    return files.reduce<PartialModule[]>((r, file) => {
      r.push(
        ...this._emitPartialModule(
          file.fileName,
          ngModuleByPipeOrDirective,
          file.directives,
          file.pipes,
          file.ngModules,
          file.injectables
        )
      );
      return r;
    }, []);
  }
\end{minted}


It calls the internal
\texttt{\_emitPartialModule}
method:

\begin{minted}{typescript}
  private _emitPartialModule(
    fileName: string,
    ngModuleByPipeOrDirective: Map<StaticSymbol, CompileNgModuleMetadata>,
    directives: StaticSymbol[],
    pipes: StaticSymbol[],
    ngModules: CompileNgModuleMetadata[],
    injectables: StaticSymbol[]
  ): PartialModule[] {
    const classes: o.ClassStmt[] = [];
    const context = this._createOutputContext(fileName);
    ..
  }
\end{minted}


After initializing context information, it loops
1
over the directives array and if a
component is found
2
, calls
\texttt{compileIvyComponent}
2
, otherwise calls
\texttt{compileIvyDirective}
3
:

\begin{minted}{ts}
// Process all components and directives
1   directives.forEach(directiveType => {
      const directiveMetadata =
this._metadataResolver.getDirectiveMetadata(directiveType);
      if (directiveMetadata.isComponent) {
        ..
        const {template: parsedTemplate} =
            this._parseTemplate(
directiveMetadata, module, module.transitiveModule.directives);
2 compileIvyComponent(
context, directiveMetadata, parsedTemplate, this._reflector);
      } else {
3 compileIvyDirective(context, directiveMetadata, this._reflector);
      }
    });

    if (context.statements) {
      return [{fileName, statements:
[...context.constantPool.statements, ...context.statements]}];
    }
    return [];
  }
\end{minted}


We note the import at the top of the file:

\begin{minted}{typescript}
import {
  compileComponent as compileIvyComponent,
  compileDirective as compileIvyDirective,
} from '../render3/r3_view_compiler';
\end{minted}


So in
\url{src/render3/r3_view_compiler.ts}
let’s track
\texttt{compileComponent}
and
\texttt{compileDirective}
.

The
\url{render3sub-directory}
in the Compiler package’s src directory is new for Render3.
It contains just two files, r3\_view\_compiler and
\url{r3_identifiers.ts}
. r3\_identifiers.ts is
imported into r3\_view\_compiler.ts with this line:

\begin{minted}{typescript}
import { Identifiers as R3 } from './r3_identifiers';
\end{minted}


So anywhere in r3\_view\_compiler.ts we see “R3” being used for naming (over 50
times), it means something in r3\_identifiers.ts is being used. r3\_identifiers.ts  is not
referenced from anywhere else in the Compiler package.

r3\_identifier.ts contains a long list of external reference identifiers for the various
instructions. Here is a sampling (note that “o” is imported from output\_ast.ts):

\begin{minted}{typescript}
import * as o from '../output/output_ast';
const CORE = '@angular/core';
export class Identifiers {
  /* Methods */
  static NEW_METHOD = 'n';
  static HOST_BINDING_METHOD = 'h';

  /* Instructions */
  static createElement: o.ExternalReference = { name: 'ɵE', moduleName: CORE };
  static elementEnd: o.ExternalReference = { name: 'ɵe', moduleName: CORE };
  static text: o.ExternalReference = { name: 'ɵT', moduleName: CORE };
  static bind: o.ExternalReference = { name: 'ɵb', moduleName: CORE };
  static bind1: o.ExternalReference = { name: 'ɵb1', moduleName: CORE };
  static bind2: o.ExternalReference = { name: 'ɵb2', moduleName: CORE };
  static projection: o.ExternalReference = { name: 'ɵP', moduleName: CORE };
  static projectionDef: o.ExternalReference = { name: 'ɵpD', moduleName: CORE };
  static injectElementRef: o.ExternalReference = {
    name: 'ɵinjectElementRef',
    moduleName: CORE,
  };
  static injectTemplateRef: o.ExternalReference = {
    name: 'ɵinjectTemplateRef',
    moduleName: CORE,
  };
  static defineComponent: o.ExternalReference = {
    name: 'ɵdefineComponent',
    moduleName: CORE,
  };
  ..
}
\end{minted}


The
\texttt{compileDirective}
function is implemented in
\url{src/render3/r3_view_compiler.ts}
as:

\begin{minted}{ts}
export function compileDirective(
    outputCtx: OutputContext,
directive: CompileDirectiveMetadata,
reflector: CompileReflector) {

  const definitionMapValues:
{key: string, quoted: boolean, value: o.Expression}[] = [];

  // e.g. 'type: MyDirective`
  definitionMapValues.push(
      {key: 'type',
value: outputCtx.importExpr(directive.type.reference),
quoted: false});

  // e.g. `factory: () => new MyApp(injectElementRef())`
1 const templateFactory =
createFactory(directive.type, outputCtx, reflector);
2 definitionMapValues.push(
{key: 'factory', value: templateFactory, quoted: false});

  // e.g 'inputs: {a: 'a'}`
  if (Object.getOwnPropertyNames(directive.inputs).length > 0) {
    definitionMapValues.push(
        {key: 'inputs',
quoted: false,
value: mapToExpression(directive.inputs)});
  }

  const className = identifierName(directive.type) !;
  className || error(`Cannot resolver the name of ${directive.type}`);

  // Create the partial class to be merged with the actual class.
3 outputCtx.statements.push( 4 new o.ClassStmt(
      /* name */ className,
      /* parent */ null,
      /* fields */[new o.ClassField(
          /* name */ 'ngDirectiveDef',
          /* type */ o.INFERRED_TYPE,
          /* modifiers */[o.StmtModifier.Static],
          /* initializer */ 5 o.importExpr(R3.defineDirective).callFn(
[o.literalMap(definitionMapValues)]))],
      /* getters */[],
      /* constructorMethod */ new o.ClassMethod(null, [], []),
      /* methods */[]));
}
\end{minted}


It first
1
creates a template factory and then pushes it on the definition map values
array
2
. Then it uses the output context
3
to push a new
\texttt{Class}
statement
4
onto the
array of statements. We note the
\texttt{initializer}
is set to
\texttt{R3.defineDirective}
5
.

The
\texttt{compileComponent}
function (also in
\url{r3_view_compiler.ts}
) is a little bit more
complex. Let’s look at it in stages. Its signature is:

\begin{minted}{typescript}
export function compileComponent(
  outputCtx: OutputContext,
  component: CompileDirectiveMetadata,
  template: TemplateAst[],
  reflector: CompileReflector
) {
  const definitionMapValues: {
    key: string;
    quoted: boolean;
    value: o.Expression;
  }[] = [];
  // e.g. `type: MyApp`
  definitionMapValues.push({
    key: 'type',
    value: outputCtx.importExpr(component.type.reference),
    quoted: false,
  });
  ...
  // some code regarding selectors (omitted)
  ..
}
\end{minted}


Then it sets up a template function expression on the definition map values:

\begin{minted}{typescript}
// e.g. `factory: function MyApp_Factory()
// { return new MyApp(injectElementRef()); }`
const templateFactory = %\step{1}% createFactory(
  component.type,
  outputCtx,
  reflector
);
definitionMapValues.push({
  key: 'factory',
  value: templateFactory,
  quoted: false,
});
\end{minted}


We note the call to the
\texttt{createFactory}
function
1
, which we need to follow up in a bit.
Then it sets up a template definition builder and again adds it to the definition map
values array:

\begin{minted}{typescript}
// e.g. `template: function MyComponent_Template(_ctx, _cm) {...}`
const templateTypeName = component.type.reference.name;
const templateName = templateTypeName ? `${templateTypeName}_Template` : null;

const templateFunctionExpression = %\step{1}% new TemplateDefinitionBuilder(
  outputCtx,
  outputCtx.constantPool,
  reflector,
  CONTEXT_NAME,
  ROOT_SCOPE.nestedScope(),
  0,
  component.template!.ngContentSelectors,
  templateTypeName,
  templateName
)
  %\step{2}%
  .buildTemplateFunction(template, []);
definitionMapValues.push({
  key: 'template',
  value: templateFunctionExpression,
  quoted: false,
});
\end{minted}


We note the use of the
\texttt{TemplateDefinitionBuilder}
class
1
, and the call to its
\texttt{buildTemplateFunction}
method
2
, both of which we will examine shortly. Then it
sets up the class name (and uses the ! non-null assertion operator to ensure it is not
null):

\begin{minted}{typescript}
const className = identifierName(component.type)!;
\end{minted}


Finally it adds the new class statement:

\begin{minted}{typescript}
// Create the partial class to be merged with the actual class.
outputCtx.statements.push(
  new o.ClassStmt(
    /* name */ className,
    /* parent */ null,
    /* fields */ [
      new o.ClassField(
        /* name */ 'ngComponentDef',
        /* type */ o.INFERRED_TYPE,
        /* modifiers */ [o.StmtModifier.Static],
        /* initializer */ %\step{1}% o
          .importExpr(R3.defineComponent)
          .callFn([o.literalMap(definitionMapValues)])
      ),
    ],
    /* getters */ [],
    /* constructorMethod */ new o.ClassMethod(null, [], []),
    /* methods */ []
  )
);
\end{minted}


We note the
\texttt{initializer}
is set to
\texttt{R3.defineComponent}
1
.

The
\texttt{createFactory}
function is defined as:

\begin{minted}{typescript}
function createFactory(
  type: CompileTypeMetadata,
  outputCtx: OutputContext,
  reflector: CompileReflector
): o.FunctionExpr {
  let args: o.Expression[] = [];
  ..
}
\end{minted}


It first resolves three reflectors:

\begin{minted}{typescript}
const elementRef = reflector.resolveExternalReference(Identifiers.ElementRef);
const templateRef = reflector.resolveExternalReference(Identifiers.TemplateRef);
const viewContainerRef = reflector.resolveExternalReference(
  Identifiers.ViewContainerRef
);
\end{minted}


Then it loops through the
\texttt{type.diDeps}
dependencies, and pushes a relevant import
expression, based on the token ref:

\begin{minted}{typescript}
  for (let dependency of type.diDeps) {
    if (dependency.isValue) {
      unsupported('value dependencies');
    }
    if (dependency.isHost) {
      unsupported('host dependencies');
    }
    const token = dependency.token;
    if (token) {
      const tokenRef = tokenReference(token);
      if (tokenRef === elementRef) {
        args.push(o.importExpr(R3.injectElementRef).callFn([]));
      } else if (tokenRef === templateRef) {
        args.push(o.importExpr(R3.injectTemplateRef).callFn([]));
      } else if (tokenRef === viewContainerRef) {
        args.push(o.importExpr(R3.injectViewContainerRef).callFn([]));
      } else {
        const value =
          token.identifier != null
            ? outputCtx.importExpr(tokenRef)
            : o.literal(tokenRef);
        args.push(o.importExpr(R3.inject).callFn([value]));
      }
    } else {
      unsupported('dependency without a token');
    }
  }

  return o.fn(
    [],
    [
      new o.ReturnStatement(
        new o.InstantiateExpr(outputCtx.importExpr(type.reference), args)
      ),
    ],
    o.INFERRED_TYPE,
    null,
    type.reference.name ? `${type.reference.name}_Factory` : null
  );
\end{minted}


The
\texttt{TemplateDefinitionBuilder}
class (also located in
\url{r3_view_compiler.ts}
) is large
(350 lines+) and can be considered the heart of Render3 compilation. It implements
the
\texttt{TemplateAstVisitor}
interface. This interface is defined in:

\begin{itemize}
  \item \href{https://github.com/angular/angular/blob/master/packages/compiler/src/template_parser/template_ast.ts}
        {<ANGULAR-MASTER>/packages/compiler/src/template\_parser/template\_ast.ts}
\end{itemize}

as follows:

\begin{minted}{typescript}
// A visitor for {@link TemplateAst} trees that will process each node.
export interface TemplateAstVisitor {
  visit?(ast: TemplateAst, context: any): any;
  visitNgContent(ast: NgContentAst, context: any): any;
  visitEmbeddedTemplate(ast: EmbeddedTemplateAst, context: any): any;
  visitElement(ast: ElementAst, context: any): any;
  visitReference(ast: ReferenceAst, context: any): any;
  visitVariable(ast: VariableAst, context: any): any;
  visitEvent(ast: BoundEventAst, context: any): any;
  visitElementProperty(ast: BoundElementPropertyAst, context: any): any;
  visitAttr(ast: AttrAst, context: any): any;
  visitBoundText(ast: BoundTextAst, context: any): any;
  visitText(ast: TextAst, context: any): any;
  visitDirective(ast: DirectiveAst, context: any): any;
  visitDirectiveProperty(ast: BoundDirectivePropertyAst, context: any): any;
}
\end{minted}


Back in
\url{r3_view_compiler.ts}
, the definition of
\texttt{TemplateDefinitionBuilder}
begins
with:

\begin{minted}{typescript}
class TemplateDefinitionBuilder implements TemplateAstVisitor, LocalResolver {
  constructor(
    private outputCtx: OutputContext,
    private constantPool: ConstantPool,
    private reflector: CompileReflector,
    private contextParameter: string,
    private bindingScope: BindingScope,
    private level = 0,
    private ngContentSelectors: string[],
    private contextName: string | null,
    private templateName: string | null
  ) {}
  ..
}
\end{minted}


We saw the call to
\texttt{buildTemplateFunction}
early in
\texttt{compileComponent}
– its has this
signature:

\begin{minted}{typescript}
  buildTemplateFunction(
    asts: TemplateAst[],
    variables: VariableAst[]
  ): o.FunctionExpr {
    ..
  }
\end{minted}


It returns an instance of
\texttt{o.FunctionExpr}
. We note the import at the top of the file:

\begin{minted}{typescript}
import * as o from '../output/output_ast';
\end{minted}


So
\texttt{o.FunctionExpr}
means the
\texttt{FunctionExpr}
class in:

\begin{itemize}
  \item \href{https://github.com/angular/angular/blob/master/packages/compiler/src/output/output_ast.ts}
        {<ANGULAR-MASTER>/packages/compiler/src/output/output\_ast.ts}
\end{itemize}

This class is defined as follows:

\begin{minted}{typescript}
export class FunctionExpr extends Expression {
  constructor(
    public params: FnParam[],
    public statements: Statement[],
    type?: Type | null,
    sourceSpan?: ParseSourceSpan | null,
    public name?: string | null
  ) {
    super(type, sourceSpan);
  }
  ...
}
\end{minted}


While we are looking at output\_ast.ts, we see this
\texttt{fn}
function:

\begin{minted}{typescript}
export function fn(
  params: FnParam[],
  body: Statement[],
  type?: Type | null,
  sourceSpan?: ParseSourceSpan | null,
  name?: string | null
): FunctionExpr {
  return new FunctionExpr(params, body, type, sourceSpan, name);
}
\end{minted}


It just makes a
\texttt{FunctionExpr}
from the supplied parameters.

An interesting function in
\url{src/template_parser/template_ast.ts}
is
\texttt{templateVisitAll}
:

\begin{minted}{typescript}
/**
 * Visit every node in a list of {@link TemplateAst}s with the given
 * {@link TemplateAstVisitor}.
 */
export function templateVisitAll(
  visitor: TemplateAstVisitor,
  asts: TemplateAst[],
  context: any = null
): any[] {
  const result: any[] = [];
  const visit = visitor.visit
    ? (ast: TemplateAst) =>
        visitor.visit!(ast, context) || ast.visit(visitor, context)
    : (ast: TemplateAst) => ast.visit(visitor, context);
  asts.forEach((ast) => {
    const astResult = visit(ast);
    if (astResult) {
      result.push(astResult);
    }
  });
  return result;
}
\end{minted}


Now let’s return to the critically important
\texttt{buildTemplateFunction}
method of
\texttt{TemplateDefinitionBuilder}
in
\url{r3_view_compiler.ts}
– a summary of its definition is:

\begin{minted}{typescript}
  buildTemplateFunction(
    asts: TemplateAst[],
    variables: VariableAst[]
  ): o.FunctionExpr {
    // Create variable bindings
    ...
    // Collect content projections
    ...
    %\step{1}% templateVisitAll(this, asts);
    ..
    %\step{2}% return o.fn(
      [
        new o.FnParam(this.contextParameter, null),
        %\step{3}%
        new o.FnParam(CREATION_MODE_FLAG, o.BOOL_TYPE),
      ],
      [
        %\step{4}% // Temporary variable declarations (i.e. let _t: any;)
        ...this._prefix,

        // Creating mode (i.e. if (cm) { ... })
        ...creationMode,

        // Binding mode (i.e. ɵp(...))
        ...this._bindingMode,

        // Host mode (i.e. Comp.h(...))
        ...this._hostMode,

        // Refresh mode (i.e. Comp.r(...))
        ...this._refreshMode,

        // Nested templates (i.e. function CompTemplate() {})
        ...this._postfix,
      ],
      %\step{5}% o.INFERRED_TYPE,
      null,
      this.templateName
    );
  }
\end{minted}


We see it first visits the template tree
1
. Then it returns the result of a call
2
to the
\texttt{fn}
function we just looked at, passing in three entries –
3
an array of
\texttt{FnParams}
,
4
an
array of statements and
5
\texttt{o.INFERRED\_TYPE}
. What is happening here is that each
node in the template tree is being visited, and where appropriate, statements are
being emitted to the output statement array with the correct Render3 instruction. The
instruction function is used to add a statement like so:

\begin{minted}{typescript}
  private instruction(
    statements: o.Statement[],
    span: ParseSourceSpan | null,
    reference: o.ExternalReference,
    ...params: o.Expression[]
  ) {
    statements.push(
      o.importExpr(reference, null, span).callFn(params, span).toStmt()
    );
  }
\end{minted}


For example, when a text node is visited, the Render3 text instruction (
\texttt{R3.text}
)
should be emitted. We see this happening with the
\texttt{visitText}
method:

\begin{minted}{typescript}
  private _creationMode: o.Statement[] = [];

  visitText(ast: TextAst) {
    // Text is defined in creation mode only.
    this.instruction(
      this._creationMode,
      ast.sourceSpan,
      R3.text,
      o.literal(this.allocateDataSlot()),
      o.literal(ast.value)
    );
  }
\end{minted}


There is an equivalent method for elements,
\texttt{visitElement}
, which is somewhat more
complex. After some setup code, it has this:

\begin{minted}{typescript}
// Generate the instruction create element instruction
this.instruction(
  this._creationMode,
  ast.sourceSpan,
  R3.createElement,
  ...parameters
);
\end{minted}


There is also a
\texttt{visitEmbeddedTemplate}
method, which emits a number of Render3
instructions:

\begin{minted}{typescript}
  visitEmbeddedTemplate(ast: EmbeddedTemplateAst) {
    ...
    // e.g. C(1, C1Template)
    this.instruction(
      this._creationMode,
      ast.sourceSpan,
      R3.containerCreate,
      o.literal(templateIndex),
      directivesArray,
      o.variable(templateName)
    );

    // e.g. Cr(1)
    this.instruction(
      this._refreshMode,
      ast.sourceSpan,
      R3.containerRefreshStart,
      o.literal(templateIndex)
    );

    // Generate directives
    this._visitDirectives(
      ast.directives,
      o.variable(this.contextParameter),
      templateIndex,
      directiveIndexMap
    );

    // e.g. cr();
    this.instruction(this._refreshMode, ast.sourceSpan, R3.containerRefreshEnd);

    // Create the template function
    const templateVisitor = new TemplateDefinitionBuilder(
      this.outputCtx,
      this.constantPool,
      this.reflector,
      templateContext,
      this.bindingScope.nestedScope(),
      this.level + 1,
      this.ngContentSelectors,
      contextName,
      templateName
    );
    const templateFunctionExpr = templateVisitor.buildTemplateFunction(
      ast.children,
      ast.variables
    );
    this._postfix.push(templateFunctionExpr.toDeclStmt(templateName, null));
  }
\end{minted}

