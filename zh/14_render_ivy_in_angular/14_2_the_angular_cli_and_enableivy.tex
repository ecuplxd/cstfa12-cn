\section{The Angular CLI And enableIvy}

% Application developers wishing to use Ivy will mostly do so via the
% \emph{enableIvy}
% command line option to Angular CLI:

\begin{itemize}
  \item \url{https://next.angular.io/guide/ivy}
\end{itemize}

% We can see from the Angular CLI 8.2 source tree this option impacts the code base in
% a few places. The schema for Angular Application options is defined here:

\begin{itemize}
  \item \href{https://github.com/angular/angular-cli/blob/master/packages/schematics/angular/application/schema.json}
        {<ANGULAR-CLI-MASTER>/packages/schematics/angular/application/schema.json}
\end{itemize}

% and includes this for
% \texttt{enableIvy}
% :

\begin{minted}{json}
{
  "$schema": "http://json-schema.org/schema",
  "id": "SchematicsAngularApp",
  "title": "Angular Application Options Schema",
  "type": "object",
  "description": "Generates a new basic app definition in the \"projects\" subfolder of the workspace.",
  "properties": {
    "enableIvy": {
      "description": "**EXPERIMENTAL** True to create a new app that uses the Ivy rendering engine.",
      "type": "boolean",
      "default": false,
      "x-user-analytics": 8
\end{minted}


% The entry point for Angular Application schematics:

\begin{itemize}
  \item \href{https://github.com/angular/angular-cli/blob/master/packages/schematics/angular/application/index.ts}
        {<ANGULAR-CLI-MASTER>/packages/schematics/angular/application/index.ts}
\end{itemize}

% has this code :

\begin{minted}{typescript}
const project = {
  root: normalize(projectRoot),
  sourceRoot,
  projectType: ProjectType.Application,
  prefix: options.prefix || 'app',
  schematics,
  targets: {
    build: {
      builder: Builders.Browser,
      options: {
        ..
        aot: !!options.enableIvy,
        ..
      }, ..
    }, ..
  }, ..
};
\end{minted}


% The schema for Angular Ng New options is defined here:
% enableIvy option set
% in tsconfig.app.json

\begin{itemize}
  \item \href{https://github.com/angular/angular-cli/blob/master/packages/schematics/angular/ng-new/schema.json}
        {<ANGULAR-CLI-MASTER>/packages/schematics/angular/ng-new/schema.json}
\end{itemize}

% and has these entries:

\begin{minted}{ts}
 {
  "$schema": "http://json-schema.org/schema",
  "id": "SchematicsAngularNgNew",
  "title": "Angular Ng New Options Schema",
  "type": "object",
  "properties": {
    "enableIvy": {
      "description":
          "When true, creates a new app that uses the Ivy rendering engine.",
      "type": "boolean",
      "default": false
    },
\end{minted}


% The description string from above ends up here in the generated documentation:

\begin{itemize}
  \item \url{https://next.angular.io/cli/new}
\end{itemize}

% The
% \texttt{ng-new}
% entrypoint:

\begin{itemize}
  \item \href{https://github.com/angular/angular-cli/blob/master/packages/schematics/angular/ng-new/index.ts}
        {<ANGULAR-CLI-MASTER>/packages/schematics/angular/ng-new/index.ts}
\end{itemize}

% accepts the
% \texttt{enableIvy}
% parameter as follows:

\begin{minted}{typescript}
const applicationOptions: ApplicationOptions = {
  projectRoot: '',
  name: options.name,
  enableIvy: options.enableIvy,
  ...
  ..
};
\end{minted}


% The template for tsconfig.app.json reacts to
% \texttt{enableIvy}
% if present:

\begin{itemize}
  \item \href{https://github.com/angular/angular-cli/blob/master/packages/schematics/angular/application/files/tsconfig.app.json.template}
        {<ANGULAR-CLI-MASTER>/packages/schematics/angular/application/files/tsconfig.app.json.template}
\end{itemize}

% as it has this entry:

\begin{minted}{shell}
<%\%% if (enableIvy) { %\%%>,
  "angularCompilerOptions": {
    "enableIvy": true
  }<%\%% } %\%%>
\end{minted}


% So that is how the entry gets into tsconfig.app.json. Now let’s see what impact it has.
% The ngtools functionality for webpack configures the bootstrap code slightly differently
% when
% \texttt{enableIvy}
% is enabled. In this file:

\begin{itemize}
  \item \href{https://github.com/angular/angular-cli/blob/master/packages/ngtools/webpack/src/transformers/replace_bootstrap.ts}
        {<ANGULAR-CLI-MASTER>/packages/ngtools/webpack/src/transformers/replace\_bootstrap.ts}
\end{itemize}

% we see:

\begin{minted}{ts}
export function replaceBootstrap(
  ..,
  enableIvy?: boolean,
): ts.TransformerFactory<ts.SourceFile> {
..
if (!enableIvy) {
        className += 'NgFactory';
        modulePath += '.ngfactory';
        bootstrapIdentifier = 'bootstrapModuleFactory';
      }
  ..
}
\end{minted}


% So when
% \texttt{enableIvy}
% is NOT present, the names for the factory artefacts are different.
% From where does
% \texttt{replaceBootstrap}
% get called? When we examine:

\begin{itemize}
  \item \href{https://github.com/angular/angular-cli/blob/master/packages/ngtools/webpack/src/angular_compiler_plugin.ts}
        {<ANGULAR-CLI-MASTER>/packages/ngtools/webpack/src/angular\_compiler\_plugin.ts}
\end{itemize}

% we see the
% \texttt{\_makeTransformers}
% method is as follows:

\begin{minted}{typescript}
  private _makeTransformers() {
    ..
    if (this._platformTransformers !== null) {
      this._transformers.push(...this._platformTransformers);
    } else {
      if (this._platform === PLATFORM.Browser) {
        ..
        if (!this._JitMode) {
          // Replace bootstrap in browser AOT.
          this._transformers.push(
            replaceBootstrap(
              isAppPath,
              getEntryModule,
              getTypeChecker,
              !!this._compilerOptions.enableIvy
            )
          );
        }
      } else if (this._platform === PLATFORM.Server) {
        ..
      }
    } ..
  }
\end{minted}


% We also see ivy used in
% \texttt{\_processLazyRoutes}
% :

\begin{minted}{typescript}
  // Process the lazy routes discovered, adding then to _lazyRoutes.
  // TODO: find a way to remove lazy routes that don't exist anymore.
  // This will require a registry of known references to a lazy route,
  // removing it when no
  // module references it anymore.
  private _processLazyRoutes(discoveredLazyRoutes: LazyRouteMap) {
    Object.keys(discoveredLazyRoutes).forEach((lazyRouteKey) => {
      ..
      if (
        this._JitMode ||
        // When using Ivy and not using allowEmptyCodegenFiles,
        // factories are not generated.
        (this._compilerOptions.enableIvy &&
          !this._compilerOptions.allowEmptyCodegenFiles)
      ) {
        modulePath = lazyRouteTSFile;
        moduleKey = `${lazyRouteModule}${moduleName ? '#' + moduleName : ''}`;
      } else {
        ..
      }
    });
  }
\end{minted}


% We also see ivy impacting on how
% \texttt{\_createOrUpdateProgram}
% works:

\begin{minted}{typescript}
  private async _createOrUpdateProgram() {
    ..
    if (!this.entryModule && !this._compilerOptions.enableIvy) {
      this._warnings.push(
        'Lazy routes discovery is not enabled. ' +
          'Because there is neither an entryModule nor a ' +
          'statically analyzable bootstrap code in the main file.'
      );
    }
    ..
  }
\end{minted}


\subsection{Impact enableIvy has on Angular CLI’s code generation}

% To best see the differences the
% \texttt{enableIvy}
% option has on code generation, we will now
% create two new projects – one with and one without the
% \texttt{--enableIvy}
% option. To save
% us some time we will use the
% \texttt{–-skipInstall}
% option, which means npm install is not
% run to download all the dependency packages.

\begin{minted}{ts}
ng new render3 --enableIvy --skipInstall
ng new render2 --skipInstall
\end{minted}


% A search for
% \texttt{ivy}
% in the generated render2 codebase reveals no hits, as expected. A
% search for
% \texttt{ivy}
% in the render3 codebase reveals 2 hits. In package.json,
% \texttt{scripts}
% has
% this additional item:

\begin{minted}{json}
  "scripts": {
    "postinstall": "ivy-ngcc",
    ..
  },
\end{minted}


% and in tsconfig.app.json,
% \texttt{angularCompilerOptions}
% has this entry:

\begin{minted}{json}
  "angularCompilerOptions": {
    "enableIvy": true
  }
\end{minted}


% Now that we have seen how Angular CLI adds enableIvy, we are ready to move on to
% explore how Compiler CLI detects and reacts to this.


\subsection{Impact enableIvy has on Angular CLI’s code generation}

% To best see the differences the
% \texttt{enableIvy}
% option has on code generation, we will now
% create two new projects – one with and one without the
% \texttt{--enableIvy}
% option. To save
% us some time we will use the
% \texttt{–-skipInstall}
% option, which means npm install is not
% run to download all the dependency packages.

\begin{minted}{ts}
ng new render3 --enableIvy --skipInstall
ng new render2 --skipInstall
\end{minted}


% A search for
% \texttt{ivy}
% in the generated render2 codebase reveals no hits, as expected. A
% search for
% \texttt{ivy}
% in the render3 codebase reveals 2 hits. In package.json,
% \texttt{scripts}
% has
% this additional item:

\begin{minted}{json}
  "scripts": {
    "postinstall": "ivy-ngcc",
    ..
  },
\end{minted}


% and in tsconfig.app.json,
% \texttt{angularCompilerOptions}
% has this entry:

\begin{minted}{json}
  "angularCompilerOptions": {
    "enableIvy": true
  }
\end{minted}


% Now that we have seen how Angular CLI adds enableIvy, we are ready to move on to
% explore how Compiler CLI detects and reacts to this.

