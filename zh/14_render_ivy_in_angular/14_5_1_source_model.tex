\subsection{Source Model}

% The source tree for the Render3 feature directly contains these source files:

\begin{itemize}
  \item assert.ts
  \item component.ts
  \item definition.ts
  \item di.ts
  \item hooks.ts
  \item index.ts
  \item instructions.ts
  \item ng\_dev\_mode.ts
  \item node\_assert.ts
  \item node\_manipulation.ts
  \item node\_selector\_matcher.ts
  \item object\_literal.ts
  \item pipe.ts
  \item query.ts
  \item util.ts
\end{itemize}

% and these documents:

\begin{itemize}
  \item perf\_notes.md
  \item TREE\_SHAKING.md
\end{itemize}

% It also has one sub-directory, interfaces, which contains these files:

\begin{itemize}
  \item container.ts
  \item definition.ts
  \item injector.ts
  \item node.ts
  \item projection.ts
  \item query.ts
  \item renderer.ts
  \item view.ts
\end{itemize}

% It could be said Render3 is a re-imagining of what rendering means for an Angular
% application. The principal change is that the DOM comes back as the main API that is
% used to render, and the idea of custom renderers goes away. In scenarios where the
% DOM does not exist (such as a web worker or on the server), then a polyfill DOM will
% be needed.

% In some pieces of Render3 code, we see use of the ‘L’ and ‘R’ prefixes. This is
% explained in a comment in the source:

\begin{minted}{md}
The "L" stands for "Logical" to differentiate between `RNodes` (actual
rendered DOM node) and our logical representation of DOM nodes, `LNodes`.
<ANGULAR-MASTER>/packages/core/src/render3/interfaces/node.ts
\end{minted}

