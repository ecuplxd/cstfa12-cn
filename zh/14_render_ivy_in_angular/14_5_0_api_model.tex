\subsection{API Model}

% There is no public API to Render3. The
% \url{Corepackage}
% contains the Render3 (and
% Render2) code. Its
% \url{index.ts}
% file just exports the contents of
% \url{public_api.ts}
% , which in
% turn exports the contents of
% \url{./src/core.ts}
% .

% Regarding the public API, this has one render-related line, an export of:

\begin{minted}{typescript}
export * from './render';
\end{minted}


% The
% \url{./src/render.ts}
% file exports no Render3 API. It does export the Render2 API, like
% so:

\input{14_render_ivy_in_angular/code/14_5_0_1.tex}

% Note that Render2 is just the
% \url{./src/render/api.ts}
% file with less than 200 lines of code
% (the
% \url{core/src/render}
% sub-directory only contains that one file)- it defines the above
% types but does not contain an implementation. You can read it in full here:

\begin{itemize}
  \item \href{https://github.com/angular/angular/blob/master/packages/core/src/render/api.ts}
        {<ANGULAR-MASTER>/packages/core/src/render/api.ts}
\end{itemize}

% Render3 does have a private API. The ./src/core.ts file contain this line:

\begin{minted}{typescript}
export * from './core_render3_private_export';
\end{minted}


% file has this:

\input{14_render_ivy_in_angular/code/14_5_0_3.tex}

% Private APIs are intended for use by other Angular packages and not by regular
% Angular applications. Hence the Greek theta character (‘ ’) is added as a prefix to
% ɵ
% private APIs, as in common with other such private APIs within Angular.

% The reason for the many very short type names is that the Angular Compiler will be
% generating lots of source code based on Render3 for your application’s Angular
% template files and it is desirable to have this as compact as possible, without the need
% to run a minifier. Typically no human reads this generated code, so compactness is
% desired rather than readability.

% If we examine:

\begin{itemize}
  \item \href{https://github.com/angular/angular/blob/master/packages/core/src/render3/index.ts}
        {<ANGULAR-MASTER>/packages/core/src/Render3/index.ts}
\end{itemize}

% we see it starts by explaining the naming scheme:

\begin{minted}{typescript}
// Naming scheme:
// - Capital letters are for creating things:
// T(Text), E(Element), D(Directive), V(View),
// C(Container), L(Listener)
// - lower case letters are for binding: b(bind)
// - lower case letters are for binding target:
//     p(property), a(attribute), k(class), s(style), i(input)
// - lower case letters for guarding life cycle hooks: l(lifeCycle)
// - lower case for closing: c(containerEnd), e(elementEnd), v(viewEnd)
\end{minted}


% Then it has a long list of exports of instructions, many with abbreviations:

\input{14_render_ivy_in_angular/code/14_5_0_5.tex}

% Each of the one- or two-letter exports corresponds to an instruction in
% \url{.src/Render3/instructions.ts}
% . In the following diagram we give the short export name
% and the full name, which more clearly explains the intent of the instruction. We have
% seen how Core’s Render3 is being used by the compiler and compiler-cli packages:

\begin{itemize}
  \item \href{https://github.com/angular/angular/tree/master/packages/compiler/src/render3}
        {<ANGULAR-MASTER>/packages/compiler/src/render3}
  \item \href{https://github.com/angular/angular/blob/master/packages/compiler-cli/src/transformers/r3_transform.ts}
        {<ANGULAR-MASTER>/packages/compiler-cli/src/transformers/r3\_transform.ts}
\end{itemize}

% It is not used by the Router or the platform- packages (which do use  Render2).
