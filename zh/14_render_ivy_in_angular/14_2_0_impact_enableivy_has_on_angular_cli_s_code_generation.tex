% \subsection{Impact enableIvy has on Angular CLI’s code generation}
\subsection{enableIvy 对 Angular CLI 生成代码的影响}

% To best see the differences the
% \texttt{enableIvy}
% option has on code generation, we will now
% create two new projects – one with and one without the
% \texttt{--enableIvy}
% option. To save
% us some time we will use the
% \texttt{–-skipInstall}
% option, which means npm install is not
% run to download all the dependency packages.

为了查看 \texttt{"enableIvy"} 选项对代码生成的不同,
我们将创建两个新项目 —— 一个启用,另一个不启用 \texttt{"--enableIvy"} 选项。
为了节省时间,我们将使用 \texttt{"–-skipInstall"} 选项,这意味着 npm install 不会下载依赖项。

\begin{minted}{ts}
ng new render3 --enableIvy --skipInstall
ng new render2 --skipInstall
\end{minted}


% A search for
% \texttt{ivy}
% in the generated render2 codebase reveals no hits, as expected. A
% search for
% \texttt{ivy}
% in the render3 codebase reveals 2 hits. In package.json,
% \texttt{scripts}
% has
% this additional item:

在 render2 代码搜索不到 \texttt{"ivy"},这符合预期。
在 render3 代码中能搜到到 2 个 \texttt{"ivy"}。
在 package.json 中,\texttt{"scripts"} 有额外的字段:

\begin{minted}{json}
  "scripts": {
    "postinstall": "ivy-ngcc",
    ..
  },
\end{minted}


% and in tsconfig.app.json,
% \texttt{angularCompilerOptions}
% has this entry:

在 tsconfig.app.json 中 \texttt{"angularCompilerOptions"},有如下字段:

\begin{minted}{json}
  "angularCompilerOptions": {
    "enableIvy": true
  }
\end{minted}


% Now that we have seen how Angular CLI adds enableIvy, we are ready to move on to
% explore how Compiler CLI detects and reacts to this.

现在我们已经看到 Angular CLI 是如何添加 enableIvy,
我们已准备好继续探索 Compiler CLI 如何检测和对此作出反应。
