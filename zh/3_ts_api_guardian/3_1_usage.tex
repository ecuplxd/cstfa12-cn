% \section{Usage}
\section{用法}

% It is used in the Angular build to check for changes to the Angular public API and to
% ensure that inadvertent changes to the public API are detected. Specifically, it you
% examine gulpfile.ts in the main Angular project:

它用于在构建 Angular 时以检查 Angular 公有 API 的更改
确保检测到对公共 API 的无意中更改。具体来说,你
检查 Angular 主项目中的 gulpfile.ts:

\begin{itemize}
  \item \href{https://github.com/angular/angular/blob/master/gulpfile.js}
        {<ANGULAR-MASTER>/gulpfile.js}
\end{itemize}

% you will see it has these two lines:

你会看到它有这样两行:

\begin{minted}{typescript}
gulp.task('public-api:enforce', loadTask('public-api', 'enforce'));
gulp.task('public-api:update', ['build.sh'], loadTask('public-api', 'update'));
\end{minted}


% and when we look at:

我们看向:

\begin{itemize}
  \item \href{https://github.com/angular/angular/blob/master/tools/gulp-tasks/public-api.js}
        {<ANGULAR-MASTER>/tools/gulp-tasks/public-api.js}
\end{itemize}

% we see two tasks named 'public-api:enforce' and 'public-api:update' and In here we
% see how ts-api-guardian is used, to
% 1
% ensure it has not been unexpectedly changed
% and to
% 2
% generate a “golden file” representing the API:

会看到名为 'public-api:enforce' 和 'public-api:update' 的两个任务,
在这里我们会看到 ts-api-guardian 是如何被使用的。\step{1} 确保没有意外的改动,
\step{2} 生成一个 “golden file” 代表 API:

\begin{minted}{ts}
  // Enforce that the public API matches the golden files
  // Note that these two commands work on built d.ts files
  // instead of the source
 1 enforce: (gulp) => (done) => {
    const platformScriptPath = require('./platform-script-path');
    const childProcess = require('child_process');
    const path = require('path');

    childProcess
        .spawn(path.join(__dirname,
           platformScriptPath(`../../node_modules/.bin/ts-api-guardian`)),
            ['--verifyDir', path.normalize(publicApiDir)]
              .concat(publicApiArgs), {stdio: 'inherit'})
        .on('close', (errorCode) => {
          if (errorCode !== 0) {
            done(new Error(
              'Public API differs from golden file. ‘ +
              ‘Please run `gulp public-api:update`.'));
          } else {
            done();
          }
        });
  },

  // Generate the public API golden files
 2 update: (gulp) => (done) => {
    const platformScriptPath = require('./platform-script-path');
    const childProcess = require('child_process');
    const path = require('path');

    childProcess
        .spawn(
            path.join(__dirname, platformScriptPath(
               `../../node_modules/.bin/ts-api-guardian`)),
            ['--outDir', path.normalize(publicApiDir)].concat(publicApiArgs),
             {stdio: 'inherit'})
        .on('close', done);
  }
\end{minted}

