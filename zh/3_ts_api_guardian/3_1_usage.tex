\section{Usage}

It is used in the Angular build to check for changes to the Angular public API and to
ensure that inadvertent changes to the public API are detected. Specifically, it you
examine gulpfile.ts in the main Angular project:

\begin{itemize}
  \item \href{https://github.com/angular/angular/blob/master/gulpfile.js}
        {<ANGULAR-MASTER>/gulpfile.js}
\end{itemize}

you will see it has these two lines:

\begin{minted}{typescript}
gulp.task('public-api:enforce', loadTask('public-api', 'enforce'));
gulp.task('public-api:update', ['build.sh'], loadTask('public-api', 'update'));
\end{minted}


and when we look at:

\begin{itemize}
  \item \href{https://github.com/angular/angular/blob/master/tools/gulp-tasks/public-api.js}
        {<ANGULAR-MASTER>/tools/gulp-tasks/public-api.js}
\end{itemize}

we see two tasks named 'public-api:enforce' and 'public-api:update' and In here we
see how ts-api-guardian is used, to
1
ensure it has not been unexpectedly changed
and to
2
generate a “golden file” representing the API:

\begin{minted}{typescript}
    // Enforce that the public API matches the golden files
    // Note that these two commands work on built d.ts files
    // instead of the source
  %\step{1}% enforce: (gulp) => (done) => {
    const platformScriptPath = require('./platform-script-path');
    const childProcess = require('child_process');
    const path = require('path');

    childProcess
      .spawn(
        path.join(
          __dirname,
          platformScriptPath(`../../node_modules/.bin/ts-api-guardian`)
        ),
        ['--verifyDir', path.normalize(publicApiDir)].concat(publicApiArgs),
        { stdio: 'inherit' }
      )
      .on('close', (errorCode) => {
        if (errorCode !== 0) {
          done(
            new Error(
              'Public API differs from golden file. Please run `gulp public-api:update`.'
            )
          );
        } else {
          done();
        }
      });
  },
  // Generate the public API golden files
  %\step{2}% update: (gulp) => (done) => {
    const platformScriptPath = require('./platform-script-path');
    const childProcess = require('child_process');
    const path = require('path');

    childProcess
      .spawn(
        path.join(
          __dirname,
          platformScriptPath(`../../node_modules/.bin/ts-api-guardian`)
        ),
        ['--outDir', path.normalize(publicApiDir)].concat(publicApiArgs),
        { stdio: 'inherit' }
      )
      .on('close', done);
  },
\end{minted}

