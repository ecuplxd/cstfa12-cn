% \section{lib}
\section{lib}

% The lib sub-directory contains these files:

lib 子目录包含这些文件:

% \begin{itemize}
%   \item cli.ts – command-line interface, processes argument list and invokes commands
%   \item main.ts – main logic for generating and verifying golden files
%   \item serializer.ts – code to serialize an API (to create the contents of a golden file)
% \end{itemize}

\begin{itemize}
  \item cli.ts —— 命令行接口,处理参数列表和包装命令
  \item main.ts —— 生成和验证 golden 文件主逻辑
  \item serializer.ts —— 将代码序列化成 API(用于创建 golden 文件的内容)
\end{itemize}

% A golden file is a textual representation of an API and the two key tasks of ts-api-
% guardian is to either create or verify golden files based on supplied command line
% arguments.

golden 文件是 API 的文本表示,
ts-api-guardian 的两个关键任务是根据提供的命令行参数创建或验证 golden 文件

% Cli.ts starts with some useful comments about how to call ts-api-guardian:

Cli.ts 以一些关于如何调用 ts-api-guardian 的有用注释开始:

\input{../output/3_ts_api_guardian/code/3_4_0.tex}

% cli.ts accepts the following command line options:

cli.ts 接受以下命令行选项:

\begin{minted}{ts}
--help                          Show this usage message
        --out <file>            Write golden output to file
        --outDir <dir>          Write golden file structure to directory
        --verify <file>         Read golden input from file
        --verifyDir <dir>       Read golden file structure from directory
        --rootDir <dir>         Specify the root directory of input files
        --stripExportPattern <regexp>
                                 Do not output exports matching the pattern
        --allowModuleIdentifiers <identifier>
                                 Whitelist identifier for "* as foo" imports
        --onStabilityMissing <warn|error|none>
                    Warn or error if an export has no stability annotation`);
\end{minted}


% The Angular API allows annotations to be attached to each API indicating whether it is
% stable, deprecated or experiemental. The
% \texttt{onStabilityMissing}
% option indicates what
% action is required if such an annotation is missing. The
% \texttt{startCli()}
% function parses
% the command line and initializes an instance of
% \texttt{SerializationOptions}
% , and then for
% generation mode calls
% \texttt{generateGoldenFile()}
% or for verification mode calls
% \texttt{verifyAgainstGoldenFile()}
% - both are in main.ts and are actually quite short
% functions:

Angular API 允许将注释附加到每个 API 上,表明它是稳定的、已弃用的还是实验性的。
\texttt{onStabilityMissing} 选项指示如果缺少这样的注释需要什么操作。
\texttt{startCli()} 函数解析命令行并初始化 \texttt{SerializationOptions} 的实例,
然后对于生成模式调用 \texttt{SerializationOptions} 或对于验证模式调用
\texttt{verifyAgainstGoldenFile()} —— 两者都在 main.ts 中,
实际上是非常短的函数:

\begin{minted}{typescript}
export function generateGoldenFile(
  entrypoint: string,
  outFile: string,
  options: SerializationOptions = {}
): void {
  const output = publicApi(entrypoint, options);
  ensureDirectory(path.dirname(outFile));
  fs.writeFileSync(outFile, output);
}
\end{minted}


% generateGoldenFile calls publicApi (from Serializer.ts) to generate the contents of the
% golden file and then writes it to a file.  VerifyAgainstGoldenFile() also calls publicApi
% and saves the result in a string called actual, and then loads the existing golden file
% data into a string called expected, and then compares then. If then are different, it
% calls createPatch (from the diff package), to create a representation of the differences
% between the actual and expected golden files.

generateGoldenFile 调用 publicApi(来自 Serializer.ts)生成 golden 文件的内容,然后将其写入文件。
VerifyAgainstGoldenFile() 也调用 publicApi 并将结果保存在名为 actual 的字符串中,
然后将现有的 golden 文件数据加载到名为 expected 的字符串中,然后进行比较。
如果不同,则调用 createPatch(来自 diff 包),以创建实际和预期 golden 文件之间差异的表示。

\begin{minted}{typescript}
export function verifyAgainstGoldenFile(
  entrypoint: string,
  goldenFile: string,
  options: SerializationOptions = {}
): string {
  const actual = publicApi(entrypoint, options);
  const expected = fs.readFileSync(goldenFile).toString();

  if (actual === expected) {
    return '';
  } else {
    const patch = createPatch(
      goldenFile,
      expected,
      actual,
      'Golden file',
      'Generated API'
    );

    // Remove the header of the patch
    const start = patch.indexOf('\n', patch.indexOf('\n') + 1) + 1;

    return patch.substring(start);
  }
}
\end{minted}


% serializer.ts defines
% \texttt{SerializationOptions}
% which has three optional properties:

serializer.ts 定义了 \texttt{SerializationOptions},它具有三个可选属性:

\begin{minted}{typescript}
export interface SerializationOptions {
  /**
   * Removes all exports matching the regular expression.
   */
  stripExportPattern?: RegExp;
  /**
   * Whitelists these identifiers as modules in the output. For example,
   * ```
   * import * as angular from './angularjs';
   *
   * export class Foo extends angular.Bar {}
   * ```
   * will produce `export class Foo extends angular.Bar {}` and requires
   * whitelisting angular.
   */
  allowModuleIdentifiers?: string[];
  /**
   * Warns or errors if stability annotations are missing on an export.
   * Supports experimental, stable and deprecated.
   */
  onStabilityMissing?: string; // 'warn' | 'error' | 'none'
}
\end{minted}


% Serializer.ts defines a public API function which just calls
% \texttt{publicApiInternal()}
% ,
% which in turn calls
% \texttt{ResolvedDeclarationEmitter()}
% , which is a 200-line class where
% the actual work is performed. It has three methods which perform the serialization:

Serializer.ts 定义了一个公共 API 函数,它只调用 \texttt{publicApiInternal()},
后者又调用 \texttt{ResolvedDeclarationEmitter()},这是一个 200 行的类,在其中执行实际工作。
它具有三种执行序列化的方法:

\begin{itemize}
  \item emit(): string
  \item private getResolvedSymbols(sourceFile: ts.SourceFile): ts.Symbol[]
  \item emitNode(node: ts.Node)
\end{itemize}
