\section{lib}

% The lib sub-directory contains these files:

\begin{itemize}
  \item cli.ts – command-line interface, processes argument list and invokes commands
  \item main.ts – main logic for generating and verifying golden files
  \item serializer.ts – code to serialize an API (to create the contents of a golden file)
\end{itemize}

% A golden file is a textual representation of an API and the two key tasks of ts-api-
% guardian is to either create or verify golden files based on supplied command line
% arguments.

% Cli.ts starts with some useful comments about how to call ts-api-guardian:

\begin{minted}{typescript}
// # Generate one declaration file
// ts-api-guardian --out api_guard.d.ts index.d.ts
//
// # Generate multiple declaration files //#(output location like typescript)
// ts-api-guardian --outDir api_guard [--rootDir .] core/index.d.ts
core / testing.d.ts;
//
// # Print usage
// ts-api-guardian --help
//
// # Check against one declaration file
// ts-api-guardian --verify api_guard.d.ts index.d.ts
//
// # Check against multiple declaration files
// ts-api-guardian --verifyDir api_guard [--rootDir .] core/index.d.ts
core / testing.d.ts;
\end{minted}


% cli.ts accepts the following command line options:

\begin{minted}{text}
--help                          Show this usage message
        --out <file>            Write golden output to file
        --outDir <dir>          Write golden file structure to directory
        --verify <file>         Read golden input from file
        --verifyDir <dir>       Read golden file structure from directory
        --rootDir <dir>         Specify the root directory of input files
        --stripExportPattern <regexp>
                                Do not output exports matching the pattern
        --allowModuleIdentifiers <identifier>
                                Whitelist identifier for "* as foo" imports
        --onStabilityMissing <warn|error|none>
                                Warn or error if an export has no stability annotation`);
\end{minted}


% The Angular API allows annotations to be attached to each API indicating whether it is
% stable, deprecated or experiemental. The
% \texttt{onStabilityMissing}
% option indicates what
% action is required if such an annotation is missing. The
% \texttt{startCli()}
% function parses
% the command line and initializes an instance of
% \texttt{SerializationOptions}
% , and then for
% generation mode calls
% \texttt{generateGoldenFile()}
% or for verification mode calls
% \texttt{verifyAgainstGoldenFile()}
% - both are in main.ts and are actually quite short
% functions:

\begin{minted}{typescript}
export function generateGoldenFile(
  entrypoint: string,
  outFile: string,
  options: SerializationOptions = {}
): void {
  const output = publicApi(entrypoint, options);
  ensureDirectory(path.dirname(outFile));
  fs.writeFileSync(outFile, output);
}
\end{minted}


% generateGoldenFile calls publicApi (from Serializer.ts) to generate the contents of the
% golden file and then writes it to a file.  VerifyAgainstGoldenFile() also calls publicApi
% and saves the result in a string called actual, and then loads the existing golden file
% data into a string called expected, and then compares then. If then are different, it
% calls createPatch (from the diff package), to create a representation of the differences
% between the actual and expected golden files.

\begin{minted}{typescript}
export function verifyAgainstGoldenFile(
  entrypoint: string,
  goldenFile: string,
  options: SerializationOptions = {}
): string {
  const actual = publicApi(entrypoint, options);
  const expected = fs.readFileSync(goldenFile).toString();

  if (actual === expected) {
    return '';
  } else {
    const patch = createPatch(
      goldenFile,
      expected,
      actual,
      'Golden file',
      'Generated API'
    );

    // Remove the header of the patch
    const start = patch.indexOf('\n', patch.indexOf('\n') + 1) + 1;

    return patch.substring(start);
  }
}
\end{minted}


% serializer.ts defines
% \texttt{SerializationOptions}
% which has three optional properties:

\begin{minted}{typescript}
export interface SerializationOptions {
  /**
   * Removes all exports matching the regular expression.
   */
  stripExportPattern?: RegExp;
  /**
   * Whitelists these identifiers as modules in the output. For example,
   * ```
   * import * as angular from './angularjs';
   *
   * export class Foo extends angular.Bar {}
   * ```
   * will produce `export class Foo extends angular.Bar {}` and requires
   * whitelisting angular.
   */
  allowModuleIdentifiers?: string[];
  /**
   * Warns or errors if stability annotations are missing on an export.
   * Supports experimental, stable and deprecated.
   */
  onStabilityMissing?: string; // 'warn' | 'error' | 'none'
}
\end{minted}


% Serializer.ts defines a public API function which just calls
% \texttt{publicApiInternal()}
% ,
% which in turn calls
% \texttt{ResolvedDeclarationEmitter()}
% , which is a 200-line class where
% the actual work is performed. It has three methods which perform the serialization:

\begin{itemize}
  \item emit(): string
  \item private getResolvedSymbols(sourceFile: ts.SourceFile): ts.Symbol[]
  \item emitNode(node: ts.Node)
\end{itemize}
