\section{API Model}

% Zone.js exposes an API for applications to use in the
% \href{https://github.com/angular/zone.js/blob/master/dist/zone.js.d.ts}
% {<ZONE-MASTER>/dist/zone.js.d.ts}
% file.

% The two main types it offers are for tasks and zones, along with some helper types. A
% zone is a (usually named) asynchronous execution context; a task is a block of
% functionality (may also be named). Tasks run in the context of a zone.

% Zone.js also supplies a const value, also called
% \texttt{Zone}
% , of type
% \texttt{ZoneType}
% :

\begin{minted}{typescript}
interface ZoneType {
  /**
   * @returns {Zone} Returns the current [Zone]. The only way to change
   * the current zone is by invoking a run() method, which will update the
   * current zone for the duration of the run method callback.
   */
  current: Zone;
  /**
   * @returns {Task} The task associated with the current execution.
   */
  currentTask: Task | null;
  /**
   * Verify that Zone has been correctly patched.
   * Specifically that Promise is zone aware.
   */
  assertZonePatched(): void;
  /**
   *  Return the root zone.
   */
  root: Zone;
}
declare const Zone: ZoneType;
\end{minted}


% Recall that TypeScript has distinct declaration spaces for values and types, so the
% \texttt{Zone}
% value is distinct from the
% \texttt{Zone}
% type. For further details, see the TypeScript
% Language Specification – Section 2.3 – Declarations:

\begin{itemize}
  \item \url{https://github.com/Microsoft/TypeScript/blob/master/doc/spec.md#2.3}
\end{itemize}

% Apart from being used to define the
% \texttt{Zone}
% value,
% \texttt{ZoneType}
% is not used further.

% When your application code wishes to find out the current zone it simply uses
% \texttt{Zone.current}
% , and when it wants to discover the current task within that zone, it
% uses
% \texttt{Zone.currentTask}
% . If you need to figure out whether Zone.js is available to your
% application (it will be for Angular applications), then just make sure
% \texttt{Zone}
% is not
% undefined. If we examine:

\begin{itemize}
  \item \href{https://github.com/angular/angular/blob/master/packages/core/src/zone/ng_zone.ts}
        {<ANGULAR-MASTER>/packages/core/src/zone/ng\_zone.ts}
\end{itemize}

% – we see that is exactly what Angular’s NgZone.ts does:

\begin{minted}{typescript}
  constructor({ enableLongStackTrace = false }) {
    if (typeof Zone == 'undefined') {
      throw new Error(`In this configuration Angular requires Zone.js`);
    }
    ..
  }
\end{minted}


% Two simple helper types used to define tasks are
% \texttt{TaskType}
% and
% \texttt{TaskData}
% .
% \texttt{TaskType}
% is just a human-friendly string to associate with a task. It is usually set to one of the
% three task types as noted in the comment:

\begin{minted}{typescript}
/**
 * Task type: `microTask`, `macroTask`, `eventTask`.
 */
declare type TaskType = 'microTask' | 'macroTask' | 'eventTask';
\end{minted}


% \texttt{TaskData}
% contains a boolean (is this task periodic, i.e. is to be repeated) and two
% numbers - delay before executing this task and a handler id from
% \texttt{setTimout}
% .

\begin{minted}{typescript}
interface TaskData {
  /**
   * A periodic [MacroTask] is such which get automatically
   * rescheduled after it is executed.
   */
  isPeriodic?: boolean;
  /**
   * Delay in milliseconds when the Task will run.
   */
  delay?: number;
  /**
   * identifier returned by the native setTimeout.
   */
  handleId?: number;
}
\end{minted}


% A task is an interface declared as:

\begin{minted}{typescript}
interface Task {
  type: TaskType;
  state: TaskState;
  source: string;
  invoke: Function;
  callback: Function;
  data?: TaskData;
  scheduleFn?: (task: Task) => void;
  cancelFn?: (task: Task) => void;
  readonly zone: Zone;
  runCount: number;
  cancelScheduleRequest(): void;
}
\end{minted}


% There are three marker interfaces derived from
% \texttt{Task}
% :

\begin{minted}{typescript}
interface MicroTask extends Task {
  type: 'microTask';
}
interface MacroTask extends Task {
  type: 'macroTask';
}
interface EventTask extends Task {
  type: 'eventTask';
}
\end{minted}


% The comments for Task nicely explains their purpose:

\begin{minted}{md}
* - [MicroTask] queue represents a set of tasks which are executing right
*   after the current stack frame becomes clean and before a VM yield. All
*   [MicroTask]s execute in order of insertion before VM yield and the next
*   [MacroTask] is executed.
* - [MacroTask] queue represents a set of tasks which are executed one at a
*   time after each VM yield. The queue is ordered by time, and insertions
*   can happen in any location.
* - [EventTask] is a set of tasks which can at any time be inserted to the
*   end of the [MacroTask] queue. This happens when the event fires.
\end{minted}


% There are three helper types used to define
% \texttt{Zone}
% .
% \texttt{HasTaskState}
% just contains
% booleans for each of the task types and a string:

\begin{minted}{typescript}
declare type HasTaskState = {
  microTask: boolean;
  macroTask: boolean;
  eventTask: boolean;
  change: TaskType;
};
\end{minted}


% \texttt{ZoneDelegate}
% is used when one zone wishes to delegate to another how certain
% operations should be performed. So for forcking (creating new tasks), scheduling,
% intercepting, invoking and error handling, the delegate may be called upon to carry
% out the action.

\begin{minted}{typescript}
interface ZoneDelegate {
  zone: Zone;
  fork(targetZone: Zone, zoneSpec: ZoneSpec): Zone;

  intercept(targetZone: Zone, callback: Function, source: string): Function;
  invoke(
    targetZone: Zone,
    callback: Function,
    applyThis?: any,
    applyArgs?: any[],
    source?: string
  ): any;
  handleError(targetZone: Zone, error: any): boolean;
  scheduleTask(targetZone: Zone, task: Task): Task;
  invokeTask(
    targetZone: Zone,
    task: Task,
    applyThis?: any,
    applyArgs?: any[]
  ): any;
  cancelTask(targetZone: Zone, task: Task): any;
  hasTask(targetZone: Zone, isEmpty: HasTaskState): void;
}
\end{minted}


% \texttt{ZoneSpec}
% is an interface that allows implementations to state what should have when
% certain actions are needed. It uses
% \texttt{ZoneDelegate}
% and the current zone:

\begin{minted}{typescript}
interface ZoneSpec {
  name: string;
  properties?: {
    [key: string]: any;
  };
  onFork?: (
    parentZoneDelegate: ZoneDelegate,
    currentZone: Zone,
    targetZone: Zone,
    zoneSpec: ZoneSpec
  ) => Zone;
  onIntercept?: (
    parentZoneDelegate: ZoneDelegate,
    currentZone: Zone,
    targetZone: Zone,
    delegate: Function,
    source: string
  ) => Function;
  onInvoke?: (
    parentZoneDelegate: ZoneDelegate,
    currentZone: Zone,

    targetZone: Zone,
    delegate: Function,
    applyThis: any,
    applyArgs?: any[],
    source?: string
  ) => any;
  onHandleError?: (
    parentZoneDelegate: ZoneDelegate,
    currentZone: Zone,
    targetZone: Zone,
    error: any
  ) => boolean;
  onScheduleTask?: (
    parentZoneDelegate: ZoneDelegate,
    currentZone: Zone,
    targetZone: Zone,
    task: Task
  ) => Task;
  onInvokeTask?: (
    parentZoneDelegate: ZoneDelegate,
    currentZone: Zone,
    targetZone: Zone,
    task: Task,
    applyThis: any,
    applyArgs?: any[]
  ) => any;
  onCancelTask?: (
    parentZoneDelegate: ZoneDelegate,
    currentZone: Zone,
    targetZone: Zone,
    task: Task
  ) => any;
  onHasTask?: (
    parentZoneDelegate: ZoneDelegate,
    currentZone: Zone,
    targetZone: Zone,
    hasTaskState: HasTaskState
  ) => void;
}
\end{minted}


% The definition of the
% \texttt{Zone}
% type is:

\begin{minted}{typescript}
interface Zone {
  parent: Zone | null;
  name: string;
  get(key: string): any;
  getZoneWith(key: string): Zone | null;
  fork(zoneSpec: ZoneSpec): Zone;
  wrap<F extends Function>(callback: F, source: string): F;
  run<T>(
    callback: Function,
    applyThis?: any,
    applyArgs?: any[],
    source?: string
  ): T;
  runGuarded<T>(
    callback: Function,
    applyThis?: any,
    applyArgs?: any[],
    source?: string
  ): T;
  runTask(task: Task, applyThis?: any, applyArgs?: any): any;
  scheduleMicroTask(
    source: string,
    callback: Function,
    data?: TaskData,
    customSchedule?: (task: Task) => void
  ): MicroTask;
  scheduleMacroTask(
    source: string,
    callback: Function,
    data?: TaskData,
    customSchedule?: (task: Task) => void,
    customCancel?: (task: Task) => void
  ): MacroTask;
  scheduleEventTask(
    source: string,
    callback: Function,
    data?: TaskData,
    customSchedule?: (task: Task) => void,
    customCancel?: (task: Task) => void
  ): EventTask;
  scheduleTask<T extends Task>(task: T): T;
  cancelTask(task: Task): any;
}
\end{minted}


\subsection{Relationship between Zone/ZoneSpec/ZoneDelegate interfaces}

Think of
\texttt{ZoneSpec}
as the processing engine that controls how a zone works. It is a
required parameter to the
\texttt{Zone.fork()}
method:

\begin{minted}{typescript}
  // Used to create a child zone.
  // @param zoneSpec A set of rules which the child zone should follow.
  // @returns {Zone} A new child zone.
  fork(zoneSpec: ZoneSpec): Zone;
\end{minted}


Often when a zone needs to perform an action, it uses the supplied
\texttt{ZoneSpec}
. Do you
want to record a long stack trace, keep track of tasks, work with WTF (discussed
later) or run async test well? For each a these a different
\texttt{ZoneSpec}
is supplied, and
each offers different features and comes with different processing costs. Zone.js
supplies one implementation of the
\texttt{Zone}
interface, and multiple implementations of
the
\texttt{ZoneSpec}
interface (in
\href{https://github.com/angular/zone.js/tree/master/lib/zone-spec}
{<ZONE-MASTER>/lib/zone-spec}
). Application code with
specialist needs could create a custom
\texttt{ZoneSpec}
.

An application can build up a hierarchy of zones and sometimes a zone needs to make
a call into another zone further up the hierarchy, and for this a
\texttt{ZoneDelegate}
is used.


\subsection{Relationship between Zone/ZoneSpec/ZoneDelegate interfaces}

Think of
\texttt{ZoneSpec}
as the processing engine that controls how a zone works. It is a
required parameter to the
\texttt{Zone.fork()}
method:

\begin{minted}{typescript}
  // Used to create a child zone.
  // @param zoneSpec A set of rules which the child zone should follow.
  // @returns {Zone} A new child zone.
  fork(zoneSpec: ZoneSpec): Zone;
\end{minted}


Often when a zone needs to perform an action, it uses the supplied
\texttt{ZoneSpec}
. Do you
want to record a long stack trace, keep track of tasks, work with WTF (discussed
later) or run async test well? For each a these a different
\texttt{ZoneSpec}
is supplied, and
each offers different features and comes with different processing costs. Zone.js
supplies one implementation of the
\texttt{Zone}
interface, and multiple implementations of
the
\texttt{ZoneSpec}
interface (in
\href{https://github.com/angular/zone.js/tree/master/lib/zone-spec}
{<ZONE-MASTER>/lib/zone-spec}
). Application code with
specialist needs could create a custom
\texttt{ZoneSpec}
.

An application can build up a hierarchy of zones and sometimes a zone needs to make
a call into another zone further up the hierarchy, and for this a
\texttt{ZoneDelegate}
is used.

