\section{Overview}

The Zone.js project provides multiple asynchronous execution contexts running within
a single JavaScript thread (i.e. within the main browser UI thread or within a single
web worker). Zones are a way of sub-dividing a single thread using the JavaScript
event loop. A single zone does not cross thread boundaries. A nice way to think about
zones is that they sub-divide the stack within a JavaScript thread into multiple mini-
stacks, and sub-divide the JavaScript event loop into multiple mini event loops that
seemingly run concurrently (but really share the same VM event loop). In effect,
Zone.js helps you write multithreaded code within a single thread.

When Zone.js is loaded by your app, it monkey-patches certain asynchronous calls
(e.g.
\texttt{setTimer}
,
\texttt{addEventListener}
) to implement zone functionality. Zone.js does
this by adding wrappers to the callbacks the application supplies, and when a timeout
occurs or an event is detected, it runs the wrapper first and then the application
callback. Chunks of executing application code form tasks and each task executes in
the context of a zone. Zones are arranged in a hierarchy and provide useful features
in areas such as error handling, performance measuring and executing configured
work items upon entering and leaving a zone (all of which might be of great interest
to implementors of change detection in a modern web UI framework, like Angular).

Zone.js is mostly transparent to application code. Zone.js runs in the background and
for the most part “just works”. For example, Angular uses Zone.js and Angular
application code usually runs inside a zone. Tens of thousands of Angular application
developers have their code execute within zones without evening knowing these exist.
More advanced applications can have custom code to make zone API calls if needed
and become more actively involved in zone management. Some application
developers may wish to take certain steps to move some non-UI performance-critical
code outside of the Angular zone – using the
\texttt{NgZone}
class).

\subsection{Project Information}

% The homepage + root of the source tree for Zone.js are in the main Angular repo at:

\begin{itemize}
  \item \url{https://github.com/angular/angular/tree/master/packages/zone.js}
\end{itemize}

% Below we assume you have got the Zone.js source tree downloaded locally under a
% directory we will call <ZONE-MASTER> and any pathnames we use will be relative to
% that. We also will look at the Angular code that calls Zone.js and we use the
% <ANGULAR-MASTER> as the root of that source tree.

% Zone.js is written in TypeScript. It has no package dependencies (its package.json has
% this entry:
% \texttt{"dependencies": {}}
% ), though it has many
% \texttt{devDependencies}
% . It is quite a
% small source tree, whose size (uncompressed) is about 3 MB.


\subsection{Project Information}

% The homepage + root of the source tree for Zone.js are in the main Angular repo at:

\begin{itemize}
  \item \url{https://github.com/angular/angular/tree/master/packages/zone.js}
\end{itemize}

% Below we assume you have got the Zone.js source tree downloaded locally under a
% directory we will call <ZONE-MASTER> and any pathnames we use will be relative to
% that. We also will look at the Angular code that calls Zone.js and we use the
% <ANGULAR-MASTER> as the root of that source tree.

% Zone.js is written in TypeScript. It has no package dependencies (its package.json has
% this entry:
% \texttt{"dependencies": {}}
% ), though it has many
% \texttt{devDependencies}
% . It is quite a
% small source tree, whose size (uncompressed) is about 3 MB.

