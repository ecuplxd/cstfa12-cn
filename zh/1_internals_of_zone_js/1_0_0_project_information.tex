% \subsection{Project Information}
\subsection{项目信息}

% The homepage + root of the source tree for Zone.js are in the main Angular repo at:

Zome.js 源码主页 + 根目录在 Angular 仓库中:

\begin{itemize}
  \item \url{https://github.com/angular/angular/tree/master/packages/zone.js}
\end{itemize}

% Below we assume you have got the Zone.js source tree downloaded locally under a
% directory we will call <ZONE-MASTER> and any pathnames we use will be relative to
% that. We also will look at the Angular code that calls Zone.js and we use the
% <ANGULAR-MASTER> as the root of that source tree.

下面我们假设你已经在被称为 <ZONE-MASTER> 的目录下将 Zone.js 源码下载到了本地,
任何将要使用到的路径名会以此作为相对路径。
我们还将查看调用 Zone.js 的 Angular 代码,我们使用
<ANGULAR-MASTER> 作为该源码的根目录。

% Zone.js is written in TypeScript. It has no package dependencies (its package.json has
% this entry:
% \texttt{"dependencies": {}}
% ), though it has many
% \texttt{devDependencies}
% . It is quite a
% small source tree, whose size (uncompressed) is about 3 MB.

Zone.js 是用 TypeScript 编写的。
它没有包依赖(它的 package.json 有这个条目:\texttt{"dependencies": {}}),
虽然它有很多 devDependencies。
这是一个相当小型源码,其大小(未压缩)约为 3 MB。
