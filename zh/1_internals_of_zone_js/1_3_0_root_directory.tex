\subsection{Root directory}

The root directory contains these markdown documentation files:

\begin{itemize}
  \item CHANGELOG.md
  \item DEVELOPER.md
  \item MODULE.md
  \item NON-STANDARD-APIS.md
  \item README.md
  \item SAMPLE.md
  \item STANDARD-APIS.md
\end{itemize}

we see it is actually very well
documented and contains plenty of detail to get us up and running writing applications
that use Zone.js. From the DEVELOPER.md document we see the contents of dist is
auto-generated (we need to explore how).

The root directory contains these JSON files:

\begin{itemize}
  \item tslint.json
  \item tsconfig[-node|-esm-node|esm|].json
  \item package.json
\end{itemize}

There are multiple files starting with tsconfig – here are the compilerOptionsfrom the
main one:

\begin{minted}{json}
  "compilerOptions": {
    "module": "commonjs",
    "target": "es5",
    "noImplicitAny": true,
    "noImplicitReturns": false,
    "noImplicitThis": false,
    "outDir": "build",
    "inlineSourceMap": true,
    "inlineSources": true,
    "declaration": false,
    "noEmitOnError": false,
    "stripInternal": false,
    "strict": true,
    "lib": [
      "es5",
      "dom",
      "es2015.promise",
      "es2015.symbol",
      "es2015.symbol.wellknown"
    ]
  }
\end{minted}


The package.json file contains metadata (including main and browser, which provide
alternative entry points depending on whether this package
1
is loaded into a server
[node] or a
2
browser app):

\begin{minted}{ts}
{
  "name": "zone.js",
  "version": "0.9.0",
  "description": "Zones for JavaScript",
1 "main": "dist/zone-node.js",
2 "browser": "dist/zone.js",
  "unpkg": "dist/zone.js",
  "typings": "dist/zone.js.d.ts",
  "files": [
    "lib",
    "dist"
  ],
  "directories": {
    "lib": "lib",
    "test": "test"
  },
\end{minted}


and a list of scripts:

\begin{minted}{json}
  "scripts": {
    "changelog": "gulp changelog",
    "ci": "npm run lint && npm run format && npm run promisetest && npm run test:single && npm run test-node",
    "closure:test": "scripts/closure/closure_compiler.sh",
    "format": "gulp format:enforce",
    "karma-jasmine": "karma start karma-build-jasmine.conf.js",
    "karma-jasmine:es2015": "karma start karma-build-jasmine.es2015.conf.js",
    "karma-jasmine:phantomjs": "karma start karma-build-jasmine-phantomjs.conf.js --single-run",
    "karma-jasmine:single": "karma start karma-build-jasmine.conf.js --single-run",
    "karma-jasmine:autoclose": "npm run karma-jasmine:single && npm run ws-client",
    "karma-jasmine-phantomjs:autoclose": "npm run karma-jasmine:phantomjs && npm run ws-client",
    "lint": "gulp lint",
    "prepublish": "tsc && gulp build",
    "promisetest": "gulp promisetest",
    "promisefinallytest": "mocha promise.finally.spec.js",
    "ws-client": "node ./test/ws-client.js",
    "ws-server": "node ./test/ws-server.js",
    "tsc": "tsc -p .",
    "tsc:w": "tsc -w -p .",
    "tsc:esm2015": "tsc -p tsconfig-esm-2015.json",
    "tslint": "tslint -c tslint.json 'lib/**/*.ts'",
    // many test scripts
    ...
  },
\end{minted}


It has no dependencies:

\begin{minted}{json}
  "dependencies": {},
\end{minted}


The root directory also contains the MIT license in a file called LICENSE, along with
the same within a comment in a file called LICENSE.wrapped.

It contains this file concerning bundling:

\begin{itemize}
  \item webpack.config.js
\end{itemize}

This has the following content:

\begin{minted}{typescript}
module.exports = {
  entry: './test/source_map_test.ts',
  output: {
    path: __dirname + '/build',
    filename: 'source_map_test_webpack.js',
  },
  devtool: 'inline-source-map',
  module: {
    loaders: [
      { test: /\.ts/, loaders: ['ts-loader'], exclude: /node_modules/ },
    ],
  },
  resolve: {
    extensions: ['', '.js', '.ts'],
  },
};
\end{minted}


Webpack is quite a popular bundler and ts-loader is a TypeScript loader for webpack.
Details on both projects can be found here:

\begin{itemize}
  \item \url{https://webpack.github.io/}
  \item \url{https://github.com/TypeStrong/ts-loader}
\end{itemize}

The root directory contains this file related to GIT:

\begin{itemize}
  \item .gitignore
\end{itemize}

It contains this task runner configuration:

\begin{itemize}
  \item gulpfile.js
\end{itemize}

It supplies a gulp task called “test/node” to run tests against the node version of
Zone.js, and a gulp task “compile” which runs the TypeScript tsc compiler in a child
process. It supplies many gulp tasks to build individual components and run tests.

All of these tasks result in a call to a local method
\texttt{generateScript}
which minifies (if
required) and calls webpack and places the result in the dist sub-directory.
