% \subsection{Use Within Angular}
\subsection{在 Angular 中使用}

% When writing Angular applications, all your application code runs within a zone, unless
% you take specific steps to ensure some of it does not. Also, most of the Angular
% framework code itself runs in a zone. When beginning Angular application
% development, you can get by simply ignoring zones, since they are set up correctly by
% default for you and applications do not have to do anything in particular to take
% advantage of them. The end of the file
% \href{https://github.com/angular/zone.js/blob/master/MODULE.md}
% {<ZONE-MASTER>/blob/master/MODULE.md}
% explains where Angular uses zones:

编写 Angular 应用时,你的所有应用代码都在一个 zone 内运行,
除非你采取具体步骤以确保其中一些不会。
此外,大多数 Angular 框架代码本身在一个 zone 中运行。
开始 Angular 应用时开发,你可以简单地忽略 zones,
它们默认为你的应用配置好了,无需做任何特别的事情就能从中受益。
文件
\href{https://github.com/angular/zone.js/blob/master/MODULE.md}
{<ZONE-MASTER>/blob/master/MODULE.md} 文件结尾
结尾解释了 Angular 在哪里使用到了 zones:

\emph{“Angular uses zone.js to manage async operations and decide when to perform}
\emph{change detection. Thus, in Angular, the following APIs should be patched, otherwise}
\emph{Angular may not work as expected.}

\begin{itemize}
  \item ZoneAwarePromise
  \item timer
  \item on\_property
  \item EventTarget
  \item XHR”
\end{itemize}

% Zones are how Angular initiates change detection – when the zone’s mini-stack is
% empty, change detection occurs. Also, zones are how Angular configures global
% exception handlers. When an error occurs in a task, its zone’s configured error handler
% is called. A default implementation is provided and applications can supply a custom
% implementation via dependency injection. For details, see here:

Zones 是 Angular 启动变更检测的方式 —— 当 zone 的小块栈是空时,发生变更检测。
此外,zones 是 Angular 配置全局异常处理的方式。
当任务发生错误时,其 zone 配置的错误处理程序被调用。
Zone 提供了默认实现,应用可以通过依赖注入提供自定义实现。
有关详细信息,请参见此处:

\begin{itemize}
  \item \url{https://angular.io/api/core/ErrorHandler}
\end{itemize}

% On that page note the code sample about setting up your own error handler:

在该页面上,提供了自定义错误处理程序的代码示例:

\begin{minted}{typescript}
class MyErrorHandler implements ErrorHandler {
  handleError(error) {
    // do something with the exception
  }
}
@NgModule({
  providers: [{ provide: ErrorHandler, useClass: MyErrorHandler }],
})
class MyModule {}
\end{minted}


% Angular provide a class,
% \texttt{NgZone}
% , which builds on zones:

Angular 提供了一个类 \texttt{NgZone},它建立在 zone 上:

\begin{itemize}
  \item \url{https://angular.io/api/core/NgZone}
\end{itemize}

% As you begin to create more advanced Angular applications, specifically those
% involving computationally intensive code that does not change the UI midway through
% the computation (but may at the end), you will see it is desirable to place such CPU-
% intensive work in a separate zone, and you would use a custom
% \texttt{NgZone}
% for that.

当你开始创建更高级的 Angular 应用程序时,
特别是那些涉及不会在中途更改 UI 的计算密集型代码计算(但可能在最后),
您会看到放置这样的 CPU 是可取的-
在单独的区域中进行密集的工作,您将为此使用自定义 \texttt{NgZone}。

% Elsewhere we will be looking in detail at
% \texttt{NgZone}
% and the use of zones within Angular in
% general when we explore the source tree for the main Angular project later, but for
% now, note the source for
% \texttt{NgZone}
% is in:

在其他地方,
我们将详细介绍 \texttt{NgZone} 和 Angular 中 zones 的使用一般当我们稍后探索主要 Angular 项目的源码时,
但对于现在,请注意 \texttt{NgZone} 的源码在:

\begin{itemize}
  \item \href{https://github.com/angular/angular/tree/master/packages/core/src/zone}
        {<ANGULAR-MASTER>/packages/core/src/zone}
\end{itemize}

% and the zone setup during bootstrap for an application is in:

应用程序引导期间的 zone 设置位于:

\begin{itemize}
  \item \href{https://github.com/angular/angular/blob/master/packages/core/src/application_ref.ts}
        {<ANGULAR-MASTER>/packages/core/src/application\_ref.ts}
\end{itemize}

% When we bootstrap our Angular applications, we either use
% \texttt{bootstrapModule<M>}
% (using the dynamic compiler) or
% \texttt{bootstrapModuleFactory<M>}
% (using the offline
% compiler). Both these functions are in application\_ref.ts.
% \texttt{bootstrapModule<M>}
% calls
% the Angular compiler
% 1
% and then calls
% \texttt{bootstrapModuleFactory<M>}
% \texttt{2}
% \texttt{.}


当我们引导 Angular 应用程序时,我们要么使用 \texttt{bootstrapModule<M>}
(使用动态编译器)或 \texttt{bootstrapModuleFactory<M>}(使用离线编译器)。
这两个函数都在 application\_ref.ts. \texttt{bootstrapModule<M>}
调用 Angular 编译器 \step{1} 然后调用 \texttt{bootstrapModuleFactory<M>} \step{2}。

\begin{minted}{typescript}
  bootstrapModule<M>(
    moduleType: Type<M>,
    compilerOptions:
      | (CompilerOptions & BootstrapOptions)
      | Array<CompilerOptions & BootstrapOptions> = []
  ): Promise<NgModuleRef<M>> {
    const options = optionsReducer({}, compilerOptions);
    %\step{1}% return compileNgModuleFactory(
      this.injector,
      options,
      moduleType
    ).then((moduleFactory) =>
      %\step{2}% this.bootstrapModuleFactory(moduleFactory, options)
    );
  }
\end{minted}


% It is in
% \texttt{bootstrapModuleFactory}
% we see how zones are initialized for Angular:

在 \texttt{bootstrapModuleFactory} 中,我们看到了如何为 Angular 初始化 zone:

\begin{minted}{typescript}
  bootstrapModuleFactory<M>(
    moduleFactory: NgModuleFactory<M>,
    options?: BootstrapOptions
  ): Promise<NgModuleRef<M>> {
    // Note: We need to create the NgZone _before_ we instantiate the module,
    // as instantiating the module creates some providers eagerly.
    // So we create a mini parent injector that just contains the new NgZone
    // and pass that as parent to the NgModuleFactory.
    const ngZoneOption = options ? options.ngZone : undefined;
    %\step{1}% const ngZone = getNgZone(ngZoneOption);
    const providers: StaticProvider[] = [{ provide: NgZone, useValue: ngZone }];
    // Attention: Don't use ApplicationRef.run here,
    // as we want to be sure that all possible constructor calls
    // are inside `ngZone.run`!
    %\step{2}% return ngZone.run(() => {
      const ngZoneInjector = Injector.create({
        providers: providers,
        parent: this.injector,
        name: moduleFactory.moduleType.name,
      });
      const moduleRef = <InternalNgModuleRef<M>>(
        moduleFactory.create(ngZoneInjector)
      );
      const exceptionHandler: ErrorHandler =
        %\step{3}% moduleRef.injector.get(ErrorHandler, null);
      if (!exceptionHandler) {
        throw new Error(
          'No ErrorHandler. Is platform module (BrowserModule) included?'
        );
      }
      moduleRef.onDestroy(() => remove(this._modules, moduleRef));
      ngZone!.runOutsideAngular(
        %\step{4}% () =>
          ngZone!.onError.subscribe({
            next: (error: any) => {
              exceptionHandler.handleError(error);
            },
          })
      );
      return _callAndReportToErrorHandler(exceptionHandler, ngZone!, () => {
        const initStatus: ApplicationInitStatus = moduleRef.injector.get(
          ApplicationInitStatus
        );
        initStatus.runInitializers();
        return initStatus.donePromise.then(() => {
          %\step{5}% this._moduleDoBootstrap(moduleRef);
          return moduleRef;
        });
      });
    });
  }
\end{minted}


% At
% 1
% we see a new
% \texttt{NgZone}
% being created and at
% 2
% its
% \texttt{run()}
% method being called, at
% 3
% we see an error handler implementation being requested from dependency injection (a
% default implementation will be returned unless the application supplies a custom one)
% and at
% 4
% , we see that error handler being used to configure error handling for the
% newly created
% \texttt{NgZone}
% . Finally at
% 5
% , we see the call to the actual bootstrapping.

在 \step{1} 我们看到一个新的 \texttt{NgZone} 被创建,
在 \step{2} 它的 \texttt{run()} 方法被调用,
在 \step{3} 我们看到依赖注入请求了一个错误处理实现
(除非应用提供自定义实现,否则将返回默认实现)
在 \step{4} 处,我们看到错误处理程序将被用于配置新创建的 \texttt{NgZone}。
最后在 \step{5} 处,我们看到对实际引导的调用。

% So in summary, Angular application developers should clearly learn about zones, since
% that is the execution context within which their application code will run.

所以总而言之,Angular 应用开发人员应该清楚地了解 zones,
因为他们的应用代码将在其中的执行上下文运行。
