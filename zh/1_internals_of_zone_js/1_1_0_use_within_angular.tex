\subsection{Use Within Angular}

When writing Angular applications, all your application code runs within a zone, unless
you take specific steps to ensure some of it does not. Also, most of the Angular
framework code itself runs in a zone. When beginning Angular application
development, you can get by simply ignoring zones, since they are set up correctly by
default for you and applications do not have to do anything in particular to take
advantage of them. The end of the file
\href{https://github.com/angular/zone.js/blob/master/MODULE.md}
{<ZONE-MASTER>/blob/master/MODULE.md}
explains where Angular uses zones:

\emph{“Angular uses zone.js to manage async operations and decide when to perform}
\emph{change detection. Thus, in Angular, the following APIs should be patched, otherwise}
\emph{Angular may not work as expected.}

\begin{itemize}
  \item ZoneAwarePromise
  \item timer
  \item on\_property
  \item EventTarget
  \item XHR”
\end{itemize}

Zones are how Angular initiates change detection – when the zone’s mini-stack is
empty, change detection occurs. Also, zones are how Angular configures global
exception handlers. When an error occurs in a task, its zone’s configured error handler
is called. A default implementation is provided and applications can supply a custom
implementation via dependency injection. For details, see here:

\begin{itemize}
  \item \url{https://angular.io/api/core/ErrorHandler}
\end{itemize}

On that page note the code sample about setting up your own error handler:

\begin{minted}{typescript}
class MyErrorHandler implements ErrorHandler {
  handleError(error) {
    // do something with the exception
  }
}
@NgModule({
  providers: [{ provide: ErrorHandler, useClass: MyErrorHandler }],
})
class MyModule {}
\end{minted}


Angular provide a class,
\texttt{NgZone}
, which builds on zones:

\begin{itemize}
  \item \url{https://angular.io/api/core/NgZone}
\end{itemize}

As you begin to create more advanced Angular applications, specifically those
involving computationally intensive code that does not change the UI midway through
the computation (but may at the end), you will see it is desirable to place such CPU-
intensive work in a separate zone, and you would use a custom
\texttt{NgZone}
for that.

Elsewhere we will be looking in detail at
\texttt{NgZone}
and the use of zones within Angular in
general when we explore the source tree for the main Angular project later, but for
now, note the source for
\texttt{NgZone}
is in:

\begin{itemize}
  \item \href{https://github.com/angular/angular/tree/master/packages/core/src/zone}
        {<ANGULAR-MASTER>/packages/core/src/zone}
\end{itemize}

and the zone setup during bootstrap for an application is in:

\begin{itemize}
  \item \href{https://github.com/angular/angular/blob/master/packages/core/src/application_ref.ts}
        {<ANGULAR-MASTER>/packages/core/src/application\_ref.ts}
\end{itemize}

When we bootstrap our Angular applications, we either use
\texttt{bootstrapModule<M>}
(using the dynamic compiler) or
\texttt{bootstrapModuleFactory<M>}
(using the offline
compiler). Both these functions are in application\_ref.ts.
\texttt{bootstrapModule<M>}
calls
the Angular compiler
1
and then calls
\texttt{bootstrapModuleFactory<M>}
\texttt{2}
\texttt{.}

\begin{minted}{typescript}
  bootstrapModule<M>(
    moduleType: Type<M>,
    compilerOptions:
      | (CompilerOptions & BootstrapOptions)
      | Array<CompilerOptions & BootstrapOptions> = []
  ): Promise<NgModuleRef<M>> {
    const options = optionsReducer({}, compilerOptions);
    %\step{1}% return compileNgModuleFactory(
      this.injector,
      options,
      moduleType
    ).then((moduleFactory) =>
      %\step{2}% this.bootstrapModuleFactory(moduleFactory, options)
    );
  }
\end{minted}


It is in
\texttt{bootstrapModuleFactory}
we see how zones are initialized for Angular:

\begin{minted}{typescript}
  bootstrapModuleFactory<M>(
    moduleFactory: NgModuleFactory<M>,
    options?: BootstrapOptions
  ): Promise<NgModuleRef<M>> {
    // Note: We need to create the NgZone _before_ we instantiate the module,
    // as instantiating the module creates some providers eagerly.
    // So we create a mini parent injector that just contains the new NgZone
    // and pass that as parent to the NgModuleFactory.
    const ngZoneOption = options ? options.ngZone : undefined;
    %\step{1}% const ngZone = getNgZone(ngZoneOption);
    const providers: StaticProvider[] = [{ provide: NgZone, useValue: ngZone }];
    // Attention: Don't use ApplicationRef.run here,
    // as we want to be sure that all possible constructor calls
    // are inside `ngZone.run`!
    %\step{2}% return ngZone.run(() => {
      const ngZoneInjector = Injector.create({
        providers: providers,
        parent: this.injector,
        name: moduleFactory.moduleType.name,
      });
      const moduleRef = <InternalNgModuleRef<M>>(
        moduleFactory.create(ngZoneInjector)
      );
      const exceptionHandler: ErrorHandler =
        %\step{3}% moduleRef.injector.get(ErrorHandler, null);
      if (!exceptionHandler) {
        throw new Error(
          'No ErrorHandler. Is platform module (BrowserModule) included?'
        );
      }
      moduleRef.onDestroy(() => remove(this._modules, moduleRef));
      ngZone!.runOutsideAngular(
        %\step{4}% () =>
          ngZone!.onError.subscribe({
            next: (error: any) => {
              exceptionHandler.handleError(error);
            },
          })
      );
      return _callAndReportToErrorHandler(exceptionHandler, ngZone!, () => {
        const initStatus: ApplicationInitStatus = moduleRef.injector.get(
          ApplicationInitStatus
        );
        initStatus.runInitializers();
        return initStatus.donePromise.then(() => {
          %\step{5}% this._moduleDoBootstrap(moduleRef);
          return moduleRef;
        });
      });
    });
  }
\end{minted}


At
1
we see a new
\texttt{NgZone}
being created and at
2
its
\texttt{run()}
method being called, at
3
we see an error handler implementation being requested from dependency injection (a
default implementation will be returned unless the application supplies a custom one)
and at
4
, we see that error handler being used to configure error handling for the
newly created
\texttt{NgZone}
. Finally at
5
, we see the call to the actual bootstrapping.

So in summary, Angular application developers should clearly learn about zones, since
that is the execution context within which their application code will run.
