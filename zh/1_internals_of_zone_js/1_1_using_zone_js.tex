% \section{Using Zone.js}
\section{使用 Zone.js}

% Before looking in detail at how Zone.js itself is implemented, we will first look at how
% it is used in production, using the example of the very popular Angular project.

在详细了解 Zone.js 自身是如何实现之前,我们首先了解在生产中如何使用它,
这里以非常流行的 Angular 项目为例。

% The use of Zone.js with Angular is optional, but used by default. Use of zones is
% mostly a good idea but there are some scenarios (e.g. using Angular Elements to build
% Web Compoenents) when this is not the case.

在 Angular 中使用 Zone.js 是可选的,不过却是默认开启。
使用 zones 通常来说是个好主意,
但存在某些场景(比如,使用 Angular Elements 构建 Web Compoenents)下是不必要的。

% To use Zone.js in your applications, you need to load it. Your package.json file will
% need (if creating a project using Angular CLI, this is added automatically for you):

要在你的应用中使用 Zone.js,你需要加载它。在 package.json 中写入
(如果你的项目是通过 Angular CLI 创建,会自动帮你添加进去):

\begin{minted}{json}
  "dependencies": {
    ..,
    "zone.js": "<version>"
  },
\end{minted}


% You should load Zone.js after loading core.js (if using that). For example, if using an
% Angular application generated via Angular CLI (as most production apps will be),
% Angular CLI will generate a file called <project-name>/src/polyfills.ts and it will
% contain:

你应该在加载 core.js 之后加载 Zone.js(如果用到了它)。
例如,如果你使用 Angular CLI 生成 Angular 应用(就像大多数生产应用一样),
Angular CLI 将生成一个名为 <project-name>/src/polyfills.ts 的文件,它将包含:

\input{../output/1_internals_of_zone_js/code/1_1_1.tex}

% Angular CLI also generates an angular.json configuration file, with this line that sets
% up polyfills:

Angular CLI 还会生成一个 angular.json 配置文件,该行设置 polyfills:

\begin{minted}{json}
  "build": {
    "builder": "@angular-devkit/build-angular:browser",
    "options": {
      ..
      "polyfills": "src/polyfills.ts",
\end{minted}


% If writing your application in TypeScript (recommended), you also need to get access
% to the ambient declarations. These define the Zone.js API and are supplied in:

如果你的应用是用 TypeScript(推荐)编写的,
您还需要配置声明访问。 这些定义了 Zone.js API 并在以下位置提供:

\begin{itemize}
  \item \href{https://github.com/angular/zone.js/blob/master/dist/zone.js.d.ts}
        {<ZONE-MASTER>/dist/zone.js.d.ts}
\end{itemize}

% (IMPORTANT: This file is particularly well documented and well worth some careful
% study by those learning Zone.js). Unlike declarations for most other libraries,
% zone.js.d.ts does not use
% \texttt{import}
% or
% \texttt{export}
% at all (those constructs do not appear
% even once in that file). That means application code wishing to use zones cannot
% simply import its .d.ts file, as is normally the case. Instead, the
% \texttt{///reference}
% construct needs to be used. This includes the referenced file at the site of the
% \texttt{///reference}
% in the containing file. The benefit of this approach is that the containing
% file itself does not have to (but may) use
% \texttt{import}
% , and thus may be a script, rather
% than having to be a module. The use of zones is not forcing the application to use
% modules (however, most larger applications, including all Angular applications - will).
% How this works is best examined with an example, so lets look at how Angular
% includes zone.d.ts. Angular contains a file, types.d.ts under its packages directory
% (and a similar one under its modules directory and tools directory):

(注意:这个文件有着良好的文档说明,值得那些了解 Zone.js 的人仔细学习)。
不同于其它库的声明文件,zone.js.d.ts 根本不使用 \texttt{import} 或 \texttt{export}
(那些命令甚至没有出现在那个文件中一次)。
这意味着通常情况下,希望使用 zones 的应用代码不能只简单地导入其\, .d.ts 文件。
相反,应该使用 \texttt{///reference} 命令。
这包括网站上的参考文件 \texttt{///reference} 包含文件中的引用。
这种方法的好处是包含文件本身不必(但可以)使用 \texttt{import},因此可能是一个脚本,
相比与其必须是一个模块。
使用 zones 并不强制应用程序使用模块(然而,大多数大型应用,包括所有 Angular 应用会使用)。
最好通过一个例子来研究它是如何工作的,
所以让我们看看 Angular 是如何包括 zone.d.ts。 Angular 包含一个文件,
types.d.ts 在包目录下(在它的模块目录和工具目录下有一个类似的)。

\begin{itemize}
  \item \href{https://github.com/angular/angular/blob/master/modules/types.d.ts}
        {<ANGULAR-MASTER>/packages/types.d.ts}
\end{itemize}

% and it has the following contents:

它包含如下内容:

\begin{minted}{typescript}
/// <reference types="hammerjs" />
/// <reference types="jasmine" />
/// <reference types="jasminewd2" />
/// <reference types="node" />
/// <reference types="zone.js" />
/// <reference lib="es2015" />
/// <reference path="./system.d.ts" />
\end{minted}


\subsection{Use Within Angular}

% When writing Angular applications, all your application code runs within a zone, unless
% you take specific steps to ensure some of it does not. Also, most of the Angular
% framework code itself runs in a zone. When beginning Angular application
% development, you can get by simply ignoring zones, since they are set up correctly by
% default for you and applications do not have to do anything in particular to take
% advantage of them. The end of the file
% \href{https://github.com/angular/zone.js/blob/master/MODULE.md}
% {<ZONE-MASTER>/blob/master/MODULE.md}
% explains where Angular uses zones:

% \emph{“Angular uses zone.js to manage async operations and decide when to perform}
% \emph{change detection. Thus, in Angular, the following APIs should be patched, otherwise}
% \emph{Angular may not work as expected.}

\begin{itemize}
  \item ZoneAwarePromise
  \item timer
  \item on\_property
  \item EventTarget
  \item XHR”
\end{itemize}

% Zones are how Angular initiates change detection – when the zone’s mini-stack is
% empty, change detection occurs. Also, zones are how Angular configures global
% exception handlers. When an error occurs in a task, its zone’s configured error handler
% is called. A default implementation is provided and applications can supply a custom
% implementation via dependency injection. For details, see here:

\begin{itemize}
  \item \url{https://angular.io/api/core/ErrorHandler}
\end{itemize}

% On that page note the code sample about setting up your own error handler:

\begin{minted}{typescript}
class MyErrorHandler implements ErrorHandler {
  handleError(error) {
    // do something with the exception
  }
}
@NgModule({
  providers: [{ provide: ErrorHandler, useClass: MyErrorHandler }],
})
class MyModule {}
\end{minted}


% Angular provide a class,
% \texttt{NgZone}
% , which builds on zones:

\begin{itemize}
  \item \url{https://angular.io/api/core/NgZone}
\end{itemize}

% As you begin to create more advanced Angular applications, specifically those
% involving computationally intensive code that does not change the UI midway through
% the computation (but may at the end), you will see it is desirable to place such CPU-
% intensive work in a separate zone, and you would use a custom
% \texttt{NgZone}
% for that.

% Elsewhere we will be looking in detail at
% \texttt{NgZone}
% and the use of zones within Angular in
% general when we explore the source tree for the main Angular project later, but for
% now, note the source for
% \texttt{NgZone}
% is in:

\begin{itemize}
  \item \href{https://github.com/angular/angular/tree/master/packages/core/src/zone}
        {<ANGULAR-MASTER>/packages/core/src/zone}
\end{itemize}

% and the zone setup during bootstrap for an application is in:

\begin{itemize}
  \item \href{https://github.com/angular/angular/blob/master/packages/core/src/application_ref.ts}
        {<ANGULAR-MASTER>/packages/core/src/application\_ref.ts}
\end{itemize}

% When we bootstrap our Angular applications, we either use
% \texttt{bootstrapModule<M>}
% (using the dynamic compiler) or
% \texttt{bootstrapModuleFactory<M>}
% (using the offline
% compiler). Both these functions are in application\_ref.ts.
% \texttt{bootstrapModule<M>}
% calls
% the Angular compiler
% 1
% and then calls
% \texttt{bootstrapModuleFactory<M>}
% \texttt{2}
% \texttt{.}

\begin{minted}{typescript}
  bootstrapModule<M>(
    moduleType: Type<M>,
    compilerOptions:
      | (CompilerOptions & BootstrapOptions)
      | Array<CompilerOptions & BootstrapOptions> = []
  ): Promise<NgModuleRef<M>> {
    const options = optionsReducer({}, compilerOptions);
    %\step{1}% return compileNgModuleFactory(
      this.injector,
      options,
      moduleType
    ).then((moduleFactory) =>
      %\step{2}% this.bootstrapModuleFactory(moduleFactory, options)
    );
  }
\end{minted}


% It is in
% \texttt{bootstrapModuleFactory}
% we see how zones are initialized for Angular:

\begin{minted}{typescript}
  bootstrapModuleFactory<M>(
    moduleFactory: NgModuleFactory<M>,
    options?: BootstrapOptions
  ): Promise<NgModuleRef<M>> {
    // Note: We need to create the NgZone _before_ we instantiate the module,
    // as instantiating the module creates some providers eagerly.
    // So we create a mini parent injector that just contains the new NgZone
    // and pass that as parent to the NgModuleFactory.
    const ngZoneOption = options ? options.ngZone : undefined;
    %\step{1}% const ngZone = getNgZone(ngZoneOption);
    const providers: StaticProvider[] = [{ provide: NgZone, useValue: ngZone }];
    // Attention: Don't use ApplicationRef.run here,
    // as we want to be sure that all possible constructor calls
    // are inside `ngZone.run`!
    %\step{2}% return ngZone.run(() => {
      const ngZoneInjector = Injector.create({
        providers: providers,
        parent: this.injector,
        name: moduleFactory.moduleType.name,
      });
      const moduleRef = <InternalNgModuleRef<M>>(
        moduleFactory.create(ngZoneInjector)
      );
      const exceptionHandler: ErrorHandler =
        %\step{3}% moduleRef.injector.get(ErrorHandler, null);
      if (!exceptionHandler) {
        throw new Error(
          'No ErrorHandler. Is platform module (BrowserModule) included?'
        );
      }
      moduleRef.onDestroy(() => remove(this._modules, moduleRef));
      ngZone!.runOutsideAngular(
        %\step{4}% () =>
          ngZone!.onError.subscribe({
            next: (error: any) => {
              exceptionHandler.handleError(error);
            },
          })
      );
      return _callAndReportToErrorHandler(exceptionHandler, ngZone!, () => {
        const initStatus: ApplicationInitStatus = moduleRef.injector.get(
          ApplicationInitStatus
        );
        initStatus.runInitializers();
        return initStatus.donePromise.then(() => {
          %\step{5}% this._moduleDoBootstrap(moduleRef);
          return moduleRef;
        });
      });
    });
  }
\end{minted}


% At
% 1
% we see a new
% \texttt{NgZone}
% being created and at
% 2
% its
% \texttt{run()}
% method being called, at
% 3
% we see an error handler implementation being requested from dependency injection (a
% default implementation will be returned unless the application supplies a custom one)
% and at
% 4
% , we see that error handler being used to configure error handling for the
% newly created
% \texttt{NgZone}
% . Finally at
% 5
% , we see the call to the actual bootstrapping.

% So in summary, Angular application developers should clearly learn about zones, since
% that is the execution context within which their application code will run.

