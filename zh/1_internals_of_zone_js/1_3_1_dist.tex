\subsection{dist}

% This single directory contains all the output from the build tasks. The zone.d.ts file is
% the ambient declarations, which TypeScript application developers will want to use.
% This is surprisingly well documented, so a new application developer getting up to
% speed with zone.js should give it a careful read. A number of implementations of Zone
% are provided, such as for the browser, for the server and for Jasmine testing:

\begin{itemize}
  \item zone.js / zone.min.js
  \item zone-node.js
  \item jasmine-patch.js / jasmine-patch.min.js
\end{itemize}

% Minified versions are supplied for the browser and jasmine builds, but not node. If you
% are using Angular in the web browser, then zone.js (or zone.min.js) is all you need
% Assuming you have created your Angular project using Angular CLI, it will
% automatically have set everything up correctly (no manual steps are needed to use
% Zone.js).

% The remaining files in the dist directory are builds of different zone specs, which for
% specialist reasons you may wish to include – for example:

\begin{itemize}
  \item async-test.js
  \item fake-async-test.js
  \item long-stack-trace-zone.js / long-stack-trace-zone.min.js
  \item proxy.js / proxy.min.js
  \item task-tracking.js / task-tracking.min.js
  \item wtf.js / wtf.min.js
\end{itemize}

% We will look in detail at what each of these does later when examining the
% \href{https://github.com/angular/zone.js/tree/master/lib/zone-spec}
% {<ZONE-MASTER>/lib/zone-spec}
% source directory.

我们在稍后详细说明
\href{https://github.com/angular/zone.js/tree/master/lib/zone-spec}
{<ZONE-MASTER>/lib/zone-spec}
目录。
