% \subsection{Relationship between Zone/ZoneSpec/ZoneDelegate interfaces}
\subsection{Zone/ZoneSpec/ZoneDelegate 之间的接口关系}

% Think of
% \texttt{ZoneSpec}
% as the processing engine that controls how a zone works. It is a
% required parameter to the
% \texttt{Zone.fork()}
% method:

把 \texttt{ZoneSpec} 看成是控制一个 zone 如何工作的处理引擎。
它是 \texttt{Zone.fork()} 方法的一个必要参数。

\begin{minted}{typescript}
  // Used to create a child zone.
  // @param zoneSpec A set of rules which the child zone should follow.
  // @returns {Zone} A new child zone.
  fork(zoneSpec: ZoneSpec): Zone;
\end{minted}


% Often when a zone needs to perform an action, it uses the supplied
% \texttt{ZoneSpec}
% . Do you
% want to record a long stack trace, keep track of tasks, work with WTF (discussed
% later) or run async test well? For each a these a different
% \texttt{ZoneSpec}
% is supplied, and
% each offers different features and comes with different processing costs. Zone.js
% supplies one implementation of the
% \texttt{Zone}
% interface, and multiple implementations of
% the
% \texttt{ZoneSpec}
% interface (in
% \href{https://github.com/angular/zone.js/tree/master/lib/zone-spec}
% {<ZONE-MASTER>/lib/zone-spec}
% ). Application code with
% specialist needs could create a custom
% \texttt{ZoneSpec}
% .

通常,当一个 zone 需要执行一个动作时,它会使用所提供的 \texttt{ZoneSpec}。
你是否想录制长堆栈调用,任务追踪,使用 WTF(稍后进行讨论)或运行异步测试?
对于每个情景提供不同的 \texttt{ZoneSpec},
每个都提供不同的功能,并具有不同的处理花销。
Zone.js 提供 \texttt{Zone} 接口的一个实现,
和多个实现 \texttt{ZoneSpec} 接口(在
\href{https://github.com/angular/zone.js/tree/master/lib/zone-spec}
{<ZONE-MASTER>/lib/zone-spec})。
具有专业需求的应用代码可以创建自定义 \texttt{ZoneSpec}。

% An application can build up a hierarchy of zones and sometimes a zone needs to make
% a call into another zone further up the hierarchy, and for this a
% \texttt{ZoneDelegate}
% is used.

一个应用程序可以建立一个 zones 的层次结构,
有时一个 zone 需要调用更高层次的另一个 zone。
为此,需要使用 \texttt{ZoneDelegate}。
