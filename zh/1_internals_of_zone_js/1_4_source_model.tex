% \section{Source Model}
\section{源码模型}

% The main source for Zone.js is in:

Zone.js 的主要源码位于:

\begin{itemize}
  \item \href{https://github.com/angular/zone.js/tree/master/lib/zone-spec}
        {<ZONE-MASTER>/lib}
\end{itemize}

% It contains a number of sub-directories:

它包含一系列的子目录:

\begin{itemize}
  \item browser
  \item closure
  \item common
  \item extra
  \item jasmine
  \item mix (a mix of browser and node)
  \item mocha
  \item node
  \item rxjs
  \item testing
  \item zone-spec
\end{itemize}

% along with one source file:

以及单一文件:

\begin{itemize}
  \item zone.ts
\end{itemize}

% It is best to think of them arranged as follows:

它们的组织整理如下:

% To enable Zone.js to function, any JavaScript APIs related to asynchronous code
% execution must be patched – a Zone.js specific implementation of the API is needed.
% So for calls such as
% \texttt{setTimeout}
% or
% \texttt{addEventListener}
% and similar, Zone.js needs its
% own handling, so that when timeouts and events and promises get triggered, zone
% code runs first.

为了使 Zone.js 能用于函数,任何和异步代码执行有关的 JavaScript API 都需要打补丁
—— 这些 API Zone.js 都有一个特定实现。
所以对于诸如 \texttt{setTimeout} 或 \texttt{addEventListener} 之类的调用,
Zone.js 都需要能够自己处理,这样,当超时、事件和 promise 触发,zone 代码就会首先运行。

% There are a number of environments supported by Zone.js that need monkey
% patching, such as the browser, the server (node) and Jasmine and each of these has a
% sub-direcory with patching code. The common patching code reside in the common
% directory. The core implementation of the Zone.js API (excluding
% \texttt{ZoneSpec}
% ) is in
% zone.ts file. The additional directory is for zone specs, which are the configurable logic
% one can add to a zone to change its behavior. There are multiple implementations of
% these, and applications could create their own.

Zone.js 支持对多种环境打补丁,browser、server(node)、Jasmine 都有一个子目录包含了补丁代码。
通用补丁代码位于 common 目录。
Zone.js API(不包括 \texttt{ZoneSpec})的核心实现位于 zone.ts 文件。
zone\_spec 目录包含可配置逻辑可以添加到 zone 以更改其行为。
有多种实现,应用可以创建自定义实现。

\subsection{zone.ts}

% The first six hundred lines of the zone.ts file is the well-commented definition of the
% Zone.js API, that will end up in zone.d.ts. The slightly larger remainder of the file is an
% implementation of the
% \texttt{Zone}
% const:

\begin{minted}{typescript}
const Zone: ZoneType = function (global: any) {
  ...
  return (global['Zone'] = Zone);
  ...
  ..
}; ..
\end{minted}


% The
% \texttt{\_ZonePrivate}
% interface defines private information that is managed per zone:

\begin{minted}{typescript}
interface _ZonePrivate {
  currentZoneFrame: () => _ZoneFrame;
  symbol: (name: string) => string;
  scheduleMicroTask: (task?: MicroTask) => void;
  onUnhandledError: (error: Error) => void;
  microtaskDrainDone: () => void;
  showUncaughtError: () => boolean;
  patchEventTarget: (global: any, apis: any[], options?: any) => boolean[];
  patchOnProperties: (
    obj: any,
    properties: string[] | null,
    prototype?: any
  ) => void;
  patchThen: (ctro: Function) => void;
  setNativePromise: (nativePromise: any) => void;
  patchMethod: ..,
  bindArguments: (args: any[], source: string) => any[];
  patchMacroTask: (
    obj: any,
    funcName: string,
    metaCreator: (self: any, args: any[]) => any
  ) => void;
  patchEventPrototype: (_global: any, api: _ZonePrivate) => void;
  isIEOrEdge: () => boolean;
  ObjectDefineProperty: (
    o: any,
    p: PropertyKey,
    attributes: PropertyDescriptor & ThisType<any>
  ) => any;
  ObjectGetOwnPropertyDescriptor: (
    o: any,
    p: PropertyKey
  ) => PropertyDescriptor | undefined;
  ObjectCreate(
    o: object | null,
    properties?: PropertyDescriptorMap & ThisType<any>
  ): any;
  ..
}
\end{minted}


% When using the client API you will not see this, but when debugging through the
% Zone.js implementation, it will crop up from time to time.

% A microtask queue is managed, which requries these variables:

\begin{minted}{typescript}
let _microTaskQueue: Task[] = [];
let _isDrainingMicrotaskQueue: boolean = false;
let nativeMicroTaskQueuePromise: any;
\end{minted}


% \texttt{\_microTaskQueue}
% is an array of microtasks, that must be executed before we give up
% our VM turn.
% \texttt{\_isDrainingMicrotaskQueue}
% is a boolean that tracks if we are in the
% process of emptying the microtask queue. When a task is run within an existing task,
% they are nested and
% \texttt{\_nativeMicroTaskQueuePromise}
% is used to access a native
% microtask queue. Which is stored as a global is not set. Two functions manage a
% microtask queue:

\begin{itemize}
  \item scheduleMicroTask
  \item drainMicroTaskQueue
\end{itemize}

% It also implements three classes:

\begin{itemize}
  \item Zone
  \item ZoneDelegate
  \item ZoneTask
\end{itemize}

% There are no implementations of
% \texttt{ZoneSpec}
% in this file. They are in the separate zone-
% spec sub-directory.

% \texttt{ZoneTask}
% is the simplest of these classes:

\begin{minted}{typescript}
class ZoneTask<T extends TaskType> implements Task {
  public type: T;
  public source: string;
  public invoke: Function;
  public callback: Function;
  public data: TaskData | undefined;
  public scheduleFn: ((task: Task) => void) | undefined;
  public cancelFn: ((task: Task) => void) | undefined;
  _zone: Zone | null = null;
  public runCount: number = 0;
  _zoneDelegates: ZoneDelegate[] | null = null;
  _state: TaskState = 'notScheduled';
  ..
}
\end{minted}


% The constructor just records the supplied parameters and sets up
% \texttt{invoke}
% :

\begin{minted}{typescript}
  constructor(
    type: T,
    source: string,
    callback: Function,
    options: TaskData | undefined,
    scheduleFn: ((task: Task) => void) | undefined,
    cancelFn: ((task: Task) => void) | undefined
  ) {
    this.type = type;
    this.source = source;
    this.data = options;
    this.scheduleFn = scheduleFn;
    this.cancelFn = cancelFn;
    this.callback = callback;
    const self = this;
    // TODO: @JiaLiPassion options should have interface
    if (type === eventTask && options && (options as any).useG) {
      this.invoke = ZoneTask.invokeTask;
    } else {
      this.invoke = function () {
        return ZoneTask.invokeTask.call(global, self, this, <any>arguments);
      };
    }
  }
\end{minted}


% The interesting activity in here is setting up the
% \texttt{invoke}
% function. It increments the
% \texttt{\_numberOfNestedTaskFrames}
% counter, calls
% \texttt{zone.runTask()}
% , and in a
% \texttt{finally}
% block,
% checks if
% \texttt{\_numberOfNestedTaskFrames}
% is 1, and if so, calls  the standalone function
% \texttt{drainMicroTaskQueue()}
% , and then decrements
% \texttt{\_numberOfNestedTaskFrames}
% .

\begin{minted}{typescript}
  static invokeTask(task: any, target: any, args: any): any {
    if (!task) {
      task = this;
    }
    _numberOfNestedTaskFrames++;
    try {
      task.runCount++;
      return task.zone.runTask(task, target, args);
    } finally {
      if (_numberOfNestedTaskFrames == 1) {
        drainMicroTaskQueue();
      }
      _numberOfNestedTaskFrames--;
    }
  }
\end{minted}


% A custom
% \texttt{toString()}
% implementation returns
% \texttt{data.handleId}
% (if available) or else the
% object’s
% \texttt{toString()}
% result:

\begin{minted}{typescript}
  public toString() {
    if (this.data && typeof this.data.handleId !== 'undefined') {
      return this.data.handleId.toString();
    } else {
      return Object.prototype.toString.call(this);
    }
  }
\end{minted}


% \texttt{drainMicroTaskQueue()}
% is defined as:

\begin{minted}{typescript}
function drainMicroTaskQueue() {
  if (!_isDrainingMicrotaskQueue) {
    _isDrainingMicrotaskQueue = true;
    while (_microTaskQueue.length) {
      const queue = _microTaskQueue;
      _microTaskQueue = [];
      for (let i = 0; i < queue.length; i++) {
        const task = queue[i];
        try {
          task.zone.runTask(task, null, null);
        } catch (error) {
          _api.onUnhandledError(error);
        }
      }
    }
    _api.microtaskDrainDone();
    _isDrainingMicrotaskQueue = false;
  }
}
\end{minted}


% The
% \texttt{\_microTaskQueue}
% gets populated via a call to
% \texttt{scheduleMicroTask}
% :

\begin{minted}{typescript}
function scheduleMicroTask(task?: MicroTask) {
  // if we are not running in any task, and there has not been anything
  // scheduled we must bootstrap the initial task creation by manually
  // scheduling the drain
  if (_numberOfNestedTaskFrames === 0 && _microTaskQueue.length === 0) {
    // We are not running in Task, so we need to kickstart the
    // microtask queue.
    if (!nativeMicroTaskQueuePromise) {
      if (global[symbolPromise]) {
        nativeMicroTaskQueuePromise = global[symbolPromise].resolve(0);
      }
    }
    if (nativeMicroTaskQueuePromise) {
      let nativeThen = nativeMicroTaskQueuePromise[symbolThen];
      if (!nativeThen) {
        // native Promise is not patchable, we need to use `then` directly
        // issue 1078
        nativeThen = nativeMicroTaskQueuePromise['then'];
      }
      nativeThen.call(nativeMicroTaskQueuePromise, drainMicroTaskQueue);
    } else {
      global[symbolSetTimeout](drainMicroTaskQueue, 0);
    }
  }
  task && _microTaskQueue.push(task);
}
\end{minted}


% If needed (not running in a task), this calls
% \texttt{setTimeout}
% with timeout set to 0, to
% enqueue a request to drain the microtask queue. Even though the timeout is 0, this
% does not mean that the
% \texttt{drainMicroTaskQueue()}
% call will execute immediately.
% Instead, this puts an event in the JavaScript’s event queue, which after the already
% scheduled events have been handled (there may be one or move already in the
% queue), will itself be handled. The currently executing function will first run to
% completion before any event is removed from the event queue. Hence in the above
% code, where
% \texttt{scheduleQueueDrain()}
% is called before
% \texttt{\_microTaskQueue.push()}
% , is not
% a problem.
% \texttt{\_microTaskQueue.push()}
% will execute first, and then sometime in future,
% the
% \texttt{drainMicroTaskQueue()}
% function will be called via the timeout.

% The
% \texttt{ZoneDelegate}
% class has to handle eight scenarios:

\begin{itemize}
  \item fork
  \item intercept
  \item invoke
  \item handleError
  \item scheduleTask
  \item invokeTask
  \item cancelTask
  \item hasTask
\end{itemize}

% It defines variables to store values for a
% \texttt{ZoneDelegate}
% and
% \texttt{ZoneSpec}
% for each of
% these, which are initialized in the constructor.

\begin{minted}{typescript}
  private _interceptDlgt: ZoneDelegate | null;
  private _interceptZS: ZoneSpec | null;
  private _interceptCurrZone: Zone | null;
\end{minted}


% \texttt{ZoneDelegate}
% also declares three variables, to store the delegates zone and parent
% delegate, and to represent task counts (for each kind of task):

\begin{minted}{typescript}
  public zone: Zone;

  private _taskCounts: {
    microTask: number;
    macroTask: number;
    eventTask: number;
  } = { microTask: 0, macroTask: 0, eventTask: 0 };

  private _parentDelegate: ZoneDelegate | null;
\end{minted}


% In
% \texttt{ZoneDelegate}
% ’s constructor, the
% \texttt{zone}
% and
% \texttt{parentDelegate}
% fields are initialized to
% the supplied parameters, and the
% \texttt{ZoneDelegate}
% and
% \texttt{ZoneSpec}
% fields for the eight
% scenarios are set (using TypeScript type guards), either to the supplied
% \texttt{ZoneSpec}
% (if
% not null), or the parent delegate’s:

\begin{minted}{typescript}
  constructor(
    zone: Zone,
    parentDelegate: ZoneDelegate | null,
    zoneSpec: ZoneSpec | null
  ) {
    this.zone = zone;
    this._parentDelegate = parentDelegate;

    this._forkZS =
      zoneSpec &&
      (zoneSpec && zoneSpec.onFork ? zoneSpec : parentDelegate!._forkZS);
    this._forkDlgt =
      zoneSpec &&
      (zoneSpec.onFork ? parentDelegate : parentDelegate!._forkDlgt);
    this._forkCurrZone =
      zoneSpec && (zoneSpec.onFork ? this.zone : parentDelegate!.zone);
    ..
  }
\end{minted}


% The
% \texttt{ZoneDelegate}
% methods for the eight scenarios just forward the calls to the
% selected
% \texttt{ZoneSpec}
% (pr parent delegate) and does some house keeping. For example,
% the invoke method checks if
% \texttt{\_invokeZS}
% is defined, and if so, calls its
% \texttt{onInvoke}
% ,
% otherwise it calls the supplied callback directly:

\begin{minted}{typescript}
  invoke(
    targetZone: Zone,
    callback: Function,
    applyThis: any,
    applyArgs?: any[],
    source?: string
  ): any {
    return this._invokeZS
      ? this._invokeZS.onInvoke!(
          this._invokeDlgt!,
          this._invokeCurrZone!,
          targetZone,
          callback,
          applyThis,
          applyArgs,
          source
        )
      : callback.apply(applyThis, applyArgs);
  }
\end{minted}


% The
% \texttt{scheduleTask}
% method is a bit different, in that it first
% 1
% tries to use the
% \texttt{\_scheduleTaskZS}
% (if defined), otherwise
% 2
% tries to use the supplied task’s
% \texttt{scheduleFn}
% (if defined), otherwise
% 3
% if a microtask calls
% \texttt{scheduleMicroTask()}
% , otherwise
% 4
% it is
% an error:

\begin{minted}{ts}
scheduleTask(targetZone: Zone, task: Task): Task {
      let returnTask: ZoneTask<any> = task as ZoneTask<any>;
      if (this._scheduleTaskZS) {
 1    if (this._hasTaskZS) {
          returnTask._zoneDelegates!.push(this._hasTaskDlgtOwner!);
        }
        returnTask = this._scheduleTaskZS.onScheduleTask!
                     (this._scheduleTaskDlgt!,
                      this._scheduleTaskCurrZone!, targetZone, task)
                      as ZoneTask<any>;
        if (!returnTask) returnTask = task as ZoneTask<any>;
      } else {
 2    if (task.scheduleFn) {
          task.scheduleFn(task);
 3    } else if (task.type == microTask) {
          scheduleMicroTask(<MicroTask>task);
 4    } else {
          throw new Error('Task is missing scheduleFn.');  }
      }
      return returnTask;
    }
\end{minted}


% The fork method is where new zones get created. If
% \texttt{\_forkZS}
% is defined, it is used,
% otherwise a new zone is created with the supplied
% \texttt{targetZone}
% and
% \texttt{zoneSpec}
% :

\begin{minted}{typescript}
  fork(targetZone: Zone, zoneSpec: ZoneSpec): AmbientZone {
    return this._forkZS
      ? this._forkZS.onFork!(this._forkDlgt!, this.zone, targetZone, zoneSpec)
      : new Zone(targetZone, zoneSpec);
  }
\end{minted}


% The internal variable
% \texttt{\_currentZoneFrame}
% is initialized to the root zone and
% \texttt{\_currentTask}
% to null:

\begin{minted}{typescript}
let _currentZoneFrame: _ZoneFrame = {
  parent: null,
  zone: new Zone(null, null),
};
let _currentTask: Task | null = null;
let _numberOfNestedTaskFrames = 0;
\end{minted}


