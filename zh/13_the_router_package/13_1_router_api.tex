% \section{Router API}
\section{Router API}

% The exported API of the Angular Router package can be represented as:

Angular Router 包的导出 API 可以表示为:

TODO:插图

% The Angular Router source tree is at:

Angular Router 源码位于:

\begin{itemize}
  \item \href{https://github.com/angular/angular/tree/master/packages/router}
        {<ANGULAR-MASTER>/packages/router}
\end{itemize}

% Its root directory contains index.ts, which just exports the contents of public\_api.ts,
% which in turn exports the contents of .src/index.ts.

根目录包含 index.ts,它只是导出了 public\_api.t 的内容,
而后者又导出 .src/index.ts 的内容。

% This lists the exports and gives an initial impression of the size of the router package:

这列出了 exports 内容并给出了路由器包大小的初步印象:

\input{../output/13_the_router_package/code/13_1_0.tex}

% It also has the line:

它还有一行:

\begin{minted}{typescript}
export * from './private_export';
\end{minted}


% As already discussed, such private exports are intended for other packages within the
% Angular Framework itself, and not to be directly used by Angular applications. The
% private\_export.ts file has these exports (note names are prepended with ’ ’);
% ɵ

正如已经讨论过的,此类私有导出旨在用于 Angular 框架本身,而不应该由 Angular 应用直接使用。
private\_export.ts 文件具有这些导出(注意,它们的名称有 “ɵ” 字符);

\begin{minted}{typescript}
export { ROUTER_PROVIDERS as ɵROUTER_PROVIDERS } from './router_module';
export { flatten as ɵflatten } from './utils/collection';
\end{minted}

