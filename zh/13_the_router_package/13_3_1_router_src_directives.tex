% \subsection{router/src/directives}
\subsection{router/src/directives}

% This directory has the following files:

该目录有如下文件:

\begin{itemize}
  \item router\_link.ts
  \item router\_link\_active.ts
  \item router\_outlet.ts
\end{itemize}

% The router\_link.ts file contains the
% \texttt{RouterLink}
% directive:

router\_link.ts 文件包含 \texttt{RouterLink} 指令:

\begin{minted}{typescript}
@Directive({ selector: ':not(a)[routerLink]' })
export class RouterLink {
  @Input() queryParams: { [k: string]: any };
  @Input() fragment: string;
  @Input() queryParamsHandling: QueryParamsHandling;
  @Input() preserveFragment: boolean;
  @Input() skipLocationChange: boolean;
  @Input() replaceUrl: boolean;
  private commands: any[] = [];
  private preserve: boolean;

  constructor(
    private router: Router,
    private route: ActivatedRoute,
    @Attribute('tabindex') tabIndex: string,
    renderer: Renderer2,
    el: ElementRef
  ) {
    if (tabIndex == null) {
      renderer.setAttribute(el.nativeElement, 'tabindex', '0');
    }
  }
  ..
}
\end{minted}


% The router link commands are set via:

router link 命令通过以下方式设置:

\begin{minted}{typescript}
  @Input()
  set routerLink(commands: any[] | string) {
    if (commands != null) {
      this.commands = Array.isArray(commands) ? commands : [commands];
    } else {
      this.commands = [];
    }
  }
\end{minted}


% When the link is clicked, the
% \texttt{onClick()}
% method is called:

当链接被点击时,会调用 \texttt{onClick()} 方法:

\begin{minted}{typescript}
  @HostListener('click')
  onClick(): boolean {
    const extras = {
      skipLocationChange: attrBoolValue(this.skipLocationChange),
      replaceUrl: attrBoolValue(this.replaceUrl),
    };
    this.router.navigateByUrl(this.urlTree, extras);
    return true;
  }
\end{minted}


% The urlTree getter uses
% \texttt{Router.createlUrlTree()}
% :

urlTree getter 使用 \texttt{Router.createlUrlTree()}:

\begin{minted}{typescript}
  get urlTree(): UrlTree {
    return this.router.createUrlTree(this.commands, {
      relativeTo: this.route,
      queryParams: this.queryParams,
      fragment: this.fragment,
      preserveQueryParams: attrBoolValue(this.preserve),
      queryParamsHandling: this.queryParamsHandling,
      preserveFragment: attrBoolValue(this.preserveFragment),
    });
  }
\end{minted}


% The same file also contains the
% \texttt{RouterLinkWithHref}
% directive:

该文件还包含了 \texttt{RouterLinkWithHref} 指令:

\begin{minted}{typescript}
@Directive({ selector: 'a[routerLink]' })
export class RouterLinkWithHref implements OnChanges, OnDestroy {
  ..
}
\end{minted}


% This has a href:

一个 href:

\begin{minted}{typescript}
  // the url displayed on the anchor element.
  @HostBinding() href: string;
\end{minted}


% and manages the
% \texttt{urlTree}
% as a field and sets it from the constructor via a call to:

并将 \texttt{urlTree} 作为字段进行管理,并通过调用从构造函数设置它:

\begin{minted}{typescript}
  private updateTargetUrlAndHref(): void {
    this.href = this.locationStrategy.prepareExternalUrl(
      this.router.serializeUrl(this.urlTree)
    );
  }
\end{minted}


% The router\_link\_active.ts file contains the
% \texttt{RouterLinkActive}
% directive:

router\_link\_active.ts 文件包含 \texttt{RouterLinkActive} 指令:

\begin{minted}{typescript}
@Directive({
  selector: '[routerLinkActive]',
  exportAs: 'routerLinkActive',
})
export class RouterLinkActive
  implements OnChanges, OnDestroy, AfterContentInit {
  ..
}
\end{minted}


% This is used to add a CSS class to an element representing an active route. Its
% constructor is defiend as:

用于向活动路由元素添加 CSS 类。
它的构造函数定义如下:

\begin{minted}{typescript}
  constructor(
    private router: Router,
    private element: ElementRef,
    private renderer: Renderer2,
    private cdr: ChangeDetectorRef
  ) {
    this.subscription = router.events.subscribe((s) => {
      if (s instanceof NavigationEnd) {
        this.update();
      }
    });
  }
\end{minted}


% Its
% \texttt{update}
% method uses the configured renderer to set the element class:

它的 \texttt{update} 方法使用配置的渲染器来设置元素的 class:

\begin{minted}{typescript}
  private update(): void {
    if (!this.links || !this.linksWithHrefs || !this.router.navigated) return;
    Promise.resolve().then(() => {
      const hasActiveLinks = this.hasActiveLinks();
      if (this.isActive !== hasActiveLinks) {
        (this as any).isActive = hasActiveLinks;
        this.classes.forEach((c) => {
          if (hasActiveLinks) {
            this.renderer.addClass(this.element.nativeElement, c);
          } else {
            this.renderer.removeClass(this.element.nativeElement, c);
          }
        });
      }
    });
  }
\end{minted}


% The router\_outlet.ts file contains the
% \texttt{RouterOutlet}
% class:

router\_outlet.ts 文件包含了 \texttt{RouterOutlet} 类:

\begin{minted}{typescript}
// Acts as a placeholder that Angular dynamically fills based on the
// current router * state.
@Directive({ selector: 'router-outlet', exportAs: 'outlet' })
export class RouterOutlet implements OnDestroy, OnInit {
  private activated: ComponentRef<any> | null = null;
  private _activatedRoute: ActivatedRoute | null = null;
  private name: string;

  @Output('activate') activateEvents = new EventEmitter<any>();
  @Output('deactivate') deactivateEvents = new EventEmitter<any>();

  constructor(
    private parentContexts: ChildrenOutletContexts,
    %\step{1}% private location: ViewContainerRef,
    %\step{2}% private resolver: ComponentFactoryResolver,
    @Attribute('name') name: string,
    private changeDetector: ChangeDetectorRef
  ) {
    this.name = name || PRIMARY_OUTLET;
    parentContexts.onChildOutletCreated(this.name, this);
  }
  ..
}
\end{minted}


% This is where application component whose lifecycle depends on the router live. We
% note the
% \texttt{ViewContainerRef}
% 1
% and
% \texttt{ComponentFactoryResolver}
% 2
% parameters to the
% constructor.

这是生命周期取决于路由器的应用程序组件所在的位置。
我们注意到构造函数的
\texttt{ViewContainerRef} \step{1} 和
\texttt{ComponentFactoryResolver} \step{2} 参数。

% \texttt{ngOnInit}
% will either call
% \texttt{attach}
% 1
% or
% \texttt{activateWith}
% 2
% , depending on whether there is
% an existing component:

\texttt{ngOnInit} 将调用
\texttt{attach} \step{1} 或 \texttt{activateWith} \step{2},
具体取决于是否存在现有组件:

\begin{minted}{typescript}
  ngOnInit(): void {
    if (!this.activated) {
      // If the outlet was not instantiated at the time the

      // route got activated we need to populate
      // the outlet when it is initialized (ie inside a NgIf)
      const context = this.parentContexts.getContext(this.name);
      if (context && context.route) {
        if (context.attachRef) {
          // `attachRef` is populated when there is an
          // existing component to mount
          %\step{1}% this.attach(context.attachRef, context.route);
        } else {
          // otherwise the component defined in the configuration is created
          %\step{2}% this.activateWith(
            context.route,
            context.resolver || null
          );
        }
      }
    }
  }
\end{minted}


% attach is defined as:

attach 定义为:

\begin{minted}{typescript}
  // Called when the `RouteReuseStrategy` instructs to
  // re-attach a previously detached subtree
  attach(ref: ComponentRef<any>, activatedRoute: ActivatedRoute) {
    this.activated = ref;
    this._activatedRoute = activatedRoute;
    this.location.insert(ref.hostView);
  }
\end{minted}


% When its activateWith method is called, the resolver will be asked to resolve a
% component factory for the component:

当它的 activateWith 方法被调用时,resolver 将被要求为组件解析一个组件工厂:

\begin{minted}{typescript}
  activateWith(
    activatedRoute: ActivatedRoute,
    resolver: ComponentFactoryResolver | null
  ) {
    if (this.isActivated) {
      throw new Error('Cannot activate an already activated outlet');
    }
    this._activatedRoute = activatedRoute;
    const snapshot = activatedRoute._futureSnapshot;
    const component = <any>snapshot.routeConfig!.component;
    resolver = resolver || this.resolver;
    const factory = resolver.resolveComponentFactory(component);
    const childContexts = this.parentContexts.getOrCreateContext(
      this.name
    ).children;
    const injector = new OutletInjector(
      activatedRoute,
      childContexts,
      this.location.injector
    );
    this.activated = this.location.createComponent(
      factory,
      this.location.length,
      injector
    );

    // Calling `markForCheck` to make sure we will run the change
    // detection when the     // `RouterOutlet` is inside a
    //  `ChangeDetectionStrategy.OnPush` component.
    this.changeDetector.markForCheck();
    this.activateEvents.emit(this.activated.instance);
  }
\end{minted}


% The location field is of type
% \texttt{ViewContainerRef,}
% which we saw being set in the
% constructor.
% \texttt{ViewContainerRef}
% is defined in:

location 字段的类型为 \texttt{ViewContainerRef},我们看到它是在构造函数中设置的。
\texttt{ViewContainerRef} 定义在:

\begin{itemize}
  \item \href{https://github.com/angular/angular/blob/master/packages/core/src/linker/view_container_ref.ts}
        {<ANGULAR-MASTER>/packages/core/src/linker/view\_container\_ref.ts}
\end{itemize}

\begin{minted}{typescript}
export abstract class ViewContainerRef {
  // Returns the number of Views currently attached to this container.
  abstract get length(): number;

  // Instantiates a single Component and inserts its Host View into this
  // container at the specified `index`.
  // The component is instantiated using its ComponentFactory which can be
  // obtained via ComponentFactoryResolver.resolveComponentFactory
  // If `index` is not specified, the new View will be inserted as the last
  // View in the container. Returns the {@link ComponentRef} of the Host
  // View created for the newly instantiated Component.
  abstract createComponent<C>(
    componentFactory: ComponentFactory<C>,
    index?: number,
    injector?: Injector,
    projectableNodes?: any[][],
    ngModule?: NgModuleRef<any>
  ): ComponentRef<C>;
}
\end{minted}


% So the component is appended as the last entry in the
% \texttt{ViewContainer}
% .

因此该组件被附加为 \texttt{ViewContainer} 中的最后一个条目。
