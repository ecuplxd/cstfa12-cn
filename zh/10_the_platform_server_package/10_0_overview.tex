\section{Overview}

Platform-Server represents a platform when the application is running on a server.

Most of the time Angular applications run in the browser and use either
Platform-Browser (with the offline template compiler) or Platform-Browser-Dynamic
(with the runtime template compiler). Running Angular applications on the server is a
more specialist deployment. It can be of interest for search engine optimization and
for pre-rendering output, which is later downloaded to browsers (this can speed up
initial display of content to users for some types of applications; and also ease
development with backend databases that might not be exposed via REST API to
browser code). Platform-Server is used by Angular Universal as its platform module.

When considering Platform-Server, there are two questions the curious software
engineer might ponder. Firstly, we wonder, if for the browser there are dynamic and
non-dynamic versions of the platform, why not for the server? The answer to this is
that for the browser, one wishes to support low-end devices serviced by poor network
connections, so reducing the burden on the browser is of great interest (hence using
the offline template compiler only); but one assumes the server has ample disk space
and RAM and a small amount of extra code is not an issue, so bundling both kinds of
server platforms (one that uses the offline template compiler, the other - “dynamic” -
that uses the runtime template compiler) in the same module simplifies matters.

The second question is how rendering works on the server (where there is no DOM)?
The answer is the rendered output is written via the Angular renderer API to an HTML
file, which then can be downloaded to a browser or made available to a search engine
- we need to explore how this rendering works (but for the curious, a library called
Domino is used which provides a DOM for node apps).
