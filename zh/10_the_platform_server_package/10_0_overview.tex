% \section{Overview}
\section{概览}

% Platform-Server represents a platform when the application is running on a server.

Platform-Server 表示应用在服务器上运行时的平台。

% Most of the time Angular applications run in the browser and use either
% Platform-Browser (with the offline template compiler) or Platform-Browser-Dynamic
% (with the runtime template compiler). Running Angular applications on the server is a
% more specialist deployment. It can be of interest for search engine optimization and
% for pre-rendering output, which is later downloaded to browsers (this can speed up
% initial display of content to users for some types of applications; and also ease
% development with backend databases that might not be exposed via REST API to
% browser code). Platform-Server is used by Angular Universal as its platform module.

大多数情况下,
Angular 应用在浏览器中运行并使用 Platform-Browser
(使用离线模板编译器)或 Platform-Browser-Dynamic(使用运行时模板编译器)。
在服务器上运行 Angular 应用是一种更专业的部署。
它可能对搜索引擎优化和预渲染输出感兴趣,这些输出后来下载到浏览器
(这可以加快向用户显示某些类型应用的内容的速度;
还可以简化后端数据库的开发,这些数据库可能不是通过 REST API 暴露给浏览器代码)。
Angular Universal 使用 Platform-Server 作为其平台模块。

% When considering Platform-Server, there are two questions the curious software
% engineer might ponder. Firstly, we wonder, if for the browser there are dynamic and
% non-dynamic versions of the platform, why not for the server? The answer to this is
% that for the browser, one wishes to support low-end devices serviced by poor network
% connections, so reducing the burden on the browser is of great interest (hence using
% the offline template compiler only); but one assumes the server has ample disk space
% and RAM and a small amount of extra code is not an issue, so bundling both kinds of
% server platforms (one that uses the offline template compiler, the other - “dynamic” -
% that uses the runtime template compiler) in the same module simplifies matters.

在考虑 Platform-Server 时,
好奇的软件工程师可能会思考两个问题。
首先,我们想知道,如果浏览器有平台的动态和非动态版本,为什么服务器没有?
这个问题的答案是,对于浏览器来说,希望支持网络连接较差的低端设备,
因此减轻浏览器的负担很重要(因此仅使用离线模板编译器);
但是假设服务器有足够的磁盘空间和 RAM 以及少量额外的代码不是问题,
所以捆绑两种服务器平台(一个使用离线模板编译器,另一个 - “动态” - 使用运行时模板编译器)
在同一模块中简化了事情。

% The second question is how rendering works on the server (where there is no DOM)?
% The answer is the rendered output is written via the Angular renderer API to an HTML
% file, which then can be downloaded to a browser or made available to a search engine
% - we need to explore how this rendering works (but for the curious, a library called
% Domino is used which provides a DOM for node apps).

第二个问题是渲染如何在服务器(没有 DOM)上工作?
答案是渲染输出通过 Angular 渲染器 API 写入 HTML 文件,
然后可以将其下载到浏览器或提供给搜索引擎 - 我们需要探索这种渲染是如何工作的
(但对于好奇的人,一个库 使用称为 Domino,它为节点应用程序提供 DOM)。
