% \section{Source Tree Layout}
\section{目录结构}

% The Tsickle source tree has these sub-directories:

Tsickle 源码有以下子目录:

\begin{itemize}
  \item src
  \item test
  \item test\_files
  \item third\_party
\end{itemize}

% The main directory has these important files:

主目录有这些重要文件:

\begin{itemize}
  \item readme.md
  \item package.json
  \item gulpfile.js
  \item tsconfig.json
\end{itemize}

% The readme.md contains useful information about the project, including this important
% guidance about the use of tsconfig.json:

% \begin{quote}
%   Project Setup
%
%   Tsickle works by wrapping tsc. To use it, you must set up your project such
%   that it builds correctly when you run tsc from the command line, by
%   configuring the settings in tsconfig.json.

%   If you have complicated tsc command lines and flags in a build file (like a
%   gulpfile etc.) Tsickle won't know about it. Another reason it's nice to put
%   everything in tsconfig.json is so your editor inherits all these settings as
%   well.
% \end{quote}

readme.md 包含有关项目的有用信息,包括这个重要的
tsconfig.json 使用指南:

\begin{quote}
  项目设置

  Tsickle 通过包装 tsc 来工作。
  要使用它,你必须通过配置 tsconfig.json 中的设置来设置您的项目,
  以便在从命令行运行 tsc 时它可以正确构建。

  如果您在构建文件(如 gulpfile 等)中有复杂的 tsc 命令行和标志,Tsickle 不会知道它。
  将所有内容都放在 tsconfig.json 中的另一个原因是你的编辑器也可以继承所有这些设置。
\end{quote}


readme.md 包含有关项目的有用信息,包括有关使用 tsconfig.json 的重要指南:

% The package.json file contains:

package.json 文件包含:

\begin{minted}{json}
  "main": "built/src/tsickle.js",
  "bin": "built/src/main.js",
\end{minted}


% The gulpfile.js file contains the following Gulp tasks:

gulpfile.js 文件包含以下 Gulp 任务:

\begin{itemize}
  \item gulp format
  \item gulp test.check-format (formatting tests)
  \item gulp test.check-lint (run tslint)
\end{itemize}
