\subsection{platform-webworker/src}

% In addition to the export files, this directory contains the following files:

\begin{itemize}
  \item worker\_app.ts
  \item worker\_render.ts
\end{itemize}

% worker\_app.ts supplies functionality for an application that runs in a worker and
% worker\_render.ts supplies functionality for the the main UI thread. They communicate
% via a message broker layered above a simple message bus.

% A web worker cannot use the DOM directly in a web browser. Therefore Angular’s
% Platform-WebWorker creates a message bus between the web worker and the main UI
% thread and rendering (and event processing) takes place over this message bus.

% The
% \texttt{platformWorkerApp}
% const creates a platform factory for a worker app:

\begin{minted}{typescript}
export const platformWorkerApp = createPlatformFactory(
  platformCore,
  'workerApp',
  [{ provide: PLATFORM_ID, useValue: PLATFORM_WORKER_APP_ID }]
);
\end{minted}


% Two important helper functions are supplied.
% \texttt{createMessageBus}
% creates the message
% bus that will supply communications between the main UI thread and the web worker:

\begin{minted}{typescript}
export function createMessageBus(zone: NgZone): MessageBus {
  const sink = new PostMessageBusSink(_postMessage);
  const source = new PostMessageBusSource();
  const bus = new PostMessageBus(sink, source);
  bus.attachToZone(zone);
  return bus;
}
\end{minted}


% \texttt{setupWebWorker}
% makes the web worker’s DOM adapter implementation as the current
% DOM adapter. The DOM renderer is used both for apps running in a normal browser UI
% thread and apps running in web workers. What is different is the DOM adapter – for a
% browser UI thread, the DOM adapter just maps to the underlying browser DOM API.
% There is no underlying DOM API in a web worker. So for an app running in a web
% worker, the worker DOM adapter needs to forward all DOM actions across the
% message bus to the browser’s main UI thread.

\begin{minted}{typescript}
export function setupWebWorker(): void {
  WorkerDomAdapter.makeCurrent();
}
\end{minted}


% Finally, the
% \texttt{NgModule}
% is defined:

\begin{minted}{typescript}
// The ng module for the worker app side.
@NgModule({
  providers: [
    BROWSER_SANITIZATION_PROVIDERS,
    Serializer,
    { provide: DOCUMENT, useValue: null },
    ClientMessageBrokerFactory,
    ServiceMessageBrokerFactory,
    WebWorkerRendererFactory2,
    { provide: RendererFactory2, useExisting: WebWorkerRendererFactory2 },
    { provide: ON_WEB_WORKER, useValue: true },
    RenderStore,
    { provide: ErrorHandler, useFactory: errorHandler, deps: [] },
    { provide: MessageBus, useFactory: createMessageBus, deps: [NgZone] },
    { provide: APP_INITIALIZER, useValue: setupWebWorker, multi: true },
  ],
  exports: [CommonModule, ApplicationModule],
})
export class WorkerAppModule {}
\end{minted}


% The worker\_render.ts file has implementations for the worker UI. This runs in the
% main UI thread and uses the
% \texttt{BrowserDomAdapter}
% , which writes to the underlying DOM
% API of the browser.
% \texttt{platformWorkerUi}
% represents a platform factory:

\begin{minted}{typescript}
export const platformWorkerUi = createPlatformFactory(
  platformCore,
  'workerUi',
  _WORKER_UI_PLATFORM_PROVIDERS
);
\end{minted}


% The providers used are as followed:

\begin{minted}{typescript}
export const _WORKER_UI_PLATFORM_PROVIDERS: StaticProvider[] = [
  { provide: NgZone, useFactory: createNgZone, deps: [] },
  {
    provide: MessageBasedRenderer2,
    deps: [
      ServiceMessageBrokerFactory,
      MessageBus,
      Serializer,
      RenderStore,
      RendererFactory2,
    ],
  },
  {
    provide: WORKER_UI_STARTABLE_MESSAGING_SERVICE,
    useExisting: MessageBasedRenderer2,
    multi: true,
  },
  BROWSER_SANITIZATION_PROVIDERS,
  { provide: ErrorHandler, useFactory: _exceptionHandler, deps: [] },
  { provide: DOCUMENT, useFactory: _document, deps: [] },
  {
    provide: EVENT_MANAGER_PLUGINS,
    useClass: DomEventsPlugin,
    deps: [DOCUMENT, NgZone],
    multi: true,
  },
  {
    provide: EVENT_MANAGER_PLUGINS,
    useClass: KeyEventsPlugin,
    deps: [DOCUMENT],
    multi: true,
  },
  {
    provide: EVENT_MANAGER_PLUGINS,
    useClass: HammerGesturesPlugin,
    deps: [DOCUMENT, HAMMER_GESTURE_CONFIG],
    multi: true,
  },
  { provide: HAMMER_GESTURE_CONFIG, useClass: HammerGestureConfig, deps: [] },
  APP_ID_RANDOM_PROVIDER,
  { provide: DomRendererFactory2, deps: [EventManager, DomSharedStylesHost] },
  { provide: RendererFactory2, useExisting: DomRendererFactory2 },
  { provide: SharedStylesHost, useExisting: DomSharedStylesHost },
  {
    provide: ServiceMessageBrokerFactory,
    useClass: ServiceMessageBrokerFactory,
    deps: [MessageBus, Serializer],
  },
  {
    provide: ClientMessageBrokerFactory,
    useClass: ClientMessageBrokerFactory,
    deps: [MessageBus, Serializer],
  },
  { provide: Serializer, deps: [RenderStore] },
  { provide: ON_WEB_WORKER, useValue: false },
  { provide: RenderStore, deps: [] },
  { provide: DomSharedStylesHost, deps: [DOCUMENT] },
  { provide: Testability, deps: [NgZone] },
  { provide: EventManager, deps: [EVENT_MANAGER_PLUGINS, NgZone] },
  { provide: WebWorkerInstance, deps: [] },
  {
    provide: PLATFORM_INITIALIZER,
    useFactory: initWebWorkerRenderPlatform,
    multi: true,
    deps: [Injector],
  },
  { provide: PLATFORM_ID, useValue: PLATFORM_WORKER_UI_ID },
  {
    provide: MessageBus,
    useFactory: messageBusFactory,
    deps: [WebWorkerInstance],
  },
];
\end{minted}


% Note the provider configuration for
% \texttt{WORKER\_UI\_STARTABLE\_MESSAGING\_SERVICE}
% is set
% to multi - thus allowing multiple messaging services to be started. Also note that
% \texttt{initWebWorkerRenderPlatform}
% is registered as a
% \texttt{PLATFORM\_INITIALIZER}
% , so it is
% going to be called when the platform launches.

% \texttt{WebWorkerInstance}
% is a simple injectable class representing the web worker and its
% message bus (note the
% \texttt{init}
% method just initializes the two fields):

\begin{minted}{typescript}
/**
 * Wrapper class that exposes the Worker
 * and underlying {@link MessageBus} for lower level message passing.
 */
@Injectable()
export class WebWorkerInstance {
  public worker: Worker;
  public bus: MessageBus;

  public init(worker: Worker, bus: MessageBus) {
    this.worker = worker;
    this.bus = bus;
  }
}
\end{minted}


% Now let’s look at
% \texttt{initWebWorkerRenderPlatform}
% :

\begin{minted}{typescript}
function initWebWorkerRenderPlatform(injector: Injector): () => void {
  return () => {
    %\step{1}% BrowserDomAdapter.makeCurrent();
    BrowserGetTestability.init();
    let scriptUri: string;
    try {
      %\step{2}% scriptUri = injector.get(WORKER_SCRIPT);
    } catch (e) {
      throw new Error(
        "You must provide your WebWorker's initialization script with the WORKER_SCRIPT token"
      );
    }
    const instance = injector.get(WebWorkerInstance);
    %\step{3}% spawnWebWorker(scriptUri, instance);
    %\step{4}% initializeGenericWorkerRenderer(injector);
  };
}
\end{minted}


% It returns a function that makes the browser Dom adatper the current adapter
% 1
% ; then
% gets the worker script from di
% 2
% ; then calls
% \texttt{spawnWebWorker}
% 3
% ; and finally calls
% \texttt{initializeGenericWorkerRenderer}
% 4
% . The
% \texttt{spawnWebWorker}
% function is defined as:

\begin{minted}{typescript}
// Spawns a new class and initializes the WebWorkerInstance
function spawnWebWorker(uri: string, instance: WebWorkerInstance): void {
  %\step{1}% const webWorker: Worker = new Worker(uri);
  %\step{2}% const sink = new PostMessageBusSink(webWorker);
  const source = new PostMessageBusSource(webWorker);
  const bus = new PostMessageBus(sink, source);
  %\step{3}% instance.init(webWorker, bus);
  ..
}
\end{minted}


% It first creates a new web worker
% 1
% ; then it create the message bus with its sinka dn
% source
% 2
% and finally it calls
% \texttt{WebWorkerInstance.init}
% 3
% which we have already seen.
% The
% \texttt{initializeGenericWorkerRenderer}
% function is defined as:

\begin{minted}{typescript}
function initializeGenericWorkerRenderer(injector: Injector) {
  const bus = injector.get(MessageBus);
  const zone = injector.get<NgZone>(NgZone);
  bus.attachToZone(zone);

  // initialize message services after the bus has been created
  const services = injector.get(WORKER_UI_STARTABLE_MESSAGING_SERVICE);
  zone.runGuarded(() => {
    services.forEach((svc: any) => {
      svc.start();
    });
  });
}
\end{minted}


% It first asks the dependency injector for a message bus and a zone and attached the
% bus to the zone. Then it also asks the dependency injector for a list of
% \texttt{WORKER\_UI\_STARTABLE\_MESSAGING\_SERVICE}
% (remember we noted it was configured as
% a multi provider), and for each service, starts it.
