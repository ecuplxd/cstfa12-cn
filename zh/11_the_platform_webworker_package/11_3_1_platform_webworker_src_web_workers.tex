\subsection{platform-webworker/src/web\_workers}

% The source files for web\_workers are provided in three sub-directories, one of which
% has shared messaging functionality use both by the UI side and worker side and the
% communication, the second refers only to the UI side and the third only to the worker
% side.

\begin{itemize}
  \item shared/api.ts
  \item shared/client\_message\_broker.ts
  \item shared/message\_bus.ts
  \item shared/messaging\_api.ts
  \item shared/post\_message\_bus.ts
  \item shared/render\_store.ts
  \item shared/serializer.ts
  \item shared/service\_message\_broker.ts
  \item ui/event\_dispatcher.ts
  \item ui/event\_serializer.ts
  \item ui/location\_providers.ts
  \item ui/platform\_location.ts
  \item ui/renderer.ts
  \item worker/location\_providers.ts
  \item worker/platform\_location.ts
  \item worker/renderer.ts
  \item worker/worker\_adapter.ts
\end{itemize}

% Communication between the UI thread and the webworker is handled by a low-level
% multi-channel message bus. The shared/messaging\_api.ts file defines the names of
% the three channels used by Angular:

\begin{minted}{typescript}
/**
 * All channels used by angular's WebWorker components are listed here.
 * You should not use these channels in your application code.
 */
export const RENDERER_2_CHANNEL = 'v2.ng-Renderer';
export const EVENT_2_CHANNEL = 'v2.ng-Events';
export const ROUTER_CHANNEL = 'ng-Router';
\end{minted}


% If you are familiar with TCP/IP, the channel name here serves the same purpose as a
% port number – it is needed to multiplex multiple independent message streams on a
% single data connection. Note the 2 in the names. This means these channels are for
% the updated renderer architecture.

% The message\_bus.ts file defines an abstract
% \texttt{MessageBus}
% class and two interfaces,
% \texttt{MessageBusSource}
% and
% \texttt{MessageBusSink}
% . Let’s look at the interfaces first.
% \texttt{MessageBusSource}
% is for incoming messages. This interface describes the functionality
% a message source is expected to supply. It has methods to initialize a channel based
% MessageBus
% MessageBusSource
% MessageBusSink

% on a string name, to attach to a zone, and to return an RxJS observable
% (
% \texttt{EventEmitter}
% ) that can be observed in order to read the incoming messages.

\begin{minted}{typescript}
export interface MessageBusSource {
  /**
   * Sets up a new channel on the MessageBusSource.
   * MUST be called before calling from on the channel.
   * If runInZone is true then the source will emit events inside the
   * angular zone. if runInZone is false then the source will emit
   * events inside the global zone.
   */
  initChannel(channel: string, runInZone: boolean): void;

  /**
   * Assigns this source to the given zone.
   * Any channels which are initialized with runInZone set to true will
   * emit events that will be
   * executed within the given zone.
   */
  attachToZone(zone: NgZone): void;

  /**
   * Returns an {@link EventEmitter} that emits every time a message
   * is received on the given channel.
   */
  from(channel: string): EventEmitter<any>;
}
\end{minted}


% Similarly,
% \texttt{MessageBusSink}
% is for outgoing messages. Again it allows a named channel
% to be initialized, attacing to a zone and returns a RxJS observer (
% \texttt{EventEmitter}
% ) used
% to send messages:

\begin{minted}{typescript}
export interface MessageBusSink {
  /**
   * Sets up a new channel on the MessageBusSink.
   * MUST be called before calling to on the channel.
   * If runInZone is true the sink will buffer messages and send only
   * once the zone exits.
   * if runInZone is false the sink will send messages immediately.
   */
  initChannel(channel: string, runInZone: boolean): void;

  /**
   * Assigns this sink to the given zone.
   * Any channels which are initialized with runInZone set to true will
   * wait for the given zone to exit before sending messages.
   */
  attachToZone(zone: NgZone): void;

  /**
   * Returns an {@link EventEmitter} for the given channel
   * To publish methods to that channel just call next on the
   * returned emitter
   */
  to(channel: string): EventEmitter<any>;
}
\end{minted}


% \texttt{MessageBus}
% is an abstract class that implements both
% \texttt{MessageBusSource}
% and
% \texttt{MessageBusSink}
% :

\begin{minted}{typescript}
/**
 * Message Bus is a low level API used to communicate between the UI and
 * the background.
 * Communication is based on a channel abstraction. Messages published in a
 * given channel to one MessageBusSink are received on the same channel
 * by the corresponding MessageBusSource.
 */
export abstract class MessageBus implements MessageBusSource, MessageBusSink {
  /**
   * Sets up a new channel on the MessageBus.
   * MUST be called before calling from or to on the channel.
   * If runInZone is true then the source will emit events inside the
   * angular zone and the sink will buffer messages and send only once
   * the zone exits.
   * if runInZone is false then the source will emit events inside the
   * global zone and the sink will send messages immediately.
   */
  abstract initChannel(channel: string, runInZone?: boolean): void;

  /**
   * Assigns this bus to the given zone.
   * Any callbacks attached to channels where runInZone was set to
   * true on initialization
   * will be executed in the given zone.
   */
  abstract attachToZone(zone: NgZone): void;

  /**
   * Returns an {@link EventEmitter} that emits every time a message
   * is received on the given channel.
   */
  abstract from(channel: string): EventEmitter<any>;

  /**
   * Returns an {@link EventEmitter} for the given channel
   * To publish methods to that channel just call next on the
   * returned emitter
   */
  abstract to(channel: string): EventEmitter<any>;
}
\end{minted}


% So far the message bus description has only specified what the functionality that
% needs to be supplied. A single implementation is supplied, in post\_message\_bus.ts,
% based on the
% \texttt{postMessage}
% API. This defines three classes,
% \texttt{PostMessageBusSource}
% ,
% \texttt{PostMessageBusSink}
% and
% \texttt{PostMessageBus}
% , that implement the above similarly
% named types.

\begin{minted}{typescript}
/**
 * A TypeScript implementation of {@link MessageBus} for communicating
 * via JavaScript's  postMessage API.
 */
@Injectable()
export class PostMessageBus implements MessageBus {
  ..
}
\end{minted}


% A useful private class is supplied called
% \texttt{\_Channel}
% that keeps track of two pieces of
% data:

\begin{minted}{typescript}
/**
 * Helper class that wraps a channel's {@link EventEmitter} and
 * keeps track of if it should run in the zone.
 */
class _Channel {
  constructor(public emitter: EventEmitter<any>, public runInZone: boolean) {}
}
\end{minted}


% Channels are initialized with
% \texttt{PostMessageBusSource.initChannel()}
% :

\begin{minted}{ts}
 initChannel(channel: string, runInZone: boolean = true) {
    if (this._channels.hasOwnProperty(channel)) {
      throw new Error(`${channel} has already been initialized`);
    }initChannel(channel: string, runInZone: boolean = true) {
    if (this._channels.hasOwnProperty(channel)) {
      throw new Error(`${channel} has already been initialized`);
    }

    const emitter = new EventEmitter(false);
    const channelInfo = new _Channel(emitter, runInZone);
    this._channels[channel] = channelInfo;
  }
\end{minted}


% So we see a channel is nothing more that a name that maps to a
% \texttt{\_Channel}
% , which is
% just an
% \texttt{EventEmitter}
% and a boolean (
% \texttt{runInZone}
% ).
% \texttt{PostMessageBusSource}
% manages
% a map of these called \_channels:

\begin{minted}{typescript}
export class PostMessageBusSource implements MessageBusSource {
  private _zone: NgZone;
  private _channels: { [key: string]: _Channel } = {};
  ..
}
\end{minted}


% The
% \texttt{from()}
% method just returns the appropriate channel emitter:

\begin{minted}{typescript}
  from(channel: string): EventEmitter<any> {
    if (this._channels.hasOwnProperty(channel)) {
      return this._channels[channel].emitter;
    } else {
      throw new Error(`${channel} is not set up. Did you forget to call
initChannel?`);
    }
  }
\end{minted}


% The constructor calls
% \texttt{addEventListener}
% to add an event listener:

\begin{minted}{typescript}
  constructor(eventTarget?: EventTarget) {
    if (eventTarget) {
      eventTarget.addEventListener('message', (ev: MessageEvent) =>
        this._handleMessages(ev)
      );
    } else {
      // if no eventTarget is given we assume we're in a
      // WebWorker and listen on the global scope
      const workerScope = <EventTarget>self;
      workerScope.addEventListener('message', (ev: MessageEvent) =>
        this._handleMessages(ev)
      );
    }
  }
\end{minted}


% The
% \texttt{\_handleMessages}
% function passes on each message to the
% \texttt{\_handleMessage}
% function, which emits it on the approapriate channel:

\begin{minted}{typescript}
  private _handleMessage(data: any): void {
    const channel = data.channel;
    if (this._channels.hasOwnProperty(channel)) {
      const channelInfo = this._channels[channel];
      if (channelInfo.runInZone) {
        this._zone.run(() => {
          channelInfo.emitter.emit(data.message);
        });
      } else {
        channelInfo.emitter.emit(data.message);
      }
    }
  }
\end{minted}


% The
% \texttt{PostMessageBusSink}
% implementation is slightly different because it needs to use
% \texttt{postMessageTarget}
% to post messages. Its constructor creates a field based on the
% supplied
% \texttt{PostMessageTarget}
% parameter:

\begin{minted}{typescript}
export class PostMessageBusSink implements MessageBusSink {
  private _zone: NgZone;
  private _channels: { [key: string]: _Channel } = {};
  private _messageBuffer: Array<Object> = [];

  constructor(private _postMessageTarget: PostMessageTarget) {}
  ..
}
\end{minted}


% The
% \texttt{initChannel}
% method subscribes to the emitter with a next handler that either
% (when runnign inside the Angular zone) adds the message to the
% \texttt{messageBuffer}
% where its sending is deferred, or (if running outside the Angular zone), calls
% \texttt{\_sendMessages}
% , to immediately send the message:

\begin{minted}{typescript}
  initChannel(channel: string, runInZone: boolean = true): void {
    if (this._channels.hasOwnProperty(channel)) {
      throw new Error(`${channel} has already been initialized`);
    }

    const emitter = new EventEmitter(false);
    const channelInfo = new _Channel(emitter, runInZone);
    this._channels[channel] = channelInfo;
    emitter.subscribe((data: Object) => {
      const message = { channel: channel, message: data };
      if (runInZone) {
        this._messageBuffer.push(message);
      } else {
        this._sendMessages([message]);
      }
    });
  }
\end{minted}


% \texttt{\_sendMessages()}
% just sends the message array via
% \texttt{PostMessageTarget}
% :

\begin{minted}{typescript}
  private _sendMessages(messages: Array<Object>) {
    this._postMessageTarget.postMessage(messages);
  }
\end{minted}


% So we saw with
% \texttt{initChannel}
% that the subscription to the emitter either calls
% \texttt{\_sendMessages}
% immediately or parks the message in a message buffer, for later
% transmission. So two questions arise – what triggers that transmission and how does
% it work. Well, to answer the second question first,
% \texttt{\_sendMessages}
% is also called for the
% bulk transmission, from inside the
% \texttt{\_handleOnEventDone}
% message:

\begin{minted}{typescript}
  private _handleOnEventDone() {
    if (this._messageBuffer.length > 0) {
      this._sendMessages(this._messageBuffer);
      this._messageBuffer = [];
    }
  }
\end{minted}


% So, what calls
% \texttt{\_handleOnEventDone}
% ? Let’s digress to look at the
% \texttt{NgZone}
% class in

\begin{itemize}
  \item \href{fix: href loss url}
        {fix: href loss url}
\end{itemize}

% which has this getter:

\begin{minted}{typescript}
  /**
   * Notifies when the last `onMicrotaskEmpty` has run and there are no
   * more microtasks, which implies we are about to relinquish VM turn.
   * This event gets called just once.
   */
  readonly onStable: EventEmitter<any> = new EventEmitter(false);
\end{minted}


% So when the zone has no more work to immediately carry out, it emits a message via
% \texttt{onStable}
% . Back to
% \texttt{PostMessageBusSink}
% – which has this code, that subscribes to the
% \texttt{onStable}
% event emitter:

\begin{minted}{typescript}
  attachToZone(zone: NgZone): void {
    this._zone = zone;
    this._zone.runOutsideAngular(() => {
      this._zone.onStable.subscribe({
        next: () => {
          this._handleOnEventDone();
        },
      });
    });
  }
\end{minted}


% With all the hard work done in
% \texttt{PostMessageBusSource}
% and
% \texttt{PostMessageBusSink}
% , the
% implementation of
% \texttt{PostMessageBus}
% is quite simple:

\begin{minted}{typescript}
/**
 * A TypeScript implementation of {@link MessageBus} for communicating
 * via JavaScript's postMessage API.
 */
@Injectable()
export class PostMessageBus implements MessageBus {
  constructor(
    public sink: PostMessageBusSink,
    public source: PostMessageBusSource
  ) {}

  attachToZone(zone: NgZone): void {
    this.source.attachToZone(zone);
    this.sink.attachToZone(zone);
  }

  initChannel(channel: string, runInZone: boolean = true): void {
    this.source.initChannel(channel, runInZone);
    this.sink.initChannel(channel, runInZone);
  }

  from(channel: string): EventEmitter<any> {
    return;
    this.source.from(channel);
  }

  to(channel: string): EventEmitter<any> {
    return this.sink.to(channel);
  }
}

/**
 * Helper class that wraps a channel's {@link EventEmitter} and
 * keeps track of if it should run in the zone.
 */
class _Channel {
  constructor(public emitter: EventEmitter<any>, public runInZone: boolean) {}
}
\end{minted}


% A different way of looking at the message bus is as a set of independent channels:

% In summary, the message bus in Angular is a simple efficient messaging passing
% mechanism with web workers. It is based on a single connection, with opaque
% messages consisting of a channel name (string) and message data:

\begin{minted}{typescript}
var msg = { channel: channel, message: data };
\end{minted}


% Angular also supplies a richer message broker layered above this simple message bus.
% UI main thread
% Web Worker
% Communicating Over The Message Bus (actual)

% Communicating Over The Message Bus (conceptual)

% The files client\_message\_broker.ts and service\_message\_broker.ts along with a
% number of helper files for serialization implement the message broker.

% The service\_message\_broker.ts file defines the
% \texttt{ReceivedMessage}
% class that
% represents a message:

\begin{minted}{typescript}
export interface ReceivedMessage {
  method: string;
  args: any[];
  id: string;
  type: string;
}
\end{minted}


% The class
% \texttt{ServiceMessageBrokerFactory}
% provide a factory for the service message
% broker:

\begin{minted}{typescript}
@Injectable()
export class ServiceMessageBrokerFactory {
  /** @internal */
  _serializer: Serializer;

  /** @internal */
  constructor(private _messageBus: MessageBus, _serializer: Serializer) {
    this._serializer = _serializer;
  }

  /**
   * Initializes the given channel and attaches a new {@link
ServiceMessageBroker} to it.
   */
  createMessageBroker(
    channel: string,
    runInZone: boolean = true
  ): ServiceMessageBroker {
    this._messageBus.initChannel(channel, runInZone);
    return new ServiceMessageBroker(
      this._messageBus,
      this._serializer,
      channel
    );
  }
}
\end{minted}


% The service message broker is created based on the supplied message bus, serializer
% and channel name. The abstract
% \texttt{ServiceMessageBroker}
% class contains just one
% ClientMessageBroker
% ServiceMessageBroker

% abstract class declaration,
% \texttt{registerMethod}
% :

\begin{minted}{typescript}
/**
 * Helper class for UIComponents that allows components to register methods.
 * If a registered method message is received from the broker on the worker,
 * the UIMessageBroker deserializes its arguments and calls the
 * registered method. If that method returns a promise, the UIMessageBroker
 * returns the result to the worker.
 */
export class ServiceMessageBroker {
  private _sink: EventEmitter<any>;
  private _methods = new Map<string, Function>();

  /** @internal */
  constructor(
    messageBus: MessageBus,
    private _serializer: Serializer,
    private channel: string
  ) {
    this._sink = messageBus.to(channel);
    const source = messageBus.from(channel);
    source.subscribe({ next: (message: any) => this._handleMessage(message) });
  }
  ..
}
\end{minted}


% It has two private fields, an event emitter
% \texttt{\_sink}
% and a map from string to function
% \texttt{\_methods}
% . In its constructor it subscribes its internal method
% \texttt{\_ handleMessage}
% to
% \texttt{messageBus.from}
% (this handles incoming messages), and set
% \texttt{\_sink}
% to
% \texttt{messageBus.to}
% (this will be used to send messages).

% \texttt{\_handleMessage()}
% message creates a
% \texttt{ReceivedMessage}
% based on the map
% parameter, and then if message.method is listed as a supported message in
% \texttt{\_methods}
% , looks up
% \texttt{\_methods}
% for the appropriate function to execute, to handle the
% message:

\begin{minted}{typescript}
  private _handleMessage(message: ReceivedMessage): void {
    if (this._methods.has(message.method)) {
      this._methods.get(message.method)!(message);
    }
  }
\end{minted}


% The next question is how methods get registered in \_methods. That is the job of
% \texttt{registerMethod()}
% , whose implementation adds an entry to the
% \texttt{\_methods}
% map based
% on the supplied parameters. Its key is the
% \texttt{methodName}
% supplied as a parameter and its
% value is an anonymous method, with a single parameter, message, of type
% \texttt{ReceivedMessage}
% :

\begin{minted}{typescript}
  registerMethod(
    methodName: string,
    signature: Array<Type<any> | SerializerTypes> | null,
    method: (..._: any[]) => Promise<any> | void,
    returnType?: Type<any> | SerializerTypes
  ): void {
    this._methods.set(methodName, (message: ReceivedMessage) => {
      const serializedArgs = message.args;
      const numArgs = signature ? signature.length : 0;
      const deserializedArgs = new Array(numArgs);
      for (let i = 0; i < numArgs; i++) {
        const serializedArg = serializedArgs[i];
        %\step{1}% deserializedArgs[i] = this._serializer.deserialize(
          serializedArg,
          signature![i]
        );
      }
      %\step{2}% const promise = method(...deserializedArgs);
      if (returnType && promise) {
        %\step{3}% this._wrapWebWorkerPromise(
          message.id,
          promise,
          returnType
        );
      }
    });
  }
\end{minted}


% We see at its
% 1
% deserialization occuring with the help of the serializer, and at
% 2
% method is called with the deserialized arguments, and at
% 3
% we see if a returnType is
% needed, \_ wrapWebWorkerPromise is called, to handle the then of the promise, which
% emits the result to the sink:

\begin{minted}{typescript}
  private _wrapWebWorkerPromise(
    id: string,
    promise: Promise<any>,
    type: Type<any> | SerializerTypes
  ): void {
    promise.then((result: any) => {
      this._sink.emit({
        type: 'result',
        value: this._serializer.serialize(result, type),
        id: id,
      });
    });
  }
\end{minted}


% The client side of this messaging relationship is handled by client\_message\_broker.ts.
% It defines some simple helper types:

\begin{minted}{typescript}
interface PromiseCompleter {
  resolve: (result: any) => void;
  reject: (err: any) => void;
}
interface RequestMessageData {
  method: string;
  args?: any[];
  id?: string;
}

interface ResponseMessageData {
  type: 'result' | 'error';
  value?: any;
  id?: string;
}
export class FnArg {
  constructor(
    public value: any,
    public type: Type<any> | SerializerTypes = SerializerTypes.PRIMITIVE
  ) {}
}
export class UiArguments {
  constructor(public method: string, public args?: FnArg[]) {}
}
\end{minted}


% To construct a client message borker it defines the
% \texttt{ClientMessageBrokerFactory}
% which, in its
% \texttt{createMessageBroker}
% method, calls
% \texttt{initChannel}
% on the message bus
% and returns a new
% \texttt{ClientMessageBroker}
% :

\begin{minted}{typescript}
@Injectable()
export class ClientMessageBrokerFactory {
  _serializer: Serializer;
  constructor(private _messageBus: MessageBus, _serializer: Serializer) {
    this._serializer = _serializer;
  }

  // Initializes the given channel and attaches a new
  // {@link ClientMessageBroker} to it.
  createMessageBroker(
    channel: string,
    runInZone: boolean = true
  ): ClientMessageBroker {
    this._messageBus.initChannel(channel, runInZone);
    return new ClientMessageBroker(this._messageBus, this._serializer, channel);
  }
}
\end{minted}


% \texttt{ClientMessageBroker}
% has an important method to run on service, which calls a
% method with the supplied
% \texttt{UiArguments}
% on the remote service side and returns a
% promise with the return value (if any).
% \texttt{ClientMessageBroker}
% is implemented as
% follows:

\begin{minted}{typescript}
export class ClientMessageBroker {
  private _pending = new Map<string, PromiseCompleter>();
  private _sink: EventEmitter<any>;
  public _serializer: Serializer;

  constructor(
    messageBus: MessageBus,
    _serializer: Serializer,
    private channel: any
  ) {
    this._sink = messageBus.to(channel);
    this._serializer = _serializer;
    const source = messageBus.from(channel);

    source.subscribe({
      next: (message: ResponseMessageData) => this._handleMessage(message),
    });
  }
  ..
}
\end{minted}


% It has three fields,
% \texttt{\_pending}
% ,
% \texttt{\_sink}
% and
% \texttt{\_serializer}
% .
% \texttt{\_pending}
% is a map from string
% to
% \texttt{PromiseCompleter}
% . It is used to keep track of method calls that require a return
% value and are outstanding – the message has been set to the service, and the result is
% awaited. In its constructor
% \texttt{\_sink}
% is set to the
% \texttt{messageBus.to}
% and serializer set to the
% \texttt{Serializer}
% parameter. Also a source subscription is set for
% \texttt{\_handleMessage}
% .

% Messages for which a return value is expected have a message id generated for them
% via:

\begin{minted}{typescript}
  private _generateMessageId(name: string): string {
    const time: string = stringify(new Date().getTime());
    let iteration: number = 0;
    let id: string = name + time + stringify(iteration);
    while (this._pending.has(id)) {
      id = `${name}${time}${iteration}`;
      iteration++;
    }
    return id;
  }
\end{minted}


% It is this string that is the key into the \_pending map. We will now see how it is set up
% in
% \texttt{runOnService}
% and used in
% \texttt{\_handleMessage}
% .
% \texttt{RunOnService}
% is what the client calls
% when it wants the service to execute a method. It returns a promise, which is a return
% value is required, it is completed when the service returns it.
% Let’s first examine
% \texttt{runOnService}
% when
% \texttt{returnType}
% is null. This creates an array of
% serialized arguments in
% \texttt{fnArgs}
% 1
% , sets up an object literal called message with
% properties “method” and “args”
% 2
% , and then calls
% \texttt{\_sink.emit(message)}
% 3
% :

\begin{minted}{typescript}
  runOnService(
    args: UiArguments,
    returnType: Type<any> | SerializerTypes | null
  ): Promise<any> | null {
    const fnArgs: any[] = [];
    if (args.args) {
      args.args.forEach((argument) => {
        if (argument.type != null) {
          %\step{1}% fnArgs.push(
            this._serializer.serialize(argument.value, argument.type)
          );
        } else {
          fnArgs.push(argument.value);
        }
      });
    }

    let promise: Promise<any> | null;
    let id: string | null = null;

    if (returnType != null) {
      ..
    }
    %\step{2}% const message: RequestMessageData = {
      method: args.method,
      args: fnArgs,
    };
    if (id != null) {
      message['id'] = id;
    }
    %\step{3}% this._sink.emit(message);

    return promise;
  }
\end{minted}


% Things are a little more complex when a return value is required. A promise and a
% promise completer are created
% 1
% ;
% \texttt{\_generateMessageId()}
% is called
% 2
% to generate a
% unique message id for this message; an entry is made into
% \texttt{\_pending}
% 3
% , whose key is
% the id and whose value is the promise completer. The then of the promise
% 4
% returns
% the result
% 5a
% (deserialized if needed
% 5b
% ). Before the message is sent via
% \texttt{sink.emit}
% ,
% the generated message id is attached to the message
% 6
% .

\begin{minted}{typescript}
if (returnType != null) {
  %\step{1}% let completer: PromiseCompleter = undefined!;
  promise = new Promise((resolve, reject) => {
    completer = { resolve, reject };
  });
  %\step{2}% id = this._generateMessageId(args.method);
  %\step{3}% this._pending.set(id, completer);

  promise.catch((err) => {
    if (console && console.error) {
      // tslint:disable-next-line:no-console
      console.error(err);
    }

    completer.reject(err);
  });

  %\step{4}% promise = promise.then(
    %\step{5a}% (v: any) =>
      this._serializer
        ? %\step{5b}% this._serializer.deserialize(v, returnType)
        : v
  );
} else {
  promise = null;
}
const message: RequestMessageData = {
  method: args.method,
  args: fnArgs,
};
if (id != null) {
  %\step{6}% message['id'] = id;
}
this._sink.emit(message);
\end{minted}


% The
% \texttt{\_handleMessage}
% accepts a
% \texttt{ResponseMessageData}
% parameter. It extracts
% 1
% the
% message id, which it uses to look up
% \texttt{\_pending}
% 2
% for the promise. If message data has
% a result field, this is used in a call to the
% \texttt{promise.resolve}
% 3
% , otherwise
% \texttt{promise.reject}
% 4
% is called:

\begin{minted}{typescript}
  private _handleMessage(message: ResponseMessageData): void {
    if (message.type === 'result' || message.type === 'error') {
      %\step{1}% const id = message.id!;
      %\step{2}% if (this._pending.has(id)) {
        if (message.type === 'result') {
          %\step{3}% this._pending.get(id)!.resolve(message.value);
        } else {
          %\step{4}% this._pending.get(id)!.reject(message.value);
        }
        this._pending.delete(id);
      }
    }
  }
\end{minted}


% Two helper files are involved with serialization – render\_store.ts and serializer.ts. The
% \texttt{RenderStore}
% class, define in render\_store.ts, manages two maps – the first,
% \texttt{\_lookupById}
% , from number to any, and the second,
% \texttt{lookupByObject}
% , from any to
% number. It supplies methods to store and remove objects and serialize and de-
% serialize them.

% The
% \texttt{Serializer}
% class is defined in serializer.ts and has methods to serialize and
% deserialize. The
% \texttt{serialize}
% method is:

\input{../output/11_the_platform_webworker_package/code/11_3_1_32.tex}

% Based on the type an appropriate serialization helper method is called.

% The last file in the shared directory is api.ts, which has this one line:

\begin{minted}{typescript}
export const ON_WEB_WORKER = new InjectionToken<boolean>(
  'WebWorker.onWebWorker'
);
\end{minted}


% It is used to store a boolean value with dependency injection, stating whether the
% current code is running in a webworker or not. At this point it would be helpful to
% review how shared functionality is actually configured with dependency injection. On
% the worker side:

\begin{itemize}
  \item \href{https://github.com/angular/angular/blob/master/packages/platform-webworker/src/worker_app.ts}
        {<ANGULAR-MASTER>/packages/platform-webworker/src/worker\_app.ts}
\end{itemize}

% has this:

\begin{minted}{typescript}
// The ng module for the worker app side.
@NgModule({
  providers: [
    ..
    { provide: ON_WEB_WORKER, useValue: true },
    RenderStore,
    ..
  ],
  exports: [CommonModule, ApplicationModule],
})
export class WorkerAppModule {}
\end{minted}


% Note
% \texttt{ON\_WEB\_WORKER}
% is set to true.

% On the ui main thread side:

\begin{itemize}
  \item \href{https://github.com/angular/angular/blob/master/packages/platform-webworker/src/worker_render.ts}
        {<ANGULAR-MASTER>/packages/platform-webworker/src/worker\_render.ts}
\end{itemize}

% has this:

\begin{minted}{typescript}
export const _WORKER_UI_PLATFORM_PROVIDERS: StaticProvider[] = [
  ..
  { provide: Serializer, deps: [RenderStore] },
  { provide: ON_WEB_WORKER, useValue: false },
  { provide: RenderStore, deps: [] },
  ..
];

export const platformWorkerUi = createPlatformFactory(
  platformCore,
  'workerUi',
  _WORKER_UI_PLATFORM_PROVIDERS
);
\end{minted}


% Note
% \texttt{ON\_WEB\_WORKER}
% is set to false.

% Now we will move on to looking at the worker-specific code in:

\begin{itemize}
  \item \href{https://github.com/angular/angular/tree/master/packages/platform-webworker/src/web_workers}
        {<ANGULAR-MASTER>/packages/platform-webworker/src/web\_workers}
\end{itemize}

% In worker\_adapter.ts,
% \texttt{WorkerDomAdapter}
% extends
% \texttt{DomAdapter}
% but only implements
% the logging functionality – for everything else an exception is raised.
% \texttt{WorkerDomAdapter}
% is not how workers render, instead is just logs data to the console.

\begin{minted}{typescript}
/**
 * This adapter is required to log error messages.
 *
 * Note: other methods all throw as the DOM is not accessible
 * directly in web worker context.
 */
export class WorkerDomAdapter extends DomAdapter {
  static makeCurrent() {
    setRootDomAdapter(new WorkerDomAdapter());
  }

  log(error: any) {
    // tslint:disable-next-line:no-console
    console.log(error);
  }

  // everything erlse throws an exception
  insertAllBefore(parent: any, el: any, nodes: any) {
    throw 'not implemented';
  }
  insertAfter(parent: any, el: any, node: any) {
    throw 'not implemented';
  }
  setInnerHTML(el: any, value: any) {
    throw 'not implemented';
  }

  ..
}
\end{minted}


% The renderer.ts file supplies the
% \texttt{WebWorkerRendererFactory2}
% and
% \texttt{WebWorkerRenderer2}
% classes and this is where worker-based rendering is managed.

% \texttt{WebWorkerRenderer2}
% implements the Renderer API and forwards all calls to
% runOnService from the root renderer.
% \texttt{WebWorkerRenderer2}
% is running the webworker
% where there is no DOM, so any rendering needs to be forwarded to the main UI
% thread, and that is what
% \texttt{runOnService}
% is doing here.

% This is a sampling of the calls:

\begin{minted}{typescript}
export class WebWorkerRenderer2 implements Renderer2 {
  data: { [key: string]: any } = Object.create(null);

  constructor(private _rendererFactory: WebWorkerRendererFactory2) {}

  private asFnArg = new FnArg(this, SerializerTypes.RENDER_STORE_OBJECT);

  destroy(): void {
    this.callUIWithRenderer('destroy');
  }

  createElement(name: string, namespace?: string): any {
    const node = this._rendererFactory.allocateNode();
    this.callUIWithRenderer('createElement', [
      new FnArg(name),
      new FnArg(namespace),
      new FnArg(node, SerializerTypes.RENDER_STORE_OBJECT),
    ]);
    return node;
  }

  createComment(value: string): any {
    const node = this._rendererFactory.allocateNode();
    this.callUIWithRenderer('createComment', [
      new FnArg(value),
      new FnArg(node, SerializerTypes.RENDER_STORE_OBJECT),
    ]);
    return node;
  }

  appendChild(parent: any, newChild: any): void {
    this.callUIWithRenderer('appendChild', [
      new FnArg(parent, SerializerTypes.RENDER_STORE_OBJECT),
      new FnArg(newChild, SerializerTypes.RENDER_STORE_OBJECT),
    ]);
  }
  ..
}
\end{minted}


% \texttt{WebWorkerRendererFactory2}
% implements Core’s
% \texttt{RendererFactory2}
% by setting up the
% client message broker factory.

\begin{minted}{typescript}
@Injectable()
export class WebWorkerRendererFactory2 implements RendererFactory2 {
  globalEvents = new NamedEventEmitter();

  private _messageBroker: ClientMessageBroker;

  constructor(
    messageBrokerFactory: ClientMessageBrokerFactory,
    bus: MessageBus,
    private _serializer: Serializer,
    public renderStore: RenderStore
  ) {
    this._messageBroker =
      messageBrokerFactory.createMessageBroker(RENDERER_2_CHANNEL);
    %\step{1}% bus.initChannel(EVENT_2_CHANNEL);
    %\step{2}% const source = bus.from(EVENT_2_CHANNEL);
    %\step{3}% source.subscribe({
      next: (message: any) => this._dispatchEvent(message),
    });
  }
  ..
}
\end{minted}


% An additional and very important role of
% \texttt{WebWorkerRendererFactory2}
% is to configure
% event handling. We see at
% 1
% \texttt{initChannel()}
% being called for the
% \texttt{EVENT\_2\_CHANNEL}
% , at
% 2
% the message source being accessed, and at
% 3
% a subscription being set up with the
% \texttt{\_dispatchEvent()}
% method, which is implemented as:

\begin{minted}{typescript}
  private _dispatchEvent(message: { [key: string]: any }): void {
    const element: WebWorkerRenderNode = this._serializer.deserialize(
      message['element'],
      SerializerTypes.RENDER_STORE_OBJECT
    );

    const eventName = message['eventName'];
    const target = message['eventTarget'];
    const event = message['event'];

    if (target) {
      this.globalEvents.dispatchEvent(
        eventNameWithTarget(target, eventName),
        event
      );
    } else {
      element.events.dispatchEvent(eventName, event);
    }
  }
\end{minted}


% The ui specific code is in:

\begin{itemize}
  \item \href{https://github.com/angular/angular/tree/master/packages/platform-webworker/src/web_workers/ui}
        {<ANGULAR-MASTER>/packages/platform-webworker/src/web\_workers/ui}
\end{itemize}

% The renderer.ts file implements the
% \texttt{MessageBasedRenderer2}
% class. Note this class,
% despite its name, does not implement any types from Core’s rendering API. Instead, it
% accepts a
% \texttt{RendererFactory2}
% instance as a constructor parameter, and forwards all
% rendering requests to it. Its constructor also has a few other parameters which it uses
% as fields and it also defines one extra, an
% \texttt{EventDispatcher}
% .

\begin{minted}{typescript}
@Injectable()
export class MessageBasedRenderer2 {
  private _eventDispatcher: EventDispatcher;

  constructor(
    private _brokerFactory: ServiceMessageBrokerFactory,

    private _bus: MessageBus,
    private _serializer: Serializer,
    private _renderStore: RenderStore,
    private _rendererFactory: RendererFactory2
  ) {}
  ..
}
\end{minted}


% Its
% \texttt{start()}
% method initializes the
% \texttt{EVENT\_2\_CHANNEL}
% , creates a new
% \texttt{EventDispatcher}
% , and then has a very long list of calls to
% \texttt{registerMethod}
% – when
% such messages are received, then the configured local method is called. Here is a
% short selection of the
% \texttt{registerMethod}
% calls:

\begin{minted}{typescript}
  start(): void {
    const broker = this._brokerFactory.createMessageBroker(RENDERER_2_CHANNEL);

    this._bus.initChannel(EVENT_2_CHANNEL);
    this._eventDispatcher = new EventDispatcher(
      this._bus.to(EVENT_2_CHANNEL),
      this._serializer
    );

    const [RSO, P, CRT] = [
      SerializerTypes.RENDER_STORE_OBJECT,
      SerializerTypes.PRIMITIVE,
      SerializerTypes.RENDERER_TYPE_2,
    ];

    const methods: any[][] = [
      ['createRenderer', this.createRenderer, RSO, CRT, P],
      ['createElement', this.createElement, RSO, P, P, P],
      ['createComment', this.createComment, RSO, P, P],
      ['createText', this.createText, RSO, P, P],
      ['appendChild', this.appendChild, RSO, RSO, RSO],
      ['insertBefore', this.insertBefore, RSO, RSO, RSO, RSO],
      ['removeChild', this.removeChild, RSO, RSO, RSO],
      ['selectRootElement', this.selectRootElement, RSO, P, P],
      ['parentNode', this.parentNode, RSO, RSO, P],
      ['nextSibling', this.nextSibling, RSO, RSO, P],
      ['setAttribute', this.setAttribute, RSO, RSO, P, P, P],
      ['removeAttribute', this.removeAttribute, RSO, RSO, P, P],
      ['addClass', this.addClass, RSO, RSO, P],
      ['removeClass', this.removeClass, RSO, RSO, P],
      ['setStyle', this.setStyle, RSO, RSO, P, P, P],
      ['removeStyle', this.removeStyle, RSO, RSO, P, P],
      ['setProperty', this.setProperty, RSO, RSO, P, P],
      ['setValue', this.setValue, RSO, RSO, P],
      ['listen', this.listen, RSO, RSO, P, P, P],
      ['unlisten', this.unlisten, RSO, RSO],
      ['destroy', this.destroy, RSO],
      ['destroyNode', this.destroyNode, RSO, P],
    ];

    methods.forEach(([name, method, ...argTypes]: any[]) => {
      broker.registerMethod(name, argTypes, method.bind(this));
    });
  }
\end{minted}


% The local methods usually call the equivalent method in the configured renderer, and
% often stores the result in the
% \texttt{RenderStore}
% . Here is
% \texttt{\_createElement()}
% :

\begin{minted}{typescript}
  private createElement(
    r: Renderer2,
    name: string,
    namespace: string,
    id: number
  ) {
    this._renderStore.store(r.createElement(name, namespace), id);
  }
\end{minted}


% For event listening, the event dispatcher is used:

\begin{minted}{typescript}
  private listen(
    r: Renderer2,
    el: any,
    elName: string,
    eventName: string,
    unlistenId: number
  ) {
    const listener = (event: any) => {
      return this._eventDispatcher.dispatchRenderEvent(
        el,
        elName,
        eventName,
        event
      );
    };

    const unlisten = r.listen(el || elName, eventName, listener);
    this._renderStore.store(unlisten, unlistenId);
  }
\end{minted}


% The
% \texttt{EventDispatcher}
% class is defined in event\_dispatcher.ts and has a constructor
% and one method,
% \texttt{dispatchRenderEvent()}
% :

\begin{minted}{typescript}
export class EventDispatcher {
  constructor(
    private _sink: EventEmitter<any>,
    private _serializer: Serializer
  ) {}
  dispatchRenderEvent(
    element: any,
    eventTarget: string,
    eventName: string,
    event: any
  ): boolean {
    let serializedEvent: any;

    switch (event.type) {
      ..
      case 'keydown':
      case 'keypress':
      case 'keyup':
        serializedEvent = serializeKeyboardEvent(event);
        break;
      ..
      default:
        throw new Error(eventName + ' not supported on WebWorkers');
    }
    this._sink.emit({
      element: this._serializer.serialize(
        element,
        SerializerTypes.RENDER_STORE_OBJECT
      ),
      eventName: eventName,
      eventTarget: eventTarget,
      event: serializedEvent,
    });
    return false;
  }
  ..
}
\end{minted}


% The switch in the middle is a long list of all the supported events and appropriate calls
% to an event serializer. Above we show what is has for keyboard events.
