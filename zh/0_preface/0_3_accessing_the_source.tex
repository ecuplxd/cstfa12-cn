% \section{Accessing the Source}
\section{获取源码}

% To study the source, you can browse it online, get a copy of the repo via git (usual) or
% download a zip. Some packages may provide extra detail about getting the source –
% for example, for the Angular project, read “Getting the Sources” here:

要研究源,你可以通过 Git(通常)或下载 zip 获取仓库副本然后进行浏览。
有些软件包可能会提供关于获取源码的额外详细信息 ——
例如,对于 Angular 项目,阅读 “Getting the Sources”:

\begin{itemize}
  \item \url{https://github.com/angular/angular/blob/master/docs/DEVELOPER.md}
\end{itemize}

% We first need to decide which branch to use. For master, we use this:

我们首先需要决定要使用哪个分支。对于 master,我们使用这个:

\begin{itemize}
  \item \url{https://github.com/angular/angular/tree/master}
\end{itemize}

% Specifically for the Angular main project, an additional way to access the source is in
% the Angular API Reference (
% \url{https://angular.io/api}
% ), the API page for each Angular
% type has a hyperlink at the top of the page to the relevant source file (this resolves to
% the latest stable version, which may or may not be the same source as master).

专门针对 Angular 主项目,
额外的访问原阿门的方法是 Angular API 参考(\url{https://angular.io/api}),
每个 Angular 类型的 API 页面在顶部有一个超链接关联源文件的页面
(这解析为最新的稳定版本,可能会或可能与主站的源不同)。
