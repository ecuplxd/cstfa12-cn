% \section{Benefits of Understanding The Source}
\section{理解源码的好处}

% There are many good reasons for intermediate- to advanced-developers to become
% familiar with the source trees of the projects that provide the foundation for their daily
% application development.

中高级开发人员有很多充分的理由需要熟悉 Angular 的源码,
因为他们日常的应用开发都需要和 Angular 打交道。

% Enhanced Understanding - As in common with many software packages, descriptions
% of software concepts and API documentation may or may not be present, and if
% present may not be as comprehensive as required, and what is present may or may
% not accurately reflect what the code currently does - the doc for a particular concept
% may well once have been up-to-date, but code changes, the doc not necessarily so,
% and certainly not in lock-step, and this applies to any fast evolving framework.

增强理解 —— 与许多软件包一样,
关于概念设计和 API 的文档可能存在也可能不存在,
如果存在也可能没有要求的那么全面,或多或少不能准确反映当前代码的真实意图 —— 文档过时了,
毕竟代码改动后,并不总是同步地更新文档,这适用于任何快速发展的框架。

% Advanced Debugging – When things go wrong, as they must assuredly will (and
% usually just become an important presentation to potential customers), application
% developers scrambling to fix problems will be much quicker when they have better
% insight via knowing the source.

高级调试 —— 当出现问题时,
因为它们肯定会出现(并且通常只是向潜在客户展示的一个重要展示),
如果深入了解过源码,能更快地解决问题。

% Optimization – for large-scale production applications with high data throughput,
% knowing how your application’s substrate actually works can be really useful in
% deciding how / what to optimize (for example, when application developers really
% understand how Angular’s
% \texttt{NgZone}
% works, how it is optional (from Angular v5 onwards)
% and how it is intertwined with Angular’s change detection, then they can place CPU-
% intensive but non-UI code in a different zone within the same thread, and this can
% result in much better performance).

优化 —— 对于具有高数据吞吐量的大规模生产应用,
了解应用的基板实际如何工作对于决定如何/优化什么非常有用
(例如,当应用开发人员真正了解 Angular 的 \texttt{NgZone} 的工作原理时,
它是如何可选的(从 Angular v5 开始)以及它如何与 Angular 的变更检测交织在一起,
然后他们可以将 CPU 密集型但非 UI 代码放置在同一线程内的不同区域中,
这可以带来更好的性能)。

% Productivity - Familiarity with the source is important for maximum productivity with
% any framework and is one part of a developer accumulating a broader understanding
% of the substrate upon which their applications execute.

生产力 —— 熟悉源码对于任何框架最大化生产力都是重要的,
是开发人员积累更广泛的理解的一部分以及应用如何执行的基础。

% Good practices – Studying large codebases that are well tested and subjected to
% detailed code reviews is a great way for “regular” developers to pick up good coding
% habits and use in their own application source trees.

良好实践 —— 研究经过良好测试和详细 review 的大型代码库
可以让开发人员养成良好编码习惯,并在自己的应用中实际运用。