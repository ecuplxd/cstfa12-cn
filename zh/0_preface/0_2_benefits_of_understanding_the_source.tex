\section{Benefits of Understanding The Source}

% There are many good reasons for intermediate- to advanced-developers to become
% familiar with the source trees of the projects that provide the foundation for their daily
% application development.

% Enhanced Understanding - As in common with many software packages, descriptions
% of software concepts and API documentation may or may not be present, and if
% present may not be as comprehensive as required, and what is present may or may
% not accurately reflect what the code currently does - the doc for a particular concept
% may well once have been up-to-date, but code changes, the doc not necessarily so,
% and certainly not in lock-step, and this applies to any fast evolving framework.

% Advanced Debugging – When things go wrong, as they must assuredly will (and
% usually just become an important presentation to potential customers), application
% developers scrambling to fix problems will be much quicker when they have better
% insight via knowing the source.

% Optimization – for large-scale production applications with high data throughput,
% knowing how your application’s substrate actually works can be really useful in
% deciding how / what to optimize (for example, when application developers really
% understand how Angular’s
% \texttt{NgZone}
% works, how it is optional (from Angular v5 onwards)
% and how it is intertwined with Angular’s change detection, then they can place CPU-
% intensive but non-UI code in a different zone within the same thread, and this can
% result in much better performance).

% Productivity - Familiarity with the source is important for maximum productivity with
% any framework and is one part of a developer accumulating a broader understanding
% of the substrate upon which their applications execute.

% Good practices – Studying large codebases that are well tested and subjected to
% detailed code reviews is a great way for “regular” developers to pick up good coding
% habits and use in their own application source trees.
