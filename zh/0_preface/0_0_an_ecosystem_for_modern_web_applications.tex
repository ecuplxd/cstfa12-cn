% \section{An Ecosystem for Modern Web Applications}
\section{现代 Web 应用生态}

% The Angular ecosystem consists of multiple projects that work together to provide a
% comprehensive foundation for developing modern web applications.  With a few
% exceptions, all of these projects are written with TypeScript (even the TypeScript
% compiler is itself written in TypeScript). The exceptions are the Bazel main project
% (which uses Java), fsevents for native access to macOS FSEvents along with the
% Jasmine unit test framework and Karma test runner (all these use JavaScript). The
% dependency on other projects varies.

Angular 生态由多个项目组成,这些项目协同工作,为开发现代 Web 应用提供全面的基础。
基本上,所有这些项目都是用 TypeScript 编写的(甚至 TypeScript 编译器本身也是用 TypeScript 编写的)。
除了 Bazel 主项目(使用 Java)、用于本地访问 macOS FSEvents 的 fsevents 以及 Jasmine 单元测试框架
和 Karma 测试运行器(使用 JavaScript)。
对其他项目的依赖各不相同。

% Note that the Java-based Bazel main project is used to build Angular itself. Bazel and
% Java are not needed to build regular Angular applications. Bazel is great for
% incremental and multi-core builds so it has advantages when used with large
% codebases (such as Angular itself). In future Bazel is likely to become a sensible build
% option for larger Angular applications.

请注意,基于 Java 的 Bazel 主项目用于构建 Angular 本身。
构建常规 Angular 应用不需要 Bazel 和 Java。
Bazel 非常适合增量和多核构建,因此在与大型代码库(例如 Angular 本身)一起使用时具有优势。
未来,Bazel 很可能成为构建大型 Angular 应用的明智选项。

% None of these projects are dependent on Visual Studio Code. These projects only need
% a code editor that can work with the TypeScript tsc compiler – and there are many -
% see middle of main page at:

上面提到的那些项目都不依赖于 Visual Studio Code。
这些项目只需要一个可以与 TypeScript tsc 编译器一起使用的代码编辑器 —— 更多信息 —— 请参见主页中间部分:

\begin{itemize}
  \item \url{http://typescriptlang.org}
\end{itemize}

% Visual Studio Code is one such editor, that is particularly good, open source, freely
% available for macOS, Linux and Windows, and it too is written in TypeScript.

Visual Studio Code 就是一个非常好、开源、可免费用于 macOS、Linux 和 Windows 的编辑器,
而且它也是用 TypeScript 编写的。

\begin{itemize}
  \item \url{https://code.visualstudio.com}
\end{itemize}

% Of choose you can use any editor to investigate the source trees of these projects.

当然你也可以选择使用任何编辑器来查阅这些项目的源码。

\subsection{Viewing Markdown Files}

% In addition to source in TypeScript, all these projects also contain documentation files
% in markdown format (*.md). Markdown is a domain specific language (DSL) for
% represent HTML documents. It is easier / quicker to manually author compared to
% HTML. When transformed to HTML it provides professional documentation. One
% question you might have is how to view them (as HTML, rather that as .md text). One
% easy way to see them as html is just to view the .md files on Github, e.g.:

\begin{itemize}
  \item \url{https://github.com/angular/angular}
\end{itemize}

% Alternatively, on your own machine, most text editors have either direct markdown
% support or it is available through extensions. When examining a source tree with .md
% files, it is often useful when your code editor can also open markdown files and
% transform them to HTML when requested. StackOverflow covers this issue here:

\begin{itemize}
  \item \url{http://stackoverflow.com/questions/9843609/view-markdown-files-offline}
\end{itemize}

% For example, Visual Studio Code supports Markdown natively and if you open a .md
% file and select CTRL+SHIFT+V, you get to see nice HTML:

\begin{itemize}
  \item \url{https://code.visualstudio.com/docs/languages/markdown}
\end{itemize}

% Finally, if you want to learn markdown, try here:

\begin{itemize}
  \item \url{http://www.markdowntutorial.com}
\end{itemize}

