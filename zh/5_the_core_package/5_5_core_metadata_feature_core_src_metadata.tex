\section{Core/Metadata Feature (core/src/metadata)}

% Think of metadata as little nuggets of information we would like to attach to other
% things. The
% \href{https://github.com/angular/angular/blob/master/packages/core/src/metadata.ts}
% {<ANGULAR-MASTER>/core/src/metadata.ts}
% file exports a variety of types
% from the core/src/metadata sub-directory:

\begin{minted}{typescript}
export {
  ANALYZE_FOR_ENTRY_COMPONENTS,
  Attribute,
  ContentChild,
  ContentChildDecorator,
  ContentChildren,
  ContentChildrenDecorator,
  Query,
  ViewChild,
  ViewChildDecorator,
  ViewChildren,
  ViewChildrenDecorator,
} from './metadata/di';
export {
  Component,
  ComponentDecorator,
  Directive,
  DirectiveDecorator,
  HostBinding,
  HostListener,
  Input,
  Output,
  Pipe,
} from './metadata/directives';
export {
  AfterContentChecked,
  AfterContentInit,
  AfterViewChecked,
  AfterViewInit,
  DoCheck,
  OnChanges,
  OnDestroy,
  OnInit,
} from './metadata/lifecycle_hooks';
export {
  CUSTOM_ELEMENTS_SCHEMA,
  ModuleWithProviders,
  NO_ERRORS_SCHEMA,
  NgModule,
  SchemaMetadata,
} from './metadata/ng_module';
export { ViewEncapsulation } from './metadata/view';
\end{minted}


% The source files in the
% \href{https://github.com/angular/angular/tree/master/packages/core/src/metadata}
% {<ANGULAR-MASTER>/packages/core/src/metadata}
% sub-
% directory are:

\begin{itemize}
  \item di.ts
  \item directives.ts
  \item lifecycle\_hooks.ts
  \item ng\_module.ts
  \item view.ts
\end{itemize}

% The di.ts file defines a range of interfaces for decorators – these include
% \texttt{AttributeDecorator, ContentChildrenDecorator, ContentChildDecorator,}
% \texttt{ViewChildrenDecorator, ViewChildDecorator}
% . It also defines and exports a
% number of consts that are imp[lementations of those interfaces, created using
% \texttt{makePropDecorator}
% .

% The view.ts file defines an enum, a var and a class. The
% \texttt{ViewEncapsulation}
% enum is
% defined as:

\begin{minted}{typescript}
export enum ViewEncapsulation {
  Emulated = 0,
  Native = 1,
  None = 2,
}
\end{minted}


% These represent how template and style encapsulation should work.
% \texttt{None}
% means don’t
% use encapsulation,
% \texttt{Native}
% means use what the renderer offers (specifically the
% Shadow DOM) and
% \texttt{Emulated}
% is best explained by the comment:

\begin{minted}{typescript}
/**
 * Emulate `Native` scoping of styles by adding an attribute containing
 * surrogate id to the Host Element and pre-processing the style rules
 * provided via {@link Component#styles styles} or
 * {@link Component#styleUrls styleUrls}, and adding the new
 * Host Element attribute to all selectors.
 *
 * This is the default option.
 */
\end{minted}


% The directives.ts file exports interfaces related to directive metadata. They include:

\begin{itemize}
  \item DirectiveDecorator
  \item Directive
  \item ComponentDecorator
  \item Component
  \item PipeDecorator
  \item Pipe
  \item InputDecorator
  \item Input
  \item OutputDecorator
  \item Output
  \item HostBindingDecorator
  \item HostBinding
  \item HostListenerDecorator
  \item HostListener
\end{itemize}

% The lifecycle\_hooks.ts file defines a number of interfaces used for lifecycle hooks:

\begin{minted}{typescript}
export interface SimpleChanges {
  [propName: string]: SimpleChange;
}
export interface OnChanges {
  ngOnChanges(changes: SimpleChanges): void;
}
export interface OnInit {
  ngOnInit(): void;
}
export interface DoCheck {
  ngDoCheck(): void;
}
export interface OnDestroy {
  ngOnDestroy(): void;
}
export interface AfterContentInit {
  ngAfterContentInit(): void;
}
export interface AfterContentChecked {
  ngAfterContentChecked(): void;
}
export interface AfterViewInit {
  ngAfterViewInit(): void;
}
export interface AfterViewChecked {
  ngAfterViewChecked(): void;
}
\end{minted}


% These define the method signatures for handlers that application component interest
% in the lifecycle hooks must implement.

% The ng\_module.ts file contains constructs used to define Angular modules. An Angular
% application is built from a set of these and
% \texttt{NgModule}
% is used to tie them together:

\begin{minted}{typescript}
export interface NgModule {
  providers?: Provider[];
  declarations?: Array<Type<any> | any[]>;
  imports?: Array<Type<any> | ModuleWithProviders | any[]>;
  exports?: Array<Type<any> | any[]>;
  entryComponents?: Array<Type<any> | any[]>;
  bootstrap?: Array<Type<any> | any[]>;
  schemas?: Array<SchemaMetadata | any[]>;
  id?: string;
}
\end{minted}


% An
% \texttt{NgModuleDecorator}
% is how
% \texttt{NgModule}
% is attached to a type:

\begin{minted}{typescript}
export interface NgModuleDecorator {
  (obj?: NgModule): TypeDecorator;
  new (obj?: NgModule): NgModule;
}
\end{minted}


% We see their usage when we explore the boilerplate code that Angular CLI generates
% for us when we use it to create a new Angular project. Look inside

\begin{itemize}
  \item <my-project>/src/app/app.module.ts
\end{itemize}

% and we will see:

\begin{minted}{typescript}
@NgModule({
  declarations: [<your-components>],
  imports: [BrowserModule, FormsModule, HttpModule],
  providers: [Title, <your-custom-providers>],
  bootstrap: [AppComponent],
})
export class AppModule {}
\end{minted}


% and when we look at:

\begin{itemize}
  \item <my-project>/src/main.ts
\end{itemize}

% we see:

\begin{minted}{typescript}
import { platformBrowserDynamic } from '@angular/platform-browser-dynamic';
import { AppModule } from './app/app.module';
..
platformBrowserDynamic()
  .bootstrapModule(AppModule)
  .catch((err) => console.log(err));
\end{minted}

