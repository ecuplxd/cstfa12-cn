\subsection{Core/Testability Implementation}

% The implementation of the testability feature is in the testability sub-directory, which
% contains two file:

\begin{itemize}
  \item testability.externs.js
  \item testability.ts
\end{itemize}

% Its provides functionality for testing Angular components, and includes use of
% \texttt{NgZone}
% .

% \texttt{GetTestability}
% is simply defined as:

\begin{minted}{typescript}
/**
 * Adapter interface for retrieving the `Testability` service associated
 *  for a particular context.
 *
 * @experimental Testability apis are primarily intended to be used by
 * e2e test tool vendors like the Protractor team.
 */
export interface GetTestability {
  addToWindow(registry: TestabilityRegistry): void;
  findTestabilityInTree(
    registry: TestabilityRegistry,
    elem: any,
    findInAncestors: boolean
  ): Testability | null;
}
\end{minted}


% \texttt{setTestabilityGetter}
% is used to set the getter:

\begin{minted}{typescript}
/**
 * Set the GetTestability implementation used by
 * the Angular testing framework.
 */
export function setTestabilityGetter(getter: GetTestability): void {
  _testabilityGetter = getter;
}
\end{minted}


% The
% \texttt{TestabilityRegistry}
% class is “A global registry of Testability instances for
% specific elements”.
% \texttt{TestabilityRegistry}
% maintains a map from any element to an
% instance of a testability. It provides a
% \texttt{registerApplication()}
% method which allows
% an entry to be added to this map, and a
% \texttt{getTestability()}
% method that is a lookup:

\begin{minted}{typescript}
@Injectable()
export class TestabilityRegistry {
  _applications = new Map<any, Testability>();
  constructor() {
    _testabilityGetter.addToWindow(this);
  }
  registerApplication(token: any, testability: Testability) {
    this._applications.set(token, testability);
  }
  unregisterApplication(token: any) {
    this._applications.delete(token);
  }
  unregisterAllApplications() {
    this._applications.clear();
  }
  getTestability(elem: any): Testability | null {
    return this._applications.get(elem) || null;
  }
  getAllTestabilities(): Testability[] {
    return Array.from(this._applications.values());
  }
  getAllRootElements(): any[] {
    return Array.from(this._applications.keys());
  }
  findTestabilityInTree(
    elem: Node,
    findInAncestors: boolean = true
  ): Testability | null {
    ..
  }
}
\end{minted}


% The
% \texttt{Testability}
% class is structured as follows:

\begin{minted}{typescript}
/**
 * The Testability service provides testing hooks that can be accessed from
 * the browser and by services such as Protractor. Each bootstrapped Angular
 * application on the page will have an instance of Testability.
 */
@Injectable()
export class Testability implements PublicTestability {
  _pendingCount: number = 0;
  _isZoneStable: boolean = true;
  /**
   * Whether any work was done since the last 'whenStable' callback. This is
   * useful to detect if this could have potentially destabilized another
   * component while it is stabilizing.
   */
  _didWork: boolean = false;
  _callbacks: Function[] = [];
  constructor(private _ngZone: NgZone) {
    this._watchAngularEvents();
  }
  _watchAngularEvents(): void {}
  increasePendingRequestCount(): number {}
  decreasePendingRequestCount(): number {}
  isStable(): boolean {}
  whenStable(callback: Function): void {}
  getPendingRequestCount(): number {
    return this._pendingCount;
  }
  findProviders(using: any, provider: string, exactMatch: boolean): any[] {}
}
\end{minted}

