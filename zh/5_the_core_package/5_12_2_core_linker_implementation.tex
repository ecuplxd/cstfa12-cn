\subsection{Core/Linker Implementation}

% The source files in
% \url{core/src/linker}
% implement the Core/Linker feature:

\begin{itemize}
  \item compiler.ts
  \item component\_factory\_resolver.ts
  \item component\_factory.ts
  \item element\_ref.ts
  \item ng\_module\_factory\_loader.ts
  \item ng\_module\_factory.ts
  \item query\_list.ts
  \item system\_js\_ng\_module\_factory\_loader.ts
  \item template\_ref.ts
  \item view\_container\_ref.ts
  \item view\_ref.ts
\end{itemize}

% A number of these files introduce types who names end in
% \texttt{Ref}
% :

\begin{itemize}
  \item ElementRef
  \item TemplateRef
  \item ViewRef
  \item ViewContainerRef
  \item ComponentRef
\end{itemize}

% If you are used to programming in a language such as C\# where the phrase reference
% type has a specific meaning, note that in Angular this is not a language concept, and
% more a naming convention. These references are merely collections of methods and
% properties used to interact with the referenced construct.

% The elements in the DOM are exposed to Angular apps as
% \texttt{ElementRef}
% s. In general, it
% is recommended not to work directly with
% \texttt{ElementRef}
% s, as a certain level of
% abstraction is very useful to enable Angular apps works on different rendering targets.

% As Angular application developers, the template syntax files we create are processed
% by the Angular template compiler to produce a number of template refs. If a template
% representation has a simple set of elements (
% \texttt{<p>hello world</p>}
% ) then a single
% \texttt{TemplateRef}
% will be created. If directives such as
% \texttt{NgFor}
% are used then the outer
% content will be in one
% \texttt{TemplateRef}
% and what is inside the
% \texttt{NgFor}
% will be in another. A
% \texttt{TemplateRef}
% is instantiated one or more times to make a
% \texttt{ViewRef}
% . There is not
% necessarily a one-to-one mapping between
% \texttt{TemplateRef}
% s and
% \texttt{ViewRef}
% s. Depending
% on the number of iterations to be applied to an
% \texttt{NgFor}
% ,  there will be that number of
% instances of
% \texttt{ViewRef}
% s, all based on the same
% \texttt{TemplateRef}
% .

% A view is a hierarchy and child views can be inserted using
% \texttt{ViewContainerRef}
% (literally a container for a child view). A view is the unit of hierarchy that Angular
% exposes to app developers to dynamically modify what is displayed to users. So think
% in terms of adding and removing embedded views, not adding and removing individual
% HTML elements.

% A
% \texttt{ComponentRef}
% contains a HostView which in turn can contain a number of
% embedded views.

% Now we will move on to looking at the source, starting with looking at
% \url{core/src/linker/compiler.ts}
% .
% \texttt{CompilerOptions}
% provides a list of configuration options
% for a compiler (all are marked as optional):

\begin{minted}{typescript}
export type CompilerOptions = {
  useJit?: boolean;
  defaultEncapsulation?: ViewEncapsulation;
  providers?: StaticProvider[];
  missingTranslation?: MissingTranslationStrategy;
  enableLegacyTemplate?: boolean;
  preserveWhitespaces?: boolean;
};
\end{minted}


% \texttt{COMPILER\_OPTIONS}
% is an exported const:

\begin{minted}{typescript}
// Token to provide CompilerOptions in the platform injector.
export const COMPILER_OPTIONS = new InjectionToken<CompilerOptions[]>(
  'compilerOptions'
);
\end{minted}


% \texttt{CompilerFactory}
% is an abstract class used to construct a compiler, via its
% \texttt{createCompiler()}
% abstract method:

\begin{minted}{typescript}
export abstract class CompilerFactory {
  abstract createCompiler(options?: CompilerOptions[]): Compiler;
}
\end{minted}


% \texttt{ModuleWithComponentFactories}
% combines a
% \texttt{NgModuleFactory}
% and a
% \texttt{ComponentFactory}
% :

\begin{minted}{typescript}
// Combination of NgModuleFactory and ComponentFactory.
export class ModuleWithComponentFactories<T> {
  constructor(
    public ngModuleFactory: NgModuleFactory<T>,
    public componentFactories: ComponentFactory<any>[]
  ) {}
}
\end{minted}


% The
% \texttt{Compiler}
% class is used to perform template compilation:

\begin{minted}{typescript}
@Injectable()
export class Compiler {
  compileModuleSync<T>(moduleType: Type<T>): NgModuleFactory<T> {
    ..
  }
  compileModuleAsync<T>(moduleType: Type<T>): Promise<NgModuleFactory<T>> {
    ..
  }
  compileModuleAndAllComponentsSync<T>(
    moduleType: Type<T>
  ): ModuleWithComponentFactories<T> {
    ..
  }
  compileModuleAndAllComponentsAsync<T>(
    moduleType: Type<T>
  ): Promise<ModuleWithComponentFactories<T>> {
    ..
  }
  clearCache(): void {}
  clearCacheFor(type: Type<any>) {}
}
\end{minted}


% The two methods
% \texttt{clearCache}
% and
% \texttt{clearCacheFor}
% are both empty, the other methods
% just throw an error. So to use a compiler, an actual implementation is needed and
% that is where the separate Compiler package comes in. The
% \texttt{Compiler}
% class defined
% here is the base class for template compilers, and it is these derived classes where the
% actual template compilation occurs.

% The four compile generic methods either synchronously or asynchronously compile an
% individual component, or an entire NgModule. All take a
% \texttt{Type<T>}
% as a parameter. The
% async versions returns the promise of one or more factories, whereas the sync
% versions return the actual factories.

% The query\_list.ts file defines the generic
% \texttt{QueryList<T>}
% class, which provides
% controlled access to a read-only array of items of type T.

% The
% \texttt{ElementRef}
% class is used as a reference to the actual rendered element - what
% that is depends on the renderer.

\begin{minted}{typescript}
export class ElementRef {
  public nativeElement: any;
  constructor(nativeElement: any) {
    this.nativeElement = nativeElement;
  }
}
\end{minted}


% In general, application developers are advised not to work directly with
% \texttt{ElementRef}
% s,
% as it makes their code renderer-specific, and may introduce security issues. We bring
% your attention to theses lines in the source:

\begin{minted}{ts}
* @security Permitting direct access to the DOM can make your application more
* vulnerable to XSS attacks. Carefully review any use of `ElementRef` in your
* code. For more detail, see the [Security Guide](http://g.co/ng/security).
\end{minted}


% The view\_ref.ts file defines the
% \texttt{ViewRef}
% and
% \texttt{EmbeddedViewRef}
% abstract classes which
% are exported by the package’s
% \url{core/src/core.ts}
% file (via
% \url{core/src/linker.ts}
% ), and an
% interface,
% \texttt{InternalViewRef}
% , that is used internally with the core package itself.

% \texttt{ViewRef}
% is defined as:

\begin{minted}{typescript}
export abstract class ViewRef extends ChangeDetectorRef {
  abstract destroy(): void;
  abstract get destroyed(): boolean;
  abstract onDestroy(callback: Function): any;
}
\end{minted}


% \texttt{EmbeddedViewRef}
% is defined as:

\begin{minted}{typescript}
export abstract class EmbeddedViewRef<C> extends ViewRef {
  abstract get context(): C;
  abstract get rootNodes(): any[];
}
\end{minted}


% \texttt{InternalViewRef}
% is defined as:

\begin{minted}{typescript}
export interface InternalViewRef extends ViewRef {
  detachFromAppRef(): void;
  attachToAppRef(appRef: ApplicationRef): void;
}
\end{minted}


% The
% \texttt{attachToAppRef/detachFromAppRef}
% methods are called from
% \texttt{ApplicationRef}
% ’s
% \texttt{attachView}
% and
% \texttt{detachView}
% methods:

\begin{minted}{typescript}
  attachView(viewRef: ViewRef): void {
    const view = viewRef as InternalViewRef;
    this._views.push(view);
    view.attachToAppRef(this);
  }
  detachView(viewRef: ViewRef): void {
    const view = viewRef as InternalViewRef;
    remove(this._views, view);
    view.detachFromAppRef();
  }
\end{minted}


% The view\_container\_ref.ts file defines the abstract
% \texttt{ViewContainerRef}
% class.
% \texttt{ViewContainerRef}
% is a container for views. It defines four getters –
% \texttt{element}
% (for the
% anchor element of the container),
% \texttt{injector}
% ,
% \texttt{parentInjector}
% and
% \texttt{length}
% (number of
% views attached to container). It defines two important create methods –
% \texttt{createEmbeddedView}
% and
% \texttt{createComponent}
% , which create the two variants of views
% supported. Finally, it has a few abstract helper methods –
% \texttt{clear}
% ,
% \texttt{get}
% ,
% \texttt{insert}
% ,
% \texttt{indexOf}
% ,
% \texttt{remove}
% and
% \texttt{detach}
% – which work on the views within the container.

\begin{minted}{typescript}
export abstract class ViewContainerRef {
  abstract get element(): ElementRef;
  abstract get injector(): Injector;
  abstract get parentInjector(): Injector;
  abstract clear(): void;
  abstract get(index: number): ViewRef | null;
  abstract get length(): number;
  abstract createEmbeddedView<C>(
    templateRef: TemplateRef<C>,
    context?: C,
    index?: number
  ): EmbeddedViewRef<C>;
  abstract createComponent<C>(
    componentFactory: ComponentFactory<C>,
    index?: number,
    injector?: Injector,
    projectableNodes?: any[][],
    ngModule?: NgModuleRef<any>
  ): ComponentRef<C>;
  abstract insert(viewRef: ViewRef, index?: number): ViewRef;
  abstract move(viewRef: ViewRef, currentIndex: number): ViewRef;
  abstract indexOf(viewRef: ViewRef): number;
  abstract remove(index?: number): void;
  abstract detach(index?: number): ViewRef | null;
}
\end{minted}


% The difference between
% \texttt{createEmbeddedView}
% and
% \texttt{createComponent}
% is that the former
% takes a template ref as a parameter and creates an embedded view from it, whereas
% the latter takes a component factory as a parameter and uses the host view of the
% newly created component.

% The component\_factory.ts file exports two abstract classes –
% \texttt{ComponentRef}
% and
% \texttt{ComponentFactory}
% :

\begin{minted}{ts}
export abstract class ComponentFactory<C> {}
  abstract get selector(): string;
  abstract get componentType(): Type<any>;
  abstract get ngContentSelectors(): string[];
  abstract get inputs(): {propName: string, templateName: string}[];
  abstract get outputs(): {propName: string, templateName: string}[];
  abstract create(
      injector: Injector, projectableNodes?: any[][], rootSelectorOrNode?:
\end{minted}


% The
% \texttt{ComponentRef}
% abstract class is defined as:

\begin{minted}{typescript}
export abstract class ComponentRef<C> {
  abstract get location(): ElementRef;
  abstract get injector(): Injector;
  abstract get instance(): C;
  abstract get hostView(): ViewRef;
  abstract get changeDetectorRef(): ChangeDetectorRef;
  abstract get componentType(): Type<any>;
  abstract destroy(): void;
  abstract onDestroy(callback: Function): void;
}
\end{minted}


% The template\_ref.ts file defines the
% \texttt{TemplateRef}
% abstract  class and the
% \texttt{TemplateRef\_}
% concrete class.

\begin{minted}{typescript}
export abstract class TemplateRef<C> {
  abstract get elementRef(): ElementRef;
  abstract createEmbeddedView(context: C): EmbeddedViewRef<C>;
}
\end{minted}


% The component\_factory\_resolver.ts file defines the
% \texttt{ComponentFactoryResolver}
% abstract class, which is part of the public API:

% \texttt{ComponentFactoryResolver}
% is defined as:

\begin{minted}{typescript}
export abstract class ComponentFactoryResolver {
  static NULL: ComponentFactoryResolver = new _NullComponentFactoryResolver();
  abstract resolveComponentFactory<T>(component: Type<T>): ComponentFactory<T>;
}
\end{minted}


% this file also defines
% \texttt{CodegenComponentFactoryResolver}
% , a concrete class that is
% exported via:

\begin{itemize}
  \item \href{https://github.com/angular/angular/blob/master/packages/core/src/codegen_private_exports.ts}
        {<ANGULAR-MASTER>/packages/core/src/codegen\_private\_exports.ts}
\end{itemize}

% as it is used elsewhere within the Angular ecosystem.

% \texttt{CodegenComponentFactoryResolver}
% is defined as:

\begin{minted}{typescript}
export class CodegenComponentFactoryResolver
  implements ComponentFactoryResolver
{
  %\step{1}% private _factories = new Map<any, ComponentFactory<any>>();

  constructor(
    %\step{2}% factories: ComponentFactory<any>[],
    private _parent: ComponentFactoryResolver,
    private _ngModule: NgModuleRef<any>
  ) {
    %\step{3}% for (let i = 0; i < factories.length; i++) {
      const factory = factories[i];
      this._factories.set(factory.componentType, factory);
    }
  }

  %\step{4}% resolveComponentFactory<T>(component: {
    new (...args: any[]): T;
  }): ComponentFactory<T> {
    %\step{5}% let factory = this._factories.get(component);
    if (!factory && this._parent) {
      %\step{6}% factory = this._parent.resolveComponentFactory(component);
    }
    if (!factory) {
      throw noComponentFactoryError(component);
    }
    %\step{7}% return new ComponentFactoryBoundToModule(
      factory,
      this._ngModule
    );
  }
}
\end{minted}


% \texttt{CodegenComponentFactoryResolver}
% has a private map field,
% \texttt{\_factories}
% \texttt{1}
% , and its
% constructor takes an array,
% \texttt{factories}
% \texttt{2}
% . Don’t mix them up! The constructor iterates
% 3
% over the array (
% \texttt{factories}
% ) and for each item, adds an entry to the map
% (
% \texttt{\_factories}
% ) that maps the factory’s
% \texttt{componentType}
% to the factory. In its
% \texttt{resolveComponentFactory()}
% 4
% method,
% \texttt{CodegenComponentFactoryResolver}
% looks
% up the map for a matching factory, and if present
% 5
% selects that as the factory and if
% not, calls the
% \texttt{resolveComponentFactory}
% method
% 6
% of the parent, and passes the
% selected factory to a new instance of the
% \texttt{ComponentFactoryBoundToModule}
% helper
% class
% 7
% .

% The ng\_module\_factory\_loader.ts file defines the
% \texttt{NgModuleFactoryLoader}
% abstract
% class as:

\begin{minted}{typescript}
export abstract class NgModuleFactoryLoader {
  abstract load(path: string): Promise<NgModuleFactory<any>>;
}
\end{minted}


% It is use for lazy loading. We will see an implementation in
% system\_js\_ng\_module\_factory\_loader.ts.

% \texttt{SystemJsNgModuleLoader}
% class, which loads NgModule factories using SystemJS.

\begin{minted}{typescript}
// NgModuleFactoryLoader that uses SystemJS to load NgModuleFactory
@Injectable()
export class SystemJsNgModuleLoader implements NgModuleFactoryLoader {
  private _config: SystemJsNgModuleLoaderConfig;
  ..
}
\end{minted}


% Its constructor takes a
% \texttt{\_compiler}
% as an optional parameter.

\begin{minted}{typescript}
  constructor(
    private _compiler: Compiler,
    @Optional() config?: SystemJsNgModuleLoaderConfig
  ) {
    this._config = config || DEFAULT_CONFIG;
  }
\end{minted}


% In its
% \texttt{load()}
% method, if
% \texttt{\_compiler}
% was provided, it calls
% \texttt{loadAndCompile()}
% ,
% otherwise it calls
% \texttt{loadFactory()}
% .

\begin{minted}{typescript}
  load(path: string): Promise<NgModuleFactory<any>> {
    const offlineMode = this._compiler instanceof Compiler;
    return offlineMode ? this.loadFactory(path) : this.loadAndCompile(path);
  }
\end{minted}


% It is within
% \texttt{loadAndCompile()}
% that
% \texttt{\_compiler.compileAppModuleAsync()}
% is called.

\begin{minted}{typescript}
  private loadAndCompile(path: string): Promise<NgModuleFactory<any>> {
    let [module, exportName] = path.split(_SEPARATOR);
    if (exportName === undefined) {
      exportName = 'default';
    }
    return System.import(module)
      .then((module: any) => module[exportName])
      .then((type: any) => checkNotEmpty(type, module, exportName))
      .then((type: any) => this._compiler.compileModuleAsync(type));
  }
\end{minted}


% The ng\_module\_factory.ts file defines the
% \texttt{NgModuleRef}
% and
% \texttt{NgModuleInjector}
% abstract classes and the
% \texttt{AppModuleFactory}
% concrete class.

% \texttt{NgModuleRef}
% is defined as:

\begin{minted}{typescript}
/**
 * Represents an instance of an NgModule created via a NgModuleFactory.
 * `NgModuleRef` provides access to the NgModule Instance as well other
 *  objects related to this NgModule Instance.
 */
export abstract class NgModuleRef<T> {
  abstract get injector(): Injector;
  abstract get componentFactoryResolver(): ComponentFactoryResolver;
  abstract get instance(): T;
  abstract destroy(): void;
  abstract onDestroy(callback: () => void): void;
}
\end{minted}


% \texttt{NgModuleFactory}
% is used to create an NgModuleRef instance:

\begin{minted}{typescript}
export abstract class NgModuleFactory<T> {
  abstract get moduleType(): Type<T>;
  abstract create(parentInjector: Injector | null): NgModuleRef<T>;
}
\end{minted}

