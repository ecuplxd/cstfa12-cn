\section{Core Render Feature (core/src/render)}

% \emph{Note: There is a new view compliation and rendering architecture  (called}
% \emph{Render3, better known by its code name, Ivy) under intensive development}
% \emph{by the Angular team. A good place to keep an eye on progress and how it is}
% \emph{evolving is:}
% \emph{https://github.com/angular/angular/tree/master/packages/core/src/render3}

% \emph{What we describe below is Render2, which is the current stable view}
% \emph{engine. Please see the appendix for detailed coverage of Render3.}

% Layering for Angular applications involves your application code talking to the Angular
% framework, which is layered into an application layer and a renderer layer, with a
% renderer API in between. The core/src/render/api.ts file defines this thin API and
% nothing else. The API consists of these abstract classes -
% \texttt{RendererType2}
% ,
% \texttt{Renderer2,}
% \texttt{RendererFactory2}
% and
% \texttt{RendererStyleFlags2.}
% Implementation of this API is not
% part of Core. Instead, the various Platform-X packages need to provide the actual
% implementations for different scenarios.

% Scenarios with diverse rendering requirements include:

\begin{itemize}
  \item UI web apps in regular browser
  \item web worker apps
  \item server apps
  \item native apps for mobile devices
  \item testing
\end{itemize}

% The Renderer API is defined in terms of elements – and provides functionality e.g. to
% create elements, set their properties and listen for their events. The Renderer API is
% not defined in terms of a DOM. Indeed, the term “DOM” is not part of any of the
% method names in this API (though it is mentioned in some comments). In that way,
% how rendering is provided is an internal implementation detail, easily replaced in
% different scenarios if needed. Obviously, for a web app running in the main UI thread
% in a regular browser, the platform used for that needs to implement the Renderer API
% in terms of the browser’s DOM (and platform-browser does). But take a web worker
% as an alternative, where there simply is no browser DOM – a different platform needs
% to provide an alternative rendering solution. We will be examining in detail how
% rendering implementations work when we cover platforms later.

% A notable characteristic of the Renderer API is that, even though it is defined in terms
% of elements, it does not list anywhere what those elements are. Elements are
% identified in terms of string names, but what are valid names is not part of the
% renderer. Instead, there is an element schema registry defined in the template
% compiler (
% \href{https://github.com/angular/angular/tree/master/packages/compiler/src/schema}
% {<ANGULAR-MASTER>/packages/compiler/src/schema}
% ) and we will examine
% it further when looking at the template compiler.

% Now we will move on to looking at the renderer API. This API is exported from the

\begin{itemize}
  \item \href{https://github.com/angular/angular/blob/master/packages/core/src/render.ts}
        {<ANGULAR-MASTER>/packages/core/src/render.ts}
\end{itemize}

% and it contains this export:

\begin{minted}{typescript}
// Public API for render
export {
  RenderComponentType,
  Renderer,
  Renderer2,
  RendererFactory2,
  RendererStyleFlags2,
  RendererType2,
  RootRenderer,
} from './render/api';
\end{minted}


% We note a number of the exports have “2” in their name and when we examine the
% actual definitions we see those that do not are deprecated so will will not be covering
% those here. If looking at older documentation you may encounter them, but
% specifically for rendering, for up to date code you should be looking at rendering
% classes with 2 in the name. Also note ‘2’ for rendering is not tied to Angular version 2
% (actually that version of Angular used the older version of the rendering APIs).

% The core/src/render directory has just one file:

\begin{itemize}
  \item \href{https://github.com/angular/angular/blob/master/packages/core/src/render/api.ts}
        {<ANGULAR-MASTER>/packages/core/src/render/api.ts}
\end{itemize}

% and this contains the rendering definitions we will see used elsewhere by platform-
% implementations.

% \texttt{RendererType2}
% is used to identify component types for which rendering is needed,
% and is defined as:

\begin{minted}{typescript}
/**
 * Used by `RendererFactory2` to associate custom rendering data and styles
 * with a rendering implementation.
 */
export interface RendererType2 {
  id: string;
  encapsulation: ViewEncapsulation;
  styles: (string | any[])[];
  data: { [kind: string]: any };
}
\end{minted}


% The
% \texttt{RendererFactory2}
% class is used to register a provider with dependency injection
% and is defined as:

\begin{minted}{typescript}
/**
 * Creates and initializes a custom renderer that implements
 * the `Renderer2` base class.
 */
export abstract class RendererFactory2 {
  abstract createRenderer(
    hostElement: any,
    type: RendererType2 | null
  ): Renderer2;
  abstract begin?(): void;
  abstract end?(): void;
  abstract whenRenderingDone?(): Promise<any>;
}
\end{minted}


% Essentially, its main method,
% \texttt{createRenderer()}
% , is used to answer this - “for a given
% host element, please give me back a renderer that I can use”. This is key to wiring up
% flexible rendering of components via dependency injection.

% An example usage is in:

\begin{itemize}
  \item \href{https://github.com/angular/angular/blob/master/packages/platform-browser/src/browser.ts}
        {<ANGULAR-MASTER>/packages/platform-browser/src/browser.ts}
\end{itemize}

% where
% \texttt{browserModule}
% is defined as:

\begin{minted}{typescript}
export const BROWSER_MODULE_PROVIDERS: StaticProvider[] = [
  ..,
  {
    provide: DomRendererFactory2,
    useClass: DomRendererFactory2,
    deps: [EventManager, DomSharedStylesHost, APP_ID],
  },
  { provide: RendererFactory2, useExisting: DomRendererFactory2 },
];
\end{minted}


% Another example usage is in:

\begin{itemize}
  \item \href{https://github.com/angular/angular/blob/master/packages/platform-webworker/src/worker_app.ts}
        {<ANGULAR-MASTER>/packages/platform-webworker/src/worker\_app.ts}
\end{itemize}

% where
% \texttt{WorkerAppModule}
% is defined as:

\begin{minted}{typescript}
@NgModule({
  providers: [
    ..
    WebWorkerRendererFactory2,
    { provide: RendererFactory2, useExisting: WebWorkerRendererFactory2 },
    RenderStore,
    ..
  ],
  ..
})
export class WorkerAppModule {}
\end{minted}


% The result of this is that different root renderers can be supplied via dependency
% injection for differing scenarios, and client code using the renderer API can use a
% suitable implementation. If that is how the
% \texttt{RendererFactory2}
% gets into dependency
% injection system, then of course the next question is, how does it get out?

\begin{itemize}
  \item \href{https://github.com/angular/angular/blob/master/packages/core/src/view/services.ts}
        {<ANGULAR-MASTER>/packages/core/src/view/services.ts}
\end{itemize}

% has this:

\begin{minted}{typescript}
function createProdRootView(
  elInjector: Injector,
  projectableNodes: any[][],
  rootSelectorOrNode: string | any,
  def: ViewDefinition,
  ngModule: NgModuleRef<any>,
  context?: any
): ViewData {
  const rendererFactory: RendererFactory2 =
    ngModule.injector.get(RendererFactory2);
  return createRootView(
    createRootData(
      elInjector,
      ngModule,
      rendererFactory,
      projectableNodes,
      rootSelectorOrNode
    ),
    def,
    context
  );
}
\end{minted}


% which is called from:

\begin{minted}{typescript}
function createProdServices() {
  return {
    setCurrentNode: () => {},
    createRootView: createProdRootView, ...
    ..
  }; ..
}
\end{minted}


% which in turn is called from:

\begin{minted}{typescript}
export function initServicesIfNeeded() {
  if (initialized) {
    return;
  }
  initialized = true;
  const services = isDevMode() ? createDebugServices() : createProdServices();
  ..
}
\end{minted}


% which is finally called from inside:

\begin{itemize}
  \item \href{https://github.com/angular/angular/blob/master/packages/core/src/view/entrypoint.ts}
        {<ANGULAR-MASTER>/packages/core/src/view/entrypoint.ts}
\end{itemize}

% in
% \texttt{NgModuleFactory\_.create()}
% method:

\begin{minted}{typescript}
class NgModuleFactory_ extends NgModuleFactory<any> {
  ..
  create(parentInjector: Injector | null): NgModuleRef<any> {
    initServicesIfNeeded();
    const def = cloneNgModuleDefinition(
      resolveDefinition(this._ngModuleDefFactory)
    );
    return Services.createNgModuleRef(
      this.moduleType,
      parentInjector || Injector.NULL,
      this._bootstrapComponents,
      def
    );
  }
}
\end{minted}


% Now we’ll return to:

\begin{itemize}
  \item \href{https://github.com/angular/angular/blob/master/packages/core/src/render/api.ts}
        {<ANGULAR-MASTER>/packages/core/src/render/api.ts}
\end{itemize}

% and move on to the principal class in the Renderer API,
% \texttt{Renderer2}
% , which is abstract
% and declares the following methods:

% \texttt{Renderer2}
% in full is as follows:

\begin{minted}{typescript}
export abstract class Renderer2 {
  abstract get data(): { [key: string]: any };
  abstract destroy(): void;
  abstract createElement(name: string, namespace?: string | null): any;
  abstract createComment(value: string): any;
  abstract createText(value: string): any;
  destroyNode!: ((node: any) => void) | null;
  abstract appendChild(parent: any, newChild: any): void;
  abstract insertBefore(parent: any, newChild: any, refChild: any): void;
  abstract removeChild(
    parent: any,
    oldChild: any,
    isHostElement?: boolean
  ): void;
  abstract selectRootElement(
    selectorOrNode: string | any,
    preserveContent?: boolean
  ): any;
  abstract parentNode(node: any): any;
  abstract nextSibling(node: any): any;
  abstract setAttribute(
    el: any,
    name: string,
    value: string,
    namespace?: string | null
  ): void;
  abstract removeAttribute(
    el: any,
    name: string,
    namespace?: string | null
  ): void;
  abstract addClass(el: any, name: string): void;
  abstract removeClass(el: any, name: string): void;
  abstract setStyle(
    el: any,
    style: string,
    value: any,
    flags?: RendererStyleFlags2
  ): void;
  abstract removeStyle(
    el: any,
    style: string,
    flags?: RendererStyleFlags2
  ): void;
  abstract setProperty(el: any, name: string, value: any): void;
  abstract setValue(node: any, value: string): void;
  abstract listen(
    target: 'window' | 'document' | 'body' | any,
    eventName: string,
    callback: (event: any) => boolean | void
  ): () => void;
  static __NG_ELEMENT_ID__: () => Renderer2 = () => SWITCH_RENDERER2_FACTORY();
}
\end{minted}


% Here only the interface is being defined – for actual implementation, refer to the
% various platform renderers in the different platform modules. The renderer is a simple
% abstraction, quite suitable for a variety of rendering engine implementations – from
% the Angular team and third parties.

% Finally is
% \texttt{RendererStyleFlags2}
% defined as:

\begin{minted}{typescript}
export enum RendererStyleFlags2 {
  Important = 1 << 0,
  DashCase = 1 << 1,
}
\end{minted}


% It is used to supply a flag parameter to two of
% \texttt{Renderer2}
% ’s methods:

\begin{minted}{typescript}
  abstract setStyle(
    el: any,
    style: string,
    value: any,
    flags?: RendererStyleFlags2
  ): void;
  abstract removeStyle(
    el: any,
    style: string,
    flags?: RendererStyleFlags2
  ): void;
\end{minted}


\subsection{core/src/debug}

% This directory contains this file:

\begin{itemize}
  \item debug\_node.ts
\end{itemize}

% The debug\_node.ts file implements
% \texttt{EventListener}
% ,
% \texttt{DebugNode}
% and
% \texttt{DebugElement}
% classes along with some helper functions.
% \texttt{EventListener}
% stores a name and a
% function, to be called after events are detected:

\begin{minted}{typescript}
export class EventListener {
  constructor(public name: string, public callback: Function) {}
}
\end{minted}


% The
% \texttt{DebugNode}
% class represents a node in a tree:

\begin{minted}{typescript}
export class DebugNode {
  nativeNode: any;
  listeners: EventListener[];
  parent: DebugElement | null;

  constructor(
    nativeNode: any,
    parent: DebugNode | null,
    private _debugContext: DebugContext
  ) {
    this.nativeNode = nativeNode;
    if (parent && parent instanceof DebugElement) {
      parent.addChild(this);
    } else {
      this.parent = null;
    }
    this.listeners = [];
  }
  get injector(): Injector {
    return this._debugContext.injector;
  }
  get componentInstance(): any {
    return this._debugContext.component;
  }
  get context(): any {
    return this._debugContext.context;
  }
  get references(): { [key: string]: any } {
    return this._debugContext.references;
  }
  get providerTokens(): any[] {
    return this._debugContext.providerTokens;
  }
}
\end{minted}


% The debug node at attached as a child to the parent
% \texttt{DebugNode}
% . The
% \texttt{nativeNode}
% to
% which this
% \texttt{DebugNode}
% refers to is recorded. A private field,
% \texttt{\_debugContext}
% records
% supplies additional debugging context. It is of type
% \texttt{DebugContext}
% , defined in
% view/types.ts as:

\begin{minted}{typescript}
export abstract class DebugContext {
  abstract get view(): ViewData;
  abstract get nodeIndex(): number | null;
  abstract get injector(): Injector;
  abstract get component(): any;
  abstract get providerTokens(): any[];
  abstract get references(): { [key: string]: any };
  abstract get context(): any;
  abstract get componentRenderElement(): any;
  abstract get renderNode(): any;
  abstract logError(console: Console, ...values: any[]): void;
}
\end{minted}


% The
% \texttt{DebugElement}
% class extends
% \texttt{DebugNode}
% and supplies a debugging representation
% of an element.

\begin{minted}{typescript}
export class DebugElement extends DebugNode {
  name: string;
  properties: { [key: string]: any };
  attributes: { [key: string]: string | null };
  classes: { [key: string]: boolean };
  styles: { [key: string]: string | null };
  childNodes: DebugNode[];
  nativeElement: any;

  constructor(nativeNode: any, parent: any, _debugContext: DebugContext) {
    super(nativeNode, parent, _debugContext);
    this.properties = {};
    this.attributes = {};
    this.classes = {};
    this.styles = {};
    this.childNodes = [];
    this.nativeElement = nativeNode;
  }
  ..
}
\end{minted}


% It includes these for adding and removing children:

\begin{minted}{typescript}
  addChild(child: DebugNode) {
    if (child) {
      this.childNodes.push(child);
      child.parent = this;
    }
  }
  removeChild(child: DebugNode) {
    const childIndex = this.childNodes.indexOf(child);
    if (childIndex !== -1) {
      child.parent = null;
      this.childNodes.splice(childIndex, 1);
    }
  }
\end{minted}


% It includes this for events:

\begin{minted}{typescript}
  triggerEventHandler(eventName: string, eventObj: any) {
    this.listeners.forEach((listener) => {
      if (listener.name == eventName) {
        listener.callback(eventObj);
      }
    });
  }
\end{minted}


% The functions manage a map:

\begin{minted}{typescript}
// Need to keep the nodes in a global Map so that
// multiple angular apps are supported.
const _nativeNodeToDebugNode = new Map<any, DebugNode>();
\end{minted}


% This is used to add a node:

\begin{minted}{typescript}
export function indexDebugNode(node: DebugNode) {
  _nativeNodeToDebugNode.set(node.nativeNode, node);
}
\end{minted}


