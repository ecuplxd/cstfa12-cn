\subsection{core/src}

The core/src directory directly contains many files, which we will group into three
categories. Firstly, a number of files just export types from equivalently named sub-
directories. Files that fall into this category include:

\begin{itemize}
  \item change\_detection.ts
  \item core.ts
  \item di.ts
  \item metadata.ts
  \item linker.ts
  \item render.ts
  \item zone.ts
\end{itemize}

For example, the renderer.ts file is a one-liner that just exports from renderer/api.ts:

\begin{minted}{typescript}
export {
  RenderComponentType,
  Renderer,
  Renderer2,
  RendererFactory2,
  RendererStyleFlags2,
  RendererType2,
  RootRenderer,
} from './render/api';
\end{minted}


and zone.ts file is a one-liner that just exports from zone/ng\_zone.ts:

\begin{minted}{typescript}
export { NgZone } from './zone/ng_zone';
\end{minted}


Secondly, are files containing what we might call utility functionality:

\begin{itemize}
  \item console.ts
  \item error\_handler.ts
  \item platform\_core\_providers.ts
  \item security.ts
  \item types.ts
  \item util.ts
  \item version.ts
\end{itemize}

console.ts contains an injectable service used to write to the console:

\begin{minted}{typescript}
@Injectable()
export class Console {
  log(message: string): void {
    // tslint:disable-next-line:no-console
    console.log(message);
  }
  // Note: for reporting errors use `DOM.logError()`
  // as it is platform specific
  warn(message: string): void {
    // tslint:disable-next-line:no-console
    console.warn(message);
  }
}
\end{minted}


It is listed as an entry in
\texttt{\_CORE\_PLATFORM\_PROVIDERS}
in platform\_core\_providers.ts,
which is used to create platforms.

error\_handler.ts defines he default error handler and also, in comments, describes
how you could implement your own.

platform\_core\_providers.ts defines
\texttt{\_CORE\_PLATFORM\_PROVIDERS}
which lists the core
providers for dependency injection:

\begin{minted}{typescript}
const _CORE_PLATFORM_PROVIDERS: StaticProvider[] = [
  // Set a default platform name for platforms that don't set it explicitly.
  { provide: PLATFORM_ID, useValue: 'unknown' },
  { provide: PlatformRef, deps: [Injector] },
  { provide: TestabilityRegistry, deps: [] },
  { provide: Console, deps: [] },
];
\end{minted}


It also defines the
\texttt{platformCore}
const, used when creating platforms:

\begin{minted}{typescript}
export const platformCore = createPlatformFactory(
  null,
  'core',
  _CORE_PLATFORM_PROVIDERS
);
\end{minted}


security.ts defines the
\texttt{SecurityContext}
enum:

\begin{minted}{typescript}
export enum SecurityContext {
  NONE = 0,
  HTML = 1,
  STYLE = 2,
  SCRIPT = 3,
  URL = 4,
  RESOURCE_URL = 5,
}
\end{minted}


and the
\texttt{Sanitizer}
abstract class:

\begin{minted}{typescript}
/**
 * Sanitizer is used by the views to sanitize potentially dangerous values.
 *
 * @stable
 */
export abstract class Sanitizer {
  abstract sanitize(
    context: SecurityContext,
    value: {} | string | null
  ): string | null;
}
\end{minted}


The third category of source files directly in the src directory are files related to
platform- and application-initialization. These include:

\begin{itemize}
  \item application\_init.ts
  \item application\_module.ts
  \item application\_tokens.ts
  \item application\_ref.ts
\end{itemize}

The first three of these are small (50-70 lines of code), whereas application\_ref.ts is
larger at over 500 lines.

Let’s start with application\_tokens.ts. It contains one provider definition and a set of
opaque tokens for various uses.
\texttt{PLATFORM\_INITIALIZER}
is an opaque token that
Angular itself and application code can use to register supplied functions that will be
executed when a platform is initialized:

\begin{minted}{typescript}
// A function that will be executed when a platform is initialized.
export const PLATFORM_INITIALIZER = new InjectionToken<Array<() => void>>(
  'Platform Initializer'
);
\end{minted}


An example usage within Angular is in platform-browser:

\begin{itemize}
  \item \href{https://github.com/angular/angular/blob/master/packages/platform-browser/src/browser.ts}
        {<ANGULAR-MASTER>/packages/platform-browser/src/browser.ts}
\end{itemize}

where it is used to have
\texttt{initDomAdapter}
function called upon platform initialization
(note use of
\texttt{multi}
– which means multiple such initializer functions can be
registered):

\begin{minted}{typescript}
export const INTERNAL_BROWSER_PLATFORM_PROVIDERS: StaticProvider[] = [
  { provide: PLATFORM_ID, useValue: PLATFORM_BROWSER_ID },
  { provide: PLATFORM_INITIALIZER, useValue: initDomAdapter, multi: true },
  {
    provide: PlatformLocation,
    useClass: BrowserPlatformLocation,
    deps: [DOCUMENT],
  },
  { provide: DOCUMENT, useFactory: _document, deps: [] },
];
\end{minted}


\texttt{INTERNAL\_BROWSER\_PLATFORM\_PROVIDERS}
is used a few lines later in browser.ts to
create the browser platform (and so is used by most Angular applications):

\begin{minted}{typescript}
export const platformBrowser: (
  extraProviders?: StaticProvider[]
) => PlatformRef = createPlatformFactory(
  platformCore,
  'browser',
  INTERNAL_BROWSER_PLATFORM_PROVIDERS
);
\end{minted}


Another opaque token in application\_tokens.ts is
\texttt{PACKAGE\_ROOT\_URL}
- used to
discover the application’s root directory.

The provider definition identified by
\texttt{APP\_ID}
supplies a function that generates a
unique string that can be used as an application identifier.

\begin{minted}{typescript}
export const APP_ID = new InjectionToken<string>('AppId');
\end{minted}


A default implementation is supplied, that uses
\texttt{Math.random}
and this is then used to
create an
\texttt{APP\_ID\_RANDOM\_PROVIDER}
:

\begin{minted}{typescript}
function _randomChar(): string {
  return String.fromCharCode(97 + Math.floor(Math.random() * 25));
}

export function _appIdRandomProviderFactory() {
  return `${_randomChar()}${_randomChar()}${_randomChar()}`;
}

export const APP_ID_RANDOM_PROVIDER = {
  provide: APP_ID,
  useFactory: _appIdRandomProviderFactory,
  deps: <any[]>[],
};
\end{minted}


application\_init.ts defines one opaque token and one injectable service. The opaque
token is:

\begin{minted}{typescript}
// A function that will be executed when an application is initialized.
export const APP_INITIALIZER = new InjectionToken<Array<() => void>>(
  'Application Initializer'
);
\end{minted}


\texttt{APP\_INITIALIZER}
Its role is similar to
\texttt{PLATFORM\_INITIALIZER}
, except it is called
when an application is initialized. The injectable service is
\texttt{ApplicationInitStatus}
,
which returns the status of executing app initializers:

\begin{minted}{typescript}
// A class that reflects the state of running {@link APP_INITIALIZER}s.
@Injectable()
export class ApplicationInitStatus {
  private resolve: Function;
  private reject: Function;
  private initialized = false;
  public readonly donePromise: Promise<any>;
  public readonly done = false;

  constructor(
    @Inject(APP_INITIALIZER)
    @Optional()
    private appInits: (() => any)[]
  ) {
    this.donePromise = new Promise((res, rej) => {
      this.resolve = res;
      this.reject = rej;
    });
  }
  ..
}
\end{minted}


We will soon see how it is used in application\_ref.ts.

application\_module.ts defines the
\texttt{ApplicationModule}
class:

\begin{minted}{typescript}
/**
 * This module includes the providers of @angular/core that are needed
 * to bootstrap components via `ApplicationRef`.
 */
@NgModule({
  providers: [
    ApplicationRef,
    ApplicationInitStatus,
    Compiler,
    APP_ID_RANDOM_PROVIDER,
    { provide: IterableDiffers, useFactory: _iterableDiffersFactory },
    { provide: KeyValueDiffers, useFactory: _keyValueDiffersFactory },
    {
      provide: LOCALE_ID,
      useFactory: _localeFactory,
      deps: [[new Inject(LOCALE_ID), new Optional(), new SkipSelf()]],
    },
  ],
})
export class ApplicationModule {
  // Inject ApplicationRef to make it eager...
  constructor(appRef: ApplicationRef) {}
}
\end{minted}


Providers are supplied to Angular’s dependency injection system. The
\texttt{IterableDiffers}
and
\texttt{KeyValueDiffers}
provides related to change detection.
\texttt{ViewUtils}
is defined in the src/linker sub-directory and contains utility-style code
related to rendering.

The types in application\_ref.ts plays a pivotal role in how the entire Angular
infrastructure works. Application developers wishing to learn how Angular really works
are strongly encouraged to carefully study the code in application\_ref.ts. Let’s start
our examination by looking at the
\texttt{createPlatformFactory()}
function:

\begin{minted}{typescript}
export function createPlatformFactory(
  %\step{1}% parentPlatformFactory:
    | ((extraProviders?: StaticProvider[]) => PlatformRef)
    | null,
  %\step{2}% name: string,
  %\step{3}% providers: StaticProvider[] = []
): %\step{4}% (extraProviders?: StaticProvider[]) => PlatformRef {
  const marker = new InjectionToken(`Platform: ${name}`);
  %\step{5}% return (extraProviders: StaticProvider[] = []) => {
    let platform = getPlatform();
    if (!platform || platform.injector.get(ALLOW_MULTIPLE_PLATFORMS, false)) {
      if (parentPlatformFactory) {
        parentPlatformFactory(
          providers
            .concat(extraProviders)
            .concat({ provide: marker, useValue: true })
        );
      } else {
        %\step{6}% createPlatform(
          Injector.create(
            providers
              .concat(extraProviders)
              .concat({ provide: marker, useValue: true })
          )
        );
      }
    }
    %\step{7}% return assertPlatform(marker);
  };
}
\end{minted}


It takes three parameters –
1
\texttt{parentPlatformFactory}
,
2
\texttt{name}
and an
3
array of
providers. It returns
4
a factory function, that when called, will return a
\texttt{PlatformRef}
.

This factory function first creates an opaque token
5
to use for DI lookup based on the
supplied name; then it calls
\texttt{getPlatform()}
to see if a platform already exists (only
one is permitted at any one time), and if false is returned, it calls
6
\texttt{createPlatform()}
, passing in the result of a call to
\texttt{ReflectiveInjector}
’s
\texttt{resolveAndCreate}
(supplied with the providers parameter). Then
7
\texttt{assertPlatform}
is called with the marker and the result of that call becomes the result of the factory
function.

\texttt{PlatformRef}
is defined as:

\begin{minted}{typescript}
@Injectable()
export class PlatformRef {
  private _modules: NgModuleRef<any>[] = [];
  private _destroyListeners: Function[] = [];
  private _destroyed: boolean = false;

  /** @internal */
  constructor(private _injector: Injector) {}

  bootstrapModuleFactory<M>(
    moduleFactory: NgModuleFactory<M>,
    options?: BootstrapOptions
  ): Promise<NgModuleRef<M>> {}

  bootstrapModule<M>(
    moduleType: Type<M>,
    compilerOptions:
      | (CompilerOptions & BootstrapOptions)
      | Array<CompilerOptions & BootstrapOptions> = []
  ): Promise<NgModuleRef<M>> {}

  private _moduleDoBootstrap(moduleRef: InternalNgModuleRef<any>): void {}

  onDestroy(callback: () => void): void {}

  get injector(): Injector {
    return this._injector;
  }

  destroy() {}

  get destroyed() {
    return this._destroyed;
  }
}
\end{minted}


A platform represents the hosting environment within which one or more applications
execute. Different platforms are supported (e.g. browser UI, web worker, server and
you can create your own). For a web page, it is how application code interacts with
the page (e.g. sets URL).
\texttt{PlatformRef}
represents the platform, and we see its two
main features are supplying the root injector and module bootstrapping. The other
members are to do with destroying resources when no longer needed.

The supplied implementation of PlatformRef manages the root injector passed in to
the constructor, an array of
\texttt{NgModuleRef}
s and an array of destroy listeners. In the
constructor,  it takes in an injector. Note that calling the platform’s
\texttt{destroy()}
method
will result in all applications that use that platform having their
\texttt{destroy()}
methods
called.

The two bootstrapping methods are
\texttt{bootstrapModule}
and
\texttt{bootstrapModuleFactory}
.
An important decision for any Angular application team is to decide when to use the
runtime compilation and when to use offline compilation. Runtime compilation is
simpler to use and is demonstrated in the
\url{QuickstartonAngular.io}
. Runtime
compilation makes the application bigger (the template compiler needs to run in the
browser) and is slower (template compilation is required before the template can be
used). Applications that use runtime compilation need to call
\texttt{bootstrapModule}
.
Offline compilation involves extra build time configuration and so is a little more
complex to set up, but due to its performance advantages is likely to be used for large
production applications. Applications that use offline compilation need to call
\texttt{bootstrapModuleFactory()}
.

\begin{minted}{typescript}
  bootstrapModule<M>(
    moduleType: Type<M>,
    compilerOptions:
      | (CompilerOptions & BootstrapOptions)
      | Array<CompilerOptions & BootstrapOptions> = []
  ): Promise<NgModuleRef<M>> {
    const compilerFactory: CompilerFactory = this.injector.get(CompilerFactory);
    const options = optionsReducer({}, compilerOptions);
    const compiler = compilerFactory.createCompiler([options]);
    %\step{1}% return compiler
      .compileModuleAsync(moduleType)
      %\step{2}% .then((moduleFactory) =>
        this.bootstrapModuleFactory(moduleFactory, options)
      );
  }
\end{minted}


\texttt{bootstrapModule}
first calls
1
the template compiler and then
2
calls
\texttt{bootstrapModuleFactory}
, so after the extra runtime compilation step, both
bootstrapping approaches follow the same code path.

When examining zones we already looked at the use of zones within
\texttt{bootstrapModuleFactory}
d- other important code there includes the construction of
\texttt{moduleRef}
1
and the call to
2
\texttt{\_moduleDoBootstrap(moduleRef)}
:

\begin{minted}{typescript}
  %\step{1}% const moduleRef = <InternalNgModuleRef<M>>(
    moduleFactory.create(ngZoneInjector)
  );

  return _callAndReportToErrorHandler(exceptionHandler, ngZone!, () => {
    const initStatus: ApplicationInitStatus = moduleRef.injector.get(
      ApplicationInitStatus
    );
    initStatus.runInitializers();
    return initStatus.donePromise.then(() => {
      %\step{2}% this._moduleDoBootstrap(moduleRef);
      return moduleRef;
    });
  });
\end{minted}


It is in
\texttt{\_moduleDoBootstrap}
that we see the actual bootstrapping taking place:

\begin{minted}{typescript}
  private _moduleDoBootstrap(moduleRef: InternalNgModuleRef<any>): void {
    const appRef = moduleRef.injector.get(ApplicationRef) as ApplicationRef;
    if (moduleRef._bootstrapComponents.length > 0) {
      moduleRef._bootstrapComponents.forEach((f) => appRef.bootstrap(f));
    } else if (moduleRef.instance.ngDoBootstrap) {
      moduleRef.instance.ngDoBootstrap(appRef);
    } else {
      throw new Error(
        `The module ${stringify(moduleRef.instance.constructor)} was
 bootstrapped, but it does not declare "@NgModule.bootstrap"
 components nor a "ngDoBootstrap" method. ` + `Please define one of these.`
      );
    }
    this._modules.push(moduleRef);
  }
\end{minted}


In addition to bootstrapping functionality, there are a few simple platform-related
functions in core/src/application\_ref.ts.
\texttt{createPlatform()}
creates a platform ref
instance, or more accurately, as highlighted in the code, asks the injector for a
platform ref and then calls the initializers:

\begin{minted}{ts}
/**
 * Creates a platform.
 * Platforms have to be eagerly created via this function.
 */
export function createPlatform(injector: Injector): PlatformRef {
  if (_platform && !_platform.destroyed &&
      !_platform.injector.get(ALLOW_MULTIPLE_PLATFORMS, false)) {
    throw new Error(
        'There can be only one platform.
Destroy the previous one to create a new one.');
  }
 _platform = injector.get(PlatformRef);
  const inits = injector.get(PLATFORM_INITIALIZER, null);
  if (inits) inits.forEach((init: any) => init());
  return _platform;
}
\end{minted}


Only a single platform may be active at any one time.
\texttt{\_platform}
is defined as:

\begin{minted}{typescript}
let _platform: PlatformRef;
\end{minted}


The
\texttt{getPlatform()}
function is simply defined as:

\begin{minted}{typescript}
export function getPlatform(): PlatformRef | null {
  return _platform && !_platform.destroyed ? _platform : null;
}
\end{minted}


The
\texttt{assertPlatform()}
function ensures two things, and if either false, throws an
error. Firstly it ensures that a platform exists, and secondly that its injector has a
provider for the token specified as a parameter.

\begin{minted}{typescript}
/**
 * Checks that there currently is a platform which contains the
 * given token as a provider.
 */
export function assertPlatform(requiredToken: any): PlatformRef {
  const platform = getPlatform();
  if (!platform) {
    throw new Error('No platform exists!');
  }
  if (!platform.injector.get(requiredToken, null)) {
    throw new Error(
      'A platform with a different configuration has been created. Please destroy it first.'
    );
  }
  return platform;
}
\end{minted}


The
\texttt{destroyPlatform()}
function calls the
\texttt{destroy}
method for the platform:

\begin{minted}{typescript}
export function destroyPlatform(): void {
  if (_platform && !_platform.destroyed) {
    _platform.destroy();
  }
}
\end{minted}


The run mode specifies whether the platform is is production mode or developer
mode. By default, it is in developer mode:

\begin{minted}{typescript}
let _devMode: boolean = true;
let _runModeLocked: boolean = false;
\end{minted}


This can be set by calling
\texttt{enableProdMode()}
:

\begin{minted}{typescript}
export function enableProdMode(): void {
  if (_runModeLocked) {
    throw new Error('Cannot enable prod mode after platform setup.');
  }
  _devMode = false;
}
\end{minted}


To determine which mode is active, call
\texttt{isDevMode()}
. This always returns the same
value. In other words, whatever mode is active when this is first call, that is the mode
that is always active.

\begin{minted}{typescript}
export function isDevMode(): boolean {
  _runModeLocked = true;
  return _devMode;
}
\end{minted}


\texttt{ApplicationRef}
is defined as:

\begin{minted}{typescript}
@Injectable()
export class ApplicationRef {
  public readonly componentTypes: Type<any>[] = [];
  public readonly components: ComponentRef<any>[] = [];
  public readonly isStable: Observable<boolean>;
  bootstrap<C>(
    componentOrFactory: ComponentFactory<C> | Type<C>,
    rootSelectorOrNode?: string | any
  ): {};
  tick(): void {}
  attachView(viewRef: ViewRef): void {}
  detachView(viewRef: ViewRef): void {}
  get viewCount() {
    return this._views.length;
  }
}
\end{minted}


It main method is
\texttt{bootstrap()}
, which is a generic method with a type parameter -
which attaches the component to DOM elements and sets up the application for
execution. Note that bootstrap’s parameter is a union type, it represents either a
\texttt{ComponentFactory}
or a
\texttt{Type}
, both of which take C as a type parameter.

One implementation of
\texttt{ApplicationRef}
is supplied, called
\texttt{ApplicationRef\_}
. This is
marked as
\texttt{Injectable()}
. It maintains the following fields:

\begin{minted}{typescript}
  static _tickScope: WtfScopeFn = wtfCreateScope('ApplicationRef#tick()');
  private _bootstrapListeners: ((compRef: ComponentRef<any>) => void)[] = [];
  private _views: InternalViewRef[] = [];
  private _runningTick: boolean = false;
  private _enforceNoNewChanges: boolean = false;
  private _stable = true;
\end{minted}


Its constructor shows what it needs from an injector and sets up the observables
(code here is abbreviated):

\begin{minted}{ts}
   constructor(
      private _zone: NgZone,
private _console: Console,
private _injector: Injector,
      private _exceptionHandler: ErrorHandler,
      private _componentFactoryResolver: ComponentFactoryResolver,
      private _initStatus: ApplicationInitStatus) {


this._enforceNoNewChanges = isDevMode();

    this._zone.onMicrotaskEmpty.subscribe(
        {next: () => { this._zone.run(() => { this.tick(); }); }});

    const isCurrentlyStable =
new Observable<boolean>((observer: Observer<boolean>) => {
    });

    const isStable = new Observable<boolean>((observer: Observer<boolean>)=>{
      // Create the subscription to onStable outside the Angular Zone so that
      // the callback is run outside the Angular Zone.
      let stableSub: Subscription;
      this._zone.runOutsideAngular(()
      });

      const unstableSub: Subscription = this._zone.onUnstable.subscribe();
    });

      return () => {
        stableSub.unsubscribe();
        unstableSub.unsubscribe();
      };
    });

    (this as{isStable: Observable<boolean>}).isStable =
        merge(isCurrentlyStable, share.call(isStable));
\end{minted}


Its
\texttt{bootstrap()}
implementation passes some code to the run function (to run in the
zone) and this code calls
\texttt{componentFactory.create()}
to create the component and
then
\texttt{\_loadComponent()}
.

\begin{minted}{typescript}
  bootstrap<C>(
    componentOrFactory: ComponentFactory<C> | Type<C>,
    rootSelectorOrNode?: string | any
  ): ComponentRef<C> {
    let componentFactory: ComponentFactory<C>;
    if (componentOrFactory instanceof ComponentFactory) {
      componentFactory = componentOrFactory;
    } else {
      componentFactory =
        this._componentFactoryResolver.resolveComponentFactory(
          componentOrFactory
        )!;
    }
    this.componentTypes.push(componentFactory.componentType);

    // Create a factory associated with the current module if
    // it's not bound to some other
    const ngModule =
      componentFactory instanceof ComponentFactoryBoundToModule
        ? Null
        : this._injector.get(NgModuleRef);
    const selectorOrNode = rootSelectorOrNode || componentFactory.selector;
    const compRef = componentFactory.create(
      Injector.NULL,
      [],
      selectorOrNode,
      ngModule
    );
    ..
    this._loadComponent(compRef);
    ..
    return compRef;
  }
\end{minted}


\texttt{\_loadComponent()}
is defined as:

\begin{minted}{typescript}
  private _loadComponent(componentRef: ComponentRef<any>): void {
    this.attachView(componentRef.hostView);
    this.tick();
    this.components.push(componentRef);
    // Get the listeners lazily to prevent DI cycles.
    const listeners = this._injector
      .get(APP_BOOTSTRAP_LISTENER, [])
      .concat(this._bootstrapListeners);
    listeners.forEach((listener) => listener(componentRef));
  }
\end{minted}


Attached views are those that can be attached to a view container and are subject to
dirty checking. Such views can be attached and detached, and an array of attached
views is recorded.

\begin{minted}{typescript}
  private _views: InternalViewRef[] = [];
  attachView(viewRef: ViewRef): void {
    const view = viewRef as InternalViewRef;
    this._views.push(view);
    view.attachToAppRef(this);
  }
  detachView(viewRef: ViewRef): void {
    const view = viewRef as InternalViewRef;
    remove(this._views, view);
    view.detachFromAppRef();
  }
  get viewCount() {
    return this._views.length;
  }
\end{minted}

