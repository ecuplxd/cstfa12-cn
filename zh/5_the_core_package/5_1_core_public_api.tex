\section{Core Public API}

% The index.ts file in Core’s root directory just exports the contents of the public-api file,
% which in turn exports the contents of src/core.ts:

\begin{itemize}
  \item \href{https://github.com/angular/angular/blob/master/packages/core/src/core.ts}
        {<ANGULAR-MASTER>/packages/core/src/core.ts}
\end{itemize}

% which is where Core’s exported API is defined:

\begin{minted}{typescript}
export * from './metadata';
export * from './version';
export { TypeDecorator } from './util/decorators';
export * from './di';
export {
  createPlatform,
  assertPlatform,
  destroyPlatform,
  getPlatform,
  PlatformRef,
  ApplicationRef,
  createPlatformFactory,
  NgProbeToken,
} from './application_ref';
export { enableProdMode, isDevMode } from './util/is_dev_mode';
export {
  APP_ID,
  PACKAGE_ROOT_URL,
  PLATFORM_INITIALIZER,
  PLATFORM_ID,
  APP_BOOTSTRAP_LISTENER,
} from './application_tokens';
export { APP_INITIALIZER, ApplicationInitStatus } from './application_init';
export * from './zone';
export * from './render';
export * from './linker';
export {
  DebugElement,
  DebugEventListener,
  DebugNode,
  asNativeElements,
  getDebugNode,
  Predicate,
} from './debug/debug_node';
export {
  GetTestability,
  Testability,
  TestabilityRegistry,
  setTestabilityGetter,
} from './testability/testability';
export * from './change_detection';
export * from './platform_core_providers';
export {
  TRANSLATIONS,
  TRANSLATIONS_FORMAT,
  LOCALE_ID,
  MissingTranslationStrategy,
} from './i18n/tokens';
export { ApplicationModule } from './application_module';
export {
  wtfCreateScope,
  wtfLeave,
  wtfStartTimeRange,
  wtfEndTimeRange,
  WtfScopeFn,
} from './profile/profile';
export { AbstractType, Type } from './interface/type';
export { EventEmitter } from './event_emitter';
export { ErrorHandler } from './error_handler';
export * from './core_private_export';
export * from './core_render3_private_export';
export { Sanitizer, SecurityContext } from './sanitization/security';
export * from './codegen_private_exports';
\end{minted}


% If you are writing a normal Angular application, to use the Core package you will
% import Core’s index.ts. There are two additional files with exports, which are used
% internally within Angular, to allow other Angular packages import additional types
% from the Core package (that are not in the normal index.ts). These two files are
% named private\_exports and are located in Core’s src directory (whereas index.ts is
% located in Core’s root directory).

% The first of these is:

\begin{itemize}
  \item \href{https://github.com/angular/angular/blob/master/packages/core/src/core_private_export.ts}
        {<ANGULAR-MASTER>/packages/core/src/core\_private\_export.ts}
\end{itemize}

% and has these exports:

\begin{minted}{typescript}
export { ALLOW_MULTIPLE_PLATFORMS as ɵALLOW_MULTIPLE_PLATFORMS } from './application_ref';
export { APP_ID_RANDOM_PROVIDER as ɵAPP_ID_RANDOM_PROVIDER } from './application_tokens';
export {
  defaultIterableDiffers as ɵdefaultIterableDiffers,
  defaultKeyValueDiffers as ɵdefaultKeyValueDiffers,
} from './change_detection/change_detection';
export { devModeEqual as ɵdevModeEqual } from './change_detection/change_detection_util';
export { isListLikeIterable as ɵisListLikeIterable } from './change_detection/change_detection_util';
export {
  ChangeDetectorStatus as ɵChangeDetectorStatus,
  isDefaultChangeDetectionStrategy as ɵisDefaultChangeDetectionStrategy,
} from './change_detection/constants';
export { Console as ɵConsole } from './console';
export {
  inject,
  setCurrentInjector as ɵsetCurrentInjector,
  ɵɵinject,
} from './di/injector_compatibility';
export {
  getInjectableDef as ɵgetInjectableDef,
  ɵɵInjectableDef,
  ɵɵInjectorDef,
} from './di/interface/defs';
export { APP_ROOT as ɵAPP_ROOT } from './di/scope';
export { ivyEnabled as ɵivyEnabled } from './ivy_switch';
export { ComponentFactory as ɵComponentFactory } from './linker/component_factory';
export { CodegenComponentFactoryResolver as ɵCodegenComponentFactoryResolver } from './linker/component_factory_resolver';
export {
  clearResolutionOfComponentResourcesQueue as ɵclearResolutionOfComponentResourcesQueue,
  resolveComponentResources as ɵresolveComponentResources,
} from './metadata/resource_loading';
export { ReflectionCapabilities as ɵReflectionCapabilities } from './reflection/reflection_capabilities';
export {
  GetterFn as ɵGetterFn,
  MethodFn as ɵMethodFn,
  SetterFn as ɵSetterFn,
} from './reflection/types';
export {
  DirectRenderer as ɵDirectRenderer,
  RenderDebugInfo as ɵRenderDebugInfo,
} from './render/api';
export { _sanitizeHtml as ɵ_sanitizeHtml } from './sanitization/html_sanitizer';
export { _sanitizeStyle as ɵ_sanitizeStyle } from './sanitization/style_sanitizer';
export { _sanitizeUrl as ɵ_sanitizeUrl } from './sanitization/url_sanitizer';
export { global as ɵglobal } from './util/global';

export { looseIdentical as ɵlooseIdentical } from './util/comparison';
export { stringify as ɵstringify } from './util/stringify';
export { makeDecorator as ɵmakeDecorator } from './util/decorators';
export {
  isObservable as ɵisObservable,
  isPromise as ɵisPromise,
} from './util/lang';
export {
  clearOverrides as ɵclearOverrides,
  initServicesIfNeeded as ɵinitServicesIfNeeded,
  overrideComponentView as ɵoverrideComponentView,
  overrideProvider as ɵoverrideProvider,
} from './view/index';
export { NOT_FOUND_CHECK_ONLY_ELEMENT_INJECTOR as ɵNOT_FOUND_CHECK_ONLY_ELEMENT_INJECTOR } from './view/provider';
export {
  getLocalePluralCase as ɵgetLocalePluralCase,
  findLocaleData as ɵfindLocaleData,
} from './i18n/locale_data_api';
export {
  LOCALE_DATA as ɵLOCALE_DATA,
  LocaleDataIndex as ɵLocaleDataIndex,
} from './i18n/locale_data';
\end{minted}


% We observe that the exports are being renamed with the Greek Theta symbol (looks
% like a ‘o’ with a horizontal line through it) – all are exported as:

\begin{minted}{typescript}
export X as ɵX;
\end{minted}


% For an explanation see here -
% \url{https://stackoverflow.com/questions/45466017/%C9%B5-theta-like-symbol-in-angular-2-source-code}

% \emph{“The letter ɵ is used by the Angular team to indicate that some method is}
% \emph{private to the framework and must not be called directly by the user, as the}
% \emph{API for these method is not guaranteed to stay stable between Angular}
% \emph{versions (in fact, I would say it’s almost guaranteed to break).”}

% The second private exports file is:

\begin{itemize}
  \item \href{https://github.com/angular/angular/blob/master/packages/core/src/codegen_private_exports.ts}
        {<ANGULAR-MASTER>/packages/core/src/codegen\_private\_exports.ts}
\end{itemize}

% and has these exports:

\begin{minted}{ts}
export {CodegenComponentFactoryResolver as ɵCodegenComponentFactoryResolver}
  from './linker/component_factory_resolver';
export {registerModuleFactory as ɵregisterModuleFactory}
  from './linker/ng_module_factory_registration';
export {ArgumentType as ɵArgumentType, BindingFlags as ɵBindingFlags,
  DepFlags as ɵDepFlags, EMPTY_ARRAY as ɵEMPTY_ARRAY,
  EMPTY_MAP as ɵEMPTY_MAP, NodeFlags as ɵNodeFlags,
  QueryBindingType as ɵQueryBindingType,
  QueryValueType as ɵQueryValueType,
  ViewDefinition as ɵViewDefinition,
  ViewFlags as ɵViewFlags,
  anchorDef as ɵand
  createComponentFactory as ɵccf,
  createNgModuleFactory as ɵcmf,
  createRendererType2 as ɵcrt,
  directiveDef as ɵdid, elementDef as ɵeld,
  getComponentViewDefinitionFactory as ɵgetComponentViewDefinitionFactory,
  inlineInterpolate as ɵinlineInterpolate,
  interpolate as ɵinterpolate, moduleDef as ɵmod,
  moduleProvideDef as ɵmpd, ngContentDef as ɵncd,
  nodeValue as ɵnov, pipeDef as ɵpid,
  providerDef as ɵprd, pureArrayDef as ɵpad,
  pureObjectDef as ɵpod, purePipeDef as ɵppd,
  queryDef as ɵqud, textDef as ɵted, unwrapValue as ɵunv,
  viewDef as ɵvid} from './view/index';
\end{minted}

