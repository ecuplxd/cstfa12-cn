\section{Core Public API}

% The index.ts file in Core’s root directory just exports the contents of the public-api file,
% which in turn exports the contents of src/core.ts:

\begin{itemize}
  \item \href{https://github.com/angular/angular/blob/master/packages/core/src/core.ts}
        {<ANGULAR-MASTER>/packages/core/src/core.ts}
\end{itemize}

% which is where Core’s exported API is defined:

\input{5_the_core_package/code/5_1_0.tex}

% If you are writing a normal Angular application, to use the Core package you will
% import Core’s index.ts. There are two additional files with exports, which are used
% internally within Angular, to allow other Angular packages import additional types
% from the Core package (that are not in the normal index.ts). These two files are
% named private\_exports and are located in Core’s src directory (whereas index.ts is
% located in Core’s root directory).

% The first of these is:

\begin{itemize}
  \item \href{https://github.com/angular/angular/blob/master/packages/core/src/core_private_export.ts}
        {<ANGULAR-MASTER>/packages/core/src/core\_private\_export.ts}
\end{itemize}

% and has these exports:

\input{5_the_core_package/code/5_1_1.tex}

% We observe that the exports are being renamed with the Greek Theta symbol (looks
% like a ‘o’ with a horizontal line through it) – all are exported as:

\begin{minted}{ts}
export X as ɵX;
\end{minted}


% For an explanation see here -
% \url{https://stackoverflow.com/questions/45466017/%C9%B5-theta-like-symbol-in-angular-2-source-code}

% \emph{“The letter ɵ is used by the Angular team to indicate that some method is}
% \emph{private to the framework and must not be called directly by the user, as the}
% \emph{API for these method is not guaranteed to stay stable between Angular}
% \emph{versions (in fact, I would say it’s almost guaranteed to break).”}

% The second private exports file is:

\begin{itemize}
  \item \href{https://github.com/angular/angular/blob/master/packages/core/src/codegen_private_exports.ts}
        {<ANGULAR-MASTER>/packages/core/src/codegen\_private\_exports.ts}
\end{itemize}

% and has these exports:

\input{5_the_core_package/code/5_1_3.tex}
