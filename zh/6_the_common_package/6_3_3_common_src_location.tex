\subsection{common/src/location}

% This sub-directory contains these source files:

\begin{itemize}
  \item hash\_location\_strategy.ts
  \item location.ts
  \item location\_strategy.ts
  \item path\_location\_strategy.ts
  \item platform\_location.ts
\end{itemize}

% The location\_strategy.ts file defines the
% \texttt{LocationStrategy}
% class and the
% \texttt{APP\_BASE\_HREF}
% opaque token.

\begin{minted}{typescript}
export abstract class LocationStrategy {
  abstract path(includeHash?: boolean): string;
  abstract prepareExternalUrl(internal: string): string;
  abstract pushState(
    state: any,
    title: string,
    url: string,
    queryParams: string
  ): void;
  abstract replaceState(
    state: any,
    title: string,
    url: string,
    queryParams: string
  ): void;
  abstract forward(): void;
  abstract back(): void;
  abstract onPopState(fn: LocationChangeListener): void;
  abstract getBaseHref(): string;
}
\end{minted}


% The opaque token is defined as:

\begin{minted}{typescript}
export const APP_BASE_HREF = new InjectionToken<string>('appBaseHref');
\end{minted}


% The two implementations of
% \texttt{LocationStrategy}
% are provided in
% hash\_location\_strategy.ts and path\_location\_strategy.ts. Both share the same
% constructor signature:

\begin{minted}{typescript}
@Injectable()
export class HashLocationStrategy extends LocationStrategy {
  private _baseHref: string = '';
  constructor(
    private _platformLocation: PlatformLocation,
    @Optional() @Inject(APP_BASE_HREF) _baseHref?: string
  ) {
    super();
    if (_baseHref != null) {
      this._baseHref = _baseHref;
    }
  }
  ..
}
\end{minted}


% The
% \texttt{PlatformLocation}
% parameter is how they access actual location information.

% \texttt{PlatformLocation}
% is an abstract class used to access location (URL) information.
% Note
% \texttt{PlatformLocation}
% does not extend
% \texttt{Location}
% - which is a service class used to
% manage the browser’s URL. They have quite distinct purposes.

\begin{minted}{typescript}
export abstract class PlatformLocation {
  abstract getBaseHrefFromDOM(): string;
  abstract onPopState(fn: LocationChangeListener): void;
  abstract onHashChange(fn: LocationChangeListener): void;

  abstract get pathname(): string;
  abstract get search(): string;
  abstract get hash(): string;

  abstract replaceState(state: any, title: string, url: string): void;

  abstract pushState(state: any, title: string, url: string): void;

  abstract forward(): void;

  abstract back(): void;
}
\end{minted}


% An important part of the various (browser, server) platform representations is to
% provide a custom implementation of
% \texttt{PlatformLocation}
% .

% The final file in common/src/location is location.ts, which is where Location is defined.
% The
% \texttt{Location}
% class is a service (perhaps it would be better to actually call it
% \texttt{LocationService}
% ) used to interact with URLs. A note in the source is important:

\begin{minted}{ts}
* Note: it's better to use {@link Router#navigate} service to trigger route
* changes. Use `Location` only if you need to interact with or create
* normalized URLs outside of routing.
\end{minted}


% The
% \texttt{Location}
% class does not extend any other class, and its constructor only takes a
% \texttt{LocationStrategy}
% as as parameter:

\begin{minted}{typescript}
@Injectable()
export class Location {
  _subject: EventEmitter<any> = new EventEmitter();
  _baseHref: string;
  _platformStrategy: LocationStrategy;

  constructor(platformStrategy: LocationStrategy) {
    this._platformStrategy = platformStrategy;
    const browserBaseHref = this._platformStrategy.getBaseHref();
    this._baseHref = Location.stripTrailingSlash(
      _stripIndexHtml(browserBaseHref)
    );
    this._platformStrategy.onPopState((ev) => {
      this._subject.emit({
        url: this.path(true),
        pop: true,
        type: ev.type,
      });
    });
  }
  ..
}
\end{minted}


% One useful method is
% \texttt{subscribe()}
% , which allows your application code to be informed
% of
% \texttt{popState}
% events:

\begin{minted}{typescript}
  // Subscribe to the platform's `popState` events.
  subscribe(
    onNext: (value: PopStateEvent) => void,
    onThrow?: ((exception: any) => void) | null,

    onReturn?: (() => void) | null
  ): ISubscription {
    return this._subject.subscribe({
      next: onNext,
      error: onThrow,
      complete: onReturn,
    });
  }
\end{minted}

