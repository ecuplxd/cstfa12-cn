\subsection{common/src/pipes}

% This sub-directory contains these source files:

\begin{itemize}
  \item async\_pipe.ts
  \item case\_conversion\_pipes.ts
  \item date\_pipe.ts
  \item i18n\_plural\_pipe.ts
  \item i18n\_select\_pipe.ts
  \item invalid\_pipe\_argument\_error.ts
  \item json\_pipe.ts
  \item number\_pipe.ts
  \item slice\_pipe.ts
\end{itemize}

% All the pipe classes are marked with the
% \texttt{Pipe}
% decorator. All the pipes implement the
% \texttt{PipeTransform}
% interface. As an example, slice\_pipe.ts has the following:

\begin{minted}{typescript}
@Pipe({ name: 'slice', pure: false })
export class SlicePipe implements PipeTransform {
  transform(value: any, start: number, end?: number): any {
    if (value == null) return value;

    if (!this.supports(value)) {
      throw invalidPipeArgumentError(SlicePipe, value);
    }

    return value.slice(start, end);
  }

  private supports(obj: any): boolean {
    return typeof obj === 'string' || Array.isArray(obj);
  }
}
\end{minted}


% \texttt{COMMON\_PIPES}
% (in index.ts) lists the defined pipes and will often be used when
% creating components.

\begin{minted}{typescript}
export const COMMON_PIPES = [
  AsyncPipe,
  UpperCasePipe,
  LowerCasePipe,
  JsonPipe,
  SlicePipe,
  DecimalPipe,
  PercentPipe,
  TitleCasePipe,
  CurrencyPipe,
  DatePipe,
  I18nPluralPipe,
  I18nSelectPipe,
];
\end{minted}


% We saw its use in the
% \texttt{NgModule}
% decorator attached to
% \texttt{CommonModule}
% .

% Finally,
% \texttt{InvalidPipeArgumentError}
% extends
% \texttt{BaseError:}

\begin{minted}{typescript}
export function invalidPipeArgumentError(type: Type<any>, value: Object) {
  return Error(`InvalidPipeArgument: '${value}' for pipe '${stringify(type)}'`);
}
\end{minted}

