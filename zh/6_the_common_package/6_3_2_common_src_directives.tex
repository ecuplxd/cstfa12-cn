\subsection{common/src/directives}

% This directory has the following source files:

\begin{itemize}
  \item index.ts
  \item ng\_class.ts
  \item ng\_component\_outlet.ts
  \item ng\_for\_of.ts
  \item ng\_if.ts
  \item ng\_plural.ts
  \item ng\_style.ts
  \item ng\_switch.ts
  \item ng\_template\_outlet.ts
\end{itemize}

% The indexs.ts file exports the directive types:

\begin{minted}{typescript}
export {
  NgClass,
  NgComponentOutlet,
  NgForOf,
  NgForOfContext,
  NgIf,
  NgIfContext,
  NgPlural,
  NgPluralCase,
  NgStyle,
  NgSwitch,
  NgSwitchCase,
  NgSwitchDefault,
  NgTemplateOutlet,
};
\end{minted}


% along with a definition for
% \texttt{COMMON\_DIRECTIVES}
% , which is:

\begin{minted}{typescript}
/**
 * A collection of Angular directives that are likely to be used
 * in each and every Angular application.
 */
export const COMMON_DIRECTIVES: Provider[] = [
  NgClass,
  NgComponentOutlet,
  NgForOf,
  NgIf,
  NgTemplateOutlet,
  NgStyle,
  NgSwitch,
  NgSwitchCase,
  NgSwitchDefault,
  NgPlural,
  NgPluralCase,
];
\end{minted}


% The various ng\_ files implement the directives. Lets take a peek at one example,
% ng\_if.ts. It uses a view container to create an embedded view based on a template
% ref, if the supplied condition is true. We first see in its constructor it records the view
% container and template ref passed in as parameter:

\begin{minted}{typescript}
@Directive({ selector: '[ngIf]' })
export class NgIf {
  private _context: NgIfContext = new NgIfContext();
  private _thenTemplateRef: TemplateRef<NgIfContext> | null = null;
  private _elseTemplateRef: TemplateRef<NgIfContext> | null = null;
  private _thenViewRef: EmbeddedViewRef<NgIfContext> | null = null;
  private _elseViewRef: EmbeddedViewRef<NgIfContext> | null = null;

  constructor(
    private _viewContainer: ViewContainerRef,
    templateRef: TemplateRef<NgIfContext>
  ) {
    this._thenTemplateRef = templateRef;
  }
  ..
}
\end{minted}


% We observe that the
% \texttt{NgIf}
% class does not derive from any other class. It is made into a
% directive by using the
% \texttt{Directive}
% decorator.

% Then we see it has some setters defined as input properties, for the variations of if:

\begin{minted}{typescript}
  @Input()
  set ngIf(condition: any) {
    this._context.$implicit = this._context.ngIf = condition;
    this._updateView();
  }

  @Input()
  set ngIfThen(templateRef: TemplateRef<NgIfContext>) {
    this._thenTemplateRef = templateRef;
    this._thenViewRef = null; // clear previous view if any.
    this._updateView();
  }

  @Input()
  set ngIfElse(templateRef: TemplateRef<NgIfContext>) {
    this._elseTemplateRef = templateRef;
    this._elseViewRef = null; // clear previous view if any.
    this._updateView();
  }
\end{minted}


% Finally it has an internal method,
% \texttt{\_updateView}
% , where the view is changed as needed:

\begin{minted}{typescript}
  private _updateView() {
    if (this._context.$implicit) {
      if (!this._thenViewRef) {
        this._viewContainer.clear();
        this._elseViewRef = null;
        if (this._thenTemplateRef) {
          this._thenViewRef =
            %\step{1}% this._viewContainer.createEmbeddedView(
              this._thenTemplateRef,
              this._context
            );
        }
      }
    } else {
      if (!this._elseViewRef) {
        this._viewContainer.clear();
        this._thenViewRef = null;
        if (this._elseTemplateRef) {
          this._elseViewRef =
            %\step{2}% this._viewContainer.createEmbeddedView(
              this._elseTemplateRef,
              this._context
            );
        }
      }
    }
  }
\end{minted}


% The important calls here are (
% 1
% \&
% 2
% ) to
% \texttt{this.\_viewContainer.createEmbeddedView}
% ,
% where the embedded view is created if the
% \texttt{NgIf}
% condition is true.

% If
% \texttt{NgIf}
% creates an embedded view zero or once, then we expect
% \texttt{NgFor}
% to create
% embedded view zero or more times, depends on the count supplied to
% \texttt{NgFor}
% . We see
% this is exactly the case, when we look at
% \url{ng_for_of.ts}
% , which implements the
% \texttt{NgFor}
% class (and a helper class -
% \texttt{NgForOfContext}
% ). The helper class is implemented as:

\begin{minted}{typescript}
export class NgForOfContext<T> {
  constructor(
    public $implicit: T,
    public ngForOf: NgIterable<T>,
    public index: number,
    public count: number
  ) {}

  get first(): boolean {
    return this.index === 0;
  }

  get last(): boolean {
    return this.index === this.count - 1;
  }

  get even(): boolean {
    // TODO: fix %
    return this.index % %\step{2}% === 0;
  }

  get odd(): boolean {
    return !this.even;
  }
}
\end{minted}


% \texttt{NgFor}
% is defined as:

\begin{minted}{typescript}
@Directive({ selector: '[ngFor][ngForOf]' })
export class NgForOf<T> implements DoCheck, OnChanges {
  private _differ: IterableDiffer<T> | null = null;
  private _trackByFn: TrackByFunction<T>;

  constructor(
    private _viewContainer: ViewContainerRef,
    private _template: TemplateRef<NgForOfContext<T>>,
    private _differs: IterableDiffers
  ) {}
}
\end{minted}


% The first thing to note about
% \texttt{NgFor}
% ’s implementation is the class implements
% \texttt{DoCheck}
% and
% \texttt{OnChanges}
% lifecycle. The
% \texttt{DoCheck}
% class is a lifecycle hook defined in
% @angular/core/src/metadata/lifecycle\_hooks.ts as:

\begin{minted}{typescript}
export interface DoCheck {
  ngDoCheck(): void;
}
\end{minted}


% \texttt{OnChanges}
% is defined in the same file as:

\begin{minted}{typescript}
export interface OnChanges {
  ngOnChanges(changes: SimpleChanges): void;
}
\end{minted}


% Hence we would expect
% \texttt{NgFor}
% to provide
% \texttt{ngDoCheck}
% and
% \texttt{ngOnChanges}
% methods and it
% does.
% \texttt{ngDoCheck()}
% calls
% \texttt{\_applyChanges}
% , where for each change operation a call to
% \texttt{viewContainer.createEmbeddedView()}
% is made.
