\subsection{Platform-browser/src/browser}

% This directory contains these files:

\begin{itemize}
  \item browser\_adapter.ts
  \item generic\_browser\_adapter.ts
  \item meta.ts
  \item server-transition.ts
  \item testability.ts
  \item title.ts
  \item transfer\_state.ts
  \item location/browser\_platform\_location.ts
  \item location/history.ts
  \item tools/browser.ts
  \item tools/common\_tools.ts
  \item tools/tools.ts
\end{itemize}

% generic\_browser\_adapter.ts defines the abstract
% \texttt{GenericBrowserDomAdapter}
% class
% that extends
% \texttt{DomAdapter}
% with DOM operations suitable for general browsers:

\begin{minted}{typescript}
export abstract class GenericBrowserDomAdapter extends DomAdapter {
  ..
}
\end{minted}


% For example, to check if shadow DOM is supported, it evaluates whether the
% \texttt{createShadowRoot}
% function exists:

\begin{minted}{typescript}
  supportsNativeShadowDOM(): boolean {
    return typeof (<any>document.body).createShadowRoot === 'function';
  }
\end{minted}


% \texttt{GenericBrowserDomAdapter}
% does not provide the full implementation of
% \texttt{DomAdapter}
% and so a derived class is needed also.

% browser\_adapter.ts supplies the concrete
% \texttt{BrowserDomAdapter}
% class, which does come
% with a full implementation of
% \texttt{DomAdapter}
% :

\begin{minted}{typescript}
/**
 * A `DomAdapter` powered by full browser DOM APIs.
 * @security Tread carefully! Interacting with the DOM directly
 * is dangerous and can introduce XSS risks.
 */
export class BrowserDomAdapter extends GenericBrowserDomAdapter {
  ..
}
\end{minted}


% Generally its methods provide implementation based on window or document – here
% are some samples:

\begin{minted}{typescript}
  getUserAgent(): string {
    return window.navigator.userAgent;
  }
  getHistory(): History {
    return window.history;
  }
  getTitle(doc: Document): string {
    return doc.title;
  }
  setTitle(doc: Document, newTitle: string) {
    doc.title = newTitle || '';
  }
\end{minted}


% title.ts supplies the
% \texttt{Title}
% service:

\input{../output/8_the_platform_browser_package/code/8_3_2_4.tex}

% The tools sub-directory contains tools.ts and common\_tools.ts. The
% \texttt{AngularProfiler}
% class is defined in common\_tools.ts:

\begin{minted}{typescript}
export class AngularProfiler {
  appRef: ApplicationRef;

  constructor(ref: ComponentRef<any>) {
    this.appRef = ref.injector.get(ApplicationRef);
  }

  timeChangeDetection(config: any): ChangeDetectionPerfRecord {
    ..
    return new ChangeDetectionPerfRecord(msPerTick, numTicks);
  }
}
\end{minted}


% Tools.ts defines two functions
% \texttt{enableDebugTools}
% () and
% \texttt{disableDebugTools}
% () to add
% 1
% and remove
% 2
% a ng property to global:

\begin{minted}{typescript}
const PROFILER_GLOBAL_NAME = 'profiler';

/**
 * Enabled Angular debug tools that are accessible via your browser's
 * developer console.
 *
 * Usage:
 *
 * 1. Open developer console (e.g. in Chrome Ctrl + Shift + j)
 * 1. Type `ng.` (usually the console will show auto-complete suggestion)
 * 1. Try the change detection profiler `ng.profiler.timeChangeDetection()`
 *    then hit Enter.
 *
 * @experimental All debugging apis are currently experimental.
 */
export function enableDebugTools<T>(ref: ComponentRef<T>): ComponentRef<T> {
  %\step{1}% exportNgVar(PROFILER_GLOBAL_NAME, new AngularProfiler(ref));
  return ref;
}

/**
 * Disables Angular tools.
 *
 * @experimental All debugging apis are currently experimental.
 */
export function disableDebugTools(): void {
  %\step{2}% exportNgVar(PROFILER_GLOBAL_NAME, null);
}
\end{minted}


% The src/browser/location sub-directory contains browser\_platform\_location.ts and
% history.ts. browser\_platform\_location.ts contains the injectable class
% \texttt{BrowserPlatformLocation}
% :

\begin{minted}{typescript}
/**
 * `PlatformLocation` encapsulates all of the direct calls to platform APIs.
 * This class should not be used directly by an application developer.
Instead, use
 * {@link Location}.
 */
@Injectable()
export class BrowserPlatformLocation extends PlatformLocation {
  public readonly location: Location;
  private _history: History;

  constructor(@Inject(DOCUMENT) private _doc: any) {
    super();
    this._init();
  }
  ..
}
\end{minted}


% This manages two private fields, location and history:

\begin{minted}{typescript}
  public readonly location!: Location;
  private _history!: History;
\end{minted}


% If you are wondering what the !: means, this StackOverflow answer:

\begin{itemize}
  \item \url{https://stackoverflow.com/questions/50983838/what-does-mean-in-typescript/50984662}
\end{itemize}

% is what you are looking for:

% \emph{“Specifically, the operation x! produces a value of the type of x with null}
% \emph{and undefined excluded.”}

% They are initialized with
% \texttt{getDOM()}
% :

\begin{minted}{typescript}
  constructor(@Inject(DOCUMENT) private _doc: any) {
    super();
    this._init();
  }
  // This is moved to its own method so that
  // `MockPlatformLocationStrategy` can overwrite it
  _init() {
    (this as { location: Location }).location = getDOM().getLocation();
    this._history = getDOM().getHistory();
  }
\end{minted}


% history.ts contains this simple function:

\begin{minted}{typescript}
export function supportsState(): boolean {
  return !!window.history.pushState;
}
\end{minted}

