\subsection{Platform-browser/src/security}

% This directory contains this file:

\begin{itemize}
  \item dom\_sanitization\_service.ts
\end{itemize}

% Security sanitizers help prevent the use of dangerous constructs in HTML, CSS styles
% and URLs. Sanitizers are configured in the BrowserModule class via an
% \texttt{NgModule}
% setting. This is the relevant extract from platform-browser/src/browser.ts:

\begin{minted}{typescript}
export const BROWSER_SANITIZATION_PROVIDERS: StaticProvider[] = [
  { provide: Sanitizer, useExisting: DomSanitizer },
  { provide: DomSanitizer, useClass: DomSanitizerImpl, deps: [DOCUMENT] },
];
\end{minted}


% The dom\_sanitization\_service.ts file declares a range of
% \texttt{safeXYZ}
% interfaces,
% implementation classes for them, the
% \texttt{DomSanitizer}
% class and the
% \texttt{DomSanitizerImpl}
% class. It starts by importing the
% \texttt{SecurityContext}
% enum and
% \texttt{Sanitizer}
% abstract class
% from Core:

\begin{itemize}
  \item \href{https://github.com/angular/angular/blob/master/packages/core/src/sanitization/security.ts}
        {<ANGULAR\_MASTER>/packages/core/src/sanitization/security.ts}
\end{itemize}

% Let’s recall they are defined as:

\begin{minted}{typescript}
export enum SecurityContext {
  NONE = 0,
  HTML = 1,
  STYLE = 2,
  SCRIPT = 3,
  URL = 4,
  RESOURCE_URL = 5,
}
export abstract class Sanitizer {
  abstract sanitize(
    context: SecurityContext,
    value: {} | string | null
  ): string | null;
}
\end{minted}


% In dom\_sanitization\_service.ts the safe marker interfaces are declared as:

\begin{minted}{typescript}
// Marker interface for a value that's safe to use in a particular context.
export interface SafeValue {}

// Marker interface for a value that's safe to use as HTML.
export interface SafeHtml extends SafeValue {}

// Marker interface for a value that's safe to use as style (CSS).
export interface SafeStyle extends SafeValue {}

// Marker interface for a value that's safe to use as JavaScript.
export interface SafeScript extends SafeValue {}

// Marker interface for a value that's safe to use as a URL linking
// to a document.
export interface SafeUrl extends SafeValue {}

// Marker interface for a value that's safe to use as a URL to load
// executable code from.
export interface SafeResourceUrl extends SafeValue {}
\end{minted}


% The
% \texttt{DomSanitizer}
% abstract class implements Core’s
% \texttt{Sanitizer}
% class. Do read the
% large comment at the beginning – you really do not want to be bypassing security if at
% all possible.

\begin{minted}{typescript}
/**
 * DomSanitizer helps preventing Cross Site Scripting Security bugs (XSS) by
 * sanitizing values to be safe to use in the different DOM contexts.
 *
 * For example, when binding a URL in an `<a [href]="someValue">` hyperlink,
 * `someValue` will be sanitized so that an attacker cannot inject e.g. a
 * `javascript:` URL that would execute code on the website.
 *
 * In specific situations, it might be necessary to disable sanitization, for
 * example if the application genuinely needs to produce a `javascript:`
 * style link with a dynamic value in it. Users can bypass security by
 * constructing a value with one of the `bypassSecurityTrust...` methods, and
 * then binding to that value from the template.
 *
 * These situations should be very rare, and extraordinary care must be taken
 * to avoid creating a Cross Site Scripting (XSS) security bug!
 *
 * When using `bypassSecurityTrust...`, make sure to call the method as early
 * as possible and as close as possible to the source of the value, to make
 * it easy to verify no security bug is created by its use.
 *
 * It is not required (and not recommended) to bypass security if the value
 * is safe, e.g. a URL that does not start with a suspicious protocol, or an
 * HTML snippet that does not contain dangerous code. The sanitizer leaves
 * safe values intact.
 *
 * @security Calling any of the `bypassSecurityTrust...` APIs disables
 * Angular's built-in sanitization for the value passed in. Carefully check
 * and audit all values and code paths going into this call. Make sure any
 * user data is appropriately escaped for this security context.
 * For more detail, see the [Security Guide](http://g.co/ng/security).
 */
export abstract class DomSanitizer implements Sanitizer {
  abstract sanitize(
    context: SecurityContext,
    value: SafeValue | string | null
  ): string | null;
  abstract bypassSecurityTrustHtml(value: string): SafeHtml;
  abstract bypassSecurityTrustStyle(value: string): SafeStyle;
  abstract bypassSecurityTrustScript(value: string): SafeScript;
  abstract bypassSecurityTrustUrl(value: string): SafeUrl;
  abstract bypassSecurityTrustResourceUrl(value: string): SafeResourceUrl;
}
\end{minted}


% The
% \texttt{DomSanitizerImpl}
% injectable class is what is supplied to NgModule:

\begin{minted}{typescript}
@Injectable()
export class DomSanitizerImpl extends DomSanitizer {
  constructor(@Inject(DOCUMENT) private _doc: any) {
    super();
  }
  ..
}
\end{minted}


% It can be divided into three sections, the
% \texttt{checkNotSafeValue}
% method, the sanitize
% method and the
% \texttt{bypassSecurityTrustXYZ}
% methods. The
% \texttt{checkNotSafeValue}
% method
% throws an error is the value parameter is an instance of
% \texttt{SafeValueImpl}
% :

\begin{minted}{typescript}
  private checkNotSafeValue(value: any, expectedType: string) {
    if (value instanceof SafeValueImpl) {
      throw new Error(
        `Required a safe ${expectedType}, got a ${value.getTypeName()} ` +
          `(see http://g.co/ng/security#xss)`
      );
    }
  }
\end{minted}


% The
% \texttt{sanitize}
% method switches on the
% \texttt{securityContext}
% enum parameter, if it is
% \texttt{NONE}
% , then value is simply returned, otherwise additional checking is carried out,
% which varies depending on the security context:

\begin{minted}{typescript}
  sanitize(
    ctx: SecurityContext,
    value: SafeValue | string | null
  ): string | null {
    if (value == null) return null;
    switch (ctx) {
      case SecurityContext.NONE:
        return value as string;
      case SecurityContext.HTML:
        if (value instanceof SafeHtmlImpl)
          return value.changingThisBreaksApplicationSecurity;
        this.checkNotSafeValue(value, 'HTML');
        return _sanitizeHtml(this._doc, String(value));
      case SecurityContext.STYLE:
        if (value instanceof SafeStyleImpl)
          return value.changingThisBreaksApplicationSecurity;
        this.checkNotSafeValue(value, 'Style');
        return _sanitizeStyle(value as string);
      case SecurityContext.SCRIPT:
        if (value instanceof SafeScriptImpl)
          return value.changingThisBreaksApplicationSecurity;
        this.checkNotSafeValue(value, 'Script');
        throw new Error('unsafe value used in a script context');
      case SecurityContext.URL:
        if (
          value instanceof SafeResourceUrlImpl ||
          value instanceof SafeUrlImpl
        ) {
          // Allow resource URLs in URL contexts,
          // they are strictly more trusted.
          return value.changingThisBreaksApplicationSecurity;
        }
        this.checkNotSafeValue(value, 'URL');
        return _sanitizeUrl(String(value));
      case SecurityContext.RESOURCE_URL:
        if (value instanceof SafeResourceUrlImpl) {
          return value.changingThisBreaksApplicationSecurity;
        }
        this.checkNotSafeValue(value, 'ResourceURL');
        throw new Error(
          'unsafe value used in a resource URL context (see http://g.co/ng/security#xss)'
        );
      default:
        throw new Error(
          `Unexpected SecurityContext ${ctx} (see http://g.co/ng/security#xss)`
        );
    }
  }
\end{minted}


% The
% \texttt{bypassSecurityTrust}
% methods returns an appropriate
% \texttt{SafeImpl}
% instance:

\begin{minted}{typescript}
  bypassSecurityTrustHtml(value: string): SafeHtml {
    return new SafeHtmlImpl(value);
  }
  bypassSecurityTrustStyle(value: string): SafeStyle {
    return new SafeStyleImpl(value);
  }
  bypassSecurityTrustScript(value: string): SafeScript {
    return new SafeScriptImpl(value);
  }
  bypassSecurityTrustUrl(value: string): SafeUrl {
    return new SafeUrlImpl(value);
  }
  bypassSecurityTrustResourceUrl(value: string): SafeResourceUrl {
    return new SafeResourceUrlImpl(value);

    ..
  }
\end{minted}

