\subsection{Platform-browser/src/dom}

We will sub-divide the code in the DOM sub-directory into four categories –
adapter/renderer, animation, debug and events.

We have seen how the Core package defines a rendering API and all other parts of the
Angular framework and applicaton code uses it to have content rendered. But Core
has no implementation. Now it is time to see an implementation, based on the DOM.
That is the role of these files:

\begin{itemize}
  \item shared\_styles\_host.ts
  \item util.ts
  \item dom\_adapter.ts
  \item dom\_renderer.ts
\end{itemize}

shared\_styles\_host.ts manages a set of styles.

Utils.ts contains simple string helpers:

\begin{minted}{typescript}
export function camelCaseToDashCase(input: string): string {
  return input.replace(
    CAMEL_CASE_REGEXP,
    (...m: string[]) => '-' + m[1].toLowerCase()
  );
}

export function dashCaseToCamelCase(input: string): string {
  return input.replace(DASH_CASE_REGEXP, (...m: string[]) =>
    m[1].toUpperCase()
  );
}
\end{minted}


The two main files involved in delivering the DomRenderer are dom\_adapter.ts and
dom\_renderer.ts. A DomAdapter is a class that represents an API very close to the
HTML DOM that every web developer is familiar with. A DOM renderer is an
implementation of Core’s Rendering API in terms of a DOM adapter.

The benefit that a DomAdapter brings (compared to hard-coding calls to the actual
DOM inside a browser), is that multiple implementations of a DomAdapter can be
supplied, including in scenarios where the real DOM is not available (e.g. server-side,
or inside webworkers).

The following diagram shows how Core’s renderer API, renderer implementations and
DOM adapters are related. For many applications, the entire application will run in the
main browser UI thread and so
\texttt{BrowserDomAdapter}
will be used alongside
\texttt{DefaultDomRenderer2}
.

{ almost all methods use }

For server applications with Platform-Server, a specialist DOM adapter called
DominoAdapter will be used, and this results in content being written to an HTML file.
DominoAdapter is implemented in the Platform-Server package here:

\begin{itemize}
  \item \href{https://github.com/angular/angular/blob/master/packages/platform-server/src/domino_adapter.ts}
        {<ANGULAR-MASTER>/packages/platform-server/src/domino\_adapter.ts}
\end{itemize}

It is based in this project:

\begin{itemize}
  \item \url{https://github.com/fgnass/domino}
\end{itemize}

We’ll explore this further when examining the Platform-Server package.

For more advanced browser applications that will use web workers, things are a little
more complicated and involves these main classes:
\texttt{WorkerRenderer2}
,
\texttt{MessageBasedRender2}
and
\texttt{WorkerDomAdapter}
.
\texttt{WorkerDomAdapter}
is used merely for
logging and does not place a significant part in rendering from workers. A message
broker based on a message bus exchanges messages between the web worker and
main browser UI thread.
\texttt{WorkerRenderer2}
runs in the web worker and forward all
rendering calls over the message broker to the main browser UI thread, where an
instance of
\texttt{MessageBasedRender2}
(which, despite its name, does not implement
Core’s renderer API) receives them and calls the regular
\texttt{DefaultDomRenderer2}
. We
will shortly examine in detail how rendering works with web workers. We’ll explore
rendering and web workers in the later chapter on Platform-WebWorker.

When trying to figure out how the DOM adaptor works, the best place to start is:

\begin{itemize}
  \item \href{fix: href loss url}
        {fix: href loss url}
\end{itemize}

The dom\_adapter.ts class defines the abstract
\texttt{DomAdapter}
class, the
\texttt{\_DOM}
variable
and two functions,
\texttt{getDOM()}
and
\texttt{setDOM()}
to get and set it.

\begin{minted}{typescript}
let _DOM: DomAdapter = null!;

export function getDOM() {
  return _DOM;
}

export function setDOM(adapter: DomAdapter) {
  _DOM = adapter;
}

export function setRootDomAdapter(adapter: DomAdapter) {
  if (!_DOM) {
    _DOM = adapter;
  }
}
\end{minted}


The
\texttt{DomAdapter}
class is a very long list of methods similar to what you would find a
normal DOM API (pay attention to the security warning!)– here is just a sampling:

\begin{minted}{typescript}
/**
 * Provides DOM operations in an environment-agnostic way.
 *
 * @security Tread carefully! Interacting with the DOM directly is dangerous
 * and can introduce XSS risks.
 */
export abstract class DomAdapter {
  abstract querySelector(el: any, selector: string): any;
  abstract querySelectorAll(el: any, selector: string): any[];
  abstract nodeName(node: any): string;
  abstract nodeValue(node: any): string | null;
  abstract parentElement(el: any): Node | null;
  abstract appendChild(el: any, node: any): any;
  abstract removeChild(el: any, node: any): any;
  abstract replaceChild(el: any, newNode: any, oldNode: any): any;
  abstract remove(el: any): Node;
  abstract insertBefore(parent: any, ref: any, node: any): any;
  abstract insertAfter(parent: any, el: any, node: any): any;
  abstract getText(el: any): string | null;
  abstract setText(el: any, value: string): any;
  abstract getValue(el: any): string;
  abstract setValue(el: any, value: string): any;
  abstract createComment(text: string): any;
  abstract createTemplate(html: any): HTMLElement;
  abstract createElement(tagName: any, doc?: any): HTMLElement;
  abstract createTextNode(text: string, doc?: any): Text;
  abstract removeStyle(element: any, styleName: string): any;
  abstract getStyle(element: any, styleName: string): string;
  abstract hasAttribute(element: any, attribute: string): boolean;
  abstract getAttribute(element: any, attribute: string): string | null;
  abstract setAttribute(element: any, name: string, value: string): any;
  abstract getCookie(name: string): string | null;
  abstract setCookie(name: string, value: string): any;
  ..
}
\end{minted}


The dom\_renderer.ts file defines the DOM renderer that relies on
\texttt{getDOM()}
to supply
a
\texttt{DomAdapter}
. The main class it supplies is
\texttt{DomRendererFactory2}
which implements
\texttt{RendererFactory2}
. We saw earlier how
\texttt{DomRendererFactory2}
is used in the
\texttt{BROWSER\_MODULE\_PROVIDERS}
declaration, as used by
\texttt{NgModule}
:

\begin{minted}{typescript}
export const BROWSER_MODULE_PROVIDERS: StaticProvider[] = [
  ..
  {
    provide: DomRendererFactory2,
    useClass: DomRendererFactory2,
    deps: [EventManager, DomSharedStylesHost, APP_ID],
  },
  { provide: RendererFactory2, useExisting: DomRendererFactory2 },
  ..
];
\end{minted}


The
\texttt{DomRendererFactory2}
class manages a map of strings to
\texttt{Renderer2}
instances.

\begin{minted}{typescript}
export class DomRendererFactory2 implements RendererFactory2 {
  private rendererByCompId = new Map<string, Renderer2>();
  private defaultRenderer: Renderer2;

  constructor(
    private eventManager: EventManager,
    private sharedStylesHost: DomSharedStylesHost,
    @Inject(APP_ID) private appId: string
  ) {
    this.defaultRenderer = new DefaultDomRenderer2(eventManager);
  }
  ..
}
\end{minted}


Its
\texttt{createRenderer()}
method performs a switch on the encapsulation type and
returns an appropriate
\texttt{Renderer2}
:

\begin{minted}{typescript}
  createRenderer(element: any, type: RendererType2 | null): Renderer2 {
    if (!element || !type) {
      return this.defaultRenderer;
    }
    switch (type.encapsulation) {
      case ViewEncapsulation.Emulated: {
        let renderer = this.rendererByCompId.get(type.id);
        if (!renderer) {
          renderer = new EmulatedEncapsulationDomRenderer2(
            this.eventManager,
            this.sharedStylesHost,
            type,
            this.appId
          );
          this.rendererByCompId.set(type.id, renderer);
        }
        (<EmulatedEncapsulationDomRenderer2>renderer).applyToHost(element);
        return renderer;
      }
      case ViewEncapsulation.Native:
      case ViewEncapsulation.ShadowDom:
        return new ShadowDomRenderer(
          this.eventManager,
          this.sharedStylesHost,
          element,
          type
        );
      default: {
        if (!this.rendererByCompId.has(type.id)) {
          const styles = flattenStyles(type.id, type.styles, []);
          this.sharedStylesHost.addStyles(styles);
          this.rendererByCompId.set(type.id, this.defaultRenderer);
        }
        return this.defaultRenderer;
      }
    }
  }
\end{minted}


\texttt{DefaultDomRenderer2}
is where renderering actually occurs. It implements
\texttt{Renderer2}
. As an example, let’s look at its
\texttt{createElement()}
method:

\begin{minted}{typescript}
  createElement(name: string, namespace?: string): any {
    if (namespace) {
      // In cases where Ivy (not ViewEngine) is giving us the actual
      // namespace, the look up by key
      // will result in undefined, so we just return the namespace here.
      return document.createElementNS(
        NAMESPACE_URIS[namespace] || namespace,
        name
      );
    }

    return document.createElement(name);
  }
\end{minted}


DOM debugging is supported via:

\begin{itemize}
  \item src/dom/debug/by.ts
  \item src/dom/debug/ng\_probe.ts
\end{itemize}

The
\texttt{By}
class may be used with Core’s
\texttt{DebugElement}
to supply predicates for its query
functions. It supplies three predicates – all, css and directive:

\begin{minted}{typescript}
// Predicates for use with {@link DebugElement}'s query functions.
export class By {
  // Match all elements.
  static all(): Predicate<DebugElement> {
    return (debugElement) => true;
  }
  // Match elements by the given CSS selector.
  static css(selector: string): Predicate<DebugElement> {
    return (debugElement) => {
      return debugElement.nativeElement != null
        ? getDOM().elementMatches(debugElement.nativeElement, selector)
        : false;
    };
  }

  // Match elements that have the given directive present.
  static directive(type: Type<any>): Predicate<DebugElement> {
    return (debugElement) => debugElement.providerTokens!.indexOf(type) !== -1;
  }
}
\end{minted}


DOM events are supported via:

\begin{itemize}
  \item src/dom/events/dom\_events.ts
  \item src/dom/events/event\_manager.ts
  \item src/dom/events/hammer\_common.ts
  \item src/dom/events/hammer\_gestures.ts
  \item src/dom/events/key\_events.ts
\end{itemize}

event\_manager.ts provides two classes –
\texttt{EventManager}
and
\texttt{EventManagerPlugin}
.

\texttt{EventManager}
manages an array of
\texttt{EventManagerPlugin}
s, which is defined as:

\begin{minted}{typescript}
export abstract class EventManagerPlugin {
  constructor(private _doc: any) {}

  manager!: EventManager;

  abstract supports(eventName: string): boolean;

  abstract addEventListener(
    element: HTMLElement,
    eventName: string,
    handler: Function
  ): Function;

  addGlobalEventListener(
    element: string,
    eventName: string,
    handler: Function
  ): Function {
    const target: HTMLElement = getDOM().getGlobalEventTarget(
      this._doc,
      element
    );
    if (!target) {
      throw new Error(
        `Unsupported event target ${target} for event ${eventName}`
      );
    }
    return this.addEventListener(target, eventName, handler);
  }
}
\end{minted}


It provides two methods to add event listeners to targets represented either by an
\texttt{HTMLElement}
or a string.
\texttt{EventManager}
itself is defined as an injectable class:

\begin{minted}{typescript}
/**
 * An injectable service that provides event management for Angular
 * through a browser plug-in.
 *
 * @publicApi
 */
@Injectable()
export class EventManager {
  private _plugins: EventManagerPlugin[];
  private _eventNameToPlugin = new Map<string, EventManagerPlugin>();

  /**
   * Initializes an instance of the event-manager service.
   */
  constructor(
    @Inject(EVENT_MANAGER_PLUGINS) plugins: EventManagerPlugin[],
    private _zone: NgZone
  ) {
    plugins.forEach((p) => (p.manager = this));
    this._plugins = plugins.slice().reverse();
  }

  /**
   * Registers a handler for a specific element and event.
   *
   * @param element The HTML element to receive event notifications.
   * @param eventName The name of the event to listen for.
   * @param handler A function to call when the notification occurs. Receives
the
   * event object as an argument.
   * @returns  A callback function that can be used to remove the handler.
   */
  addEventListener(
    element: HTMLElement,
    eventName: string,
    handler: Function
  ): Function {
    const plugin = this._findPluginFor(eventName);
    return plugin.addEventListener(element, eventName, handler);
  }

  /**
   * Registers a global handler for an event in a target view.
   *
   * @param target A target for global event notifications. One of "window",
"document", or "body".
   * @param eventName The name of the event to listen for.
   * @param handler A function to call when the notification occurs. Receives
the
   * event object as an argument.
   * @returns A callback function that can be used to remove the handler.
   */
  addGlobalEventListener(
    target: string,
    eventName: string,
    handler: Function
  ): Function {
    const plugin = this._findPluginFor(eventName);
    return plugin.addGlobalEventListener(target, eventName, handler);
  }

  /**
   * Retrieves the compilation zone in which event listeners are registered.
   */
  getZone(): NgZone {
    return this._zone;
  }

  /** @internal */
  _findPluginFor(eventName: string): EventManagerPlugin {
    const plugin = this._eventNameToPlugin.get(eventName);
    if (plugin) {
      return plugin;
    }
    const plugins = this._plugins;
    for (let i = 0; i < plugins.length; i++) {
      const plugin = plugins[i];
      if (plugin.supports(eventName)) {
        this._eventNameToPlugin.set(eventName, plugin);
        return plugin;
      }
    }
    throw new Error(`No event manager plugin found for event ${eventName}`);
  }
}
\end{minted}


Its constructor is defined in such a way as to allow dependency injection to inject an
array of event manager plugins. We note this list is reversed in the constructor, which
will impact the ordering of finding a plugin for an event type.

The other files in src/dom/events supply event manager plugins.

Note the comment in
\texttt{DomEventsPlugin}
:

\begin{minted}{typescript}
@Injectable()
export class DomEventsPlugin extends EventManagerPlugin {
  ..
  // This plugin should come last in the list of plugins,
  // because it accepts all events.
  supports(eventName: string): boolean {
    return true;
  }
}
\end{minted}


Touch events via the hammer project are supported via the hammer\_gesture.ts file.
The list of supported touch events are:

\begin{minted}{typescript}
const EVENT_NAMES = {
  // pan
  pan: true,
  panstart: true,
  panmove: true,
  panend: true,
  pancancel: true,
  panleft: true,
  panright: true,
  panup: true,
  pandown: true,
  // pinch
  pinch: true,
  pinchstart: true,
  pinchmove: true,
  pinchend: true,
  pinchcancel: true,
  pinchin: true,
  pinchout: true,
  // press
  press: true,
  pressup: true,
  // rotate
  rotate: true,
  rotatestart: true,
  rotatemove: true,
  rotateend: true,
  rotatecancel: true,
  // swipe
  swipe: true,
  swipeleft: true,
  swiperight: true,
  swipeup: true,
  swipedown: true,
  // tap
  tap: true,
};
\end{minted}


This is used by
\texttt{HammerGesturesPlugin}
:

\begin{minted}{typescript}
@Injectable()
export class HammerGesturesPlugin extends EventManagerPlugin {
  constructor(
    @Inject(DOCUMENT) doc: any,
    @Inject(HAMMER_GESTURE_CONFIG) private _config: HammerGestureConfig,
    private console: Console,
    @Optional() @Inject(HAMMER_LOADER) private loader?: HammerLoader | null
  ) {
    super(doc);
  }
  ..
}
\end{minted}


Its
\texttt{addEventListener}
method is implemented as:

\begin{minted}{typescript}
  addEventListener(
    element: HTMLElement,
    eventName: string,
    handler: Function
  ): Function {
    const zone = this.manager.getZone();
    eventName = eventName.toLowerCase();
    if (!(window as any).Hammer && this.loader) {
      let cancelRegistration = false;
      let deregister: Function = () => {
        cancelRegistration = true;
      };
      this.loader(); ..
      return () => {
        deregister();
      };
    }
    %\step{1}% return zone.runOutsideAngular(() => {
      // Creating the manager bind events, must be done outside of angular
      const mc = this._config.buildHammer(element);
      const callback = function (eventObj: HammerInput) {
        zone.runGuarded(function () {
          handler(eventObj);
        });
      };
      mc.on(eventName, callback);
      return () => {
        mc.off(eventName, callback);
        // destroy mc to prevent memory leak
        if (typeof mc.destroy === 'function') {
          mc.destroy();
        }
      };
    });
  }
\end{minted}


Note the use of
\texttt{zone.runOutsideAngular()}
1
. Also note it does not implement
\texttt{addGlobalEventListener}
. Its constructor expects a
\texttt{HAMMER\_GESTURE\_CONFIG}
from
dependency injection. The Hammer package is used in the injectable
\texttt{HammerGestureConfig}
class:

\begin{minted}{typescript}
@Injectable()
export class HammerGestureConfig {
  events: string[] = [];

  overrides: { [key: string]: Object } = {};

  buildHammer(element: HTMLElement): HammerInstance {
    const mc = new Hammer(element);

    mc.get('pinch').set({ enable: true });
    mc.get('rotate').set({ enable: true });

    for (const eventName in this.overrides) {
      mc.get(eventName).set(this.overrides[eventName]);
    }

    return mc;
  }
}
\end{minted}


These event manager plugins need to be set in the
\texttt{NgModule}
configuration. We see
how this is done in
\texttt{BROWSER\_MODULE\_PROVIDERS}
:

\begin{minted}{typescript}
export const BROWSER_MODULE_PROVIDERS: StaticProvider[] = [
  {
    provide: EVENT_MANAGER_PLUGINS,
    useClass: DomEventsPlugin,
    multi: true,
    deps: [DOCUMENT, NgZone, PLATFORM_ID],
  },
  {
    provide: EVENT_MANAGER_PLUGINS,
    useClass: KeyEventsPlugin,
    multi: true,
    deps: [DOCUMENT],
  },
  {
    provide: EVENT_MANAGER_PLUGINS,
    useClass: HammerGesturesPlugin,
    multi: true,
    deps: [
      DOCUMENT,
      HAMMER_GESTURE_CONFIG,
      Console,
      [new Optional(), HAMMER_LOADER],
    ],
  },
  { provide: HAMMER_GESTURE_CONFIG, useClass: HammerGestureConfig, deps: [] },
  {
    provide: DomRendererFactory2,
    useClass: DomRendererFactory2,
    deps: [EventManager, DomSharedStylesHost, APP_ID],
  },
  {
    provide: EventManager,
    useClass: EventManager,
    deps: [EVENT_MANAGER_PLUGINS, NgZone],
  },
  ELEMENT_PROBE_PROVIDERS,
  ..
];
\end{minted}

