\subsection{http/src}

% The http/src directory has following source files:

\begin{itemize}
  \item backend.ts
  \item client.ts
  \item headers.ts
  \item interceptors.ts
  \item jsonp.ts
  \item module.ts
  \item params.ts
  \item request.ts
  \item response.ts
  \item xhr.ts
  \item xsrf.ts
\end{itemize}

% The module.ts file has a number of
% \texttt{NgModule}
% definitions, the main one being:

\begin{minted}{typescript}
@NgModule({
  imports: [
    HttpClientXsrfModule.withOptions({
      cookieName: 'XSRF-TOKEN',
      headerName: 'X-XSRF-TOKEN',
    }),
  ],
  providers: [
    HttpClient,
    // HttpHandler is the backend + interceptors and is constructed
    // using the interceptingHandler factory function.
    {
      provide: HttpHandler,
      useFactory: interceptingHandler,
      deps: [HttpBackend, [new Optional(), new Inject(HTTP_INTERCEPTORS)]],
    },
    HttpXhrBackend,
    { provide: HttpBackend, useExisting: HttpXhrBackend },
    BrowserXhr,
    { provide: XhrFactory, useExisting: BrowserXhr },
  ],
})
export class HttpClientModule {}
\end{minted}


% That
% \texttt{HttpXHRBackend}
% provider for
% \texttt{HttpClientModule}
% may need to be replaced when
% using an in-memory-web-api for testing, as explained here (search for
% \texttt{InMemoryWebApiModule}
% ):

\begin{itemize}
  \item \url{https://angular.io/docs/ts/latest/tutorial/toh-pt6.html}
\end{itemize}

% We note
% \texttt{HttpClientModule}
% uses
% \texttt{HttpClientXsrfModule}
% (we’ll see where
% \texttt{XSRF\_COOKIE\_NAME}
% and
% \texttt{XSRF\_HEADER\_NAME}
% are defined when examining XSRF
% shortly):

\begin{minted}{typescript}
@NgModule({
  providers: [
    HttpXsrfInterceptor,
    {
      provide: HTTP_INTERCEPTORS,
      useExisting: HttpXsrfInterceptor,
      multi: true,
    },
    { provide: HttpXsrfTokenExtractor, useClass: HttpXsrfCookieExtractor },
    { provide: XSRF_COOKIE_NAME, useValue: 'XSRF-TOKEN' },
    { provide: XSRF_HEADER_NAME, useValue: 'X-XSRF-TOKEN' },
  ],
})
export class HttpClientXsrfModule {
  ..
}
\end{minted}


% The headers.ts file provides the
% \texttt{HttpHeaders}
% class. This is essentially a wrapper
% around a
% \texttt{Map}
% data structure. It defines its primary data structure as:

\begin{minted}{typescript}
  private headers: Map<string, string[]>;
\end{minted}


% It offers functionality to work with that data structure.

\begin{minted}{typescript}
// Immutable set of Http headers, with lazy parsing.
export class HttpHeaders {
  /**
   * Internal map of lowercase header names to values.
   */
  private headers: Map<string, string[]>;

  /**
   * Internal map of lowercased header names to the normalized
   * form of the name (the form seen first).
   */
  private normalizedNames: Map<string, string> = new Map();

  /**
   * Complete the lazy initialization of this object (needed before reading).
   */
  private lazyInit: HttpHeaders | Function | null;

  /**
   * Queued updates to be materialized the next initialization.
   */
  private lazyUpdate: Update[] | null = null;

  constructor(headers?: string | { [name: string]: string | string[] }) {
    ..
  }
}
\end{minted}


% The request.ts file implements the
% \texttt{Request}
% class. A request is a request method, a set
% of headers, and a content type. Depending on the request method, a body may or
% may not be needed. To supply initialization parameters, this interface is defined:

\begin{minted}{typescript}
interface HttpRequestInit {
  headers?: HttpHeaders;
  reportProgress?: boolean;
  params?: HttpParams;
  responseType?: 'arraybuffer' | 'blob' | 'json' | 'text';
  withCredentials?: boolean;
}
\end{minted}


% We see its use in the constructor for
% \texttt{Request}
% :

\begin{minted}{typescript}
// Next, need to figure out which argument holds the HttpRequestInit
// options, if any.
let options: HttpRequestInit | undefined;

// Check whether a body argument is expected. The only valid way to omit
// the body argument is to use a known no-body method like GET.
if (mightHaveBody(this.method) || !!fourth) {
  // Body is the third argument, options are the fourth.
  this.body = third !== undefined ? (third as T) : null;
  options = fourth;
} else {
  // No body required, options are the third argument.
  // The body stays null.
  options = third as HttpRequestInit;
}

// If options have been passed, interpret them.
if (options) {
  ..
}
\end{minted}


% Similarly, the response.ts implements the
% \texttt{Response}
% classes, which are based on the
% \texttt{HttpResponseBase}
% class:

\begin{minted}{typescript}
export abstract class HttpResponseBase {
  // All response headers.
  readonly headers: HttpHeaders;
  // Response status code.
  readonly status: number;
  // Textual description of response status code.
  readonly statusText: string;
  //URL of the resource retrieved, or null if not available.
  readonly url: string | null;
  // Whether the status code falls in the 2xx range.
  readonly ok: boolean;
  // Type of the response, narrowed to either the full response
  // or the header.
  readonly type: HttpEventType.Response | HttpEventType.ResponseHeader;
}
\end{minted}


% The backend.ts file provides classes for pluggable connection handling. By providing
% alternative implementations of these, flexible server communication is supported.
% Note what the comments in the code say:

\begin{minted}{typescript}
/**
 * Transforms an `HttpRequest` into a stream of `HttpEvent`s, one of
 * which will likely be a  * `HttpResponse`.
 *
 * `HttpHandler` is injectable. When injected, the handler instance
 * dispatches requests to the
 * first interceptor in the chain, which dispatches to the second, etc,
 * eventually reaching the `HttpBackend`.
 *
 * In an `HttpInterceptor`, the `HttpHandler` parameter is the
 * next interceptor in the chain.
 */
export abstract class HttpHandler {
  abstract handle(req: HttpRequest<any>): Observable<HttpEvent<any>>;
}

/**
 * A final `HttpHandler` which will dispatch the request via browser HTTP
 * APIs to a backend.
 *
 * Interceptors sit between the `HttpClient` interface and the `HttpBackend`.
 *
 * When injected, `HttpBackend` dispatches requests directly to the backend,
 * without going through the interceptor chain.
 */
export abstract class HttpBackend implements HttpHandler {
  abstract handle(req: HttpRequest<any>): Observable<HttpEvent<any>>;
}
\end{minted}


% The backend is important because it means a test-oriented in-memory backend could
% be switched for the real backend as needed, without further changes to HTTP code.

% When communicating with a real remote server, the main workload is performed by
% \texttt{HttpXhrBackend}
% class which is defined in
% \url{xhr.ts}
% :

\begin{minted}{typescript}
@Injectable()
export class HttpXhrBackend implements HttpBackend {
  constructor(private xhrFactory: XhrFactory) {}
  ..
}
\end{minted}


% The
% \url{xsrf.ts}
% file defines two injectable classes that helps with XSRF (Cross Site Request
% Forgery) protection:

\begin{minted}{ts}
// `HttpXsrfTokenExtractor` which retrieves the token from a cookie.
@Injectable()
export class HttpXsrfCookieExtractor implements HttpXsrfTokenExtractor {
  private lastCookieString: string = '';
  private lastToken: string|null = null;

  constructor(
      @Inject(DOCUMENT) private doc: any,
@Inject(PLATFORM_ID) private platform: string,
      @Inject(XSRF_COOKIE_NAME) private cookieName: string) {}

  getToken(): string|null ..
  }
}

// `HttpInterceptor` which adds an XSRF token to eligible outgoing requests.
@Injectable()
export class HttpXsrfInterceptor implements HttpInterceptor {
  constructor(
      private tokenService: HttpXsrfTokenExtractor,
      @Inject(XSRF_HEADER_NAME) private headerName: string) {}
  intercept(req: HttpRequest<any>, next: HttpHandler):
Observable<HttpEvent<any>> {..
  }
}
\end{minted}


% This files also defines:

\begin{minted}{typescript}
export const XSRF_COOKIE_NAME = new InjectionToken<string>('XSRF_COOKIE_NAME');
export const XSRF_HEADER_NAME = new InjectionToken<string>('XSRF_HEADER_NAME');
\end{minted}


% These are used when defining the @NgModule
% \texttt{HttpClientXsrfModule}
% as we saw
% earlier.
