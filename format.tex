\usepackage{fontspec}
\setmainfont{Linux Libertine O}
\setsansfont{Droid Sans}
\setmonofont{CMU Typewriter Text}
\usepackage[showframe]{geometry}
% \geometry{left=2.54cm, right=2.54cm, top=3.18cm, bottom=3.18cm}
\usepackage{indentfirst}
\usepackage{titlesec}
\titleformat{\chapter}{\color{red}\LARGE\bfseries\centering}{\thechapter :}{.5em}{}[\vspace*{1em}\titlerule]
\titleformat{\section}{\color{red}\Large\bfseries}{}{0pt}{}
\titleformat{\subsection}{\color{red}\large\bfseries}{}{0pt}{}
\usepackage{titletoc}
\titlecontents{section}[2em]{\addvspace{2pt}\filright}{}{}{\titlerule*[8pt]{.}\contentspage}
\titlecontents{subsection}[4em]{\addvspace{2pt}\filright}{}{}{\titlerule*[8pt]{.}\contentspage}
% 应在 geometry 之后引入 不然页眉线不撑满整个宽度
\usepackage{fancyhdr}
\pagestyle{fancy}
\fancyhf{}
\fancyhead[LO]{\color{red}\nouppercase\rightmark}
\fancyhead[LE,RO]{\color{red}\thepage}
\fancyhead[RE]{\color{red}\nouppercase\leftmark}
\renewcommand{\headrulewidth}{0.6pt}
\renewcommand{\chaptermark}[1]{\markboth{\thechapter:#1}{}}
\renewcommand{\sectionmark}[1]{\markright{#1}}

\usepackage{xpatch}
\xpretocmd\headrule{\color{red}}{}{\PatchFailed}
\usepackage{xurl}
% make te text hyphenat
\usepackage[htt]{hyphenat}
\usepackage{xcolor}
\definecolor{lightgray}{rgb}{.9, .9, .9}
\usepackage{enumitem}
\setlist[itemize]{font=\ttfamily, itemsep=0pt, parsep=0pt}
\usepackage{float}
\usepackage{caption}
\captionsetup{font={large, bf}, labelformat=empty, justification=centering}
\usepackage{tikz}
\usepackage{graphicx}
\usepackage[all,pdf,color]{xy}
\usepackage{rotating}
\usepackage{float}
\usepackage{makeidx}
\makeindex

\usepackage{hyperref}
\hypersetup{
  bookmarksnumbered,
  colorlinks
}
\usepackage[cachedir=../_minted-main]{minted}
% xleftmargin=20pt
\setminted[typescript]{fontsize=\small, breaklines, linenos, bgcolor=lightgray}
\renewcommand{\theFancyVerbLine}{\ttfamily
  \textcolor[rgb]{0.5,0.5,1.0}{\scriptsize
    \oldstylenums{\arabic{FancyVerbLine}}}}
\usepackage{listings}
\lstset{
  backgroundcolor=\color{lightgray},
  extendedchars=true,
  basicstyle=\small\ttfamily,
  showstringspaces=false,
  showspaces=false,
  numbers=left,
  numberstyle=\small,
  numbersep=9pt,
  tabsize=2,
  breaklines=true,
  showtabs=false,
  captionpos=b,
  escapechar=\%
}
\newcommand\step[1]{\setlength\fboxsep{1.5pt}\colorbox{black}{\textcolor{white}{#1}}}
