\subsection{Relationship between Zone/ZoneSpec/ZoneDelegate interfaces}

Think of
\texttt{ZoneSpec}
as the processing engine that controls how a zone works. It is a
required parameter to the
\texttt{Zone.fork()}
method:

\begin{minted}{typescript}
  // Used to create a child zone.
  // @param zoneSpec A set of rules which the child zone should follow.
  // @returns {Zone} A new child zone.
  fork(zoneSpec: ZoneSpec): Zone;
\end{minted}


Often when a zone needs to perform an action, it uses the supplied
\texttt{ZoneSpec}
. Do you
want to record a long stack trace, keep track of tasks, work with WTF (discussed
later) or run async test well? For each a these a different
\texttt{ZoneSpec}
is supplied, and
each offers different features and comes with different processing costs. Zone.js
supplies one implementation of the
\texttt{Zone}
interface, and multiple implementations of
the
\texttt{ZoneSpec}
interface (in
\href{https://github.com/angular/zone.js/tree/master/lib/zone-spec}
{<ZONE-MASTER>/lib/zone-spec}
). Application code with
specialist needs could create a custom
\texttt{ZoneSpec}
.

An application can build up a hierarchy of zones and sometimes a zone needs to make
a call into another zone further up the hierarchy, and for this a
\texttt{ZoneDelegate}
is used.
