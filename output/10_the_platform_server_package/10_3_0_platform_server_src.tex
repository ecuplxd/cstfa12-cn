\subsection{platform-server/src}

These source files are present:

\begin{itemize}
  \item domino\_adapter.ts
  \item http.ts
  \item location.ts
  \item platform\_state.ts
  \item server\_events.ts
  \item server\_renderer.ts
  \item server.ts
  \item styles\_host.ts
  \item tokens.ts
  \item transfer\_state.ts
  \item utils.t
  \item version.ts
\end{itemize}

There are no sub-directories beneath src.

The server.ts file declares this const:

\begin{minted}{typescript}
export const INTERNAL_SERVER_PLATFORM_PROVIDERS: StaticProvider[] = [
  { provide: DOCUMENT, useFactory: _document, deps: [Injector] },
  { provide: PLATFORM_ID, useValue: PLATFORM_SERVER_ID },
  {
    provide: PLATFORM_INITIALIZER,
    useFactory: initDominoAdapter,
    multi: true,
    deps: [Injector],
  },
  {
    provide: PlatformLocation,
    useClass: ServerPlatformLocation,
    deps: [DOCUMENT, [Optional, INITIAL_CONFIG]],
  },
  { provide: PlatformState, deps: [DOCUMENT] },
  // Add special provider that allows multiple instances of
  // platformServer* to be created.
  { provide: ALLOW_MULTIPLE_PLATFORMS, useValue: true },
];
\end{minted}


The factory functions
\texttt{platformServer}
and
\texttt{platformDynamicServer}
create server
platforms that use the offline template compiler and the runtime template compiler
respectively.

It adds two additional provider configurations. Firstly,
\texttt{PLATFORM\_INITIALIZER}
, which
is an initializer function called before bootstrapping. Here we see it initializes the
domino adapter, in a call to the local function
\texttt{initDominoAdapter()}
which calls
\texttt{makeCurrent()}
for the
\texttt{DominoAdapter}
:

\begin{minted}{typescript}
function initDominoAdapter(injector: Injector) {
  return () => {
    DominoAdapter.makeCurrent();
  };
}
\end{minted}


Secondly, it adds
\texttt{PlatformLocation}
, which is used by applications to interact with
location (URL) information. It is set to a class defined in location.ts,
\texttt{ServerPlatformLocation}
, which mostly just throws exceptions:

\begin{minted}{typescript}
/**
 * Server-side implementation of URL state. Implements `pathname`, `search`,
 * and `hash` but not the state stack.
 */
@Injectable()
export class ServerPlatformLocation implements PlatformLocation {
  public readonly href: string = '/';
  public readonly hostname: string = '/';
  public readonly protocol: string = '/';
  public readonly port: string = '/';
  public readonly pathname: string = '/';
  public readonly search: string = '';
  public readonly hash: string = '';
  private _hashUpdate = new Subject<LocationChangeEvent>();
  ..
  getBaseHrefFromDOM(): string {
    return getDOM().getBaseHref(this._doc)!;
  }

  onPopState(fn: LocationChangeListener): void {
    // No-op: a state stack is not implemented, so
    // no events will ever come.
  }

  onHashChange(fn: LocationChangeListener): void {
    this._hashUpdate.subscribe(fn);
  }

  get url(): string {
    return `${this.pathname}${this.search}${this.hash}`;
  }

  forward(): void {
    throw new Error('Not implemented');
  }
  back(): void {
    throw new Error('Not implemented');
  }

  // History API isn't available on server, therefore return undefined
  getState(): unknown {
    return undefined;
  }
}
\end{minted}


server.ts declares two exported functions:

\begin{minted}{typescript}
export const platformServer = createPlatformFactory(
  platformCore,
  'server',
  INTERNAL_SERVER_PLATFORM_PROVIDERS
);
export const platformDynamicServer = createPlatformFactory(
  platformCoreDynamic,
  'serverDynamic',
  INTERNAL_SERVER_PLATFORM_PROVIDERS
);
\end{minted}


We saw earlier that
\texttt{platformCore}
is defined in:

\begin{itemize}
  \item \href{https://github.com/angular/angular/blob/master/packages/core/src/platform_core_providers.ts}
        {<ANGULAR-MASTER>/packages/core/src/platform\_core\_providers.ts}
\end{itemize}

as:

\begin{minted}{typescript}
const _CORE_PLATFORM_PROVIDERS: StaticProvider[] = [
  // Set a default platform name for platforms that don't set it explicitly.
  { provide: PLATFORM_ID, useValue: 'unknown' },
  { provide: PlatformRef, deps: [Injector] },
  { provide: TestabilityRegistry, deps: [] },
  { provide: Console, deps: [] },
];

/**
 * This platform has to be included in any other platform
 *
 * @publicApi
 */
export const platformCore = createPlatformFactory(
  null,
  'core',
  _CORE_PLATFORM_PROVIDERS
);
\end{minted}


\texttt{platformCoreDynamic}
adds additional provider config (for the dynamic compiler) to
\texttt{platformCore}
and is defined in:

\begin{itemize}
  \item \href{fix: href loss url}
        {fix: href loss url}
\end{itemize}

as:

\begin{minted}{typescript}
export const platformCoreDynamic = createPlatformFactory(
  platformCore,
  'coreDynamic',
  [
    { provide: COMPILER_OPTIONS, useValue: {}, multi: true },
    {
      provide: CompilerFactory,
      useClass: JitCompilerFactory,
      deps: [COMPILER_OPTIONS],
    },
  ]
);
\end{minted}


\texttt{createPlatformFactory()}
is defined in:

\begin{itemize}
  \item \href{https://github.com/angular/angular/blob/master/packages/core/src/application_ref.ts}
        {<ANGULAR-MASTER>/packages/core/src/application\_ref.ts}
\end{itemize}

it calls Core’s
\texttt{createPlatform()}
with the supplied parameters, resulting in a new
platform being constructed.

The remain part of Platform-Server’s server.ts file to discuss is the definition of the
\texttt{NgModule}
called
\texttt{ServerModule}
:

\begin{minted}{typescript}
/**
 * The ng module for the server.
 *
 * @publicApi
 */
@NgModule({
  exports: [BrowserModule],
  imports: [HttpClientModule, NoopAnimationsModule],
  providers: [
    SERVER_RENDER_PROVIDERS,
    SERVER_HTTP_PROVIDERS,
    { provide: Testability, useValue: null },
    { provide: ViewportScroller, useClass: NullViewportScroller },
  ],
})
export class ServerModule {}

function _document(injector: Injector) {
  let config: PlatformConfig | null = injector.get(INITIAL_CONFIG, null);
  if (config && config.document) {
    return parseDocument(config.document, config.url);
  } else {
    return getDOM().createHtmlDocument();
  }
}
\end{minted}


The domino\_adapter.ts file create a DOM adapter for Domino. The Domino library is a
DOM engine for HTML5 that runs in Node. Its project page is:

\begin{itemize}
  \item \url{https://github.com/fgnass/domino}
\end{itemize}

and states:

\emph{“As the name might suggest, domino's goal is to provide a DOM in Node.”}

The domino\_adapter.ts file provides an adapter class,
\texttt{DominoDomAdapter}
, based on
Domino’s serialization functionality that implements a
\texttt{DomAdapter}
suitable for use in
server environments.

The domino\_adapter.ts file has these functions to parse and serialze a document:

\begin{minted}{typescript}
/**
 * Parses a document string to a Document object.
 */
export function parseDocument(html: string, url = '/') {
  let window = domino.createWindow(html, url);
  let doc = window.document;
  return doc;
}

/**
 * Serializes a document to string.
 */
export function serializeDocument(doc: Document): string {
  return (doc as any).serialize();
}
\end{minted}


We see
\texttt{serializeDocument}
being called from:

\begin{itemize}
  \item \href{https://github.com/angular/angular/blob/master/packages/platform-server/src/platform_state.ts}
        {<ANGULAR-MASTER>/packages/platform-server/src/platform\_state.ts}
\end{itemize}

as:

\begin{minted}{typescript}
@Injectable()
export class PlatformState {
  constructor(@Inject(DOCUMENT) private _doc: any) {}

  /**
   * Renders the current state of the platform to string.
   */
  renderToString(): string {
    return serializeDocument(this._doc);
  }

  /**
   * Returns the current DOM state.
   */
  getDocument(): any {
    return this._doc;
  }
}
\end{minted}


\texttt{BrowserDomAdapter}
is defined in:

\begin{itemize}
  \item \href{https://github.com/angular/angular/blob/master/packages/platform-browser/src/browser/browser_adapter.ts}
        {<ANGULAR-MASTER>/packages/platfrom-browser/src/browser/browser\_adapter.ts}
\end{itemize}

\texttt{DominoAdapter}
simply extends
\texttt{BrowserDomAdapter}
:

\begin{minted}{typescript}
/**
 * DOM Adapter for the server platform based on
 * https://github.com/fgnass/domino.
 */
export class DominoAdapter extends BrowserDomAdapter {
  private static defaultDoc: Document;
  ..
}
\end{minted}


Its static
\texttt{makeCurrent()}
method, that we saw Universal Angular uses for server-side
rendering,  initializes those three variables and then calls
\texttt{setRootDomAdapter()}
:

\begin{minted}{typescript}
  static makeCurrent() {
    setDomTypes();
    setRootDomAdapter(new DominoAdapter());
  }
\end{minted}


Recall that
\texttt{setRootDomAdapter()}
is defined in:

\begin{itemize}
  \item \href{https://github.com/angular/angular/blob/master/packages/platform-browser/src/dom/dom_adapter.ts}
        {<ANGULAR-MASTER>/packages/platform-browser/src/dom/dom\_adapter.ts}
\end{itemize}

as:

\begin{minted}{typescript}
let _DOM: DomAdapter = null!;

export function getDOM() {
  return _DOM;
}

export function setDOM(adapter: DomAdapter) {
  _DOM = adapter;
}

export function setRootDomAdapter(adapter: DomAdapter) {
  if (!_DOM) {
    _DOM = adapter;
  }
}
\end{minted}


and that
\texttt{getDOM()}
is used by the
\texttt{DOMRenderer}
. Hence our
\texttt{DominoAdapter}
gets wired
into the DOM renderer.
Many of the  DOM adapter methods throw exceptions as they do not make sense
server-side:

\begin{minted}{typescript}
  getHistory(): History {
    throw _notImplemented('getHistory');
  }
  getLocation(): Location {
    throw _notImplemented('getLocation');
  }
  getUserAgent(): string {
    return 'Fake user agent';
  }
\end{minted}

