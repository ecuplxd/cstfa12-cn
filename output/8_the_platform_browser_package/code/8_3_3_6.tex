\begin{minted}{typescript}
  sanitize(
    ctx: SecurityContext,
    value: SafeValue | string | null
  ): string | null {
    if (value == null) return null;
    switch (ctx) {
      case SecurityContext.NONE:
        return value as string;
      case SecurityContext.HTML:
        if (value instanceof SafeHtmlImpl)
          return value.changingThisBreaksApplicationSecurity;
        this.checkNotSafeValue(value, 'HTML');
        return _sanitizeHtml(this._doc, String(value));
      case SecurityContext.STYLE:
        if (value instanceof SafeStyleImpl)
          return value.changingThisBreaksApplicationSecurity;
        this.checkNotSafeValue(value, 'Style');
        return _sanitizeStyle(value as string);
      case SecurityContext.SCRIPT:
        if (value instanceof SafeScriptImpl)
          return value.changingThisBreaksApplicationSecurity;
        this.checkNotSafeValue(value, 'Script');
        throw new Error('unsafe value used in a script context');
      case SecurityContext.URL:
        if (
          value instanceof SafeResourceUrlImpl ||
          value instanceof SafeUrlImpl
        ) {
          // Allow resource URLs in URL contexts,
          // they are strictly more trusted.
          return value.changingThisBreaksApplicationSecurity;
        }
        this.checkNotSafeValue(value, 'URL');
        return _sanitizeUrl(String(value));
      case SecurityContext.RESOURCE_URL:
        if (value instanceof SafeResourceUrlImpl) {
          return value.changingThisBreaksApplicationSecurity;
        }
        this.checkNotSafeValue(value, 'ResourceURL');
        throw new Error(
          'unsafe value used in a resource URL context (see http://g.co/ng/security#xss)'
        );
      default:
        throw new Error(
          `Unexpected SecurityContext ${ctx} (see http://g.co/ng/security#xss)`
        );
    }
  }
\end{minted}
