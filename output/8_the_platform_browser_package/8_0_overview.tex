\section{Overview}

A platform module is how an application interacts with its hosting environment. Duties
of a platform include rendering (deciding what is displayed and how), multitasking
(web workers), security sanitization of URLs/html/style (to detect dangerous
constructs) and location management (the URL displayed in the browser).

We have seen how the Core module provides a rendering API, but it includes no
implementations of renderers and no mention of the DOM. All other parts of Angular
that need to have content rendered talk to this rendering API and rely on an
implementation to actually deal with the content to be “displayed” (and what
“displayed” means varies depending on the platform). You will find an implementation
of renderer in the various platform packages – the main ones are defined
\url{hereinthePlatform-Browser}
package. Note that these render to a DOM adapter (and multiple of
those exist), but Core and all the features sitting on top of Core only know about the
rendering API, not the DOM.

Angular supplies five platform packages:

\begin{itemize}
  \item platform-browser (runs in the browser’s main UI thread and uses the offline template compiler),
  \item platform-browser-dynamic (runs in the browser’s main UI thread and uses the runtime template compiler),
  \item platform-webworker (runs in a web worker and uses the offline compiler),
  \item platform-webworker-dynamic (runs in a web worker and uses the runtime template compiler) and
  \item platform-server (runs in the server and can uses either the offline or runtime template compiler)
\end{itemize}

Shared functionality relating to platforms is in Platform-Browser and imported by the
other platform packages. So Platform-Browser is a much bigger packages compared
to the other platform packages. In this chapter we will explore Platform-Browser and
we will cover the others in the subsequent chapters.

Platform-Browser is how application code can interact with the browser when running
in the main UI thread and assumes the offline template compiler has been used to
pre-generate a module factory. For production use, platform-browser is likely to be
the platform of choice, as it results in the fastest display (no in-browser template
compilation needed) and smallest download size (the Angular template compiler does
not have to be downloaded to the browser).

The Platform-Browser package does not depend on the Compiler package since it
assumes compilation has occurred Ahead-Of-Time (AOT) using Compiler-CLI, the
command-line interface that wraps the Compiler package. In contrast, Platform-
Browser-Dynamic does import directly from the Compiler package – for example, see:

\begin{itemize}
  \item \href{https://github.com/angular/angular/tree/master/packages/platform-browser-dynamic/src/compiler_factory.ts}
        {<ANGULAR-MASTER>/packages/platform-browser-dynamic/src/compiler\_factory.ts}
\end{itemize}

There is exactly one platform instance per thread (main browser UI thread or web
worker). Multiple applications may run in the same thread, and they interact with the
same platform instance.
