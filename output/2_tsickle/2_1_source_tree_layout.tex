\section{Source Tree Layout}

The Tsickle source tree has these sub-directories:

\begin{itemize}
  \item src
  \item test
  \item test\_files
  \item third\_party
\end{itemize}

The main directory has these important files:

\begin{itemize}
  \item readme.md
  \item package.json
  \item gulpfile.js
  \item tsconfig.json
\end{itemize}

The readme.md contains useful information about the project, including this important
guidance about the use of tsconfig.json:

\emph{Tsickle works by wrapping tsc. To use it, you must set up your project such}
\emph{that it builds correctly when you run tsc from the command line, by}
\emph{configuring the settings in tsconfig.json.}

\emph{If you have complicated tsc command lines and flags in a build file (like a}
\emph{gulpfile etc.) Tsickle won't know about it. Another reason it's nice to put}
\emph{everything in tsconfig.json is so your editor inherits all these settings as}
\emph{well.}

The package.json file contains:

\begin{minted}{json}
  "main": "built/src/tsickle.js",
  "bin": "built/src/main.js",
\end{minted}


The gulpfile.js file contains the following Gulp tasks:

\begin{itemize}
  \item gulp format
  \item gulp test.check-format (formatting tests)
  \item gulp test.check-lint (run tslint)
\end{itemize}
