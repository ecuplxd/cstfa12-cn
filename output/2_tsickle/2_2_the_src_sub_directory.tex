\section{The src Sub-Directory}

The src sub-directory contains the following source files:

\begin{itemize}
  \item class\_decorator\_downlevel\_transformer.ts
  \item cli\_support.ts
  \item closure\_externs.js
  \item decorator-annotator.ts
  \item decorators.ts
  \item es5processor.ts
  \item fileoverview\_comment\_transformer.ts
  \item jsdoc.ts
  \item main.ts
  \item modules\_manifest.ts
  \item rewriter.ts
  \item source\_map\_utils.ts
  \item transformer\_sourcemap.ts
  \item transformer\_util.ts
  \item tsickle.ts
  \item type-translator.ts
  \item util.ts
\end{itemize}

main.ts is where the call to tsickle starts executing and tsickle.ts is where the core
logic is – the other files are helpers.

The entry point at the bottom of main.ts calls the main function passing in the
argument list as an array of strings.

\begin{minted}{ts}
function main(args: string[]): number {
1 const {settings, tscArgs} = loadSettingsFromArgs(args);
2 const config = loadTscConfig(tscArgs);
  if (config.errors.length) {
    console.error(tsickle.formatDiagnostics(config.errors));
    return 1;
  }

  if (config.options.module !== ts.ModuleKind.CommonJS) {
    // This is not an upstream TypeScript diagnostic, therefore it does not
    // go through the diagnostics array mechanism.
    console.error(
    'tsickle converts TypeScript modules to Closure modules via
        'CommonJS internally. Set tsconfig.js "module": "commonjs"');
    return 1;
  }

  // Run tsickle+TSC to convert inputs to Closure JS files.
3 const result = toClosureJS(
      config.options, config.fileNames, settings,
      (filePath: string, contents: string) => {
        mkdirp.sync(path.dirname(filePath));
4       fs.writeFileSync(filePath, contents, {encoding: 'utf-8'});
      });
  if (result.diagnostics.length) {
    console.error(tsickle.formatDiagnostics(result.diagnostics));
    return 1;
  }

5 if (settings.externsPath) {
    mkdirp.sync(path.dirname(settings.externsPath));
    fs.writeFileSync(settings.externsPath,
        tsickle.getGeneratedExterns(result.externs));
  }
  return 0;
}

// CLI entry point
if (require.main === module) {
  process.exit(main(process.argv.splice(2)));
}
\end{minted}


The main function first loads the settings
1
from the args and
2
the tsc config. Then it
calls the
\texttt{toClosureJs()}
function
3
, and outputs to a file
4
each resulting JavaScript
file. If
\texttt{externsPath}
is set in settings, they too are written out to files
5
.

The
\texttt{loadSettingsfromArgs()}
function handles the command-line arguments, which
can be a mix of tsickle-specific arguments and regular tsc arguments. The tsickle-
specific arguments are –externs (generate externs file) and –untyped (every
TypeScript type becomes a Closure {?} type).

The
\texttt{toClosureJs()}
function is where the transformation occurs. It returns
1
a map of
transformed file contents, optionally with externs information, it so configured.

\begin{minted}{typescript}
export function toClosureJS(
  options: ts.CompilerOptions,
  fileNames: string[],
  settings: Settings,
  writeFile?: ts.WriteFileCallback
): tsickle.EmitResult {
  %\step{1}% const compilerHost = ts.createCompilerHost(options);
  %\step{2}% const program = ts.createProgram(
    fileNames,
    options,
    compilerHost
  );

  %\step{3}% const transformerHost: tsickle.TsickleHost = {
    shouldSkipTsickleProcessing: (fileName: string) => {
      return fileNames.indexOf(fileName) === -1;
    },
    shouldIgnoreWarningsForPath: (fileName: string) => false,
    pathToModuleName: cliSupport.pathToModuleName,
    fileNameToModuleId: (fileName) => fileName,
    es5Mode: true,
    googmodule: true,
    prelude: '',
    transformDecorators: true,
    transformTypesToClosure: true,
    typeBlackListPaths: new Set(),
    untyped: false,
    logWarning: (warning) =>
      console.error(tsickle.formatDiagnostics([warning])),
  };
  const diagnostics = ts.getPreEmitDiagnostics(program);
  if (diagnostics.length > 0) {
    return {
      diagnostics,
      modulesManifest: new ModulesManifest(),
      externs: {},
      emitSkipped: true,
      emittedFiles: [],
    };
  }
  %\step{4}% return tsickle.emitWithTsickle(
    program,
    transformerHost,
    compilerHost,
    options,
    undefined,
    writeFile
  );
}
\end{minted}


It first creates
1
a compiler host based on the supplied options, then
2
it uses
TypeScript’s
\texttt{createProgram}
method with the original program source to ensure it is
syntatically correct and any error messages refer the original source, not the modified
source. Then it creates
\texttt{3}
a
\texttt{tsickle.TsickleHost}
instance which it passes
\texttt{4}
to
\texttt{tsickle.emitWithTsickle()}
.

The
\texttt{annotate}
function is a simple function:

\begin{minted}{typescript}
export function annotate(
  typeChecker: ts.TypeChecker,
  file: ts.SourceFile,
  host: AnnotatorHost,
  tsHost?: ts.ModuleResolutionHost,
  tsOpts?: ts.CompilerOptions,
  sourceMapper?: SourceMapper
): { output: string; diagnostics: ts.Diagnostic[] } {
  return new Annotator(
    typeChecker,
    file,
    host,
    tsHost,
    tsOpts,
    sourceMapper
  ).annotate();
}
\end{minted}


Classes called rewriters are used to rewrite the source. The rewriter.ts file has the
\texttt{Rewriter}
abstract class. An important method is
\texttt{maybeProcess()}
.

\begin{minted}{typescript}
/**
 * A Rewriter manages iterating through a ts.SourceFile, copying input
 * to output while letting the subclass potentially alter some nodes
 * along the way by implementing maybeProcess().
 */
export abstract class Rewriter {
  private output: string[] = [];
  /** Errors found while examining the code. */
  protected diagnostics: ts.Diagnostic[] = [];
  /** Current position in the output. */
  private position: SourcePosition = { line: 0, column: 0, position: 0 };
  /**
   * The current level of recursion through TypeScript Nodes.  Used in
formatting internal debug
   * print statements.
   */
  private indent = 0;
  /**
   * Skip emitting any code before the given offset.
   * E.g. used to avoid emitting @fileoverview
   * comments twice.
   */
  private skipCommentsUpToOffset = -1;
  constructor(
    public file: ts.SourceFile,
    private sourceMapper: SourceMapper = NOOP_SOURCE_MAPPER
  ) {}
  ..
}
\end{minted}


tsickle.ts has some classes that derive from
\texttt{Rewriter}
, according to this hierarchy:

\texttt{Annotator.maybeProcess()}
is where the actual rewriting occurs.
Annotator
ClosureRewriter
Rewriter
