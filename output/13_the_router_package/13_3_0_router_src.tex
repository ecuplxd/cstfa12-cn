\subsection{router/src}

The router/src directory contains:

\begin{itemize}
  \item apply\_redirects.ts
  \item config.ts
  \item create\_router\_state.ts
  \item create\_url\_tree.ts
  \item events.ts
  \item interfaces.ts
  \item index.ts
  \item interfaces.ts
  \item pre\_activation.ts
  \item private\_export.ts
  \item recognize.ts
  \item resolve.ts
  \item route\_reuse\_strategy.ts
  \item router\_config\_loader.ts
  \item router\_module.ts
  \item router\_outlet\_context.ts
  \item router\_preloader.ts
  \item router\_state.ts
  \item router.ts
  \item shared.ts
  \item url\_handling\_strategy.ts
  \item url\_tree.ts
  \item version.ts
\end{itemize}

We’ll start by looking at:

\begin{itemize}
  \item \href{https://github.com/angular/angular/blob/master/packages/router/src/router_module.ts}
        {<ANGULAR-MASTER>/packages/router/src/router\_module.ts}
\end{itemize}

which defines the RouterModule class and related types.

It defines three consts:

\begin{minted}{typescript}
const ROUTER_DIRECTIVES = [
  RouterOutlet,
  RouterLink,
  RouterLinkWithHref,
  RouterLinkActive,
];
\end{minted}


The first,
\texttt{ROUTER\_DIRECTIVES}
, is the collection of router directives that can appear in
Angular templates defining where the routed content is to be located on the page, and
how links used for routing are to be displayed.
\texttt{ROUTER\_DIRECTIVES}
is specified in the
declarations and exports of the
\texttt{@NgModule}
metadata for
\texttt{RouterModule}
:

\begin{minted}{typescript}
@NgModule({ declarations: ROUTER_DIRECTIVES, exports: ROUTER_DIRECTIVES })
export class RouterModule {
  ..
}
\end{minted}


The other two are injection tokens for DI:

\begin{minted}{typescript}
export const ROUTER_CONFIGURATION = new InjectionToken<ExtraOptions>(
  'ROUTER_CONFIGURATION'
);
export const ROUTER_FORROOT_GUARD = new InjectionToken<void>(
  'ROUTER_FORROOT_GUARD'
);
\end{minted}


It also defines the
\texttt{ROUTER\_PROVIDERS}
array of providers (which is only used by
forRoot, not forChild):

\begin{minted}{typescript}
export const ROUTER_PROVIDERS: Provider[] = [
  Location,
  { provide: UrlSerializer, useClass: DefaultUrlSerializer },
  {
    provide: Router,
    useFactory: setupRouter, ..
  },
  ChildrenOutletContexts,
  { provide: ActivatedRoute, useFactory: rootRoute, deps: [Router] },
  { provide: NgModuleFactoryLoader, useClass: SystemJsNgModuleLoader },
  RouterPreloader,
  NoPreloading,
  PreloadAllModules,
  { provide: ROUTER_CONFIGURATION, useValue: { enableTracing: false } },
];
\end{minted}


An important provider there is
\texttt{Router}
, which is the actually routing service. This is set
up in DI to return the result of the
\texttt{setupRouter}
factory method. An abbreviated
version of this is as follows:

\begin{minted}{typescript}
export function setupRouter(..) {
  const router = new Router(
    null,
    urlSerializer,
    contexts,
    location,
    injector,
    loader,
    compiler,
    flatten(config)
  );

  if (urlHandlingStrategy)
  if (routeReuseStrategy) ..
  if (opts.errorHandler) ..
  if (opts.enableTracing) ..
  if (opts.onSameUrlNavigation) ..
  if (opts.paramsInheritanceStrategy) ..
  ..
  return router;
}
\end{minted}


It instantiates the router service and for each specified option / strategy takes
appropciarte action. Then it returns the new router service instance. It is important
that there is only a single router service per application the web browser only have a
single URL per session) and we need to track how this is so.

RouterModule is defined as:

\begin{minted}{typescript}
@NgModule({ declarations: ROUTER_DIRECTIVES, exports: ROUTER_DIRECTIVES })
export class RouterModule {
  // Note: We are injecting the Router so it gets created eagerly...
  constructor(
    @Optional() @Inject(ROUTER_FORROOT_GUARD) guard: any,
    @Optional() router: Router
  ) {}
  static forRoot(routes: Routes, config?: ExtraOptions): ModuleWithProviders {
    ..
  }
  static forChild(routes: Routes): ModuleWithProviders {
    ..
  }
}
\end{minted}


Note that both
\texttt{forRoot}
and
\texttt{forChild}
return a
\texttt{ModuleWithProviders}
instance. What
they put in it is different. Recall that this type is defined in:

\begin{itemize}
  \item \href{https://github.com/angular/angular/blob/master/packages/core/src/metadata/ng_module.ts}
        {<ANGULAR-MASTER>/packages/core/src/metadata/ng\_module.ts}
\end{itemize}

as follows:

\begin{minted}{typescript}
// wrapper around a module that also includes the providers.
export interface ModuleWithProviders {
  ngModule: Type<any>;
  providers?: Provider[];
}
\end{minted}


\texttt{ForChild}
is intended for all except the root routing module. It returns
\texttt{ngModule}
and a
list of providers that only contains the result of calling
\texttt{provideRoutes}
:

\begin{minted}{typescript}
  static forChild(routes: Routes): ModuleWithProviders {
    return { ngModule: RouterModule, providers: [provideRoutes(routes)] };
  }
\end{minted}


Critically, it does not contain ROUTER\_PROVIDERS. In contrast,
\texttt{forRoot}
adds this and
many more providers:

\begin{minted}{typescript}
  static forRoot(routes: Routes, config?: ExtraOptions): ModuleWithProviders {
    return {
      ngModule: RouterModule,
      providers: [
        ROUTER_PROVIDERS,
        provideRoutes(routes),
        {
          provide: ROUTER_FORROOT_GUARD, ..
        },
        {
          provide: ROUTER_CONFIGURATION, ..
        },
        {
          provide: LocationStrategy, ..
        },
        {
          provide: PreloadingStrategy, ..
        },
        {
          provide: NgProbeToken, ..
        },
        provideRouterInitializer(),
      ],
    };
  }
\end{minted}


\texttt{ExtraOptions}
are additional options passed in to
\texttt{forRoot}
(it is not used with
\texttt{forChild}
):

\begin{minted}{typescript}
export interface ExtraOptions {
  enableTracing?: boolean;
  useHash?: boolean;
  initialNavigation?: InitialNavigation;
  errorHandler?: ErrorHandler;
  preloadingStrategy?: any;
  onSameUrlNavigation?: 'reload' | 'ignore';
  paramsInheritanceStrategy?: 'emptyOnly' | 'always';
}
\end{minted}


For example, if we wished to customize how preloading worked, we need to set the
\texttt{preloadingStrategy}
option.

The
\texttt{provideRouterInitializer()}
function providers a list of initializers:

\begin{minted}{typescript}
export function provideRouterInitializer() {
  return [
    RouterInitializer,
    {
      provide: APP_INITIALIZER,
      multi: true,
      useFactory: getAppInitializer,
      deps: [RouterInitializer],
    },
    {
      provide: ROUTER_INITIALIZER,
      useFactory: getBootstrapListener,
      deps: [RouterInitializer],
    },
    {
      provide: APP_BOOTSTRAP_LISTENER,
      multi: true,
      useExisting: ROUTER_INITIALIZER,
    },
  ];
}
\end{minted}


This uses the
\texttt{RouterInitializer}
class, whose purpose is best explained by this
comment in the code:

\begin{minted}{typescript}
/**
 * To initialize the router properly we need to do in two steps:
 *
 * We need to start the navigation in a APP_INITIALIZER to block the
 * bootstrap if a resolver or a guards executes asynchronously. Second, we
 * need to actually run activation in a BOOTSTRAP_LISTENER. We utilize the
 * afterPreactivation hook provided by the router to do that.
 *
 * The router navigation starts, reaches the point when preactivation is
 * done, and then pauses. It waits for the hook to be resolved. We then
 * resolve it only in a bootstrap listener.
 */
@Injectable()
export class RouterInitializer {
  ..
}
\end{minted}


We saw when examining the Core package that its:

\begin{itemize}
  \item \href{https://github.com/angular/angular/blob/master/packages/core/src/application_init.ts}
        {<ANGULAR\_MASTER>/packages/core/src/application\_init.ts}
\end{itemize}

defines
\texttt{APP\_INITIALIZER}
as:

\begin{minted}{typescript}
// A function that will be executed when an application is initialized.
export const APP_INITIALIZER = new InjectionToken<Array<() => void>>(
  'Application Initializer'
);
\end{minted}


Interfaces.ts declares a number of useful interfaces.

\begin{minted}{typescript}
export interface CanActivate {
  canActivate(
    route: ActivatedRouteSnapshot,
    state: RouterStateSnapshot
  ): Observable<boolean> | Promise<boolean> | boolean;
}
export interface CanActivateChild {
  canActivateChild(
    childRoute: ActivatedRouteSnapshot,
    state: RouterStateSnapshot
  ): Observable<boolean> | Promise<boolean> | boolean;
}
export interface CanDeactivate<T> {
  canDeactivate(
    component: T,
    currentRoute: ActivatedRouteSnapshot,
    currentState: RouterStateSnapshot,
    nextState?: RouterStateSnapshot
  ): Observable<boolean> | Promise<boolean> | boolean;
}
export interface Resolve<T> {
  resolve(
    route: ActivatedRouteSnapshot,
    state: RouterStateSnapshot
  ): Observable<T> | Promise<T> | T;
}
export interface CanLoad {
  canLoad(route: Route): Observable<boolean> | Promise<boolean> | boolean;
}
\end{minted}


router\_config\_loader.ts defines a class -
\texttt{RouterConfigLoader}
– and an opaque token,
\texttt{ROUTES}
.  After some bookkeeping,
\texttt{RouterConfigLoader}
creates a new instance of
LoadedRouterConfig it:

\begin{minted}{typescript}
export const ROUTES = new InjectionToken<Route[][]>('ROUTES');

export class RouterConfigLoader {
  constructor(
    private loader: NgModuleFactoryLoader,
    private compiler: Compiler,
    private onLoadStartListener?: (r: Route) => void,
    private onLoadEndListener?: (r: Route) => void
  ) {}

  load(parentInjector: Injector, route: Route): Observable<LoadedRouterConfig> {
    ..
    return new LoadedRouterConfig(flatten(module.injector.get(ROUTES)), module);
  }
  ..
}
\end{minted}


The router\_state.ts file contains these classes (and some helper functions):

\begin{itemize}
  \item RouterState
  \item ActivatedRoute
  \item ActivatedRouteSnapshot
  \item RouterStateSnapshot
\end{itemize}

\texttt{RouterState}
is defined as:

\begin{minted}{typescript}
// RouterState is a tree of activated routes.
// Every node in this tree knows about the "consumed" URL
// segments, the extracted parameters, and the resolved data.
export class RouterState extends Tree<ActivatedRoute> {
  constructor(
    root: TreeNode<ActivatedRoute>,
    public snapshot: RouterStateSnapshot
  ) {
    super(root);
    setRouterState(<RouterState>this, root);
  }
}
\end{minted}


\texttt{RouterStateSnapshot}
is defined as:

\begin{minted}{typescript}
/**
 * @whatItDoes Represents the state of the router at a moment in time.
 * RouterStateSnapshot is a tree of activated route snapshots. Every node in
 * this tree knows about the "consumed" URL segments, the extracted
 * parameters, and the resolved data.
 */
export class RouterStateSnapshot extends Tree<ActivatedRouteSnapshot> {
  constructor(public url: string, root: TreeNode<ActivatedRouteSnapshot>) {
    super(root);
    setRouterState(<RouterStateSnapshot>this, root);
  }
}
\end{minted}


The
\texttt{setRouterStateSnapshot()}
function is defined as:

\begin{minted}{typescript}
function setRouterState<U, T extends { _routerState: U }>(
  state: U,
  node: TreeNode<T>
): void {
  node.value._routerState = state;
  node.children.forEach((c) => setRouterState(state, c));
}
\end{minted}


So its sets the router state for the current node, and then recursively calls
\texttt{setRouterState()}
to set it for all children.

The
\texttt{ActivatedRoute}
class is used by the router outlet directive to describe the
component it has loaded:

\begin{minted}{typescript}
// Contains the information about a route associated with a component loaded
// in an outlet. An `ActivatedRoute` can also be used to traverse the
// router state tree
export class ActivatedRoute {
  ..
  constructor() { ..
    this._futureSnapshot = futureSnapshot;
  }

  /** The configuration used to match this route */
  get routeConfig(): Route | null {
    return this._futureSnapshot.routeConfig;
  }

  /** The root of the router state */
  get root(): ActivatedRoute {
    return this._routerState.root;
  }

  /** The parent of this route in the router state tree */
  get parent(): ActivatedRoute | null {
    return this._routerState.parent(this);
  }

  /** The first child of this route in the router state tree */
  get firstChild(): ActivatedRoute | null {
    return this._routerState.firstChild(this);
  }

  /** The children of this route in the router state tree */
  get children(): ActivatedRoute[] {
    return this._routerState.children(this);
  }

  /** The path from the root of the router state tree to this route */
  get pathFromRoot(): ActivatedRoute[] {
    return this._routerState.pathFromRoot(this);
  }

  get paramMap(): Observable<ParamMap> {
    ..
  }
  get queryParamMap(): Observable<ParamMap> {
    ..
  }
}
\end{minted}

