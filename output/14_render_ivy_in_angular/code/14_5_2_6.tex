\begin{minted}{typescript}
/**
 * Procedural style of API needed to create elements and text nodes.
 *
 * In non-native browser environments (e.g. platforms such as web-workers),
 * this is the facade that enables element manipulation. This also
 * facilitates backwards compatibility with Renderer2.
 */
export interface ProceduralRenderer3 {
  destroy(): void;
  createElement(name: string, namespace?: string | null): RElement;
  createText(value: string): RText;
  destroyNode?: ((node: RNode) => void) | null;
  appendChild(parent: RElement, newChild: RNode): void;
  insertBefore(parent: RNode, newChild: RNode, refChild: RNode | null): void;
  removeChild(parent: RElement, oldChild: RNode): void;
  selectRootElement(selectorOrNode: string | any): RElement;
  setAttribute(
    el: RElement,
    name: string,
    value: string,
    namespace?: string | null
  ): void;
  removeAttribute(el: RElement, name: string, namespace?: string | null): void;
  addClass(el: RElement, name: string): void;
  removeClass(el: RElement, name: string): void;
  setStyle(
    el: RElement,
    style: string,
    value: any,
    flags?: RendererStyleFlags2 | RendererStyleFlags3
  ): void;
  removeStyle(
    el: RElement,
    style: string,
    flags?: RendererStyleFlags2 | RendererStyleFlags3
  ): void;
  setProperty(el: RElement, name: string, value: any): void;
  setValue(node: RText, value: string): void;
  listen(
    target: RNode,
    eventName: string,
    callback: (event: any) => boolean | void
  ): () => void;
}
\end{minted}
