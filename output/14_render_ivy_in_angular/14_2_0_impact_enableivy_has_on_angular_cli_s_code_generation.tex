\subsection{Impact enableIvy has on Angular CLI’s code generation}

To best see the differences the
\texttt{enableIvy}
option has on code generation, we will now
create two new projects – one with and one without the
\texttt{--enableIvy}
option. To save
us some time we will use the
\texttt{–-skipInstall}
option, which means npm install is not
run to download all the dependency packages.

\begin{minted}{ts}
ng new render3 --enableIvy --skipInstall
ng new render2 --skipInstall
\end{minted}


A search for
\texttt{ivy}
in the generated render2 codebase reveals no hits, as expected. A
search for
\texttt{ivy}
in the render3 codebase reveals 2 hits. In package.json,
\texttt{scripts}
has
this additional item:

\begin{minted}{json}
  "scripts": {
    "postinstall": "ivy-ngcc",
    ..
  },
\end{minted}


and in tsconfig.app.json,
\texttt{angularCompilerOptions}
has this entry:

\begin{minted}{json}
  "angularCompilerOptions": {
    "enableIvy": true
  }
\end{minted}


Now that we have seen how Angular CLI adds enableIvy, we are ready to move on to
explore how Compiler CLI detects and reacts to this.
