\subsection{Interfaces}

When trying to figure out how Render3 works, a good place to start is with its
interfaces. Let’s look at this first:

\begin{itemize}
  \item \href{https://github.com/angular/angular/blob/master/packages/core/src/render3/interfaces/renderer.ts}
        {<ANGULAR-MASTER>/packages/core/src/render3/interfaces/renderer.ts}
\end{itemize}

There are some simple helper interfaces describing a node, an element and a text
node. The node is defined as have three methods to insert, append and remove a
child:

\begin{minted}{typescript}
/** Subset of API needed for appending elements and text nodes. */
export interface RNode {
  removeChild(oldChild: RNode): void;

  // Insert a child node.
  // Used exclusively for adding View root nodes into ViewAnchor location.
  insertBefore(
    newChild: RNode,
    refChild: RNode | null,
    isViewRoot: boolean
  ): void;

  //Append a child node.
  //Used exclusively for building up DOM which are static (ie not View roots)
  appendChild(newChild: RNode): RNode;
}
\end{minted}


The element allows adding and removing of listeners, working with attributes and
properties and style configuration – it is defined as:

\begin{minted}{typescript}
/**
 * Subset of API needed for writing attributes, properties, and setting up
 * listeners on Element.
 */
export interface RElement extends RNode {
  style: RCssStyleDeclaration;
  classList: RDomTokenList;
  setAttribute(name: string, value: string): void;
  removeAttribute(name: string): void;
  setAttributeNS(
    namespaceURI: string,
    qualifiedName: string,
    value: string
  ): void;
  addEventListener(
    type: string,
    listener: EventListener,
    useCapture?: boolean
  ): void;
  removeEventListener(
    type: string,
    listener?: EventListener,
    options?: boolean
  ): void;
  setProperty?(name: string, value: any): void;
}
\end{minted}


The text node adds a
\texttt{textContent}
property:

\begin{minted}{typescript}
export interface RText extends RNode {
  textContent: string | null;
}
\end{minted}


It has this factory code. Note the return type from
\texttt{createRenderer}
is
\texttt{Render3}
1
–
and that for the
\texttt{domRendererFactory3}
implementation
2
this is the normal DOM
\texttt{document}
:

\begin{minted}{typescript}
export interface RendererFactory3 {
  createRenderer(
    hostElement: RElement | null,
    rendererType: RendererType2 | null
  ): Renderer3 %\step{1}%;
  begin?(): void;
  end?(): void;
}
export const domRendererFactory3: RendererFactory3 = {
  createRenderer: (
    hostElement: RElement | null,
    rendererType: RendererType2 | null
  ): Renderer3 => {
    return document;
  } %\step{2}%,
};
\end{minted}


This is key to moving back to regular DOM usage for code that runs in the main
browser UI thread, and yet allowing alternatives elsewhere.
\texttt{Renderer3}
is a type alias:

\begin{minted}{typescript}
export type Renderer3 = ObjectOrientedRenderer3 | ProceduralRenderer3;
\end{minted}


This represents the two kinds of renderers that are supported. The bolded text in the
comment highlights the usage scenario for the first of these:

\begin{minted}{typescript}
/**
 * Object Oriented style of API needed to create elements and text nodes.
 *
 * This is the native browser API style, e.g. operations are methods on
 * individual objects like HTMLElement. With this style, no additional
 * code is needed as a facade (reducing payload size).
 */
export interface ObjectOrientedRenderer3 {
  createElement(tagName: string): RElement;
  createTextNode(data: string): RText;
  querySelector(selectors: string): RElement | null;
}
\end{minted}


\texttt{ProceduralRender3}
is intended to be used from web workers and server-side:

\begin{minted}{typescript}
/**
 * Procedural style of API needed to create elements and text nodes.
 *
 * In non-native browser environments (e.g. platforms such as web-workers),
 * this is the facade that enables element manipulation. This also
 * facilitates backwards compatibility with Renderer2.
 */
export interface ProceduralRenderer3 {
  destroy(): void;
  createElement(name: string, namespace?: string | null): RElement;
  createText(value: string): RText;
  destroyNode?: ((node: RNode) => void) | null;
  appendChild(parent: RElement, newChild: RNode): void;
  insertBefore(parent: RNode, newChild: RNode, refChild: RNode | null): void;
  removeChild(parent: RElement, oldChild: RNode): void;
  selectRootElement(selectorOrNode: string | any): RElement;
  setAttribute(
    el: RElement,
    name: string,
    value: string,
    namespace?: string | null
  ): void;
  removeAttribute(el: RElement, name: string, namespace?: string | null): void;
  addClass(el: RElement, name: string): void;
  removeClass(el: RElement, name: string): void;
  setStyle(
    el: RElement,
    style: string,
    value: any,
    flags?: RendererStyleFlags2 | RendererStyleFlags3
  ): void;
  removeStyle(
    el: RElement,
    style: string,
    flags?: RendererStyleFlags2 | RendererStyleFlags3
  ): void;
  setProperty(el: RElement, name: string, value: any): void;
  setValue(node: RText, value: string): void;
  listen(
    target: RNode,
    eventName: string,
    callback: (event: any) => boolean | void
  ): () => void;
}
\end{minted}


Let’s now look at
\url{view.ts}
. It includes the following to work with static data:

\begin{minted}{typescript}
// The static data for an LView (shared between all templates of a
// given type). Stored on the template function as ngPrivateData.
export interface TView {
  data: TData;
  firstTemplatePass: boolean;
  initHooks: HookData | null;
  checkHooks: HookData | null;
  contentHooks: HookData | null;
  contentCheckHooks: HookData | null;
  viewHooks: HookData | null;
  viewCheckHooks: HookData | null;
  destroyHooks: HookData | null;
  objectLiterals: any[] | null;
}
\end{minted}


\texttt{TData}
is defined as:

\begin{minted}{typescript}
/**
 * Static data that corresponds to the instance-specific data array on an
 * Lview. Each node's static data is stored in tData at the same index that
 * it's stored in the data array. Each directive's definition is stored here
 * at the same index as its directive instance in the data array. Any nodes
 * that do not have static data store a null value in tData to avoid a
 * sparse array.
 */
export type TData = (TNode | DirectiveDef<any> | null)[];
\end{minted}


A tree of
\texttt{LView}
s or
\texttt{LContainer}
s will be needed, so this type is a node in the
hierarchy:

\begin{minted}{typescript}
/** Interface necessary to work with view tree traversal */
export interface LViewOrLContainer {
  next: LView | LContainer | null;
  child?: LView | LContainer | null;
  views?: LViewNode[];
  parent: LView | null;
}
\end{minted}


An LView stores info relating to processing a view’s instructions. Detailed comments
(not shown here) for each of its fields are in
\url{thesource}
.

\begin{minted}{typescript}
/**
 * `LView` stores all of the information needed to process the instructions
 * as they are invoked from the template. Each embedded view and component
 * view has its own `LView`. When processing a particular view, we set the
 * `currentView` to that `LView`. When that view is done processing, the
 * `currentView` is set back to whatever the original `currentView` was
 * before (the parent `LView`).
 * Keeping separate state for each view facilities view insertion / deletion,
 * so we don't have to edit the data array based on which views are present.
 */
export interface LView {
  creationMode: boolean;
  readonly parent: LView | null;
  readonly node: LViewNode | LElementNode;
  readonly id: number;
  readonly renderer: Renderer3;
  bindingStartIndex: number | null;
  cleanup: any[] | null;
  lifecycleStage: LifecycleStage;
  child: LView | LContainer | null;
  tail: LView | LContainer | null;
  next: LView | LContainer | null;
  readonly data: any[];
  tView: TView;
  template: ComponentTemplate<{}> | null;
  context: {} | null;
  dynamicViewCount: number;
  queries: LQueries | null;
}
\end{minted}


The
\url{container.ts}
file looks at containers, which are collections of views and sub-
containers. It exports one type,
\texttt{TContainer}
:

\begin{minted}{typescript}
/**
 * The static equivalent of LContainer, used in TContainerNode.
 *
 * The container needs to store static data for each of its embedded views
 * (TViews). Otherwise, nodes in embedded views with the same index as nodes
 * in their parent views will overwrite each other, as they are in
 * the same template.
 *
 * Each index in this array corresponds to the static data for a certain
 * view. So if you had V(0) and V(1) in a container, you might have:
 *
 * [
 *   [{tagName: 'div', attrs: ...}, null],     // V(0) TView
 *   [{tagName: 'button', attrs ...}, null]    // V(1) TView
 * ]
 */
export type TContainer = TView[];
\end{minted}


and one interface,
\texttt{LContainer}
: (note
\url{thesourcefile}
contained detailed comments for
each field):

\begin{minted}{typescript}
/** The state associated with an LContainer */
export interface LContainer {
  nextIndex: number;
  next: LView | LContainer | null;
  parent: LView | null;
  readonly views: LViewNode[];
  renderParent: LElementNode | null;
  readonly template: ComponentTemplate<any> | null;
  dynamicViewCount: number;
  queries: LQueries | null;
}
\end{minted}


The
\url{query.ts}
file contains the
\texttt{QueryReadType}
class:

\begin{minted}{typescript}
export class QueryReadType<T> {
  private defeatStructuralTyping: any;
}
\end{minted}


and the
\texttt{LQuery}
interface:

\begin{minted}{typescript}
/** Used for tracking queries (e.g. ViewChild, ContentChild). */
export interface LQueries {
  child(): LQueries | null;
  addNode(node: LNode): void;
  container(): LQueries | null;
  enterView(newViewIndex: number): LQueries | null;
  removeView(removeIndex: number): void;
  track<T>(
    queryList: QueryList<T>,
    predicate: Type<any> | string[],
    descend?: boolean,
    read?: QueryReadType<T> | Type<T>
  ): void;
}
\end{minted}


The
\url{projection.ts}
file define LProjection as:

\begin{minted}{typescript}
// Linked list of projected nodes (using the pNextOrParent property).
export interface LProjection {
  head: LElementNode | LTextNode | LContainerNode | null;
  tail: LElementNode | LTextNode | LContainerNode | null;
}
\end{minted}


The injector.ts file has this:

\begin{minted}{typescript}
export interface LInjector {
  readonly parent: LInjector | null;
  readonly node: LElementNode | LContainerNode;
  bf0: number;
  bf1: number;
  bf2: number;
  bf3: number;
  cbf0: number;
  cbf1: number;
  cbf2: number;
  cbf3: number;
  injector: Injector | null;
  /** Stores the TemplateRef so subsequent injections of the TemplateRef get
the same instance. */
  templateRef: TemplateRef<any> | null;
  /** Stores the ViewContainerRef so subsequent injections of the
ViewContainerRef get the same
   * instance. */
  viewContainerRef: ViewContainerRef | null;
  /** Stores the ElementRef so subsequent injections of the ElementRef get
the same instance. */
  elementRef: ElementRef | null;
}
\end{minted}


The node.ts file is large and contains a hierarchy of node-related types.

The root,
\texttt{LNode}
, is defined (abbreviated) as:

\begin{minted}{typescript}
/**
 * LNode is an internal data structure which is used for the incremental DOM
 * algorithm. The "L" stands for "Logical" to differentiate between `RNodes`
 * (actual rendered DOM node) and our logical representation of DOM nodes,
 * `Lnodes`. The data structure is optimized for speed and size.
 *
 * In order to be fast, all subtypes of `LNode` should have the same shape.
 * Because size of the `LNode` matters, many fields have multiple roles
 * depending on the `LNode` subtype.
 */
export interface LNode {
  flags: LNodeFlags;
  readonly native: RElement | RText | null | undefined;
  readonly parent: LNode | null;
  child: LNode | null;
  next: LNode | null;
  readonly data: LView | LContainer | LProjection | null;
  readonly view: LView;
  nodeInjector: LInjector | null;
  queries: LQueries | null;
  pNextOrParent: LNode | null;
  tNode: TNode | null;
}
\end{minted}


Each of the other types adds a few additional fields to represent that node type.

Render3 as  a View Processing Unit (VPU)
