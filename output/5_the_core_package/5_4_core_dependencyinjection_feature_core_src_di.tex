\section{Core/DependencyInjection Feature (core/src/di)}

The core/src/di.ts source file exports a variety of dependency injection types:

\begin{minted}{typescript}
export * from './di/metadata';
export { forwardRef, resolveForwardRef, ForwardRefFn } from './di/forward_ref';
export { Injector } from './di/injector';
export { ReflectiveInjector } from './di/reflective_injector';
export {
  StaticProvider,
  ValueProvider,
  ExistingProvider,
  FactoryProvider,
  Provider,
  TypeProvider,
  ClassProvider,
} from './di/provider';
export {
  ResolvedReflectiveFactory,
  ResolvedReflectiveProvider,
} from './di/reflective_provider';
export { ReflectiveKey } from './di/reflective_key';
export { InjectionToken } from './di/injection_token';
\end{minted}


The core/src/di directory contains these files:

\begin{itemize}
  \item forward\_ref.ts
  \item injection\_token.ts
  \item injector.ts
  \item metadata.ts
  \item opaque\_token.ts
  \item provider.ts
  \item reflective\_errors.ts
  \item reflective\_injector.ts
  \item reflective\_key.ts
  \item reflective\_provider.ts
\end{itemize}

The metadata.ts file defines these interfaces:

\begin{minted}{typescript}
export interface InjectDecorator {
  (token: any): any;
  new (token: any): Inject;
}
export interface Inject {
  token: any;
}
export interface OptionalDecorator {
  (): any;
  new (): Optional;
}
export interface Optional {}
export interface InjectableDecorator {
  (): any;
  new (): Injectable;
}
export interface Injectable {}
export interface SelfDecorator {
  (): any;
  new (): Self;
}
export interface Self {}
export interface SkipSelfDecorator {
  (): any;
  new (): SkipSelf;
}
export interface SkipSelf {}
export interface HostDecorator {
  (): any;
  new (): Host;
}
export interface Host {}
\end{minted}


metadata.ts also defines these variables:

\begin{minted}{typescript}
export const Self: SelfDecorator = makeParamDecorator('Self');
export const SkipSelf: SkipSelfDecorator = makeParamDecorator('SkipSelf');
export const Inject: InjectDecorator = makeParamDecorator(
  'Inject',
  (token: any) => ({ token })
);
export const Optional: OptionalDecorator = makeParamDecorator('Optional');
export const Injectable: InjectableDecorator = makeDecorator('Injectable');
export const Host: HostDecorator = makeParamDecorator('Host');
\end{minted}


Forward refs are placeholders used to faciliate out-of-sequence type declarations. The
forward\_ref.ts file defines an interface and two functions:

\begin{minted}{typescript}
export interface ForwardRefFn {
  (): any;
}

export function forwardRef(forwardRefFn: ForwardRefFn): Type<any> {
  (<any>forwardRefFn).__forward_ref__ = forwardRef;
  (<any>forwardRefFn).toString = function () {
    return stringify(this());
  };
  return <Type<any>>(<any>forwardRefFn);
}

export function resolveForwardRef(type: any): any {
  if (
    typeof type === 'function' &&
    type.hasOwnProperty('__forward_ref__') &&
    type.__forward_ref__ === forwardRef
  ) {
    return (<ForwardRefFn>type)();
  } else {
    return type;
  }
}
\end{minted}


The injector.ts file defines the
\texttt{Injector}
abstract class:

\begin{minted}{typescript}
export abstract class Injector {
  static THROW_IF_NOT_FOUND = _THROW_IF_NOT_FOUND;
  static NULL: Injector = new _NullInjector();
  abstract get<T>(token: Type<T> | InjectionToken<T>, notFoundValue?: T): T;
  abstract get(token: any, notFoundValue?: any): any;
  static create(providers: StaticProvider[], parent?: Injector): Injector {
    return new StaticInjector(providers, parent);
  }
}
\end{minted}


Application code (and indeed, Angular internal code) passes a token to
\texttt{Injector.get()}
, and the implementation returns the matching instance. Concrete
implementations of this class need to override the
\texttt{get()}
method, so it actually works
as expected. See reflective\_injector.ts for a derived class.

provider.ts defines a number of
\texttt{Provider}
classes and then uses them to define the
\texttt{Provider}
type:

\begin{minted}{typescript}
export type Provider =
  | TypeProvider
  | ValueProvider
  | ClassProvider
  | ExistingProvider
  | FactoryProvider
  | any[];
\end{minted}

