\subsection{Core/View Implementation}

The source file:

\begin{itemize}
  \item \href{https://github.com/angular/angular/blob/master/packages/core/src/view/types.ts}
        {<ANGULAR-MASTER>/packages/core/src/view/types.ts}
\end{itemize}

defines helper types used elsewhere in Core/View. Let’s start with
\texttt{ViewContainerData}
which simply extends
\texttt{ViewContainerRef}
and adds an internal
property to record view data for embedded views:

\begin{minted}{typescript}
export interface ViewContainerData extends ViewContainerRef {
  _embeddedViews: ViewData[];
}
\end{minted}


The
\texttt{TemplateData}
interface extends
\texttt{TemplateRef}
and just adds an array of
\texttt{ViewData}
instances:

\begin{minted}{typescript}
export interface TemplateData extends TemplateRef<any> {
  // views that have been created from the template
  // of this element, but inserted into the embeddedViews of
  // another element. By default, this is undefined.
  _projectedViews: ViewData[];
}
\end{minted}


The source file:

\begin{itemize}
  \item \href{https://github.com/angular/angular/blob/master/packages/core/src/view/refs.ts}
        {<ANGULAR-MASTER>/packages/core/src/view/refs.ts}
\end{itemize}

provides internal implementations of references (refs).

\texttt{ViewContainerRef\_}
(note the trailing \_) implements
\texttt{ViewContainerData}
which in
turn implements the public
\texttt{ViewContainerRef}
.

It takes an
\texttt{ElementData}
for its anchor element in its constructor:

\begin{minted}{typescript}
class ViewContainerRef_ implements ViewContainerData {
  _embeddedViews: ViewData[] = [];
  constructor(
    private _view: ViewData,
    private _elDef: NodeDef,
    private _data: ElementData
  ) {}
  get element(): ElementRef {
    return new ElementRef(this._data.renderElement);
  }
  get injector(): Injector {
    return new Injector_(this._view, this._elDef);
  }
  get parentInjector(): Injector {
    ..
  }
  get(index: number): ViewRef | null {
    ..
  }
  get length(): number {
    return this._embeddedViews.length;
  }
  createEmbeddedView<C>(
    templateRef: TemplateRef<C>,
    context?: C,
    index?: number
  ): EmbeddedViewRef<C> {
    ..
  }
  createComponent<C>(
    componentFactory: ComponentFactory<C>,
    index?: number,
    injector?: Injector,
    projectableNodes?: any[][],
    ngModuleRef?: NgModuleRef<any>
  ): ComponentRef<C> {
    ..
  }
  insert(viewRef: ViewRef, index?: number): ViewRef {
    ..
  }
  move(viewRef: ViewRef_, currentIndex: number): ViewRef {
    ..
  }
  clear(): void {
    ..
  }
}
\end{minted}


The two create methods are implemented as follows:

\begin{minted}{typescript}
  createEmbeddedView<C>(
    templateRef: TemplateRef<C>,
    context?: C,
    index?: number
  ): EmbeddedViewRef<C> {
    %\step{1}% const viewRef = templateRef.createEmbeddedView(
      context || <any>{}
    );
    %\step{2}% this.insert(viewRef, index);
    return viewRef;
  }

  createComponent<C>(
    componentFactory: ComponentFactory<C>,
    index?: number,
    injector?: Injector,
    projectableNodes?: any[][],
    ngModuleRef?: NgModuleRef<any>
  ): ComponentRef<C> {
    const contextInjector = injector || this.parentInjector;
    if (
      !ngModuleRef &&
      !(componentFactory instanceof ComponentFactoryBoundToModule)
    ) {
      ngModuleRef = contextInjector.get(NgModuleRef);
    }
    const componentRef = %\step{3}% componentFactory.create(
      contextInjector,
      projectableNodes,
      undefined,
      ngModuleRef
    );
    %\step{4}% this.insert(componentRef.hostView, index);
    return componentRef;
  }
\end{minted}


We see the difference between them –
\texttt{createEmbeddedView()}
calls
1
the
\texttt{TemplateRef}
’s
\texttt{createEmbeddedView()}
method and inserts
2
the resulting viewRef;
whereas
\texttt{createComponent()}
calls
3
the component factory’s
\texttt{create}
method, and
with the resulting
\texttt{ComponentRef}
, inserts
4
its
\texttt{HostView}
. Note the return type is
different for each create method – the first returns an
\texttt{EmbededViewRef}
whereas the
second returns a
\texttt{ComponentRef}
.

The implementations of
\texttt{insert}
,
\texttt{indexOf}
,
\texttt{remove}
and
\texttt{detach}
result in use of
appropriate view management APIs:

\begin{minted}{typescript}
  insert(viewRef: ViewRef, index?: number): ViewRef {
    const viewRef_ = <ViewRef_>viewRef;
    const viewData = viewRef_._view;
    attachEmbeddedView(this._view, this._data, index, viewData);
    viewRef_.attachToViewContainerRef(this);
    return viewRef;
  }
  move(viewRef: ViewRef_, currentIndex: number): ViewRef {
    const previousIndex = this._embeddedViews.indexOf(viewRef._view);
    moveEmbeddedView(this._data, previousIndex, currentIndex);
    return viewRef;
  }
  indexOf(viewRef: ViewRef): number {
    return this._embeddedViews.indexOf((<ViewRef_>viewRef)._view);
  }
  remove(index?: number): void {
    const viewData = detachEmbeddedView(this._data, index);
    if (viewData) {
      Services.destroyView(viewData);
    }
  }
  detach(index?: number): ViewRef | null {
    const view = detachEmbeddedView(this._data, index);
    return view ? new ViewRef_(view) : null;
  }
\end{minted}


The
\texttt{ComponentRef\_}
concrete class extends
\texttt{ComponentRef}
. Its constructor takes in an
\texttt{ViewRef}
, which is used in the
\texttt{destroy}
method and to set the component’s change
detector ref and host view:

\begin{minted}{typescript}
class ComponentRef_ extends ComponentRef<any> {
  public readonly hostView: ViewRef;
  public readonly instance: any;
  public readonly changeDetectorRef: ChangeDetectorRef;
  private _elDef: NodeDef;

  constructor(
    private _view: ViewData,
    private _viewRef: ViewRef,
    private _component: any
  ) {
    super();
    this._elDef = this._view.def.nodes[0];
    this.hostView = _viewRef;
    this.changeDetectorRef = _viewRef;
    this.instance = _component;
  }
  get location(): ElementRef {
    return new ElementRef(
      asElementData(this._view, this._elDef.nodeIndex).renderElement
    );
  }
  get injector(): Injector {
    return new Injector_(this._view, this._elDef);
  }
  get componentType(): Type<any> {
    return <any>this._component.constructor;
  }

  destroy(): void {
    this._viewRef.destroy();
  }
  onDestroy(callback: Function): void {
    this._viewRef.onDestroy(callback);
  }
}
\end{minted}


The
\texttt{TemplateRef\_}
implementation has a constructor that takes a
\texttt{ViewData}
for the
parent and a
\texttt{NodeDef}
. Its
\texttt{createEmbeddedView}
method returns a new
\texttt{ViewRef\_}
based on those parameters.

\begin{minted}{typescript}
class TemplateRef_ extends TemplateRef<any> implements TemplateData {
  _projectedViews: ViewData[];
  constructor(private _parentView: ViewData, private _def: NodeDef) {
    super();
  }

  createEmbeddedView(context: any): EmbeddedViewRef<any> {
    return new ViewRef_(
      Services.createEmbeddedView(
        this._parentView,
        this._def,
        this._def.element!.template!,
        context
      )
    );
  }

  get elementRef(): ElementRef {
    return new ElementRef(
      asElementData(this._parentView, this._def.nodeIndex).renderElement
    );
  }
}
\end{minted}

