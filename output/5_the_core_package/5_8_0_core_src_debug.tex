\subsection{core/src/debug}

This directory contains this file:

\begin{itemize}
  \item debug\_node.ts
\end{itemize}

The debug\_node.ts file implements
\texttt{EventListener}
,
\texttt{DebugNode}
and
\texttt{DebugElement}
classes along with some helper functions.
\texttt{EventListener}
stores a name and a
function, to be called after events are detected:

\begin{minted}{typescript}
export class EventListener {
  constructor(public name: string, public callback: Function) {}
}
\end{minted}


The
\texttt{DebugNode}
class represents a node in a tree:

\begin{minted}{typescript}
export class DebugNode {
  nativeNode: any;
  listeners: EventListener[];
  parent: DebugElement | null;

  constructor(
    nativeNode: any,
    parent: DebugNode | null,
    private _debugContext: DebugContext
  ) {
    this.nativeNode = nativeNode;
    if (parent && parent instanceof DebugElement) {
      parent.addChild(this);
    } else {
      this.parent = null;
    }
    this.listeners = [];
  }
  get injector(): Injector {
    return this._debugContext.injector;
  }
  get componentInstance(): any {
    return this._debugContext.component;
  }
  get context(): any {
    return this._debugContext.context;
  }
  get references(): { [key: string]: any } {
    return this._debugContext.references;
  }
  get providerTokens(): any[] {
    return this._debugContext.providerTokens;
  }
}
\end{minted}


The debug node at attached as a child to the parent
\texttt{DebugNode}
. The
\texttt{nativeNode}
to
which this
\texttt{DebugNode}
refers to is recorded. A private field,
\texttt{\_debugContext}
records
supplies additional debugging context. It is of type
\texttt{DebugContext}
, defined in
view/types.ts as:

\begin{minted}{typescript}
export abstract class DebugContext {
  abstract get view(): ViewData;
  abstract get nodeIndex(): number | null;
  abstract get injector(): Injector;
  abstract get component(): any;
  abstract get providerTokens(): any[];
  abstract get references(): { [key: string]: any };
  abstract get context(): any;
  abstract get componentRenderElement(): any;
  abstract get renderNode(): any;
  abstract logError(console: Console, ...values: any[]): void;
}
\end{minted}


The
\texttt{DebugElement}
class extends
\texttt{DebugNode}
and supplies a debugging representation
of an element.

\begin{minted}{typescript}
export class DebugElement extends DebugNode {
  name: string;
  properties: { [key: string]: any };
  attributes: { [key: string]: string | null };
  classes: { [key: string]: boolean };
  styles: { [key: string]: string | null };
  childNodes: DebugNode[];
  nativeElement: any;

  constructor(nativeNode: any, parent: any, _debugContext: DebugContext) {
    super(nativeNode, parent, _debugContext);
    this.properties = {};
    this.attributes = {};
    this.classes = {};
    this.styles = {};
    this.childNodes = [];
    this.nativeElement = nativeNode;
  }
  ..
}
\end{minted}


It includes these for adding and removing children:

\begin{minted}{typescript}
  addChild(child: DebugNode) {
    if (child) {
      this.childNodes.push(child);
      child.parent = this;
    }
  }
  removeChild(child: DebugNode) {
    const childIndex = this.childNodes.indexOf(child);
    if (childIndex !== -1) {
      child.parent = null;
      this.childNodes.splice(childIndex, 1);
    }
  }
\end{minted}


It includes this for events:

\begin{minted}{typescript}
  triggerEventHandler(eventName: string, eventObj: any) {
    this.listeners.forEach((listener) => {
      if (listener.name == eventName) {
        listener.callback(eventObj);
      }
    });
  }
\end{minted}


The functions manage a map:

\begin{minted}{typescript}
// Need to keep the nodes in a global Map so that
// multiple angular apps are supported.
const _nativeNodeToDebugNode = new Map<any, DebugNode>();
\end{minted}


This is used to add a node:

\begin{minted}{typescript}
export function indexDebugNode(node: DebugNode) {
  _nativeNodeToDebugNode.set(node.nativeNode, node);
}
\end{minted}

